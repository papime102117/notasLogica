\documentclass[11pt,letterpaper]{article}
\usepackage{../packageslc}
\usepackage{../optionslc}

%%%% Macros para las notas de lenguajes de programacion


%%%% math

\newcommand{\vphi}{\varphi}
\newcommand{\vp}{\varphi}

% \newcommand{\dn}{\mathsf{DN}}
% \newcommand{\dnC}{\mathsf{DN_C}}
% \newcommand{\dnM}{\mathsf{DN_M}}
% \newcommand{\dnp}{\mathsf{DN_p}}
% \newcommand{\dnm}{\mathsf{DN_p^M}}
% \newcommand{\dnc}{\mathsf{DN_p^C}}

%\newcommand{\case}{\mathsf{case}}
%\renewcommand\labelitemi{$\circ$}

\newcommand{\imp}{\rightarrow}
\newcommand{\Imp}{\Rightarrow}
\renewcommand{\iff}{\leftrightarrow}
\newcommand{\Iff}{\Leftrightarrow}
\newcommand{\G}{\Gamma}
\newcommand{\D}{\Delta}


\newcommand{\De}{\mathcal{D}}
\newcommand{\F}{\mathcal{F}}
\newcommand{\Ge}{\mathcal{G}}
\newcommand{\Pe}{\mathcal{P}}
\newcommand{\I}{\mathcal{I}}
\newcommand{\C}{\mathcal{C}}
\newcommand{\K}{\mathcal{K}}
\renewcommand{\L}{\mathcal{L}}
\newcommand{\M}{\mathcal{M}}
\newcommand{\Nc}{\mathcal{N}}
%\newcommand{\E}{\mathcal{E}}
%\newcommand{\R}{\mathcal{R}}
%\newcommand{\Q}{\mathcal{Q}}
\newcommand{\Sc}{\mathcal{S}}
\newcommand{\Te}{\mathcal{T}}
\newcommand{\W}{\mathcal{W}}

\newcommand{\Db}{\mathbb{D}}
\newcommand{\Fb}{\mathbb{F}}
\newcommand{\Kb}{\mathbb{K}}
\newcommand{\Eb}{\mathbb{E}}
\newcommand{\Ebs}{\mathbb{E}^\star}
\newcommand{\Ob}{\mathbb{O}}
\newcommand{\Ib}{\mathbb{I}}
\newcommand{\Rb}{\mathbb{R}}
\newcommand{\Qb}{\mathbb{Q}}
\newcommand{\Kbb}{\mathbb{K}}
\newcommand{\T}{\mathbb{\Theta}}


\newcommand{\kb}{\bbkappa}

\newcommand{\Sf}{\mathsf{\Sigma}}

\newcommand{\fa}{\forall}
\newcommand{\ex}{\exists}

\newcommand{\inc}{\subseteq}

\newcommand{\Lb}{\Lambda}
\newcommand{\Om}{\Omega}
\newcommand{\lb}{\lambda}
\newcommand{\al}{\alpha}
\newcommand{\ga}{\gamma}


\newcommand{\mg}{\mathbb{m}}

\newcommand{\cg}{\mathbb{C}}
\newcommand{\dg}{\mathbb{D}}
\newcommand{\jg}{\mathbb{J}}
\newcommand{\Ha}{\mathcal{H}}
%\newcommand{\A}{\mathcal{A}}
\newcommand{\sg}{\mathbb{S}}

\newcommand{\Bc}{\mathcal{B}}
\newcommand{\Df}{\mathfrak{D}}
\newcommand{\Dc}{\mathcal{D}}
%\newcommand{\Tc}{\mathcal{T}}
\newcommand{\Mf}{\mathfrak{M}}

\newcommand{\Sg}{\mathbb{S}}

\newcommand{\Q}{\ensuremath{\mathbb{Q}}}
\newcommand{\Z}{\ensuremath{\mathbb{Z}}}
\newcommand{\N}{\ensuremath{\mathbb{N}}}
\newcommand{\R}{\ensuremath{\mathbb{R}}}
\renewcommand{\S}{\mathbb{\Sigma}}
\newcommand{\A}{\mathcal{A}}
\newcommand{\E}{\ensuremath{\exists}}
\newcommand{\iso}{\ensuremath{\cong}}
\newcommand{\union}{\ensuremath{\cup}}
\newcommand{\morinyec}{\ensuremath{\precapprox}}

\newcommand{\nin}{\ensuremath{\notin}}
\newcommand{\tog}{\makebox[7mm][l]}
\newcommand{\toge}{\makebox[11mm][l]}
\newcommand{\toget}{\makebox[13mm][l]}
\newcommand{\togeth}{\makebox[14mm][l]}
\newcommand{\togethe}{\makebox[15mm][l]}
\newcommand{\together}{\makebox[17mm][l]}
\newcommand{\niso}{\ensuremath{\not \cong}}


\newcommand{\Mg}{\mathbb{M}}
\newcommand{\Bg}{\mathbb{B}}
\newcommand{\Lg}{\mathbb{L}}
\newcommand{\Tg}{\mathbb{T}}

\newcommand{\sketch}{\Red{{\sc sketch}}}

\newcommand{\restr}[2]{#1\!\!\boldsymbol{\restriction}\!#2}

\newcommand{\vacio}{\varnothing}
\newcommand{\done}{\ensuremath{\checkmark}}

\newcommand{\ida}{$\Rightarrow \; )$ }
\newcommand{\regr}{$\Leftarrow \; )$ }

\newcommand{\ol}[1]{\overline{#1}}

\newcommand{\Tsf}{\mathsf{T}}

\newcommand{\inds}[1]{\index[simb]{#1}}

\newcommand{\B}{\mathbb{B}}
%\newcommand{\N}{\mathbb{N}}

\newcommand{\vx}{\vec{x}}
\newcommand{\vy}{\vec{y}}
\newcommand{\vz}{\vec{z}}
\newcommand{\vt}{\vec{t}}
\newcommand{\vf}{\vec{f}}


% \newcommand{\propo}{\ensuremath{\mathsf{PROP}}}
% \newcommand{\atom}{\ensuremath{\mathsf{ATOM}}}
\newcommand{\term}{\ensuremath{\mathsf{TERM}}}
\newcommand{\form}{\mathsf{FORM}}

\newcommand{\true}{\mathop{\mathsf{true}}}

%\newcommand{\id}{\mathsf{Id}}

%\newcommand{\uc}{\mathcal{U}}
%\newcommand{\Ic}{\mathcal{I}}
%\newcommand{\pc}{\mathcal{P}}
%\newcommand{\qc}{\mathcal{Q}}
%\newcommand{\mc}{\mathcal{M}}
\newcommand{\supc}{\supseteq}
\newcommand{\limo}{\mathop{\mathpzc{Lim}}}
\newcommand{\ord}{\mathsf{OR}}

\newcommand{\pt}[1]{\langle #1 \rangle}


%%%% frames
\newcommand{\titulos}[2]{\frametitle{#1}\framesubtitle{#2}}
\newcommand{\fot}[1]{\footnote{\scriptsize{#1}}}


%%%% ambientes

\newcommand{\cb}[2]{\colorbox{#1}{#2}}

\newcommand{\bc}{\begin{center}}
\newcommand{\ec}{\end{center}}
\newcommand{\be}{\begin{enumerate}}
\newcommand{\ee}{\end{enumerate}}
\newcommand{\bi}{\begin{itemize}}
\newcommand{\ei}{\end{itemize}}
\newcommand{\beq}{\begin{equation}}
\newcommand{\eeq}{\end{equation}}
\newcommand{\beqs}{\begin{equation*}}
\newcommand{\eeqs}{\end{equation*}}
\newcommand{\ba}{\begin{array}}
\newcommand{\ea}{\end{array}}


% \newtheorem{theorem}{Teorema}
% \newcommand{\teo}[1]{\begin{theorem} #1 \end{theorem}}
% \newtheorem{proposition}{Proposici\'on}
% \newcommand{\prop}[1]{\begin{proposition} #1 \end{proposition}}
% \newtheorem{definition}{Definici\'on}
% \newcommand{\defin}[1]{\begin{definition} #1 \end{definition}}
% \newtheorem{corollary}{Corolario}
% \newcommand{\cor}[1]{\begin{corollary} #1 \end{corollary}}
% \newtheorem{lemma}{Lema}
% \newcommand{\lema}[1]{\begin{lemma} #1 \end{lemma}}
% \newcommand{\dem}[1]{\begin{proof} #1 \end{proof}}

%\renewcommand{\qed}{\qedsymbol{$\mathbf{\dashv}$}}

%\newcommand{\proof}{\hfill\\\noindent\textbf{\textit{Demostraci\'on. }}}

\newcommand{\hint}{\emph{Sugerencia: }}


\newcounter{EjempCtr}[section]
\newenvironment{enumrom}{\renewcommand{\theenumi}{\roman{enumi}}%
\renewcommand{\theenumii}{\roman{enumii}}
\renewcommand{\theenumiii}{\roman{enumiii}}
\renewcommand{\theenumiv}{\roman{enumiv}}
\begin{enumerate}}{\end{enumerate}}
\newenvironment{Ejemplo}
        {\stepcounter{EjempCtr}%
        \begin{description}\item[Ejemplo \thesection.\arabic{EjempCtr}]}%
        {\end{description}}
\newenvironment{demostr}{%
             {\em Demostración:}
                \begin{quotation}}{\end{quotation}}

   \newcommand{\beje}{\begin{Ejemplo}}
\newcommand{\eeje}{\end{Ejemplo}}


\newtheorem{eje}{Ejemplo}[section]
\newcommand{\ejem}[1]{\begin{eje}\normalfont #1 \end{eje}}

% \renewcommand\contentsname{\'Indice}
%\renewcommand\chaptername{Cap\'itulo}
% \renewcommand\indexname{\'Indice}

%%\newcommand{\qed}{\hfill$\mathbb{Qed}$}
%\newcommand{\qed}{\hfill$\mathsf{\boldsymbol{\dashv}}$}
%\renewcommand{\qed}{\hfill$\boldsymbol{\dashv}$}


\newenvironment{prueba}{\vspace{-5mm}\noindent\textbf{Demostraci\'on}\\}{
\noindent$\blacksquare$\\}

\newcommand{\Ejercicios}{\section*{Ejercicios}}

 \newenvironment{manitas}{%
      \renewcommand{\labelitemi}{\ding{44}}%
      \vspace{-0.5cm}%
      \begin{itemize}%
      \setlength{\itemsep}{0pt}\setlength{\parsep}{0pt}\setlength{\topsep}{0pt}%
      }{\end{itemize}}
\newenvironment{malitos}{%
      \renewcommand{\labelitemi}%
            {\raisebox{1.5ex}{\makebox[0.3cm][l]{\begin{rotate}{-90}%
            \ding{43}\end{rotate}}}}%
      \vspace{-0.5cm}%
      \begin{itemize}%
      \setlength{\itemsep}{0pt}\setlength{\parsep}{0pt}\setlength{\topsep}{0pt}%
      }{\end{itemize}}
\newenvironment{ejercs}{
     \renewcommand{\labelenumi}{\thesection.\theenumi.-}
     \renewcommand{\labelenumii}{\theenumii)}
     \begin{enumerate}}
     {\end{enumerate}}

   \newcommand{\bej}{\begin{ejercs}}
\newcommand{\eej}{\end{ejercs}}


%\newenvironment{leterize}{%
%        \renewcommand{\theenumi}{\alph{enumi}}
%        \begin{enumerate}}{\end{enumerate}}

%\newenvironment{manitas}{%
%      \renewcommand{\labelitemi}{\ding{44}}%
%      \vspace{-0.5cm}%
%      \begin{itemize}%
%      
% \setlength{\itemsep}{0pt}\setlength{\parsep}{0pt}\setlength{\topsep}{0pt}%
%      }{\end{itemize}}



%%=============================================================================

\def\stackunder#1#2{\mathrel{\mathop{#2}\limits_{#1}}}


%%%% notas

\newcommand{\doubt}{\Red{{\LARGE {\sf ??}}}}

\newcommand{\coment}[1]{\hfill\\ \Big[{\bf Comentario Privado:} #1\Big]}
\newcommand{\preg}[1]{\hfill\\ \BrickRed{{\bf Pregunta:} #1}}
\newcommand{\conjet}[1]{\hfill\\ \OliveGreen{{\bf Conjecura:} #1}}

\newcommand{\pendiente}{\BrickRed{{\sc Pendiente}}}
\newcommand{\verifpendiente}{\BrickRed{{\sc Verificación pendiente}}}


%--------------------------------------------------------------------------

\DeclareMathAlphabet{\mathpzc}{OT1}{pzc}{m}{it}



\title{Tableaux Sem\'anticos \\ 
L\'ogica Computacional 2018-2, Nota de clase 8}
\author{Favio Ezequiel Miranda Perea\and Araceli Liliana Reyes Cabello\and
Lourdes Del Carmen Gonz\'alez Huesca \and Pilar Selene Linares Ar\'evalo}
\date{ Facultad de Ciencias UNAM \\ 12 de abril de 2018 \\
Material desarrollado bajo el proyecto UNAM-PAPIME PE102117}
\begin{document}

\maketitle

\section{Introducción}
Los \emph{tableaux} o \emph{tablas semánticas} son un método de
demostración por refutación, el cual fue introducido en los años 1950 
de manera independiente por Hintikka y Beth. 
Este método se basa en la semántica más que en la sintáxis de las fórmulas. Sin
embargo, se requiere de una clasificación de las fórmulas basada en sintaxis.

Esta teor\'ia de prueba utiliza el principio de refutación: dado un 
conjunto de premisas~$\Gamma$ y una conclusión~$\varphi$, se debe mostrar que el
conjunto $\Gamma\cup \{ \lnot \varphi \}$ no tiene un 
modelo~\footnote{Recordemos que un modelo de un conjunto de fórmulas $\Gamma$, 
es una interpretación $\I$ tal que para cualquier estado de las 
variables~$\sigma$ se cumple $\I_\sigma(\Gamma)=1$.},
lo cual es equivalente a que $\G\models\vp$ es decir, a que
$\varphi$ sea consecuencia lógica de $\Gamma$.

Es Smullyan quien estandariza la notación y realiza la clasificación de las
fórmulas en cuatro tipos identificando su esquema:
\bi
\item \emph{tipo $\alpha$}: las fórmulas conjuntivas.
\item \emph{tipo $\beta$}: las fórmulas disyuntivas.
\item \emph{tipo $\gamma$}: las fórmulas universales.
\item \emph{tipo $\delta$}: las f\'ormulas existenciales.
\ei
De acuerdo a su tipo, una fórmula genera una extensión en el tableau. 
Veamos ahora los preliminares para despu\'es detallar el método de prueba.

% Las siguientes equivalencias lógicas nos servirán para clasificar las
% fórmulas:
% \beqs
% \lnot(\varphi \lor \psi)\equiv\lnot \varphi \land \lnot \psi\;\;\;\;\;
% \lnot(\varphi \imp \psi)\equiv\varphi \land \lnot \psi
% \eeqs
% \beqs
% \lnot(\varphi \land\psi)\equiv\lnot \varphi \lor \lnot \psi\;\;\;\;\;
% \varphi \imp \psi\equiv\lnot \varphi \lor  \psi
% \eeqs
% \beqs
% \lnot(\varphi \syss \psi)\equiv\lnot \varphi \syss \psi\equiv
% (\lnot \varphi \land  \psi) \lor \lnot ( \varphi \land \lnot \psi)
% \;\;\;\;\;\;\varphi \syss \psi\equiv
%        (\varphi \land \psi) \lor (\lnot \varphi \land\lnot \psi)
% \eeqs
% \beqs
% \neg \fa x\vp\equiv\ex x\vp\;\;\;\;\;\neg\ex x \vp\equiv\fa x\lnot\vp
% \eeqs


\begin{definicion}
  Una literal es una fórmula atómica o la negación de una fórmula atómica.
\end{definicion}

\begin{definicion} El conjunto de fórmulas que no son literales se clasifica en
 cuatro tipos, en cada tipo de fórmula se distinguen ciertas
 subfórmulas necesarias posteriormente:
 \begin{enumerate}
  \item[] \textbf{Tipo $\alpha$}: si la fórmula es conjuntiva o es equivalente
   a una fórmula conjuntiva. \\
   Una fórmula $\chi \in \form$ es tipo $\alpha$, si tiene alguna de las
   siguientes formas:
   \begin{enumerate} 
   \item  $\chi = \vp \land \psi$, con subfórmulas $\alpha_{1} = \vp$ y
     $\alpha_{2} = \psi$.

   \item  $\chi = \lnot(\vp \lor \psi)$, con subfórmulas 
   $\alpha_{1} = \lnot \vp$ y  $\alpha_{2} = \lnot \psi$.

   \item   $\chi = \lnot(\varphi \imp \psi)$, con subfórmulas  
   $\alpha_{1} = \vp$ y  $\alpha_{2} = \lnot \psi$.
   \end{enumerate}
% Las fórmulas $\alpha_{1}$ y $\alpha_{2}$ son llamadas 
% \emph{$\alpha$-subfórmulas} de $\varphi$.

  \item[] \textbf{Tipo $\beta$}: si la fórmula es disyuntiva o es equivalente
   a una fórmula disyuntiva. \\
   Una fórmula $\chi \in \form$ es tipo $\beta$, si tiene alguna de las
   siguientes formas:
   \begin{enumerate}
   \item  $\chi = \vp \lor \psi$, con subfórmulas $\beta_{1} = \varphi$ y
    $\beta_{2} = \psi$.

   \item $\chi = \lnot(\vp \land \psi)$, con subfórmulas  
    $\beta_{1} = \lnot \varphi$ y $\beta_{2} = \lnot \psi$.

   \item $\chi = \vp \imp \psi$, con subfórmulas $\beta_{1} = \lnot \vp$
    y $\beta_{2} = \psi$.

   \item $\chi = \varphi \syss \psi$, con subfórmulas 
   $\beta_{1} = \varphi \land \psi$ y $\beta_{2} = \lnot \vp \land \lnot \psi$.

   \item $\chi = \lnot ( \varphi \syss \psi)$, con subfórmulas  
   $\beta_{1} = \lnot \varphi \land  \psi$ y $\beta_{2} = \vp\land \lnot \psi$.
   \end{enumerate}
% Las fórmulas $\beta_{1}$ y $\beta_{2}$ se llaman \emph{$\beta$-subfórmulas} 
% de $\chi$.

  \item[] \textbf{Tipo $\gamma$}: si la fórmula está cuantificada universalmente 
  o es equivalente a una fórmula cuantificada universalmente. \\
  Una fórmula $\varphi$ es de tipo $\gamma$, si tiene alguna de las 
  siguientes formas:
  \begin{enumerate}
   \item  $\chi = \forall x \varphi$.
   \item  $\chi = \lnot \exists x \varphi$.
    \end{enumerate}
% Cualquier fórmula $\varphi[x:=t]$ con $t$ un término cerrado se llama 
% \emph{$\gamma$-subfórmula} de $\chi$.     

 \item[] \textbf{Tipo $\delta$}: si la fórmula está cuantificada 
  existencialmente o es equivalente a una fórmula cuantificada 
  existencialmente. \\
  Una fórmula $\varphi$ es tipo $\gamma$, si tiene alguna de las  siguientes 
  formas:
  \begin{enumerate}
   \item  $\chi = \exists x \varphi$.
   \item  $\chi = \lnot \forall x \varphi$.
  \end{enumerate}
% Cualquier fórmula $\varphi[x:=t]$ con $t$ un término cerrado se
% llama \emph{$\delta$-subfórmula} de $\chi$.     
\end{enumerate}
\end{definicion}
De esta manera cualquier fórmula pertenece a una de cinco categor\'ias:
literales, $\al,\;\beta,\;\gamma$ ó $\delta$.\\
Obsérvese que algunas equivalencias en fórmulas tipo~$\beta$ o sus negaciones
pueden clasificarse como fórmulas~$\al$, pero la clasificación elegida aquí es 
óptima para construir los tableaux, en particular, la negación de una fórmula 
$\beta$ no es necesariamente una fórmula $\alpha$.

% Notemos que de la clasificación presentada se sigue el siguiente resultado.

% \lema{\label{correctezReglas}
%   Sea $\varphi$ una fórmula, $\mathcal{M} = \langle M , \mathcal{I} \rangle$ y
%   $\nu$ una asignación de variables,  se cumple que:
%   \begin{enumerate}
%   \item $\mathcal{I}_{\nu}(\varphi) = 1$  con $\varphi$ una fórmula de tipo
%     $\alpha$  syss $\mathcal{I}_{\nu}(\alpha_{1}) = 1$ y 
% $\mathcal{I}_{\nu}(\alpha_{2}) = 1$ 
%     con $\alpha_{1}$ y $\alpha_{2}$ $\alpha$-subfórmulas de $\varphi$.

%   \item $\mathcal{I}_{\nu}(\varphi) = 1$  con $\varphi$ una fórmula de tipo
%     $\beta$  syss $\mathcal{I}_{\nu}(\beta_{1}) = 1$ o 
% $\mathcal{I}_{\nu}(\beta_{2}) = 1$ 
%     con $\beta_{1}$ y $\beta_{2}$ $\beta$-subfórmulas de $\varphi$.

%  \item $\mathcal{I}_{\nu}(\varphi) = 1$  con $\varphi=\forall x\psi$ una 
% fórmula de tipo
%     $\gamma$  syss $\mathcal{I}_{\nu}(\psi([x:=t])) = 1$ para todo término
%     cerrado\footnote{Recordemos que un término cerrado es un término
%       que no contiene variables.} $t$.

%  \item $\mathcal{I}_{\nu}(\varphi) = 1$  con $\varphi = \exists \psi$ una 
% fórmula de tipo
%     $\beta$  syss $\mathcal{I}_{\nu}(\psi([x:=t])) = 1$ con $t$ un término 
% cerrado
%     nuevo que no aparecía en la ramma a la cual pertenece $\varphi$.
%   \end{enumerate}
% }

% \emph{Demostración}: La demostración es directa de la definición de
% satisfacción de fórmulas. Y se deja como ejercicio al lector.\qed

%\begin{proposicion}
% Sean $\varphi$ una fórmula y  $\mathcal{I}$ una interpretación, se cumple:
% \begin{enumerate}
% \item Si $\varphi$ es una fórmula de tipo
%   $\alpha$, entonces $\mathcal{I}$ satisface $\varphi$ syss
%   $\mathcal{I}$ satisface la $\alpha_{1}$-subfórmula y la 
% $\alpha_{2}$-subfórmula de $\varphi$.

% \item Si $\varphi$ es una fórmula de tipo $\beta$, entonces $\mathcal{I}$ 
% satisface $\varphi$  syss $\mathcal{I}$ satisface a alguna de las dos
%   $\beta$-subfórmulas  de $\varphi$,
%   $\beta_{1}$ o $\beta_{2}$.
% \end{enumerate}
%\end{proposicion}

\section{Tableaux Semánticos}

Un tableau es un árbol donde cada nodo está etiquetado con una
fórmula y tiene a lo más dos hijos. La construcción del árbol la dictan las reglas de
extensión que introducimos enseguida. Inicialmente el árbol únicamente
contiene una rama en donde cada uno de los nodos que la conforman
están etiquetados con una de las fórmulas del conjunto $\Gamma \cup
\{\lnot \varphi \}$, si lo que queremos probar es que 
$\Gamma \models\varphi$; o bien, $\Gamma$ si lo que queremos probar es que este 
conjunto es insatisfacible.  
% A cada uno de estos conjuntos le llamaremos \emph{conjunto inicial}.

\begin{definicion}
 Las reglas de extensión de tableaux se definen de acuerdo a la
 clasificación antes mencionada, y son:
 \begin{enumerate}
 \item Regla de extensión $\al$: 
% Para las fórmulas del tipo $\alpha$ (conjuntivas), la \emph{$\alpha$-regla}
% se define de la siguiente forma, dependiendo de la sintáxis de $\varphi$:
 \begin{center}
%   \boldmath
\tikzstyle{level 1}=[level distance=10mm]
   \tikz{\node{$\vp\land \psi$}
       child{node{$\vp$}
             child{node{$\psi$}}
            }
     ;
     }
\end{center}
%   \begin{center}
%      \begin{displaymath}
%        \begin{array}{ccccc}
%          \underline{\varphi \land \psi} & \hspace{30pt} &
%          \underline{\lnot ( \varphi \lor
%            \psi)}  & \hspace{30pt} & \underline{\lnot(\varphi \imp
%            \psi )} \\
%          \varphi  & \hspace{30pt} & \lnot \varphi & \hspace{30pt} &
%          \varphi \\
%          \psi & \hspace{30pt}& \lnot \psi & \hspace{30pt}&\lnot
%          \psi \\
%        \end{array}
%      \end{displaymath}
%  \end{center}
%  Se agregan dos nuevos nodos a la rama $\Gamma$, es decir a la rama que 
%  contiene a todas las fórmulas de $\Gamma$.  que contiene a
%  $\varphi$, etiquetados con las $\alpha$-subfórmulas $\alpha_{1}$ y
%  $\alpha_{2}$. % En símbolos, $\Gamma + \alpha_{1} + \alpha_{2}$, donde 
%  $\Gamma$ corresponde a la rama que contiene a $\varphi$.
% \newpage
 \item Regla de extensión $\beta$:% -regla correspondiente a las fórmulas del 
% tipo $\beta$ (disyuntivas),
%    opera de la siguiente manera, dependiendo de la forma que tenga $\varphi$:
 \begin{center}
%   \boldmath
\tikzstyle{level 1}=[level distance=15mm]
   \tikz{\node{$\vp\lor \psi$}
     child{node{$\vp$}}
     child{node{$\psi$}}
     ;
     }
\end{center}
% \begin{center}
% %\;\;\;\;\;\;\;\;\;
%  %  \boldmath
%    \tikz{\node{$\vp\iff \psi$}
%      child{node{$\vp\land\psi$}}
%      child{node{$\neg\vp\land\neg\psi$}}
%      ;
%      }
% \;\;\;\;\;\;\;\;\;
%  %  \boldmath
%    \tikz{\node{$\neg(\vp\iff \psi)$}
%      child{node{$\neg\vp\land\psi$}}
%      child{node{$\vp\land\neg\psi$}}
%      ;
%      }
%  \end{center}

%    \begin{center}
%      \begin{displaymath}
%        \begin{array}{ccccccccc}
%          \underline{\varphi \lor \psi} & \hspace{10pt} &
%          \underline{\lnot(\varphi \land \psi)} & \hspace{5pt} &
%          \underline{\varphi \imp \psi} & \hspace{5pt} &
%          \underline{\hspace{25pt}  \varphi \syss \psi
%            \hspace{27pt}} &
%          \underline{\hspace{20pt}  \lnot (\varphi \syss \psi)
%            \hspace{20pt}} \\

%          \varphi \textrm{ } | \textrm{ } \psi & \hspace{5pt} & \lnot
%          \varphi \textrm{ } | \textrm{ } \lnot \psi &  &
%          \lnot \varphi \textrm{ } | \textrm{ } \psi  &  &
%          \varphi \land \psi \textrm{ } | \textrm{ } \lnot \varphi \land  \lnot 
% \psi    &
%          \lnot \varphi \land \psi \textrm{ } | \textrm{ }
%          \varphi \land           \lnot \psi
%        \end{array}
%      \end{displaymath}
%    \end{center}

%        La regla $\beta$ agrega un nodo izquierdo etiquetado con la 
% $\beta_{1}$-subfórmula y 
%    un nodo derecho etiquetado con la $\beta_{2}$-subfórmula. % en símbolos,
% %    $\Gamma  + \beta_{1}$ y $\Gamma + \beta_{2}$, donde $\Gamma$ es la rama
% %   que contiene a la formula $\varphi$.


%  \begin{itemize}
  \item Regla de extensión $\ga$:
% Si $\varphi$ es de tipo $\gamma$, la regla que se aplicará es la
% \emph{$\gamma$-regla}:
  \begin{center}
  \tikzstyle{level 1}=[level distance=12mm]
    \tikz{\node{$\fa x\vp$}
       child{node{$\vp[x:=t]$}
            }
     ;
     }
\end{center}
%     \begin{displaymath}
%         \begin{array}{c}
%           \forall x \; \varphi\\ \hline
%           \varphi[x:= t] \\
%         \end{array}
%        \end{displaymath}
% La regla $\ga$ agrega un nuevo nodo etiquetado con la $\ga$-subfórmula 
% $\varphi[x:=t]$, 
donde $t$ es cualquier término cerrado.  Decimos que las fórmulas
tipo $\gamma$ siempre están activas.
% Cualquier nodo etiquetado 
% % con una fórmula tipo $\gamma$ puede ser reutilizado para realizar una
% % extensión.
% En otras palabras, si $\Gamma$ es la rama que contiene a  $\varphi$ y 
% $\Gamma$ no es cerrada siempre es posible extender a $\Gamma$
% utilizando la  $\gamma$-regla, obteniendo, $\Gamma + \varphi[x:=t]$, con
% $t$ cualquier término cerrado.\footnote{Podemos decir que las fórmulas
% tipo $\gamma$ siempre están activas.}
          
  \item Regla de extensión $\delta$: 
% Si $\varphi$ es de tipo $\delta$, la regla que debe aplicarse es la
% \emph{$\delta$-regla}:
  \begin{center}
  \tikzstyle{level 1}=[level distance=12mm]
    \tikz{\node{$\ex x\vp$}
       child{node{$\vp[x:=c]$}
            }
     ;
     }
\end{center}
donde $c$ es una constante nueva, es decir una constante que no figura
antes en la rama.
%       \begin{displaymath}
%         \begin{array}{c}
%           \exists x \; \varphi \\ \hline
%           \varphi[x:=c] 
%         \end{array}
%       \end{displaymath}
% La regla $\delta$ agrega un nuevo nodo etiquetado con la 
% $\delta$-subfórmula $\varphi[x:=c]$, donde  $c$
% es una {\bf constante nueva}, es decir, una constante que no figura
% antes en la rama. % La extensión es la siguiente: $\Gamma + \psi[x:=c]$. 
%  % \end{itemize}
 \end{enumerate}
\end{definicion}
Como se puede apreciar, cada extensi\'on de la regla agreaga a lo m\'as dos 
nodos con las subf\'ormulas correspondientes.\\


A pesar de que se consideran algunas f\'ormulas equivalentes en la 
clasificaci\'on, se debe enfatizar que su uso respeta las reglas de extensi\'on 
y no debe procederse a una transformaci\'on de una f\'ormula en un principio.
Veamos los casos de las reglas aplicadas a las f\'ormulas equivalentes:
\begin{enumerate}
 \item Extensión de f\'ormulas equivalentes tipo~$\al$: 
 \begin{center}
\tikzstyle{level 1}=[level distance=10mm]
  \tikz{\node{$\neg(\vp\lor \psi)$}
       child{node{$\neg\vp$}
             child{node{$\neg\psi$}}
            }
     ;
     }
\qquad \qquad 
  \tikz{\node{$\neg(\vp\to\psi)$}
       child{node{$\vp$}
             child{node{$\neg\psi$}}
            }
     ;
     }
  
\end{center}
\item Extensión de f\'ormulas equivalentes tipo~$\beta$:
\begin{center}
\tikzstyle{level 1}=[level distance=15mm]
\tikzstyle{level 1}=[sibling distance= 20mm]
   \tikz{\node{$\neg(\vp\land \psi)$}
     child{node{$\neg \vp$}}
     child{node{$\neg \psi$}}
     ;
     }
 \quad \quad
   \tikz{\node{$\vp\to \psi$}
     child{node{$\neg \vp$}}
     child{node{$\psi$}}
     ;
     }
\quad \qquad
   \tikz{\node{$\vp\iff \psi$}
     child{node{$\vp\land \psi$}}
     child{node{$\lnot\vp\land\lnot\psi$}}
     ;
     }
\quad \qquad
   \tikz{\node{$\lnot(\vp\iff \psi)$}
     child{node{$\neg \vp \land \psi$}}
     child{node{$\vp \land \lnot \psi$}}
     ;
     }

     
\end{center}

\item Extensión de f\'ormulas equivalentes tipo~$\gamma$:
\begin{center}
  \tikzstyle{level 1}=[level distance=12mm]
    \tikz{\node{$\lnot\exists x\vp$}
       child{node{$\lnot \vp[x:=t]$}
            }
     ;
     }
\end{center}

\item Extensión de f\'ormulas equivalentes tipo~$\delta$:
\begin{center}
  \tikzstyle{level 1}=[level distance=12mm]
    \tikz{\node{$\lnot \fa x\vp$}
       child{node{$\lnot \vp[x:=a]$}
            }
     ;
     }
\end{center}


\end{enumerate}

A continuación definimos formalmente un tableau semántico:

 \begin{definicion}[\emph{Tableaux semánticos}]
 Sea $\Gamma = \{\varphi_{1},\varphi_{2},\ldots , \varphi_{n}\}$ un conjunto de
 fórmulas, un tableau para $\Gamma$ denotado $\mathcal{T}(\Gamma)$ es una 
 extensión obtenida recursivamente como sigue:
 \begin{enumerate}
 \item El árbol formado por la rama cuyos nodos son $\varphi_{1}, \varphi_{2},
   \ldots , \varphi_{n}$ es un tableau para $\Gamma$.

 \item Si $\mathcal{T}_{1}$ es un  tableau para $\Gamma$, entonces la extensión
   que se obtiene después de aplicarle alguna de las reglas $\alpha,\;
   \beta,\;\gamma,\;\delta$ es un tableau para $\Gamma$.

   \item Son todos 
   % Únicamente son tableaux de $\Gamma$ aquellos que se obtienen
   %  utilizando las reglas mencionadas en 1 y 2.
 \end{enumerate}
\end{definicion}

Dado un conjunto inicial $\Gamma$ denotamos con $\mathcal{T}(\Gamma)$ a
cualquier extensión de $\Gamma$, obtenida mediante las reglas arriba 
definidas, la cual por supuesto no es única. \\
Las siguientes observaciones acerca del uso de las reglas de extensión son de 
gran importancia:

\bi
 \item Las reglas de extensión $\al,\beta,\delta$ se utilizan una 
  \textbf{única} vez para cada fórmula. Es decir, si $\vp$ es una fórmula a la 
  que se puede aplicar alguna extensión $\al,\beta,\delta$ entonces una vez 
  hecha la extensión la fórmula $\vp$ no puede usarse nuevamente para una 
  extensión. 
 \item En contraste la regla $\ga$ puede utilizarse para realizar una extensión 
  cualquier número de veces, por supuesto utilizando distintos términos 
  cerrados $t$ en cada ocasión. Por esto decimos que la regla $\ga$ permanece 
  siempre activa.
 \item Cualquier extensión de una fórmula $\vp$ debe realizarse sobre 
  \textbf{todas} las ramas del tableau a las que pertenece la fórmula $\vp$ y 
  únicamente sobre éstas.
 \item Para extender un tableau es conveniente preferir el uso de fórmulas 
  tipo~$\al$ sobre fórmulas tipo~$\beta$ para buscando ramificar el tableau lo menos posible.
 \item Similarmente el uso de la regla $\ga$ se debe posponer hasta que sea 
  estrictamente necesario.
 \item La regla $\ga$ debe utilizar en su extensión términos cerrados ya 
  presentes en el tableau, excepto cuando estos aún no existan. Únicamente en 
  este caso se permite aplicar la regla $\ga$ usando una nueva constante $a$.
\ei

\begin{definicion}[Tableau cerrado]
 Una rama $\rho$ % $\Gamma$ 
 de un tableau es \emph{cerrada } si tanto una literal (fórmula 
 atómica) $\varphi$ como su negación figuran en $\rho$. Un tableau es 
 \emph{cerrado} si todas sus ramas están cerradas.
\end{definicion}

\begin{definicion}[Tableau abierto]
 Una rama $\rho$ de una tableau es \emph{abierta} si  no es cerrada. Un
 tableau se considera \emph{abierto} si contiene al menos una rama abierta.
\end{definicion}


\subsection{Los teoremas de correctud y completud refutacional}

El método de tableaux se relaciona con la semántica de la lógica mediante los 
siguientes teoremas de correctud y completud refutacional.

\begin{teorema}[Correctud]
Si hay un tableau cerrado para $\Gamma$ entonces $\Gamma$ no tiene un modelo.
% $\Gamma$ es un conjunto de fórmulas de $\propo$ insatisfacible   syss
% existe un tablero $\mathcal{T}(\Gamma)$ cerrado.
\end{teorema}
\noindent
Es decir, si un tableaux con conjunto inicial $\G$ se cierra entonces
dicho conjunto es insatisfacible.

\begin{teorema}[Completud refutacional]
Si $\Gamma$ no tiene un modelo entonces existe un tableau $\mathcal{T}(\Gamma)$
cerrado. 
\end{teorema}
\noindent
Es decir si un conjunto inicial $\G$ es insatisfacible entonces 
existe una extensión $\Te(\G)$ cerrada.\\

\noindent
Con respecto a la consecuencia lógica se tienen los siguientes corolarios:

\begin{corolario}
 $\Gamma \models \varphi$ si y s\'olo si existe una extensión
 $\mathcal{T}(\Gamma \cup \{ \lnot\varphi \})$ cerrada.
\end{corolario}

\begin{corolario}
 $\models\varphi$  si y s\'olo si existe una extensión $\mathcal{T}(\lnot
 \varphi)$ cerrada.
\end{corolario}


% \section{Decidibilidad de la lógica de proposiciones}

% En esta sección mostraremos la decidibilidad de la lógica de
% proposiciones $\propo$ usando el método de tableaux. De manera que
% cualquier resultado se referirá exclusivamente a proposiciones.


% \begin{definicion}[Tableau proposicional completo]
%  Una rama $\Gamma$ es \emph{completa} si cumple lo siguiente:
%  \begin{itemize}
%  \item Para cada una de las fórmulas de $\Gamma$ de tipo $\alpha$, 
% sus $\alpha$-subfórmulas, $\alpha_{1}$ y $\alpha_{2}$, también están
% en $\Gamma$.
%  \item Para cada una de las fórmulas de $\Gamma$ de tipo $\beta$,
%    $\Gamma$ contiene alguna de las dos $\beta$-subfórmulas, $\beta_{1}$ o 
% $\beta_{2}$.
%  \end{itemize}
%  Un tableau es \emph{completo} si todas sus ramas son completas o cerradas.
% \end{definicion}

% %\begin{definicion}[Tableau abierto]
% % Una rama $\Gamma$ de una tableau es \emph{abierta} si  no es cerrada. Un
% % tableau se considera \emph{abierto} si contiene al menos una rama abierta.
% %\end{definicion}

% \subsection{Construcción de modelos}
% \renewcommand{\R}{\mathcal{R}}
% Los tableaux semánticos nos permiten construir modelos para los conjuntos de
% fórmulas satisfacibles y contraejemplos para argumentos no válidos.
% Si $\Gamma$ es un conjunto satisfacible, entonces  existe un  tableau de
% $\Gamma$ abierto y completo, de ahí que $\mathcal{T}(\Gamma)$ contenga una 
% rama
% $\R$  completa y abierta. Luego entonces, para construir un modelo que
% satisfaga $\Gamma$ basta con dar una asignación de
% variables $\mathcal{I}$ que satisfaga a la rama $\R$.

% \begin{definicion}
%  Una rama $\R$ de un $\mathcal{T}(\Gamma)$ es satisfacible si existe una
%  interpretación $\mathcal{I}$ que satisfaga cada una de las fórmulas que
%  ocurren en $\R$. Un tableaux $\mathcal{T}(\Gamma)$ es satisfacible si tiene
%  una rama satisfacible.
% \end{definicion}

% Si una rama $\R$ es completa y abierta entonces es satisfacible, siendo
% una interpretación $\mathcal{I}$ posible la siguiente:
% %\begin{center}
%  \begin{displaymath}
%     \mathcal{I}(p) = \left \{
%       \begin{array}{ll}
%         1 & \textrm{ si la variable } p \textrm{ es uno de los nodos
%           de la rama } \R \textrm{ o } p \textrm{ no figura en }
%         \R \\
%         0 & \textrm{ si la fórmula } \lnot p \textrm{ es uno de los nodos de
%           la rama } \R
%       \end{array} \right.
%  \end{displaymath}
% %\end{center}


% Obsérvese que podrían existir otros modelos, en realidad, si una
% variable $q$ no figura en la rama $\R$ entonces el valor de verdad de
% $q$ es irrelevante para el modelo de $\R$ de manera que $q$ puede ser
% verdadera o falsa generandose así dos modelos distintos para $\R$.\\
 
% El siguiente resultado es consecuencia inmediata de los teoremas anteriores y
% de la definición anterior.

% \begin{corolario}
% Un conjunto $\Gamma$ es satisfacible syss existe un tableaux completo
% $\mathcal{T}(\Gamma)$ satisfacible. 
% \end{corolario}

% %Los tableaux sirven para implementar algoritmos de decisión para diversos
% %problemas de la lógica de proposiciones, a continuación mencionamos algunos
% %de ellos.

% De lo anterior podemos concluir la decidibilidad de la lógica de 
% proposiciones.

% \begin{teorema}[Decidibilidad de $\propo$]
% La lógica proposicional $\propo$ es decidible. Es decir, existe un algoritmo
% que decide, dada una fórmula $\vp$, si esta es o no una tautología.
% \end{teorema}

% Podemos dar el siguiente algoritmo:
% \bi
% \item Entrada: $\vp$.
% \item Construir $\mathcal{T}(\neg\vp)$
% \bi
% \item Si $\mathcal{T}(\neg\vp)$ se cierra entonces $\vp$ es tautología.
% \item En otro caso existe una rama abierta y completa en
%   $\mathcal{T}(\neg\vp)$ la cual genera un modelo de $\neg\vp$, por lo que
%   $\vp$ no es tautología.
% \ei
% \ei

% Finalmente mencionamos como  resolver algoritmicamente otros problemas de la
% lógica proposicional:

% \subsection{Satisfacibilidad de un conjunto de fórmulas $\Gamma$}
% \bi
% \item Entrada: un conjunto de fórmulas $\Gamma$.
% \item Construir $\mathcal{T}(\Gamma)$.
% \bi
% \item Si $\mathcal{T}(\Gamma)$ se cierra entonces $\Gamma$ es insatisfacible
% \item En otro caso existe una rama abierta y completa en
%   $\mathcal{T}(\Gamma)$ la cual genera un modelo, por lo que
%   $\Gamma$ es satisfacible.
% \ei
% \ei


% \subsection{Correctud de un argumento lógico $\Gamma\models\vp$}
% \bi
% \item Entrada:  Un conjunto de fórmulas $\Gamma$ y una fórmula $\vp$.
% \item Construir $\mathcal{T}(\Gamma\cup\{\neg\vp\})$.
% \bi
% \item Si $\mathcal{T}(\Gamma\cup\{\neg\vp\})$ se cierra entonces la
%   consecuencia lógica $\Gamma\models\vp$ se da y el argumento es correcto.
% \item En otro caso existe una rama abierta y completa en
%   $\mathcal{T}(\Gamma\cup\{\neg\vp\})$ por lo que la consecuencia es inválida 
%  y se genera un modelo de las premisas $\Gamma$ donde la conclusión $\vp$ es
%   falsa.  
% \ei
% \ei


% \subsection{Clasificación de una fórmula $\vp$}
% \bi
% \item Entrada:  Una fórmula $\vp$ que deseamos clasificar como tautología,
%   contradicción o contingencia.
% \item Construir $\mathcal{T}(\vp)$.
% \bi
% \item Si $\mathcal{T}(\vp)$ se cierra entonces $\vp$ es contradicción
% \item En otro caso existe una rama abierta y completa en
%   $\mathcal{T}(\vp)$. 
% \bi
% \item Construir $\mathcal{T}(\neg \vp)$.
% \item Si $\mathcal{T}(\neg \vp)$ se cierra entonces $\vp$ es tautología.
% \item En otro caso $\mathcal{T}(\neg \vp)$ tiene una rama abierta y completa
%   y $\vp$ es contingente.
% \ei
% \ei
% \ei


\section{Tableaux en lógica de predicados: indecidibilidad}

Es bien sabido que la lógica de predicados es indecidible, es decir 
\textit{no existe un algoritmo que reciba una fórmula~$\vp$ y decida si ésta
es o no una fórmula universalmente válida}. En particular:
\bi
\item No existe un algoritmo que reciba un conjunto de premisas $\G$ y
una fórmula $\vp$ y decida si se cumple o no $\G\models\vp$.
\item Podemos preguntarnos entonces por qu\'e los algoritmos de decisión para
  la lógica de proposiciones fallan al usarlos o trasladarlos a la l\'ogica de 
  predicados, en particular por qu\'e el método de tableaux no nos permitirá 
  decidir la consecuencia lógica en la lógica de predicados. 
  ?`Donde está la falla?
\ei

Considérese el siguiente tableau:
\begin{center}
%   \boldmath
 \tikz{
  \tikzstyle{level 1}=[level distance=10mm]
  \tikzstyle{level 2}=[level distance=10mm]
  \tikzstyle{level 3}=[level distance=10mm]
  \tikzstyle{level 4}=[level distance=10mm]
  \node{$\fa x\ex y P(x,y)$}
       child{node{$\ex y P(c_1,y)$}
             child{node{$P(c_1,c_2)$}
                    child{node{$\ex y P(c_2,y)$}
                           child{node{$P(c_2,c_3)$}
                                   child{node{$\ex y P(c_3,y)$}
                                          child{node{$P(c_3,c_4)$}
                                                 child{node{$\vdots$}
                                                      }
                                               } 
                                        }
                                 }
                         }  
                  }
            }
     ;
     }
\end{center}
%  \begin{center}
%    \pstree[nodesep=2pt,levelsep=20pt]
%    {\TR{$\forall x \exists y P(x,y)$}}
%    {
%      \pstree{\TR{$\exists y P(a_1,y)$}}{
%        \pstree{\TR{$P(a_1,a_2)$}}{
%          \pstree{\TR{$\exists y P(a_2,y)$}}{
%            \pstree{\TR{$P(a_2,a_3)$}}{
%              \pstree{\TR{$\exists y P(a_3,y)$}}{
%                \pstree{\TR{$P(a_3,a_4)$}}{
%                  \TR{$\vdots$}
%                }
%              }

%            }
%          }
         
%        }
%      }
%    }
%  \end{center}
\bi
 \item Al aplicar la regla $\ga$ a la primera fórmula hacemos la extensión con 
  una nueva constante $c_1$ obteniendo una fórmula existencial $\ex y P(c_1,y)$.
 \item Al aplicar la regla $\delta$ a $\ex y P(c_1,y)$ debemos elegir una 
  constante $c_2$ distinta de $c_1$ obteniendo $ P(c_1,c_2)$.
 \item Como las fórmulas universales están siempre activas entonces podemos 
  aplicar la regla $\gamma$ nuevamente a la primera fórmula pero ahora con la 
  constante $c_2$ obteniendo $\ex y P(c_2,y)$, etc.
 \item Se observa que este proceso se puede continuar indefinidamente
  por lo que podemos obtener una rama infinita.
 \item Al existir ramas infinitas no tiene sentido hablar de 
  ramas completas como en la lógica de proposiciones. Pueden existir ramas, como 
  la de arriba que sigan extendi\'endose indefinidamente, por lo que nunca 
  sabremos si la rama se podrá cerrar o no.
\ei

De este ejemplo debe quedar claro el porque los tableaux en lógica de 
predicados únicamente pueden ser utilizados  para trata de establecer  
\emph{insatisfacibilidad, validez} ó \emph{consecuencia lógica} y  por lo 
general \textbf{no} son un método apto para construir modelos como en   el caso 
de la lógica de proposiciones. 

Existen formas de construir modelos a partir de un tableaux en lógica de   
predicados pero son m\'as bien casos particulares basados en ciertas   
\textit{heurísticas}.

En lógica de proposiciones la construcción de un modelo se sirve del concepto 
de rama completa y abierta, pero en lógica de predicados no existe un concepto 
de rama completa.
Esto debe convencernos de la indecidibilidad de la lógica de predicados. \\


\noindent Veamos otro ejemplo, consid\'erese lo siguiente:
$$ \fa x\ex y P(x,y),\; R(a,b)\;/\;\therefore \ex x R(x,b) $$
este argumento es claramente correcto pues la conclusión es consecuencia de la 
segunda premisa, la primera no aporta nada pues habla de otra relación. \\
Aunque existe un tableau cerrado generado por el argumento (utilizando la 
segunda premisa y la negaci\'on de la conclusión) también existe un tableau 
infinito muy semejante al que analizamos en el ejemplo anterior (utilizando la 
primer premisa). \\
Por lo tanto si en el análisis de un argumento nos encontramos ante 
un tableau que no se ha cerrado no podemos asegurar nada.



\end{document}

