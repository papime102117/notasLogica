\documentclass[11pt,letterpaper]{article}
\usepackage{../packageslc}
\usepackage{../optionslc}

%%%% Macros para las notas de lenguajes de programacion


%%%% math

\newcommand{\vphi}{\varphi}
\newcommand{\vp}{\varphi}

% \newcommand{\dn}{\mathsf{DN}}
% \newcommand{\dnC}{\mathsf{DN_C}}
% \newcommand{\dnM}{\mathsf{DN_M}}
% \newcommand{\dnp}{\mathsf{DN_p}}
% \newcommand{\dnm}{\mathsf{DN_p^M}}
% \newcommand{\dnc}{\mathsf{DN_p^C}}

%\newcommand{\case}{\mathsf{case}}
%\renewcommand\labelitemi{$\circ$}

\newcommand{\imp}{\rightarrow}
\newcommand{\Imp}{\Rightarrow}
\renewcommand{\iff}{\leftrightarrow}
\newcommand{\Iff}{\Leftrightarrow}
\newcommand{\G}{\Gamma}
\newcommand{\D}{\Delta}


\newcommand{\De}{\mathcal{D}}
\newcommand{\F}{\mathcal{F}}
\newcommand{\Ge}{\mathcal{G}}
\newcommand{\Pe}{\mathcal{P}}
\newcommand{\I}{\mathcal{I}}
\newcommand{\C}{\mathcal{C}}
\newcommand{\K}{\mathcal{K}}
\renewcommand{\L}{\mathcal{L}}
\newcommand{\M}{\mathcal{M}}
\newcommand{\Nc}{\mathcal{N}}
%\newcommand{\E}{\mathcal{E}}
%\newcommand{\R}{\mathcal{R}}
%\newcommand{\Q}{\mathcal{Q}}
\newcommand{\Sc}{\mathcal{S}}
\newcommand{\Te}{\mathcal{T}}
\newcommand{\W}{\mathcal{W}}

\newcommand{\Db}{\mathbb{D}}
\newcommand{\Fb}{\mathbb{F}}
\newcommand{\Kb}{\mathbb{K}}
\newcommand{\Eb}{\mathbb{E}}
\newcommand{\Ebs}{\mathbb{E}^\star}
\newcommand{\Ob}{\mathbb{O}}
\newcommand{\Ib}{\mathbb{I}}
\newcommand{\Rb}{\mathbb{R}}
\newcommand{\Qb}{\mathbb{Q}}
\newcommand{\Kbb}{\mathbb{K}}
\newcommand{\T}{\mathbb{\Theta}}


\newcommand{\kb}{\bbkappa}

\newcommand{\Sf}{\mathsf{\Sigma}}

\newcommand{\fa}{\forall}
\newcommand{\ex}{\exists}

\newcommand{\inc}{\subseteq}

\newcommand{\Lb}{\Lambda}
\newcommand{\Om}{\Omega}
\newcommand{\lb}{\lambda}
\newcommand{\al}{\alpha}
\newcommand{\ga}{\gamma}


\newcommand{\mg}{\mathbb{m}}

\newcommand{\cg}{\mathbb{C}}
\newcommand{\dg}{\mathbb{D}}
\newcommand{\jg}{\mathbb{J}}
\newcommand{\Ha}{\mathcal{H}}
%\newcommand{\A}{\mathcal{A}}
\newcommand{\sg}{\mathbb{S}}

\newcommand{\Bc}{\mathcal{B}}
\newcommand{\Df}{\mathfrak{D}}
\newcommand{\Dc}{\mathcal{D}}
%\newcommand{\Tc}{\mathcal{T}}
\newcommand{\Mf}{\mathfrak{M}}

\newcommand{\Sg}{\mathbb{S}}

\newcommand{\Q}{\ensuremath{\mathbb{Q}}}
\newcommand{\Z}{\ensuremath{\mathbb{Z}}}
\newcommand{\N}{\ensuremath{\mathbb{N}}}
\newcommand{\R}{\ensuremath{\mathbb{R}}}
\renewcommand{\S}{\mathbb{\Sigma}}
\newcommand{\A}{\mathcal{A}}
\newcommand{\E}{\ensuremath{\exists}}
\newcommand{\iso}{\ensuremath{\cong}}
\newcommand{\union}{\ensuremath{\cup}}
\newcommand{\morinyec}{\ensuremath{\precapprox}}

\newcommand{\nin}{\ensuremath{\notin}}
\newcommand{\tog}{\makebox[7mm][l]}
\newcommand{\toge}{\makebox[11mm][l]}
\newcommand{\toget}{\makebox[13mm][l]}
\newcommand{\togeth}{\makebox[14mm][l]}
\newcommand{\togethe}{\makebox[15mm][l]}
\newcommand{\together}{\makebox[17mm][l]}
\newcommand{\niso}{\ensuremath{\not \cong}}


\newcommand{\Mg}{\mathbb{M}}
\newcommand{\Bg}{\mathbb{B}}
\newcommand{\Lg}{\mathbb{L}}
\newcommand{\Tg}{\mathbb{T}}

\newcommand{\sketch}{\Red{{\sc sketch}}}

\newcommand{\restr}[2]{#1\!\!\boldsymbol{\restriction}\!#2}

\newcommand{\vacio}{\varnothing}
\newcommand{\done}{\ensuremath{\checkmark}}

\newcommand{\ida}{$\Rightarrow \; )$ }
\newcommand{\regr}{$\Leftarrow \; )$ }

\newcommand{\ol}[1]{\overline{#1}}

\newcommand{\Tsf}{\mathsf{T}}

\newcommand{\inds}[1]{\index[simb]{#1}}

\newcommand{\B}{\mathbb{B}}
%\newcommand{\N}{\mathbb{N}}

\newcommand{\vx}{\vec{x}}
\newcommand{\vy}{\vec{y}}
\newcommand{\vz}{\vec{z}}
\newcommand{\vt}{\vec{t}}
\newcommand{\vf}{\vec{f}}


% \newcommand{\propo}{\ensuremath{\mathsf{PROP}}}
% \newcommand{\atom}{\ensuremath{\mathsf{ATOM}}}
\newcommand{\term}{\ensuremath{\mathsf{TERM}}}
\newcommand{\form}{\mathsf{FORM}}

\newcommand{\true}{\mathop{\mathsf{true}}}

%\newcommand{\id}{\mathsf{Id}}

%\newcommand{\uc}{\mathcal{U}}
%\newcommand{\Ic}{\mathcal{I}}
%\newcommand{\pc}{\mathcal{P}}
%\newcommand{\qc}{\mathcal{Q}}
%\newcommand{\mc}{\mathcal{M}}
\newcommand{\supc}{\supseteq}
\newcommand{\limo}{\mathop{\mathpzc{Lim}}}
\newcommand{\ord}{\mathsf{OR}}

\newcommand{\pt}[1]{\langle #1 \rangle}


%%%% frames
\newcommand{\titulos}[2]{\frametitle{#1}\framesubtitle{#2}}
\newcommand{\fot}[1]{\footnote{\scriptsize{#1}}}


%%%% ambientes

\newcommand{\cb}[2]{\colorbox{#1}{#2}}

\newcommand{\bc}{\begin{center}}
\newcommand{\ec}{\end{center}}
\newcommand{\be}{\begin{enumerate}}
\newcommand{\ee}{\end{enumerate}}
\newcommand{\bi}{\begin{itemize}}
\newcommand{\ei}{\end{itemize}}
\newcommand{\beq}{\begin{equation}}
\newcommand{\eeq}{\end{equation}}
\newcommand{\beqs}{\begin{equation*}}
\newcommand{\eeqs}{\end{equation*}}
\newcommand{\ba}{\begin{array}}
\newcommand{\ea}{\end{array}}


% \newtheorem{theorem}{Teorema}
% \newcommand{\teo}[1]{\begin{theorem} #1 \end{theorem}}
% \newtheorem{proposition}{Proposici\'on}
% \newcommand{\prop}[1]{\begin{proposition} #1 \end{proposition}}
% \newtheorem{definition}{Definici\'on}
% \newcommand{\defin}[1]{\begin{definition} #1 \end{definition}}
% \newtheorem{corollary}{Corolario}
% \newcommand{\cor}[1]{\begin{corollary} #1 \end{corollary}}
% \newtheorem{lemma}{Lema}
% \newcommand{\lema}[1]{\begin{lemma} #1 \end{lemma}}
% \newcommand{\dem}[1]{\begin{proof} #1 \end{proof}}

%\renewcommand{\qed}{\qedsymbol{$\mathbf{\dashv}$}}

%\newcommand{\proof}{\hfill\\\noindent\textbf{\textit{Demostraci\'on. }}}

\newcommand{\hint}{\emph{Sugerencia: }}


\newcounter{EjempCtr}[section]
\newenvironment{enumrom}{\renewcommand{\theenumi}{\roman{enumi}}%
\renewcommand{\theenumii}{\roman{enumii}}
\renewcommand{\theenumiii}{\roman{enumiii}}
\renewcommand{\theenumiv}{\roman{enumiv}}
\begin{enumerate}}{\end{enumerate}}
\newenvironment{Ejemplo}
        {\stepcounter{EjempCtr}%
        \begin{description}\item[Ejemplo \thesection.\arabic{EjempCtr}]}%
        {\end{description}}
\newenvironment{demostr}{%
             {\em Demostración:}
                \begin{quotation}}{\end{quotation}}

   \newcommand{\beje}{\begin{Ejemplo}}
\newcommand{\eeje}{\end{Ejemplo}}


\newtheorem{eje}{Ejemplo}[section]
\newcommand{\ejem}[1]{\begin{eje}\normalfont #1 \end{eje}}

% \renewcommand\contentsname{\'Indice}
%\renewcommand\chaptername{Cap\'itulo}
% \renewcommand\indexname{\'Indice}

%%\newcommand{\qed}{\hfill$\mathbb{Qed}$}
%\newcommand{\qed}{\hfill$\mathsf{\boldsymbol{\dashv}}$}
%\renewcommand{\qed}{\hfill$\boldsymbol{\dashv}$}


\newenvironment{prueba}{\vspace{-5mm}\noindent\textbf{Demostraci\'on}\\}{
\noindent$\blacksquare$\\}

\newcommand{\Ejercicios}{\section*{Ejercicios}}

 \newenvironment{manitas}{%
      \renewcommand{\labelitemi}{\ding{44}}%
      \vspace{-0.5cm}%
      \begin{itemize}%
      \setlength{\itemsep}{0pt}\setlength{\parsep}{0pt}\setlength{\topsep}{0pt}%
      }{\end{itemize}}
\newenvironment{malitos}{%
      \renewcommand{\labelitemi}%
            {\raisebox{1.5ex}{\makebox[0.3cm][l]{\begin{rotate}{-90}%
            \ding{43}\end{rotate}}}}%
      \vspace{-0.5cm}%
      \begin{itemize}%
      \setlength{\itemsep}{0pt}\setlength{\parsep}{0pt}\setlength{\topsep}{0pt}%
      }{\end{itemize}}
\newenvironment{ejercs}{
     \renewcommand{\labelenumi}{\thesection.\theenumi.-}
     \renewcommand{\labelenumii}{\theenumii)}
     \begin{enumerate}}
     {\end{enumerate}}

   \newcommand{\bej}{\begin{ejercs}}
\newcommand{\eej}{\end{ejercs}}


%\newenvironment{leterize}{%
%        \renewcommand{\theenumi}{\alph{enumi}}
%        \begin{enumerate}}{\end{enumerate}}

%\newenvironment{manitas}{%
%      \renewcommand{\labelitemi}{\ding{44}}%
%      \vspace{-0.5cm}%
%      \begin{itemize}%
%      
% \setlength{\itemsep}{0pt}\setlength{\parsep}{0pt}\setlength{\topsep}{0pt}%
%      }{\end{itemize}}



%%=============================================================================

\def\stackunder#1#2{\mathrel{\mathop{#2}\limits_{#1}}}


%%%% notas

\newcommand{\doubt}{\Red{{\LARGE {\sf ??}}}}

\newcommand{\coment}[1]{\hfill\\ \Big[{\bf Comentario Privado:} #1\Big]}
\newcommand{\preg}[1]{\hfill\\ \BrickRed{{\bf Pregunta:} #1}}
\newcommand{\conjet}[1]{\hfill\\ \OliveGreen{{\bf Conjecura:} #1}}

\newcommand{\pendiente}{\BrickRed{{\sc Pendiente}}}
\newcommand{\verifpendiente}{\BrickRed{{\sc Verificación pendiente}}}


%--------------------------------------------------------------------------

\DeclareMathAlphabet{\mathpzc}{OT1}{pzc}{m}{it}



\title{Formas Normales en Lógica de Predicados \\
L\'ogica Computacional 2018-2, Nota de clase 10}
\author{Favio Ezequiel Miranda Perea\and Araceli Liliana Reyes Cabello\and
Lourdes Del Carmen Gonz\'alez Huesca \and Pilar Selene Linares Ar\'evalo}
\date{Facultad de Ciencias UNAM \\ 24 de abril de 2018 \\
Material desarrollado bajo el proyecto UNAM-PAPIME PE102117}
\begin{document}

\maketitle

%En esta nota presentamos las herramientas básicas de los sistemas de
%programación lógica, a saber la transformación de cualquier fórmula al
%lenguaje de cláusulas, la unificación de literales y la regla de resolución
%binaria para la lógica de predicados.

%\section{Formas Normales}

En esta nota desarrollaremos diversas formas normales para la lógica de
predicados, las cuales son transformaciones sintácticas de una f\'ormula que 
conservan ciertas propiedades sem\'anticas. Las fórmulas en forma normal, si
bien pueden ser sintácticamente más complejas, ayudan a semiautomatizar en
ciertos casos el proceso para tratar de verificar si un argumento es
correcto. Además son de gran importancia en programación lógica.

\bigskip

Antes de construir formas normales se aplica un proceso de
simplificación a las f\'ormulas llamado \emph{rectificaci\'on}, el cual permite 
simplificar la transformación a la forma normal de 
interés.\index{rectificaci\'on}

\defin{Una f\'ormula $\vp$ esta rectificada si y sólo si se cumplen las
siguientes condiciones:\index{f\'orm11ula!rectificada}
\bi
\item $\vp$ no tiene presencias libres y ligadas de una misma
  variable, es decir, $FV(\vp)\cap BV(\vp)=\varnothing$. 
\item $\vp$ no tiene cuantificadores de la misma variable con alcances
  ajenos.
\item $\vp$ no tiene cuantificadores múltiples de la misma variable, es decir, 
  subfórmulas de la forma \\
  $ \fa x\fa x\psi,\;\ex x\ex x\psi,\;\ex x\fa x\psi,\;\fa x\ex x\psi $
\item $\vp$ no tiene cuantificadores vácuos (tontos), es decir, cuantificadores
de la forma $\fa x\psi$ o $\ex x\psi$ donde $x\notin FV(\psi)$.
% tales que la variable del cuantificador no figure en el alcance del mismo.
\ei
}

\noindent Nuestro primer objetivo es mostrar que cualquier fórmula es
rectificable para lo cual nos serviremos del siguiente:
\lema{\label{lemareva}
Se cumplen las siguientes equivalencias l\'ogicas:
\be
\item Eliminaci\'on de cuantificadores m\'ultiples.\index{eliminaci\'on!del
cuantificador m\'ultiple} 
\bi
\item $\fa x\fa x\vp\equiv\fa x\vp$
\item $\ex x\ex x\vp\equiv\ex x\vp$
\item $\ex x\fa x\vp\equiv\fa x\vp$
\item $\fa x\ex x\vp\equiv\ex x\vp$
\ei
\item Renombre de variables. Si $y$ no figura libre en $\vp$, es
  decir, $y\notin FV(\vp)$ %  y $\vp\{x/y\}$ es admisible
  entonces:\index{renombre de variables}
\bi
\item $\fa x\vp\equiv\fa y(\vp[x:=y])$
\item $\ex x\vp\equiv\ex y(\vp[x:=y])$
\ei
\item Eliminaci\'on de cuantificadores vácuos. Si $x\notin FV(\vp)$
  entonces:\index{eliminaci\'on!del cuantificador vacuo}
\bi
\item $\fa x\vp\equiv\vp$
\item $\ex x\vp\equiv\vp$
\ei
\ee
}
\proof
Se deja como ejercicio.
\qed
\bigskip\\

Una consecuencia inmediata de los lemas anteriores es la siguiente proposición.

\prop{
Siempre es posible transformar mediante un algoritmo una f\'ormula
dada $\vp$ en una f\'ormula $\mathsf{rec}(\vp)$ de manera que 
$\mathsf{rec}(\vp)$ est\'a
rectificada y $\mathsf{rec}(\vp)\equiv\vp$.
}
\proof
Sea $\vp$ una f\'ormula. Si $FV(\vp)\cap BV(\vp)\neq\varnothing$
podemos aplicar el lema~\ref{lemareva} (parte 2) para renombrar las
presencias ligadas de todas las variables que figuraban libres y
ligadas en $\vp$, an\'alogamente si existían en $\vp$ cuantificadores
con alcances ajenos de la misma variable. \\
Los cuantificadores m\'ultiples se eliminan mediante el lema 
\ref{lemareva} (parte 1) y los cuantificadores v\'acuos con el lema 
\ref{lemareva} (parte 3). \\
Cualquier f\'ormula obtenida al terminar estos procesos se denota 
$\mathsf{rec}(\vp)$,
obs\'ervese que no es \'unica.
\qed

\vspace{.5cm}

Nuestro objetivo principal, dirigido a tener un método de prueba en lógica de 
predicados usando resolución, es transformar una f\'ormula dada en la 
lógica de predicados en una f\'ormula esencialmente proposicional que sea en 
cierto sentido equivalente a la f\'ormula original. Con las siguientes 
transformaciones esto se logra parcialmente.

\section{Forma Normal Negativa}

El objetivo de esta forma normal es obtener una f\'ormula equivalente a una
f\'ormula dada de manera que las negaciones sólo afecten a fórmulas atómicas,
es decir a predicados.  


\defin{Una f\'ormula~$\vp$ est\'a en forma normal negativa si y sólo si se 
cumplen las siguientes condiciones:
\be
\item $\vp$ no contiene símbolos de equivalencia ni implicaciones y
\item las negaciones que figuran en~$\vp$ afectan sólo a f\'ormulas at\'omicas. 
\ee
\index{forma normal!negativa}
}

La transformación a forma normal negativa es prácticamente igual que en el
caso de la lógica proposicional, basta considerar adicionalmente las leyes de
negación para los cuantificadores.

% \newpage

\prop{[Leyes de la Negaci\'on]
Se cumplen las siguientes equivalencias l\'ogicas.
\bi
\item Doble Negaci\'on: $\neg\neg\vp\equiv\vp$.
\item De Morgan: $\neg(\vp\lor\psi)\equiv \neg\vp\land\neg\psi$.
\item De Morgan: $\neg(\vp\land\psi)\equiv \neg\vp\lor\neg\psi$.
\item $\neg(\vp\imp\psi)\equiv\vp\land\neg\psi$
\item $\neg(\vp\iff\psi)\equiv\neg\vp\iff\psi\equiv\vp\iff\neg\psi$.
\item $\neg\fa x \vp\equiv\ex x\neg \vp$
\item $\neg\ex x \vp\equiv\fa x\neg \vp$
\ei
\index{leyes!de la negaci\'on}
\index{leyes!de De Morgan}
\label{leyneg}
}
\proof Ejercicio
\qed

\prop{Sea $\vp$ una f\'ormula. Podemos encontrar de manera algorítmica una
f\'ormula~$\psi$ que es lógicamente equivalente a~$\vp$ y tal que $\psi$ est\'a 
en forma normal negativa. En tal caso $\psi$ se denota con 
$\mathsf{fnn}(\psi)$. 
}
\proof
Dada la fórmula~$\vp$, aplicarle exhaustivamente las leyes de negaci\'on de
manera adecuada. La f\'ormula $\psi$ obtenida es~$\mathsf{fnn}(\vp)$.  
\qed

\ejem{Sea 
$\vp=\fa x(Px\lor\neg\ex y (Qy\land Rxy))\lor\neg (Py\land\neg\fa x Px)$.\\
La forma normal negativa de $\neg\ex y (Qy\land Rxy)$ se obtiene as\'{\i}:
\[
\neg\ex y (Qy\land Rxy) \equiv \fa y\neg(Qy\land Rxy) \equiv
\fa y(\neg Qy\lor\neg Rxy)
\]
La forma normal negativa de $\neg (Py\land\neg\fa x Px)$ se obtiene
as\'{\i}:
\[
\neg (Py\land\neg\fa x Px) \equiv \neg Py\lor \neg\neg\fa x Px\equiv 
\neg Py\lor\fa x Px
\]
Por lo tanto tenemos que
$$ \mathsf{fnn}(\vp) = 
\fa x (Px\lor\;\fa y(\neg Qy\lor\neg Rxy))\lor (\neg Py\lor\fa x Px)$$
\label{ejfnn}
}
 
\section{Forma Normal Prenex}

Al buscar una fórmula en lógica de predicados que se parezca a una 
proposicional se deben eliminar los cuantificadores. 
El primer paso para eliminarlos de una f\'ormula consiste en 
\emph{factorizarlos} de manera que todos queden juntos al principio de la
f\'ormula, esto se logra con la llamada forma normal prenex.\\
El siguiente lema describe las propiedades básicas de factorización o
prenexación de cuantificadores.

\lema{Se cumplen las siguientes equivalencias l\'ogicas.\label{lemafacu}
\be
\item Si $x$ no figura libre en $\vp$ entonces:
\bi
\item $\vp\land \fa x\psi\equiv\fa x(\vp\land\psi)$.
\item $\vp\land \ex x\psi\equiv\ex x(\vp\land\psi)$.
\item $\vp\lor \fa x\psi\equiv\fa x(\vp\lor\psi)$.
\item $\vp\lor \ex x\psi\equiv\ex x(\vp\lor\psi)$.
\item $\vp\imp \fa x\psi\equiv\fa x(\vp\imp\psi)$.
\item $\vp\imp \ex x\psi\equiv\ex x(\vp\imp\psi)$.
\ei

\item Si $x$ no figura libre en $\psi$ entonces:
\bi
\item $\fa x\vp\imp\psi\equiv\ex x(\vp\imp\psi)$.
\item $\ex x\vp\imp\psi\equiv\fa x(\vp\imp\psi)$.
\ei
\ee
}
\proof
Se deja como ejercicio.
\qed

\vspace{.5cm}

Obsérvese que no se dieron equivalencias lógicas para factorizar un
cuantificador en una equivalencia, como en la fórmula~$\vp\iff\fa x\psi$. 
De manera que primero se deben eliminar las equivalencias mediante la 
implicación y la conjunción, y luego prenexar. \\

De mayor importancia es tener en cuenta que al prenexar un
cuantificador en el antecedente de una implicación éste se intercambia
por el cuantificador contrario o dual.

\defin{Una fórmula~$\vp$ está en forma normal prenex si y sólo si $\vp$ es de la
  forma $Q_1x_1\ldots Q_nx_n\psi$ donde $\psi$ es una
  fórmula sin cuantificadores y $Q_i\in\{\fa,\ex\}$ para toda $1\leq i\leq n$.
  En tal caso, la cadena de cuantificadores $Q_1x_1\ldots Q_nx_n$ se conoce 
  como el {\em prefijo} de $\vp$ mientras que $\psi$ es la {\em matriz} de 
  $\vp$.
\index{forma normal!prenex!prefijo de la}
\index{forma normal!prenex!matriz de la}
}


\prop{
Sea $\vp$ una f\'ormula. Se puede construir mediante un algoritmo
una f\'ormula $\psi$ en forma normal prenex tal que $\vp\equiv\psi$. A
$\psi$ se le denota con $\mathsf{fnp}(\vp)$. 
Obs\'ervese que $\mathsf{fnp}(\vp)$ no es \'unica.
}
\proof
Dada $\vp$ eliminar las bicondicionales o equivalencias~\footnote{De la manera 
más conveniente en cada caso particular, ya sea usando la 
equivalencia~$\vp\iff\psi \equiv (\vp\to\psi)\land (\psi\to\vp)$ o bien 
$\vp\iff\psi \equiv (\vp\land\psi) \lor (\neg\vp\land\neg\psi)$} obteniendo una 
fórmula $\vp'$.
Aplicarle el lema \ref{lemafacu} a la fórmula rectificada $\mathsf{rec}(\vp')$ 
hasta factorizar todos sus cuantificadores. La f\'ormula obtenida es 
$\mathsf{fnp}(\vp)$. 
\qed

\ejem{Sea $\vp$ la misma f\'ormula del ejemplo \ref{ejfnn}.\\
Tenemos
$\mathsf{rec}(\vp) =
\fa x(Px\lor\neg\ex v (Qv\land Rxv))\lor\neg (Py\land\neg\fa z Pz)$.\\
La forma normal prenex de $\neg\ex v (Qv\land Rxv)$ es 
$\fa v\neg (Qv\land Rxv)$.\\
La forma normal prenex de $\fa x(Px\lor\fa v\neg (Qv\land Rxv))$ es 
$\fa x\fa v (Px\lor\neg(Qv\land Rxv))$.\\
La forma normal prenex de $\neg\fa z Pz$ es $\ex z\neg Pz$.\\
La forma normal prenex de $Py\land\ex z\neg Pz$ es $\ex z (Py\land\neg Pz)$.\\
La forma normal prenex de $\neg \ex z (Py\land\neg Pz)$ es 
$\fa z\neg(Py\land\neg Pz)$.\\
Por lo tanto la forma normal prenex de $\vp$ es
$$ \mathsf{fnp}(\vp) = 
\fa x\fa v\fa z (Px\lor\neg (Qv\land Rxv)\lor\neg(Py\land\neg Pz))$$
}

\ejem{Sea $\vp=\fa x Px\imp\ex y\fa z Ryz$. Entonces
$$
\ba{ll}
\fa x Px\imp\ex y\fa z Ryz & \equiv \\
\ex y (\fa x Px\imp\fa z Ryz) & \equiv \\
\ex y\fa z(\fa x Px\imp Ryz) & \equiv \\
\ex y\fa z\ex x(Px\imp Ryz).
\ea
$$
$\therefore\;\;\mathsf{fnp}(\vp)=\ex y\fa z\ex x(Px\imp Ryz).$
}


\section{Forma Normal de Skolem}

Ahora nuestro inter\'es se centra en los cuantificadores existenciales.
Buscamos un m\'etodo para eliminarlos de una f\'ormula, de manera
que se preserve la satisfacibilidad. Para esto utilizaremos el proceso 
de \emph{skolemizaci\'on}~\footnote{El nombre es debido a su creador,
Thoralf Skolem (1887-1963).}, el cual consiste en substituir las f\'ormulas
existenciales por \textbf{testigos} nuevos de las cuantificaciones 
existenciales de la f\'ormula, de la siguiente manera:\index{skolemizaci\'on}
\[
\ba{rcl}
\ex x\vp & \text{ se sustituye por } & \vp[x:=c] \\ \\
\fa x_1\ldots\fa x_n\ex y\vp & \text{ se sustituye por } &
\fa x_1\ldots\fa x_n\vp[y:=gx_1\ldots x_n]
\ea
\] 
aqu\'{\i}, $c$ es un s\'{\i}mbolo de constante {\bf nuevo},
llamado\index{constante de Skolem}
\emph{constante de Skolem} y $g^{(n)}$ es un s\'{\i}mbolo de funci\'on 
{\bf nuevo} llamado \emph{funci\'on de Skolem}\index{funci\'on! de Skolem}. 
Con $\mathsf{Sko}(\vp)$ denotamos a la skolemizaci\'on de $\vp$ que se obtiene 
al e\-li\-mi\-nar mediante este proceso, de izquierda a derecha, todos los
cuantificadores existenciales de la f\'ormula~$\vp$. 
\ejem{
\[
\ba{ll}
\vp & \mathsf{Sko}(\vp) \\\hline
\ex y Ry & Rc \\
\fa x\ex y Pxy & \fa x Pxgx \\ 
\fa x\fa z\ex y Rxgyfz & \fa x\fa z Rxghxzfz
\ea
\]
}

Hasta ahora hemos visto que una f\'ormula $\vp$ es l\'ogicamente
equivalente a cualquiera de sus formas normales, situaci\'on muy
conveniente pues si hallamos un modelo de una de las formas normales,
\'este mismo nos servirá como modelo de $\vp$. Desafortunadamente no sucede
lo mismo con la skolemizaci\'on, ya que tenemos una relaci\'on más debil
entre $\vp$ y $\mathsf{Sko}(\vp)$, pero la cual resultará suficiente para 
nuestros propósitos. 

\defin{Sean $\vp,\psi$ dos f\'ormulas, decimos que $\vp,\psi$ son
equisatisfacibles cuando se cumple que: \\ $\vp$ es satisfacible si y
s\'olo si $\psi$ es satisfacible. \\ 
Esta propiedad será denotada con 
$\vp\sim_{sat}\psi$.\index{f\'ormulas!satisfaciblemente equivalentes}
\label{defisat}
} 
Es claro que $\vp\equiv\psi$ implica que $\vp\sim_{sat}\psi$, mas no al
revés, veamos unos ejemplos.

\ejem{Claramente $\fa xPxa\sim_{sat}\fa xPax$ pero si
interpretamos la constante~$a$ como $0$ y el predicado~$P$ como $\leq$ sobre 
$\mathbb{N}$ observamos que $\fa xPxa\not\equiv\fa xPax$.
}

\ejem{Considere la fórmula~$\ex x Qx\sim_{sat} Qc$, este ejemplo muestra que 
una f\'ormula equisatisfacible a la f\'ormula existencial dada, resulta
mucho m\'as f\'acil de analizar.
}


%Con ayuda del siguiente teorema demostraremos que $\vp\sim_{sat}sko(\vp)$.

%\prop{(AE). (Forma Normal de Skolem). Sean $\Sigma$ una signatura, $g$ un 
% s\'{\i}mbolo de
%funci\'on que no est\'a en $\Sigma$ y $\fa x_1\ldots\fa x_n\ex y\vp$ una
%$\Sigma$-f\'ormula ce\-rra\-da tal que $x_1\ldots x_n$ no est\'an acotadas en
%$\vp$ (de manera que la sustituci\'on $\{y/g(x_1\ldots x_n)\}$ es
%admisible). Sea $\Sigma_g=\Sigma\cup\{g\}$. Entonces todo $\Sigma_g$-modelo
%de $\fa x_1\ldots\fa x_n\vp\{y/g(x_1\ldots x_n)\}$ es un $\Sigma_g$-modelo de  
%$\fa x_1\ldots\fa x_n\ex y\vp$. Inversamente, todo $\Sigma$-modelo de 
%$\fa x_1\ldots\fa x_n\ex y\vp$ puede expanderse a un $\Sigma_g$-modelo de 
%$\fa x_1\ldots\fa x_n\vp\{y/g(x_1\ldots x_n)\}$.
%\label{teofns}
%\index{teorema!de Skolem}
%}
%\proof
%Se omite.
%\qed

%\cor{$\fa x_1\ldots\fa x_n\ex y\vp$ tiene un modelo syss  
%$\fa x_1\ldots\fa x_n\vp\{y/g(x_1,\ldots,\allowbreak x_n)\}$ tiene un modelo.
%}

\prop{Si $\vp$ es una f\'ormula entonces $\vp\sim_{sat}\mathsf{Sko}(\vp)$
}
\proof 
Ejercicio.
\qed
\\
Finalmente podemos construir la forma normal de Skolem de cualquier fórmula.

\defin{Una enunciado de la forma $\fa  x_1\ldots\fa x_n\vp$ donde $\vp$ no 
  tiene cuantificadores y además $\vp$ est\'a en forma normal conjuntiva es 
  una f\'ormula en forma normal de Skolem.\index{forma normal!de Skolem}
}

Obsérvese que en la definición anterior se pide que la fórmula sea un enunciado,
es decir, que no tenga variables libres. Si de entrada existiera alguna 
variable libre se deberá cuantificar universalmente, situación que no cambia en 
absoluto los resultados pues el buscar modelos para fórmulas con variables 
libres equivale a buscar modelos para sus cerraduras universales.

\prop{Para cada f\'ormula $\vp$ podemos construir efectivamente una
f\'ormula~$\psi$ en forma normal de Skolem de manera que $\vp\sim_{sat}\psi$. 
En tal caso $\psi$ se denota con $\mathsf{fns}(\vp)$.
\label{teosko}
}
\proof Es consecuencia inmediata de los resultados anteriores de esta
nota.
\qed

\ejem{Sea $\vp=\fa x\ex y(\ex z\fa w Rxyzw\imp \ex v Pv)$, obtener la forma 
normal de Skolem para $\vp$. \\
Primero obtenemos la forma normal prenex de $\vp$:
\[
\vp \equiv \fa x\ex y\ex v(\ex z\fa w Rxyzw\imp Pv)\equiv 
\fa x\ex y\ex v\fa z(\fa w Rxyzw\imp Pv)\equiv \\ \\
\fa x\ex y\ex v\fa z\ex w(Rxyzw \imp Pv) = \mathsf{fnp}(\vp)
\]

Ahora aplicamos el proceso de skolemizaci\'on a 
$\mathsf{fnp}(\vp)$:\label{ejfns}
\be
\item Eliminamos el primer existencial $\ex y$ mediante la funci\'on de
skolem $f^{(1)}$, obteniendo 
\[\fa x\ex v\fa z\ex w(Rxfxzw\imp Pv)\]
\item Eliminamos el segundo existencial $\ex v$ mediante la funci\'on de
skolem $g^{(1)}$, obteniendo 
\[\fa x\fa z\ex w(Rxfxzw\imp Pgx)\]
\item Eliminamos el tercer existencial $\ex w$ mediante la funci\'on de
skolem $h^{(2)}$, obteniendo 
\[\fa x\fa z(Rxfxzhxz\imp Pgx)\]
\ee
Por lo tanto, $\mathsf{Sko}(\vp)=\fa x\fa z(Rxfxzhxz\imp Pgx)$. \\
Finalmente para obtener $\mathsf{fns}(\vp)$ basta poner la matriz de $sko(\vp)$ 
en forma normal conjuntiva~\footnote{La transformación a forma normal conjuntiva
es exactamente igual que en el caso de la lógica de proposiciones sólo que
en lugar de variables proposicionales ahora se manejan predicados.}.
\[
\mathsf{fnc}(Rxfxzhxz\imp Pgx)\equiv\neg Rxfxzhxz\lor Pgx.
\]
Por lo tanto la forma normal de Skolem buscada es
\[
\mathsf{fns}(\vp)=\fa x\fa z(\neg Rxfxzhxz\lor Pgx)
\]
}

\section{Forma Clausular}

Finalmente llegamos a la forma normal necesaria para el método de resolución 
binaria y para la programación lógica, la llamada forma clausular.

%\defin{Una cl\'ausula es una disyunci\'on de literales.
%}
%Con esta definici\'on podemos redefinir una forma normal conjuntiva
%como una conjunci\'on de cl\'ausulas.

\defin{Sean $\vp$ una fórmula y $\mathsf{fns}(\vp)=\fa x_1\ldots\fa x_n\psi$ la 
forma normal de Skolem de~$\vp$. Si $\psi$ es $\C_1\land\ldots\land\C_n$ 
entonces la forma clausular de $\vp$, denotada $Cl(\vp)$ es la secuencia de 
cláusulas
$$ Cl(\vp)=\C_1,\ldots,\C_n $$
donde {\em sin perder generalidad} las variables en cualesquiera dos cláusulas 
son distintas. Es decir, cualesquiera dos cláusulas tienen variables ajenas.
% tales que $\mathsf{fnc}(\vp)=\C_1\land\ldots\land C_n$
% forma clausular de $\vp$ y se denota 
% con $Cl(\vp)$.\index{forma clausular}
}

\noindent De manera que la forma clausular de una fórmula $\vp$ se obtiene de 
la matriz de su forma normal de Skolem.\\
Observemos que $Cl(\vp)$ es una secuencia (conjunción) de cl\'ausulas, libre de
cuantificadores; adem\'as es inmediato que existe un proceso efectivo para
construir $Cl(\vp)$ y que $Cl(\vp)\sim_{sat}\vp$. Con esto hemos logrado el
objetivo planteado, cada f\'ormula $\vp$ puede transformarse en una
f\'ormula esencialmente proposicional $Cl(\vp)$, satisfaciblemente
equivalente a $\vp$.
 
%M\'as adelante trabajaremos con las formas clausulares que nos ofrecen
%diversas ventajas y veremos su importancia en la programaci\'on l\'ogica. 
%\index{programaci\'on!l\'ogica}

\ejem{Sea $\vp$ la f\'ormula del ejemplo \ref{ejfns}. 
La forma cl\'ausular de $\vp$ es 
$$ Cl(\vp)=\neg Rxfxzhxz\lor Pgx $$
}

% Obsérvese que en este caso $Cl(\vp)$ consta de una única claúsula,
% pero en el caso general es de la forma
% $Cl(\vp)=\C_1\land\C_2\land\ldots\land\C_n$ y es costumbre referirse
% también con $Cl(\vp)$ a la lista de cláusulas $\C_1,\ldots,\C_n$.

Esto concluye nuestra exposición de formas normales, por medio de la cual
ahora podemos transformar cualquier fórmula en su forma clausular, la cual es
una fórmula en forma normal conjuntiva sin cuantificadores y equisatisfacible
a la fórmula original.


\section{Algoritmo de conversión a la Forma Clausular}

La conversión a forma clausular de cualquier fórmula~$\vp$ se obtiene al 
aplicar todas las formas normales anteriores. 
Damos aquí un algoritmo que se sirve además de ciertas heurísticas para 
facilitar el proceso.

\be
\item Rectificar $\vp$.
% \item Eliminar los conectivos $\imp,\iff$ mediante
% \beqs
% A\to B\equiv \neg A\lor B\;\;\;\;\;\;
% A\iff B\equiv (A\to B)\land (B\to A)
% \eeqs
\item Transformar a la forma normal negativa.
\item Prenexar obteniendo una fórmula de la forma $Q z_1\ldots Q z_m\chi$.
\item Skolemizar obteniendo una fórmula de la forma $\fa x_1\ldots\fa x_n\psi$.
\item Transformar $\psi$ (la matriz de la skolemización anterior) a la forma 
normal conjuntiva obteniendo $\mathsf{fnc}(\psi)=\C_1\land\ldots\land\C_k$.
\item Renombrar las variables de cada cláusula $\C_i$ obteniendo una clásula 
$\C_i'$ de manera que las variables en cada una de éstas sean distintas.
% digamos $\C_1',\ldots,\C_k'$
\item La forma clausular de $\vp$ es
$$ Cl(\vp)=\C_1',\ldots,\C_k' $$
\ee


De esta manera concluimos que cualquier fórmula de la lógica de predicados puede 
transformarse en un conjunto de fórmulas de la llama lógica clausular.\\

Ahora ya estamos preparados para nuestro siguiente tema, que versa sobre otro 
método sintáctico para verificar la correctud de un argumento dentro de la 
lógica clausular, la {\em resolución binaria}, método que es también el 
fundamento principal del paradigma de programación lógica.


\end{document}
