\documentclass[11pt,letterpaper]{article}
\usepackage{../packageslc}
\usepackage{../optionslc}

%%%% Macros para las notas de lenguajes de programacion


%%%% math

\newcommand{\vphi}{\varphi}
\newcommand{\vp}{\varphi}

% \newcommand{\dn}{\mathsf{DN}}
% \newcommand{\dnC}{\mathsf{DN_C}}
% \newcommand{\dnM}{\mathsf{DN_M}}
% \newcommand{\dnp}{\mathsf{DN_p}}
% \newcommand{\dnm}{\mathsf{DN_p^M}}
% \newcommand{\dnc}{\mathsf{DN_p^C}}

%\newcommand{\case}{\mathsf{case}}
%\renewcommand\labelitemi{$\circ$}

\newcommand{\imp}{\rightarrow}
\newcommand{\Imp}{\Rightarrow}
\renewcommand{\iff}{\leftrightarrow}
\newcommand{\Iff}{\Leftrightarrow}
\newcommand{\G}{\Gamma}
\newcommand{\D}{\Delta}


\newcommand{\De}{\mathcal{D}}
\newcommand{\F}{\mathcal{F}}
\newcommand{\Ge}{\mathcal{G}}
\newcommand{\Pe}{\mathcal{P}}
\newcommand{\I}{\mathcal{I}}
\newcommand{\C}{\mathcal{C}}
\newcommand{\K}{\mathcal{K}}
\renewcommand{\L}{\mathcal{L}}
\newcommand{\M}{\mathcal{M}}
\newcommand{\Nc}{\mathcal{N}}
%\newcommand{\E}{\mathcal{E}}
%\newcommand{\R}{\mathcal{R}}
%\newcommand{\Q}{\mathcal{Q}}
\newcommand{\Sc}{\mathcal{S}}
\newcommand{\Te}{\mathcal{T}}
\newcommand{\W}{\mathcal{W}}

\newcommand{\Db}{\mathbb{D}}
\newcommand{\Fb}{\mathbb{F}}
\newcommand{\Kb}{\mathbb{K}}
\newcommand{\Eb}{\mathbb{E}}
\newcommand{\Ebs}{\mathbb{E}^\star}
\newcommand{\Ob}{\mathbb{O}}
\newcommand{\Ib}{\mathbb{I}}
\newcommand{\Rb}{\mathbb{R}}
\newcommand{\Qb}{\mathbb{Q}}
\newcommand{\Kbb}{\mathbb{K}}
\newcommand{\T}{\mathbb{\Theta}}


\newcommand{\kb}{\bbkappa}

\newcommand{\Sf}{\mathsf{\Sigma}}

\newcommand{\fa}{\forall}
\newcommand{\ex}{\exists}

\newcommand{\inc}{\subseteq}

\newcommand{\Lb}{\Lambda}
\newcommand{\Om}{\Omega}
\newcommand{\lb}{\lambda}
\newcommand{\al}{\alpha}
\newcommand{\ga}{\gamma}


\newcommand{\mg}{\mathbb{m}}

\newcommand{\cg}{\mathbb{C}}
\newcommand{\dg}{\mathbb{D}}
\newcommand{\jg}{\mathbb{J}}
\newcommand{\Ha}{\mathcal{H}}
%\newcommand{\A}{\mathcal{A}}
\newcommand{\sg}{\mathbb{S}}

\newcommand{\Bc}{\mathcal{B}}
\newcommand{\Df}{\mathfrak{D}}
\newcommand{\Dc}{\mathcal{D}}
%\newcommand{\Tc}{\mathcal{T}}
\newcommand{\Mf}{\mathfrak{M}}

\newcommand{\Sg}{\mathbb{S}}

\newcommand{\Q}{\ensuremath{\mathbb{Q}}}
\newcommand{\Z}{\ensuremath{\mathbb{Z}}}
\newcommand{\N}{\ensuremath{\mathbb{N}}}
\newcommand{\R}{\ensuremath{\mathbb{R}}}
\renewcommand{\S}{\mathbb{\Sigma}}
\newcommand{\A}{\mathcal{A}}
\newcommand{\E}{\ensuremath{\exists}}
\newcommand{\iso}{\ensuremath{\cong}}
\newcommand{\union}{\ensuremath{\cup}}
\newcommand{\morinyec}{\ensuremath{\precapprox}}

\newcommand{\nin}{\ensuremath{\notin}}
\newcommand{\tog}{\makebox[7mm][l]}
\newcommand{\toge}{\makebox[11mm][l]}
\newcommand{\toget}{\makebox[13mm][l]}
\newcommand{\togeth}{\makebox[14mm][l]}
\newcommand{\togethe}{\makebox[15mm][l]}
\newcommand{\together}{\makebox[17mm][l]}
\newcommand{\niso}{\ensuremath{\not \cong}}


\newcommand{\Mg}{\mathbb{M}}
\newcommand{\Bg}{\mathbb{B}}
\newcommand{\Lg}{\mathbb{L}}
\newcommand{\Tg}{\mathbb{T}}

\newcommand{\sketch}{\Red{{\sc sketch}}}

\newcommand{\restr}[2]{#1\!\!\boldsymbol{\restriction}\!#2}

\newcommand{\vacio}{\varnothing}
\newcommand{\done}{\ensuremath{\checkmark}}

\newcommand{\ida}{$\Rightarrow \; )$ }
\newcommand{\regr}{$\Leftarrow \; )$ }

\newcommand{\ol}[1]{\overline{#1}}

\newcommand{\Tsf}{\mathsf{T}}

\newcommand{\inds}[1]{\index[simb]{#1}}

\newcommand{\B}{\mathbb{B}}
%\newcommand{\N}{\mathbb{N}}

\newcommand{\vx}{\vec{x}}
\newcommand{\vy}{\vec{y}}
\newcommand{\vz}{\vec{z}}
\newcommand{\vt}{\vec{t}}
\newcommand{\vf}{\vec{f}}


% \newcommand{\propo}{\ensuremath{\mathsf{PROP}}}
% \newcommand{\atom}{\ensuremath{\mathsf{ATOM}}}
\newcommand{\term}{\ensuremath{\mathsf{TERM}}}
\newcommand{\form}{\mathsf{FORM}}

\newcommand{\true}{\mathop{\mathsf{true}}}

%\newcommand{\id}{\mathsf{Id}}

%\newcommand{\uc}{\mathcal{U}}
%\newcommand{\Ic}{\mathcal{I}}
%\newcommand{\pc}{\mathcal{P}}
%\newcommand{\qc}{\mathcal{Q}}
%\newcommand{\mc}{\mathcal{M}}
\newcommand{\supc}{\supseteq}
\newcommand{\limo}{\mathop{\mathpzc{Lim}}}
\newcommand{\ord}{\mathsf{OR}}

\newcommand{\pt}[1]{\langle #1 \rangle}


%%%% frames
\newcommand{\titulos}[2]{\frametitle{#1}\framesubtitle{#2}}
\newcommand{\fot}[1]{\footnote{\scriptsize{#1}}}


%%%% ambientes

\newcommand{\cb}[2]{\colorbox{#1}{#2}}

\newcommand{\bc}{\begin{center}}
\newcommand{\ec}{\end{center}}
\newcommand{\be}{\begin{enumerate}}
\newcommand{\ee}{\end{enumerate}}
\newcommand{\bi}{\begin{itemize}}
\newcommand{\ei}{\end{itemize}}
\newcommand{\beq}{\begin{equation}}
\newcommand{\eeq}{\end{equation}}
\newcommand{\beqs}{\begin{equation*}}
\newcommand{\eeqs}{\end{equation*}}
\newcommand{\ba}{\begin{array}}
\newcommand{\ea}{\end{array}}


% \newtheorem{theorem}{Teorema}
% \newcommand{\teo}[1]{\begin{theorem} #1 \end{theorem}}
% \newtheorem{proposition}{Proposici\'on}
% \newcommand{\prop}[1]{\begin{proposition} #1 \end{proposition}}
% \newtheorem{definition}{Definici\'on}
% \newcommand{\defin}[1]{\begin{definition} #1 \end{definition}}
% \newtheorem{corollary}{Corolario}
% \newcommand{\cor}[1]{\begin{corollary} #1 \end{corollary}}
% \newtheorem{lemma}{Lema}
% \newcommand{\lema}[1]{\begin{lemma} #1 \end{lemma}}
% \newcommand{\dem}[1]{\begin{proof} #1 \end{proof}}

%\renewcommand{\qed}{\qedsymbol{$\mathbf{\dashv}$}}

%\newcommand{\proof}{\hfill\\\noindent\textbf{\textit{Demostraci\'on. }}}

\newcommand{\hint}{\emph{Sugerencia: }}


\newcounter{EjempCtr}[section]
\newenvironment{enumrom}{\renewcommand{\theenumi}{\roman{enumi}}%
\renewcommand{\theenumii}{\roman{enumii}}
\renewcommand{\theenumiii}{\roman{enumiii}}
\renewcommand{\theenumiv}{\roman{enumiv}}
\begin{enumerate}}{\end{enumerate}}
\newenvironment{Ejemplo}
        {\stepcounter{EjempCtr}%
        \begin{description}\item[Ejemplo \thesection.\arabic{EjempCtr}]}%
        {\end{description}}
\newenvironment{demostr}{%
             {\em Demostración:}
                \begin{quotation}}{\end{quotation}}

   \newcommand{\beje}{\begin{Ejemplo}}
\newcommand{\eeje}{\end{Ejemplo}}


\newtheorem{eje}{Ejemplo}[section]
\newcommand{\ejem}[1]{\begin{eje}\normalfont #1 \end{eje}}

% \renewcommand\contentsname{\'Indice}
%\renewcommand\chaptername{Cap\'itulo}
% \renewcommand\indexname{\'Indice}

%%\newcommand{\qed}{\hfill$\mathbb{Qed}$}
%\newcommand{\qed}{\hfill$\mathsf{\boldsymbol{\dashv}}$}
%\renewcommand{\qed}{\hfill$\boldsymbol{\dashv}$}


\newenvironment{prueba}{\vspace{-5mm}\noindent\textbf{Demostraci\'on}\\}{
\noindent$\blacksquare$\\}

\newcommand{\Ejercicios}{\section*{Ejercicios}}

 \newenvironment{manitas}{%
      \renewcommand{\labelitemi}{\ding{44}}%
      \vspace{-0.5cm}%
      \begin{itemize}%
      \setlength{\itemsep}{0pt}\setlength{\parsep}{0pt}\setlength{\topsep}{0pt}%
      }{\end{itemize}}
\newenvironment{malitos}{%
      \renewcommand{\labelitemi}%
            {\raisebox{1.5ex}{\makebox[0.3cm][l]{\begin{rotate}{-90}%
            \ding{43}\end{rotate}}}}%
      \vspace{-0.5cm}%
      \begin{itemize}%
      \setlength{\itemsep}{0pt}\setlength{\parsep}{0pt}\setlength{\topsep}{0pt}%
      }{\end{itemize}}
\newenvironment{ejercs}{
     \renewcommand{\labelenumi}{\thesection.\theenumi.-}
     \renewcommand{\labelenumii}{\theenumii)}
     \begin{enumerate}}
     {\end{enumerate}}

   \newcommand{\bej}{\begin{ejercs}}
\newcommand{\eej}{\end{ejercs}}


%\newenvironment{leterize}{%
%        \renewcommand{\theenumi}{\alph{enumi}}
%        \begin{enumerate}}{\end{enumerate}}

%\newenvironment{manitas}{%
%      \renewcommand{\labelitemi}{\ding{44}}%
%      \vspace{-0.5cm}%
%      \begin{itemize}%
%      
% \setlength{\itemsep}{0pt}\setlength{\parsep}{0pt}\setlength{\topsep}{0pt}%
%      }{\end{itemize}}



%%=============================================================================

\def\stackunder#1#2{\mathrel{\mathop{#2}\limits_{#1}}}


%%%% notas

\newcommand{\doubt}{\Red{{\LARGE {\sf ??}}}}

\newcommand{\coment}[1]{\hfill\\ \Big[{\bf Comentario Privado:} #1\Big]}
\newcommand{\preg}[1]{\hfill\\ \BrickRed{{\bf Pregunta:} #1}}
\newcommand{\conjet}[1]{\hfill\\ \OliveGreen{{\bf Conjecura:} #1}}

\newcommand{\pendiente}{\BrickRed{{\sc Pendiente}}}
\newcommand{\verifpendiente}{\BrickRed{{\sc Verificación pendiente}}}


%--------------------------------------------------------------------------

\DeclareMathAlphabet{\mathpzc}{OT1}{pzc}{m}{it}



\title{Lógica de Primer Orden: Introducción y Sintaxis \\ 
L\'ogica Computacional 2018-2, Nota de clase 4}
\author{Favio Ezequiel Miranda Perea\and Araceli Liliana Reyes Cabello\and
Lourdes Del Carmen Gonz\'alez Huesca \and Pilar Selene Linares Ar\'evalo}
\date{Facultad de Ciencias UNAM \\ 6 de marzo de 2018 \\
Material desarrollado bajo el proyecto UNAM-PAPIME PE102117}

\begin{document}

\maketitle

\section{Introducción}

La lógica ha sido estudiada desde los tiempos de Aristóteles. En sus inicios
se dedicó a fundamentar de manera sólida el análisis del razonamiento
humano, formalizando las \enquote{leyes del pensamiento}, es decir, las 
nociones de argumento válido o inválido. 
Aristóteles aisló principios lógicos llamados
\emph{silogismos} los cuales fueron utilizados para analizar y verificar
argumentos hasta la edad media por lógicos como Peter Abelard o Guillermo de 
Ockham.\\

En el siglo 19 George Boole y otros desarrollaron versiones algebráicas de la
lógica debido a las limitaciones de la teoría del silogismo para manejar
relaciones binarias, por ejemplo \enquote{x es hijo de y}. 
Esto dio nacimiento a la lógica algebraica, formalismo que aun existe aunque ha 
perdido popularidad.\\

Al final del siglo 19 Gottlob Frege desarrollo la lógica cuantificada
sentando así los cimientos de gran parte de la lógica moderna, aún cuando
Bertrand Russell le mostró que su sistema era inconsistente. Al mismo tiempo
Charles S. Pierce de manera independiente, propuso una lógica similar.\\

La \emph{paradoja de Russell} generó una crisis en los fundamentos de la
matemática la cual se resolvío al inicio del siglo XX usando la lógica de
primer orden, la cual funge hasta la actualidad como un fundamento
de las matemáticas modernas.
Debido a su gran poder, la lógica de primer orden también tiene grandes
limitaciones que fueron descubiertas en los años 1930 por Kurt Gödel,
Alan Turing y Alonzo Church, entre otros. Gödel dio un sistema
deductivo completo y correcto para la lógica de primer orden, Alfred Tarski
le dio la semántica formal que estudiaremos y que es la base para el estudio
de las semánticas denotacionales de los actuales lenguajes de programación y
Gerhard Gentzen y otros desarrollarón la teoría de la demostración. A partir de
este punto la lógica de primer orden toma su forma actual.\\

Durante la siguiente parte del curso nos dedicaremos a estudiar los siguientes 
aspectos de la lógica de primer orden:

\bi
\item Su sintaxis y semántica formal.
\item El proceso de especificación formal del español a la lógica.
% \item Generalizar la noción de validez de argumentos de la lógica
%   proposicional a la lógica de primer orden.
\item Los métodos para establecer válidez de argumentos:
\be
%\item Tablas de verdad (\textbf{No funcionan!})
\item Argumentación semántica  (método \'util pero prácticamente imposible de
  implementar de manera eficaz).
\item Tableaux semánticos (método fácilmente implementable)
\item Resolución binaria (método fundamental de la programación lógica)
\item Deducción natural (método completo y correcto, fundamento de los
  sistemas de tipos para lenguajes de programación).
\ee
\ei


\subsection{Lógica proposicional: expresividad vs. decidibilidad}

Por un lado, observemos que la lógica proposicional no es lo suficientemente 
expresiva. 
Considérese por ejemplo el siguiente razonamiento:
\bc
\emph{Algunas personas van al teatro. Todos los que van al teatro se
divierten. \\ De manera que algunas personas se divierten.}
\ec
La intuición nos dice que el argumento es correcto. Sin embargo la 
representación correspondiente en lógica proposicional es:
$$ p,\;q\;/\therefore\;r\qquad \text{ !`Incorrecto! } $$

\espc

Otros ejemplos de expresiones que no pueden ser formalizadas adecuadamente en
la lógica proposicional son:
\bi
\item[] \emph{La lista está ordenada.}
\item[] \emph{Cualquier empleado tiene un jefe.}
\item[] \emph{Si los caballos son animales entonces las cabezas de caballos son
  cabezas de animales.}
\ei
Este problema de expresividad se soluciona al introducir la lógica de 
predicados.

\espc

Por otro lado, la lógica proposicional es decidible, es decir, existen
diversos algoritmos para responder a las preguntas 
\bc
?`$\;\models\vp\;$?, ?` Existe  $\I$ tal que $\I\models\vp\;$?,  \\
?` Es $\Gamma$ satisfacible?,  ?`$\;\Gamma\models\vp\;$?
\ec
Como vimos anteriormente, su estudio es de gran importancia en la teoría de 
complejidad computacional (problema {\sf SAT}).
Esta propiedad, de gran importancia desde el punto de vista computacional, se
pierde en la lógica de predicados.  

As\'i, la l\'ogica proposicional nos ofrece decibilidad de algunas propiedades 
pero nos limita respecto a la expresividad. La l\'ogica de predicados o de 
primer orden que estudiaremos, nos ofrecer\'a mayor expresividad para 
formalizar expresiones m\'as complejas.\\

Antes de empezar el estudio formal de la l\'ogica de primer orden discutiremos
con detalle un ejemplo de especificación no trivial en lógica de proposiciones. 
De aquí podremos observar una aplicación directa del problema {\sf SAT} fuera 
del ámbito de la lógica. Esta clase de representaciones son típicas de los 
métodos de reducción de problemas en teoría de la complejidad.

\newpage

\subsection{Reducción del Sudoku al problema {\sf SAT}\protect\footnote{
Esta sección se basa en el artículo \textit{Sudoku as a SAT Problem} de In\^es 
Lynce y Jo\"el Ouaknine. 9th International Symposium on Artificial Intelligence 
and Mathematics, January 2006.}}

\noindent Consideramos aquí Sudokus que cumplen las siguientes dos 
propiedades:
\be
\item Tienen una solución única.
\item Pueden resolverse con razonamiento únicamente, es decir, sin búsqueda.
\ee

La representación se sirve de $729$ variables proposicionales denotadas 
$s_{xyz}$ con $x,y,z\in\{1,\ldots,9\}$, cuyo significado es
que el número~$z$ está en la entrada o celda~$xy$. 
Debe pensarse el tablero de Sudoku como una matriz de manera que la 
posición~$xy$ denota a la celda en el renglón~$x$ y columna~$y$. \\
La \textbf{especificación} es la siguiente:
\bi
\item Hay al menos un número en cada celda
$$ \bigwedge_{x=1}^9\,\bigwedge_{y=1}^9\,\bigvee_{z=1}^9 s_{xyz} $$
% \beqs
% \fa x \fa y\Big( 1\leq x,y\leq 9 \to \exists z \big(1\leq z\leq 9 \land 
% S(x,y,z)\big)\Big)
% \eeqs
Esta fórmula nos dice que para cada renglón~$x$ y para cada columna~$y$ hay un 
número~$z$ en la celda~$xy$.

\item Cada número figura a los más una vez en cada renglón
$$ \bigwedge_{y=1}^9\,\bigwedge_{z=1}^9\,\bigwedge_{x=1}^8\,\bigwedge_{i=x+1}^9 
(\neg s_{xyz}\lor \neg s_{iyz}) $$
Esta fórmula nos dice que para cada columna~$y$, para cada número~$z$ y para 
cada renglón~$x$, excepto el último, si el número~$z$ está en el renglón~$x$ 
(en la celda~$xy$) entonces no sucede que el número~$z$ está en alguno de los 
renglones posteriores a~$x$ (los denotados por~$i$).

\item Cada número figura a lo más una vez en cada columna
$$ \bigwedge_{x=1}^9\,\bigwedge_{z=1}^9\,\bigwedge_{y=1}^8\,\bigwedge_{i=y+1}^9 
(\neg s_{xyz}\lor \neg s_{xiz})$$
Esta fórmula nos dice que para cada renglón~$x$, para cada número~$z$ y para 
cada columna~$y$, excepto la última, si el número~$z$ está en la columna~$y$ 
(en la celda~$xy$) entonces no sucede que el número~$z$ está en alguna de las 
columnas posteriores a $y$ (las denotadas por~$i$).

\item Cada número figura a lo más una vez en cada cuadrante de $3\times 3$: 
\bi
\item Dentro de cada cuadrante, si un número~$z$ está en un renglón entonces 
  no está en las celdas siguientes en el mismo renglón. Aquí $i,j$ delimitan 
  las celdas válidas del cuadrante y $k$ se encarga de verificar las columnas 
  siguientes a la de la celda en consideración. Todo dentro del mismo cuadrante.
  \[ 
   \bigwedge_{z=1}^9\,\bigwedge_{i=0}^2\,\bigwedge_{j=0}^2\,\bigwedge_{x=1}^3\,
   \bigwedge_ { y=1}^3\,\bigwedge_{k=y+1}^3 (\neg s_{(3i+x)(3j+y)z}\lor \neg 
    s_{(3i+x)(3j+k)z})
  \]
  
\item Dentro de cada cuadrante, si un número~$z$ está en una columna entonces 
  no está en los renglones superiores ni en la misma columna ni en las columnas 
  siguientes. Aquí $i,j$ delimitan las celdas válidas del cuadrante, $k$ se 
  encarga de verificar los renglones superiores y $l$ verifica la columna 
actual 
  y las columnas siguientes.
  \[
   \bigwedge_{z=1}^9\,\bigwedge_{i=0}^2\,\bigwedge_{j=0}^2\,\bigwedge_{x=1}^3\,
   \bigwedge_ { y=1}^3\,\bigwedge_{k=x+1}^3 \bigwedge_{l=1}^3 
    (\neg s_{(3i+x)(3j+y)z}\lor \neg s_{(3i+k)(3j+l)z})
  \]
\ei
\ei

Las cláusulas anteriores conforman una codificación m\'inima. Un bonito cálculo 
combinatorio lleva a concluir que a partir de las fórmulas anteriores la 
fórmula 
resultante de la transformación a FNC tendrá 8829 cláusulas de las cuales 81 
cláusulas tienen 9 literales, y las restantes 8748 son binarias, es decir, 
tienen 2 literales. \\

Adicionalmente se deben considerar las cláusulas unitarias 
correspondientes a las entradas del Sudoku que ya tienen un número asignado.
Al agregar las siguientes cláusulas modelamos la \textbf{unicidad} de un número 
en cada celda:
\bi
\item Hay a lo más un numero en cada celda.
\[
\bigwedge_{x=1}^9\,\bigwedge_{y=1}^9\,\bigwedge_{z=1}^8\,\bigwedge_{i=z+1}^9 
(\neg s_{xyz} \lor \neg s_{xyi})
\]
Esta fórmula nos dice que para cada renglón~$x$, para cada columna~$y$, para 
cada número no $9$, si la celda~$xy$ tiene al numero~$z$ entonces no tiene a 
los números mayores que~$z$.

\item Cada número figura al menos una vez en cada columna
$$ \bigwedge_{y=1}^9\,\bigwedge_{z=1}^9\,\bigvee_{x=1}^9 s_{xyz} $$
Está fórmula nos dice que para cada columna~$y$ y cada número~$z$, existe un 
renglón~$x$, tal que~$z$ está en la celda~$xy$.

\item Cada número figura al menos una vez en cada renglón
$$ \bigwedge_{x=1}^9\,\bigwedge_{z=1}^9\,\bigvee_{y=1}^9 s_{xyz} $$
Está fórmula nos dice que para cada renglón~$x$ y cada número~$z$, existe una 
columna~$y$, tal que~$z$ está en la celda~$xy$.

\item Cada número figura al menos una vez en cada cuadrante de $3\times 3$.
\[
\bigvee_{z=1}^9\,\bigwedge_{i=0}^2\,\bigwedge_{j=0}^2\,\bigwedge_{x=1}^3\, 
\bigwedge_ { y=1 }^3 s_{(3i+x)(3j+y)z}
\]
Está fórmula nos dice que hay un número~$z$ en cada uno de los cuadrantes, los 
cuales qse generan por las celdas~$(3i+x)(3j+y)$.
\ei

La fórmula resultante de transformar todas las anteriores consta de 11988 
cláusulas, 324 de ellas con 9 literales y las restantes 
11664 binarias. Nuevamente debemos agregar cláusulas unitarias para las 
entradas 
previamente asignadas del Sudoku.

\espc

De los c\'alculos anteriores podemos ver que a pesar de ser un problema 
pesado, la implementaci\'on de la teor\'ia de prueba resoluci\'on binaria puede 
hacer a la computadora una herramienta \'util para resolver sudokus.



\section{Lógica de predicados de manera informal}


En la lógica proposicional debemos representar a ciertas frases del español
como por ejemplo
\bc
\emph{Algunos programas usan ciclos anidados}
\ec
mediante fórmulas proposicionales atómicas, es decir mediante una simple 
variable proposicional~$p$. 
En la lógica de predicados descomponemos a este tipo de
frases distinguiendo dos categorias: objetos y relaciones entre
objetos. Ambas categorias construidas sobre un universo de discurso fijo el 
cual varía según el problema particular que se desee representar. 
Estudiemos con mas detalle estas categorias:

\subsection{Objetos y universo de discurso}
\bc
 \emph{Algunas \textbf{personas} van al teatro} \\
 \emph{Todos los \textbf{programas} en Java usan clases.}
\ec
  
\bi
 \item Universo o dominio de discurso: conjunto de objetos que se consideran 
  en el razonamiento, pueden ser cosas, personas, datos, objetos abstractos, 
  etc.
   
 \item En el caso de los enunciados anteriores en el universo podemos tener 
  personas y programas en Java, y  posiblemente el teatro y las clases.
\ei


\subsection{Relaciones entre objetos}
\bi
 \item Las propiedades o relaciones atribuibles a los objetos del dominio de 
  discurso se representan con predicados.

 \item En el caso de los enunciados anteriores tenemos \emph{ir al teatro, usar 
  clases}.
\ei

\subsection{Cuantificadores}
Los cuantificadores especifican el número de individuos que cumplen con el 
predicado. Por ejemplo:
\bi
\item \emph{Todos} los estudiantes trabajan duro.
\item \emph{Algunos} estudiantes se duermen en clase.
\item \emph{La mayoría} de los maestros están locos
\item \emph{Ocho de cada diez} gatos lo prefieren.
\item \emph{Nadie} es más tonto que yo.
\item \emph{Al menos seis} estudiantes están despiertos.
\item \emph{Hay una infinidad} de números primos.
\item \emph{Hay más} PCs que Macs.
\ei

En lógica de primer orden sólo consideraremos dos cuantificadores:
\emph{todos} y \emph{algunos}. 


\subsection{Proposiciones vs. Predicados}

El uso de predicados en lugar de proposiciones podría parecer simplemente
otra manera de escribir las cosas, por ejemplo, la proposición:
\bc
\textit{Chubaka recita poesía nordica}
\ec
se representa con predicados como
\bc
\textit{Recita(Chubaka, poesía nordica)}
\ec

\noindent La gran ventaja es que el predicado puede cambiar de argumentos, por 
ejemplo
\bc
\textit{Recita(Licantro, odas en sánscrito)} 
\ec
o en el caso general podemos usar variables para denotar individuos:
\bc
\textit{Recita(x,y)}
\ec


Obsérvese que está última expresión no es una proposición. 
En este sentido un mismo predicado está representando un número
potencialmente infinito de proposiciones.


\espc
\noindent
Reconozcamos ahora las componentes anteriores en un argumento particular:
\bc
 \emph{Todos los matemáticos estudian algún teorema.
   Los teoremas son elementales o trascendentes. \\
   Giuseppe es  matemático. Luego entonces, Giuseppe estudia algo 
   elemental o trascendente.}
\ec
    
\bi
 \item Universo: \emph{matemáticos y teoremas}
 \item Predicados: \emph{estudiar, elemental, trascendente.}
 \item Cuantificadores: \emph{todos, los, algo.}
 \item Los individuos se representarán formalmente mediante dos categor\'ias:
  \bi
   \item Constantes: las cuales denotan individuos particulares,
     como Giuseppe.
   \item Variables: las cuales denotan individuos genéricos o
     indeterminados como los matemáticos o el teorema. \\
     \textbf{Obsérvese que} las variables no figuran en el lenguaje natural 
como 
     el español, pero son necesarias para la formalización y representaci\'on 
     de individuos no fijos.
  \ei
\ei


Formalicemos ahora todos los conceptos discutidos anteriormente.


\section{Sintaxis de la lógica de primer orden}

En contraposición al lenguaje de la lógica proposicional que está
determinado de manera única, no es posible hablar de un solo lenguaje
para la lógica de predicados. Dependiendo de la estructura semántica
que tengamos en mente será necesario agregar símbolos particulares
para denotar objetos y relaciones entre objetos. De esta manera
el alfabeto consta de dos partes ajenas entre sí, la parte común a
todos los lenguajes está determinada por los símbolos lógicos y auxiliares
y la parte particular, llamada tipo de semejanza o signatura del lenguaje.

\noindent La parte común a todos los lenguajes consta de:
\bi
 \item Un conjunto infinito de variables 
  $\mathsf{Var}=\{x_1,\ldots,x_n,\ldots\}$
 \item Constantes lógicas: $\bot,\top$
 \item Conectivos u operadores lógicos: $\neg,\land,\lor,\imp,\iff$
 \item Cuantificadores: $\fa,\ex$.
  \item Símbolos auxiliares: $(\;,\;)$ y $,$ (coma).
 \item Si se agrega el símbolo de igualdad~$=$, decimos que el lenguaje tiene 
  igualdad.
\ei

\noindent La signatura de un lenguaje en particular está dada por:
\bi
 \item Un conjunto~$\Pe$, posiblemente vacío, de símbolos o letras de
  predicado:  $ P_1,\ldots,P_n,\ldots$\\
  A cada símbolo se le asigna un índice~\footnote{Este índice suele llamarse 
  también \enquote{aridad}, pero dado que tal palabra no existe en el 
  diccionario de la lengua española, trataremos de evitar su uso.} o número 
  de argumentos~$m$, el cual se hace explícito escribiendo~$P_n^{(m)}$ que  
  significa que el símbolo~$P_n$ necesita de~$m$ argumentos.
  
 \item Un conjunto~$\F$, posiblemente vacío, de símbolos o letras de
  función:  $f_1,\ldots,f_n,\ldots$\\ 
  Análogamente a los símbolos de predicado cada símbolo de función tiene un 
  índice asignado, $f_n^{(m)}$ significará que el símbolo~$f_n$ necesita de~$m$ 
  argumentos.   
  
 \item Un conjunto~$\C$, posiblemente vacío, de símbolos de constante:
  $c_1,\ldots,c_n,\ldots$\\
  En algunos libros los símbolos de constante se consideran como parte del
  conjunto de símbolos de función, puesto que pueden verse como funciones de
  índice cero, es decir, funciones sin argumentos.
\ei

Dado que un lenguaje de primer orden queda determinado de manera única por su 
signatura, abusaremos de la notación y escribiremos $\L=\Pe\cup\F\cup\C $
para denotar al lenguaje dado por tal signatura.


\subsection{Términos}  
Los términos del lenguaje son aquellas expresiones que representarán
objetos en la semántica y su definición es:
\bi
 \item Los símbolos de constante $c_1,\ldots,c_n,\ldots$ son términos.
 \item Las variables $x_1,\ldots,x_n,\ldots$ son términos. 
 \item Si $f^{(m)}$ es un símbolo de función y $t_1,\ldots,t_m$ son términos 
  entonces $f(t_1,\ldots,t_m)$ es un término.
 \item Son todos.
\ei
Es decir, los términos están dados por la siguiente gramática:
$$ t ::= x \mid c \mid f(t_1,\ldots,t_m) $$
Y el conjunto de términos de un lenguaje dado se denota con~$\term_\L$, o
simplemente~$\term$ si es claro cuál es el lenguaje. \\

En algunas ocasiones se encuentra una definición de términos que no incluye 
constantes. Esta omisión no existe en realidad puesto que las constantes se 
consideran casos particulares de símbolos de función de índice cero como hemos 
mencionado antes, es decir un símbolo de función que no recibe argumentos. 
Considerando lo anterior, presentamos la siguiente implementación.

\subsubsection{Implementación}

\noindent Emplearemos la siguiente representación en \textsc{Haskell}:

\begin{verbatim}
type Nombre = String

data Term = V Nombre | F Nombre [Term]
\end{verbatim}

\noindent Veamos algunos ejemplos:
\bi
\item $c$ se implementa como  \verb-  F ``c'' [ ]-
\item $f(x,y)$ se implementa como \verb- F ``f'' [V ``x'', V ``y'']-
\item $g(a)$ se implementa como  \verb- F ``g'' [F ``a'' []]-
\item $h(f(x))$ se implementa como  \verb- F ``h'' [F ``f'' [V ``x'']]}-
\item $h(b,f(a),z)$ se implementa como  
  \verb-``h'' [F ``b'' [], F ``f'' [F ``a'' []], V ``z'']-
\ei

\subsection{Fórmulas}

\noindent El siguiente paso es determinar las expresiones o fórmulas atómicas,
dadas por:
\bi
\item Las constantes lógicas $\bot,\top$.
\item Las expresiones de la forma: $P_1(t_1,\ldots,t_n)$ donde
  $t_1,\ldots,t_n$ son términos 
\item Las expresiones de la forma $t_1=t_2$, si el lenguaje cuenta con igualdad
\ei
El conjunto de expresiones atómicas se denotará con $\mathsf{ATOM}_\L$. \\

As\'i el conjunto $\form_\L$ de expresiones compuestas aceptadas en un lenguaje 
$\L$, llamadas usualmente fórmulas, se define recursivamente como sigue: 
\bi
\item Si $\vp\in\mathsf{ATOM}_\L$ entonces $\vp\in\form_\L$. Es decir, 
toda fórmula atómica es una fórmula.
\item Si $\vp\in\form_\L$ entonces $(\neg\vp)\in\form_\L$.
\item Si $\vp,\psi\in\form_\L$ entonces 
$(\vp\land\psi),(\vp\lor\psi),(\vp\imp\psi),(\vp\iff\psi)\in\form_\L$.
\item Si $\vp\in\form_\L$ y $x\in\mathsf{Var}$ entonces 
$(\fa x\vp),(\exists x\vp)\in\form_\L$.
\item Son todas.
\ei

%La última cláusula de la definición garantiza que el conjunto $\form$ es el
%mínimo conjunto cerrado bajo las tres primeras reglas.
%La definición anterior puede darse en la llamada forma de Backus-Naur para
%definir grámaticas como sigue:
La gramática para las fórmulas en forma de Backus-Naur es:
\[
\begin{array}{rcl}
  \vp & ::= & at \mid (\neg\vp) \mid (\vp\land\psi) 
    \mid (\vp\lor\psi) \mid (\vp\imp\psi) \mid (\vp\iff\psi) 
    \mid (\fa x\vp) \mid (\ex x\vp) \\
  at & ::= & \bot \mid \top \mid P(t_1,\ldots,t_m) \mid t_1=t_2
\end{array}
\]
Cada lenguaje definido a partir de una signatura y con forme a la 
gram\'atica recién descrita será un \emph{lenguaje de primer orden}. 

\noindent Las siguientes observaciones son de suma importancia:
\bi
 \item Los términos y las fórmulas son dos categorías sintácticas ajenas, es 
  decir \textbf{ningún} término es fórmula y \textbf{ninguna} fórmula es 
  término.
 
 \item Los términos denotan exclusivamente individuos u objetos.
 
 \item Las fórmulas atómicas denotan proposiciones o propiedades acerca de los
    términos. 
    
 \item \textbf{Sólo} los individuos u objetos son cuantificables y       
  \textbf{únicamente} sobre ellos se puede \enquote{predicar}. 
  Esta característica justifica la denominación \enquote{primer orden}.
\ei


\subsection{Precedencia}

Si bien los cuantificadores no son propiamente operadores entre
fórmulas y por lo tanto no se puede hablar de una precedencia en el
sentido usual, usamos la siguiente convención:
\bc
 Los cuantificadores se aplican a la \textit{mínima} expresión sintácticamente 
posible.
\ec
de manera que
\[
\ba{rll}
\fa x\vp\imp \psi  & \text{ es} &  (\fa x\vp)\imp \psi \\
\ex y\vp \land \fa w\psi\imp\chi& \text{ es} & (\ex y\vp)\land(\fa 
w\psi)\imp\chi\\
\ea
\]

\subsection{Implementación}
La implementación de f\'ormulas es la siguiente:
\begin{verbatim}
data Form = TrueF | FalseF | Pr Nombre [Term] | Eq Term Term | 
            Neg Form | Conj Form Form | Disy Form Form | Imp Form Form | 
            Equi Form Form | All Nombre Form | Ex Nombre Form 
\end{verbatim}

Algunos ejemplos son:
\bi
 \item $P(x,y)$ se implementa como \verb-Pr ``P'' [V ``x'', V ``y'']-
 \item $Q(a,h(w))$ se implementa como 
  \verb-Pr ``Q'' [F ``a'' [], F ``h'' [V ``w'']]-
 \item $R(x)\imp x=b$ se implementa como 
  \verb-Imp (Pr ``R'' [V ``x'']) (Eq (V ``x'') (F ``b'' []))-
 \item $\neg R(z)\lor (S(a)\land f(x)=z)$ se implementa como
 \begin{verbatim}
  Disy (Neg (Pr ``R'' [V ``z'']))
       (Conj (Pr ``S'' [F ``a'' []]) 
             (Eq (F ``f'' [V ``x'']) (V ``z''))
       )
 \end{verbatim}
 
 \vspace{-25pt}
 
 \item $\fa x Q(x,g(x))$ se implementa como
  \verb-All ``x'' (Pr ``Q'' [V ``x'',F ``g'' [V ``x'']])-
 \item $\fa x \ex y P(x,y)$ se implementa como
  \verb-All ``x'' (Ex ``y'' (Pr ``P'' [V ``x'',V ``y'']))-
\ei


%subst (All y f) x t = if not (elem x (y:var t) then All y (sterm f x t) else 
% (All y f)


\section{Inducción y Recursión}
Las definiciones recursivas para los elementos del lenguaje generan principios 
de inducci\'on. 
Dado que ahora tenemos dos categorías sintácticas, términos y fórmulas,
tenemos también principios de inducción correspondientes a cada categoría.

\defin{[Principio de Inducci\'on Estructural para $\term_\L$]
Sea $\Pe$ una propiedad acerca de t\'erminos. Para demostrar que $\Pe$ es
v\'alida para todos los t\'erminos, basta seguir los siguientes pasos:
\bi
\item Base de la inducción: mostrar que
\be
\item Si $x\in\mathsf{Var}$ entonces $\Pe$ es v\'alida para $x$.
\item Si $c\in\C$ entonces $\Pe$ es v\'alida para $c$.
\ee
\item Hipótesis de inducción: suponer $\Pe$ para cualesquiera 
$t_1,\ldots,t_n\in\term_\L$.
\item Paso inductivo: usando la Hipótesis de inducción mostrar que 
\be
\item Si $f\in\F$ es un símbolo de función
  de índice $n$ entonces $f(t_1,\ldots,t_n)$ cumple $\Pe$.
\ee
\ei
}

\defin{[Principio de Inducción Estructural para $\form_\L$]
Sea $\Pe$ una propiedad acerca de fórmulas. Para probar que toda fórmula $
\vp\in\form_\L$ tiene la propiedad $\Pe$ basta seguir los siguientes pasos:
\bi
\item Caso base: mostrar que toda fórmula atómica tiene la propiedad~$\Pe$.
\item Hipótesis de inducción: suponer que $\vp$ y $\psi$ cumplen $\Pe$.
\item Paso inductivo: mostrar usando la Hipótesis de inducci\'on que
\be
\item $(\neg\vp)$ también cumple $\Pe$.
\item  $(\vp\star\psi)$ tiene la propiedad $\Pe$, donde 
$\star\in\{\imp,\land,\lor,\iff\}$
\item $\fa x\vp$ y $\ex x\vp$ cumplen $\Pe$.
\ee
\ei
}

Con respecto a la recursión, tambi\'en podemos definir funciones sobre los 
términos y las fórmulas usando los siguientes principios:

\paragraph{Definici\'on Recursiva para T\'erminos} 
Para definir una funci\'on~$h:\term_\L\imp A$, basta definir~$h$ como sigue:
\bi
\item Definir $h(x)$ para $x\in\mathsf{Var}$.
\item Definir $h(c)$ para cada constante $c\in\C$.
\item Suponiendo que $h(t_1),\ldots,h(t_n)$ est\'an definidas, definir
$h(f(t_1,\ldots,\allowbreak t_n))$ para cada s\'{\i}mbolo de funci\'on $f\in\F$ 
de índice $n$, utilizando $h(t_1),\ldots,\allowbreak h(t_n)$.
\ei


\paragraph{Definici\'on Recursiva para F\'ormulas}.
Para definir una funci\'on~$h:\form_\L\imp A$, basta definir~$h$ como sigue:
\bi
\item Definir $h$ para cada f\'ormula at\'omica, es decir, definir $h(\bot), 
h(\top)$, $h(P(t_1,\ldots,t_n))$ y $h(t_1=t_2)$ si el lenguaje tiene igualdad.
\item Suponiendo definidas $h(\vp)$ y $h(\psi)$, definir a partir de
ellas a\\
$h(\neg\vp),\;h(\vp\lor\psi),\;h(\vp\land\psi),\;h(\vp\imp\psi),
\;h(\vp\iff\psi) 
,\;h(\fa x\vp)$ y $h(\ex x\vp)$. 
\ei

\espc

\noindent Veamos algunos ejemplos:
\ejem{
Definimos el conjunto de subt\'erminos de un t\'ermino $t$ recursivamente
como sigue:
\bi
\item $\mathsf{Subt}(x)=\{x\}$
\item $\mathsf{Subt}(c)=\{c\}$
\item $\mathsf{Subt}(f(t_1,\ldots,t_n))=\{f(t_1,\ldots,t_n)\}\cup
\mathsf{Subt}(t_1)\cup\ldots\cup \mathsf{Subt}(t_n)$
\ei
}
\ejem{Definimos recursivamente el número de conectivos y cuantificadores de
  una f\'ormula, conocido como el peso de una fórmula, como sigue:
\bi
\item $\mathsf{peso}(\bot)=\mathsf{peso}(\top)=0$.
\item $\mathsf{peso}(P(t_1,\ldots,t_n))=0$.
\item $\mathsf{peso}(\neg\vp)=\mathsf{peso}(\vp)+1$.
\item $\mathsf{peso}(\vp\land\psi)=\mathsf{peso}(\vp)+\mathsf{peso}(\psi)+1$
\item $\mathsf{peso}(\fa x\vp)=\mathsf{peso}(\vp)+1$
\ei

}

\begin{comment}
\subsubsection{Ejercicio}

Definir recursivamente e implementar las siguientes funciones:

\bi
\item \verb-consT :: Term -> [Nombre]} tal que \verb-ConsT t} devuelve la lista 
con todos los nombres de constante que figuran en $t$.
\item \verb-varT :: Term -> [Nombre]} tal que \verb-VarT t} devuelve la lista 
con 
todos los nombres de variables que figuran en $t$.
\item \verb-funT :: Term -> [Nombre]} tal que \verb-FunT t} devuelve la lista 
con 
todos los nombres de función que figuran en $t$.
\item Análogamente para el caso de las fórmulas: \verb-consF, varF,funF:: Form 
-> 
[Nombre]}
\item \verb-subT :: Term -> [Term]} tal que \verb-subT t} devuelve la lista de 
subtérminos de $t$.
\item \verb-subF :: Form -> [Form]} tal que \verb-subF A} devuelve la lista de 
subfórmulas de $A$.
\item \verb-cuantF :: Form -> [Formm]} tal que \verb-cuantF A} devuelve la 
lista 
de cuantificaciones de $A$, es decir, la lista de subfórmulas de $A$ que son 
cuantificaciones.

\ei

\end{comment}
\end{document}
