\documentclass[11pt,letterpaper]{article}
\usepackage{../packageslc}
\usepackage{../optionslc}

%%%% Macros para las notas de lenguajes de programacion


%%%% math

\newcommand{\vphi}{\varphi}
\newcommand{\vp}{\varphi}

% \newcommand{\dn}{\mathsf{DN}}
% \newcommand{\dnC}{\mathsf{DN_C}}
% \newcommand{\dnM}{\mathsf{DN_M}}
% \newcommand{\dnp}{\mathsf{DN_p}}
% \newcommand{\dnm}{\mathsf{DN_p^M}}
% \newcommand{\dnc}{\mathsf{DN_p^C}}

%\newcommand{\case}{\mathsf{case}}
%\renewcommand\labelitemi{$\circ$}

\newcommand{\imp}{\rightarrow}
\newcommand{\Imp}{\Rightarrow}
\renewcommand{\iff}{\leftrightarrow}
\newcommand{\Iff}{\Leftrightarrow}
\newcommand{\G}{\Gamma}
\newcommand{\D}{\Delta}


\newcommand{\De}{\mathcal{D}}
\newcommand{\F}{\mathcal{F}}
\newcommand{\Ge}{\mathcal{G}}
\newcommand{\Pe}{\mathcal{P}}
\newcommand{\I}{\mathcal{I}}
\newcommand{\C}{\mathcal{C}}
\newcommand{\K}{\mathcal{K}}
\renewcommand{\L}{\mathcal{L}}
\newcommand{\M}{\mathcal{M}}
\newcommand{\Nc}{\mathcal{N}}
%\newcommand{\E}{\mathcal{E}}
%\newcommand{\R}{\mathcal{R}}
%\newcommand{\Q}{\mathcal{Q}}
\newcommand{\Sc}{\mathcal{S}}
\newcommand{\Te}{\mathcal{T}}
\newcommand{\W}{\mathcal{W}}

\newcommand{\Db}{\mathbb{D}}
\newcommand{\Fb}{\mathbb{F}}
\newcommand{\Kb}{\mathbb{K}}
\newcommand{\Eb}{\mathbb{E}}
\newcommand{\Ebs}{\mathbb{E}^\star}
\newcommand{\Ob}{\mathbb{O}}
\newcommand{\Ib}{\mathbb{I}}
\newcommand{\Rb}{\mathbb{R}}
\newcommand{\Qb}{\mathbb{Q}}
\newcommand{\Kbb}{\mathbb{K}}
\newcommand{\T}{\mathbb{\Theta}}


\newcommand{\kb}{\bbkappa}

\newcommand{\Sf}{\mathsf{\Sigma}}

\newcommand{\fa}{\forall}
\newcommand{\ex}{\exists}

\newcommand{\inc}{\subseteq}

\newcommand{\Lb}{\Lambda}
\newcommand{\Om}{\Omega}
\newcommand{\lb}{\lambda}
\newcommand{\al}{\alpha}
\newcommand{\ga}{\gamma}


\newcommand{\mg}{\mathbb{m}}

\newcommand{\cg}{\mathbb{C}}
\newcommand{\dg}{\mathbb{D}}
\newcommand{\jg}{\mathbb{J}}
\newcommand{\Ha}{\mathcal{H}}
%\newcommand{\A}{\mathcal{A}}
\newcommand{\sg}{\mathbb{S}}

\newcommand{\Bc}{\mathcal{B}}
\newcommand{\Df}{\mathfrak{D}}
\newcommand{\Dc}{\mathcal{D}}
%\newcommand{\Tc}{\mathcal{T}}
\newcommand{\Mf}{\mathfrak{M}}

\newcommand{\Sg}{\mathbb{S}}

\newcommand{\Q}{\ensuremath{\mathbb{Q}}}
\newcommand{\Z}{\ensuremath{\mathbb{Z}}}
\newcommand{\N}{\ensuremath{\mathbb{N}}}
\newcommand{\R}{\ensuremath{\mathbb{R}}}
\renewcommand{\S}{\mathbb{\Sigma}}
\newcommand{\A}{\mathcal{A}}
\newcommand{\E}{\ensuremath{\exists}}
\newcommand{\iso}{\ensuremath{\cong}}
\newcommand{\union}{\ensuremath{\cup}}
\newcommand{\morinyec}{\ensuremath{\precapprox}}

\newcommand{\nin}{\ensuremath{\notin}}
\newcommand{\tog}{\makebox[7mm][l]}
\newcommand{\toge}{\makebox[11mm][l]}
\newcommand{\toget}{\makebox[13mm][l]}
\newcommand{\togeth}{\makebox[14mm][l]}
\newcommand{\togethe}{\makebox[15mm][l]}
\newcommand{\together}{\makebox[17mm][l]}
\newcommand{\niso}{\ensuremath{\not \cong}}


\newcommand{\Mg}{\mathbb{M}}
\newcommand{\Bg}{\mathbb{B}}
\newcommand{\Lg}{\mathbb{L}}
\newcommand{\Tg}{\mathbb{T}}

\newcommand{\sketch}{\Red{{\sc sketch}}}

\newcommand{\restr}[2]{#1\!\!\boldsymbol{\restriction}\!#2}

\newcommand{\vacio}{\varnothing}
\newcommand{\done}{\ensuremath{\checkmark}}

\newcommand{\ida}{$\Rightarrow \; )$ }
\newcommand{\regr}{$\Leftarrow \; )$ }

\newcommand{\ol}[1]{\overline{#1}}

\newcommand{\Tsf}{\mathsf{T}}

\newcommand{\inds}[1]{\index[simb]{#1}}

\newcommand{\B}{\mathbb{B}}
%\newcommand{\N}{\mathbb{N}}

\newcommand{\vx}{\vec{x}}
\newcommand{\vy}{\vec{y}}
\newcommand{\vz}{\vec{z}}
\newcommand{\vt}{\vec{t}}
\newcommand{\vf}{\vec{f}}


% \newcommand{\propo}{\ensuremath{\mathsf{PROP}}}
% \newcommand{\atom}{\ensuremath{\mathsf{ATOM}}}
\newcommand{\term}{\ensuremath{\mathsf{TERM}}}
\newcommand{\form}{\mathsf{FORM}}

\newcommand{\true}{\mathop{\mathsf{true}}}

%\newcommand{\id}{\mathsf{Id}}

%\newcommand{\uc}{\mathcal{U}}
%\newcommand{\Ic}{\mathcal{I}}
%\newcommand{\pc}{\mathcal{P}}
%\newcommand{\qc}{\mathcal{Q}}
%\newcommand{\mc}{\mathcal{M}}
\newcommand{\supc}{\supseteq}
\newcommand{\limo}{\mathop{\mathpzc{Lim}}}
\newcommand{\ord}{\mathsf{OR}}

\newcommand{\pt}[1]{\langle #1 \rangle}


%%%% frames
\newcommand{\titulos}[2]{\frametitle{#1}\framesubtitle{#2}}
\newcommand{\fot}[1]{\footnote{\scriptsize{#1}}}


%%%% ambientes

\newcommand{\cb}[2]{\colorbox{#1}{#2}}

\newcommand{\bc}{\begin{center}}
\newcommand{\ec}{\end{center}}
\newcommand{\be}{\begin{enumerate}}
\newcommand{\ee}{\end{enumerate}}
\newcommand{\bi}{\begin{itemize}}
\newcommand{\ei}{\end{itemize}}
\newcommand{\beq}{\begin{equation}}
\newcommand{\eeq}{\end{equation}}
\newcommand{\beqs}{\begin{equation*}}
\newcommand{\eeqs}{\end{equation*}}
\newcommand{\ba}{\begin{array}}
\newcommand{\ea}{\end{array}}


% \newtheorem{theorem}{Teorema}
% \newcommand{\teo}[1]{\begin{theorem} #1 \end{theorem}}
% \newtheorem{proposition}{Proposici\'on}
% \newcommand{\prop}[1]{\begin{proposition} #1 \end{proposition}}
% \newtheorem{definition}{Definici\'on}
% \newcommand{\defin}[1]{\begin{definition} #1 \end{definition}}
% \newtheorem{corollary}{Corolario}
% \newcommand{\cor}[1]{\begin{corollary} #1 \end{corollary}}
% \newtheorem{lemma}{Lema}
% \newcommand{\lema}[1]{\begin{lemma} #1 \end{lemma}}
% \newcommand{\dem}[1]{\begin{proof} #1 \end{proof}}

%\renewcommand{\qed}{\qedsymbol{$\mathbf{\dashv}$}}

%\newcommand{\proof}{\hfill\\\noindent\textbf{\textit{Demostraci\'on. }}}

\newcommand{\hint}{\emph{Sugerencia: }}


\newcounter{EjempCtr}[section]
\newenvironment{enumrom}{\renewcommand{\theenumi}{\roman{enumi}}%
\renewcommand{\theenumii}{\roman{enumii}}
\renewcommand{\theenumiii}{\roman{enumiii}}
\renewcommand{\theenumiv}{\roman{enumiv}}
\begin{enumerate}}{\end{enumerate}}
\newenvironment{Ejemplo}
        {\stepcounter{EjempCtr}%
        \begin{description}\item[Ejemplo \thesection.\arabic{EjempCtr}]}%
        {\end{description}}
\newenvironment{demostr}{%
             {\em Demostración:}
                \begin{quotation}}{\end{quotation}}

   \newcommand{\beje}{\begin{Ejemplo}}
\newcommand{\eeje}{\end{Ejemplo}}


\newtheorem{eje}{Ejemplo}[section]
\newcommand{\ejem}[1]{\begin{eje}\normalfont #1 \end{eje}}

% \renewcommand\contentsname{\'Indice}
%\renewcommand\chaptername{Cap\'itulo}
% \renewcommand\indexname{\'Indice}

%%\newcommand{\qed}{\hfill$\mathbb{Qed}$}
%\newcommand{\qed}{\hfill$\mathsf{\boldsymbol{\dashv}}$}
%\renewcommand{\qed}{\hfill$\boldsymbol{\dashv}$}


\newenvironment{prueba}{\vspace{-5mm}\noindent\textbf{Demostraci\'on}\\}{
\noindent$\blacksquare$\\}

\newcommand{\Ejercicios}{\section*{Ejercicios}}

 \newenvironment{manitas}{%
      \renewcommand{\labelitemi}{\ding{44}}%
      \vspace{-0.5cm}%
      \begin{itemize}%
      \setlength{\itemsep}{0pt}\setlength{\parsep}{0pt}\setlength{\topsep}{0pt}%
      }{\end{itemize}}
\newenvironment{malitos}{%
      \renewcommand{\labelitemi}%
            {\raisebox{1.5ex}{\makebox[0.3cm][l]{\begin{rotate}{-90}%
            \ding{43}\end{rotate}}}}%
      \vspace{-0.5cm}%
      \begin{itemize}%
      \setlength{\itemsep}{0pt}\setlength{\parsep}{0pt}\setlength{\topsep}{0pt}%
      }{\end{itemize}}
\newenvironment{ejercs}{
     \renewcommand{\labelenumi}{\thesection.\theenumi.-}
     \renewcommand{\labelenumii}{\theenumii)}
     \begin{enumerate}}
     {\end{enumerate}}

   \newcommand{\bej}{\begin{ejercs}}
\newcommand{\eej}{\end{ejercs}}


%\newenvironment{leterize}{%
%        \renewcommand{\theenumi}{\alph{enumi}}
%        \begin{enumerate}}{\end{enumerate}}

%\newenvironment{manitas}{%
%      \renewcommand{\labelitemi}{\ding{44}}%
%      \vspace{-0.5cm}%
%      \begin{itemize}%
%      
% \setlength{\itemsep}{0pt}\setlength{\parsep}{0pt}\setlength{\topsep}{0pt}%
%      }{\end{itemize}}



%%=============================================================================

\def\stackunder#1#2{\mathrel{\mathop{#2}\limits_{#1}}}


%%%% notas

\newcommand{\doubt}{\Red{{\LARGE {\sf ??}}}}

\newcommand{\coment}[1]{\hfill\\ \Big[{\bf Comentario Privado:} #1\Big]}
\newcommand{\preg}[1]{\hfill\\ \BrickRed{{\bf Pregunta:} #1}}
\newcommand{\conjet}[1]{\hfill\\ \OliveGreen{{\bf Conjecura:} #1}}

\newcommand{\pendiente}{\BrickRed{{\sc Pendiente}}}
\newcommand{\verifpendiente}{\BrickRed{{\sc Verificación pendiente}}}


%--------------------------------------------------------------------------

\DeclareMathAlphabet{\mathpzc}{OT1}{pzc}{m}{it}


\title{Introducci\'on y Sintaxis de la L\'ogica de Proposiciones \\
Lógica Computacional 2018-2, Nota de clase 1}
\author{Favio Ezequiel Miranda Perea\and Araceli Liliana Reyes Cabello\and
Lourdes Del Carmen Gonz\'alez Huesca \and Pilar Selene Linares Ar\'evalo}
\date{ Facultad de Ciencias UNAM}

\begin{document}
\maketitle

\section{Introducci\'on}
En muchos aspectos cotidianos hacemos referencia a la l\'ogica, ya sea de forma 
expl\'icita o impl\'icita pero estricta y formalmente 
\begin{center}
 \textbf{?`Qué es la l\'ogica?}
\end{center}
\bi
\item La habilidad para determinar respuestas correctas mediante un proceso
  estandarizado.
\item El estudio de la inferencia formal.
\item Una secuencia de afirmaciones verificadas
\item Razonamiento, lo opuesto a la intuici\'on.
\item La deducción o derivación de afirmaciones a partir de afirmaciones 
previas.
\item Rama de la filosof\'ia que trata de las formas de razonamiento y 
pensamiento, especialmente la inferencia y el método científico.
\item Estudio de los principios de inferencia v\'alida.
\ei
 
\begin{center}
 \textbf{?`Por qué estudiar l\'ogica?}
\end{center}
\bi
 \item La l\'ogica computacional, que es la que veremos en este curso, 
  refleja el uso de la l\'ogica en computación, en cierto sentido es el 
 \emph{cálculo de la computación}, un fundamento matemático para tratar la 
  información y razonar acerca del comportamiento de programas.
 \item Además proporciona un entrenamiento que lleva a formas correctas y 
  precisas de razonamiento, en particular permite generar descripciones libres 
  de ambigüedades.
 \item Es relevante para cursos posteriores como inteligencia artificial, bases 
  de datos o ingeniería de software.
 \item Muchos problemas computacionales involucran l\'ogica en su representación
  o solución.
 \item Si bien la l\'ogica es una ciencia muy antigua y la computación de 
  reciente estudio, ambas se han fundido durante el siglo pasado, de manera 
  que hoy en d\'ia es difícil marcar una frontera entre ambas ciencias. 
\ei

\subsection{Motivación para la formalización del razonamiento correcto}
\noindent La l\'ogica ha sido pieza clave para estructurar el pensamiento y el 
razonamiento:
\bi
\item Dar un fundamento a las matemáticas.
\item Eliminar errores del razonamiento.
\item Encontrar una forma eficiente para llegar a una justificación de una 
conclusión, dada cierta información en forma de premisas.
\ei

\section{Argumentos lógicos}
Un \emph{argumento lógico} es una colección finita de afirmaciones 
(proposiciones) dividida en premisas y conclusión. Las premisas y conclusión 
deben ser susceptibles de recibir un valor de verdad. El argumento lógico 
puede ser correcto o incorrecto.

Un problema central de la l\'ogica es verificar la correctud de un argumento 
lógico.
Y una aplicación principal en computación es: \\
?`Cómo sabemos que el código que nos proporcione un programador es correcto, 
es decir, realmente el programa siempre hará lo que queremos?

Veamos algunos ejemplos:
\bi
 \item \emph{Todos los hombres son mortales, Socrates es
  hombre. Por lo tanto Socrates es Mortal.} ?`Es correcto?
 \item \emph{Nada es mejor que Superman, Un taco es mejor que nada. Por lo 
  tanto un taco es mejor que Superman.} ?`Es correcto?
 \item \emph{Los borogrovos se ponen fefos durante el brilgo. Pac es un 
  borogrovo y hay brilgo. Por lo tanto Pac está fefo.}?`Es correcto?
\ei      

Pero, ?`cómo decidir si un argumento es correcto? 
?` Por sentido común, por votación, por autoridad ? 

Nosotros usaremos la l\'ogica como un medio para decidir la validez de un
argumento.
Un argumento es correcto o válido si {\bf suponiendo} que sus premisas son 
ciertas, entonces necesariamente la conclusión también lo es.   
Obsérvese que las premisas se suponen siempre ciertas. En el análisis de un   
argumento lógico no importa el contenido sino la forma de las premisas.


\paragraph{La isla de los caballeros y los bribones}

En la isla de los caballeros y bribones sólo hay dos clases de habitantes, 
los caballeros que siempre dicen la verdad y los bribones que siempre mienten. 
Un n\'aufrago llega a la isla y encuentra dos habitantes: A y B. 
El habitante A afirma : Yo soy un bribón o B es un caballero. \\
El acertijo consiste en averiguar c\'omo son A y B.\\

\noindent Representación l\'ogica: 
\bi
 \item[] $p$ := ``A es un bribón'', $q$ := ``B es un caballero''.\\
  Entonces A dijo $s$ := p $\lor$ q.
\ei
Para resolver el acertijo suponemos que A es bribón.
 \bi
  \item En tal caso $s$ es mentira por lo que ambas $p$ y $q$ son falsas.
  \item Si $p$ es falsa entonces A no es bribón.
  \item Contradicción: A no puede ser bribón y no bribón a la vez.
 \ei
 De manera que A no es bribón y entonces es caballero y $s$ es cierta. 
 Pero como $p$ es falsa, dado que A es caballero entonces $q$ debe ser cierta 
 de manera que también B es caballero.


\paragraph{Especificación de un sistema de cómputo}
?`Puede existir un sistema de cómputo y un estado particular del mismo con las 
siguientes características?
\begin{enumerate}
 \item El sistema está en modo multiusuario si y sólo si se comporta 
  normalmente.
 \item Si el sistema se comporta normalmente, su núcleo funciona.
 \item El núcleo no funciona o el sistema está en modo de interrupción.
 \item Si el sistema no está en modo multiusuario entonces está en modo de 
  interrupción.
 \item El sistema no está en modo de interrupción.
\end{enumerate}

?`Cómo resolvemos este problema? 


\subsection{Características de un argumento lógico}

\bi
 \item Los argumentos involucran individuos (personas, objetos, cosas en
  general): \emph{los hombres, Socrates, un taco, Superman, etc..}
 \item Los individuos tienen propiedades: \emph{ser mortal, ser mejor, etc}
 \item Los argumentos se forman mediante proposiciones, clasificadas como
  premisas y conclusión del argumento. 
 \item Una proposición es una oración que puede calificarse como verdadera o 
  falsa.
 \item Las proposiciones pueden ser compuestas.
 \item Un argumento puede ser correcto (válido) o incorrecto (inválido)
 \item Un argumento nunca es verdadero ni falso.
\ei



\section{Sistema l\'ogico}

\subsection{Componentes de un sistema lógico}

Cualquier sistema lógico consta al menos de las siguientes tres componentes:
\bi
 \item Sintaxis: lenguaje formal que se utilizará como medio de expresión.
 \item Semántica: mecanismo que proporciona significado al lenguaje formal dado 
  por la sintaxis.
 \item Teoría de la prueba: colección de mecanismos puramente sintácticos cuyo 
  propósito fundamental es obtener o identificar las expresiones válidas, 
  respecto a la semántica, de un lenguaje, en particular se encarga de decidir 
  la correctud de un argumento lógico por medios puramtente sintácticos.
\ei

\subsection{Propiedades de un sistema l\'ogico}
\bi
\item \textbf{Consistencia}: se refiere a que en el sistema l\'ogico no hay
contradicciones.
\item \textbf{Correctud}: las reglas del sistema no pueden obtener una 
inferencia falsa a partir de una verdadera.
\item \textbf{Completud}: no hay sentencias verdaderas que no se puedan 
demostrar en el sistema. Es decir, todo lo verdadero es demostrable.
\ei


\section{Lógica proposicional}
La l\'ogica proposicional es el sistema lógico más simple. Se encarga del 
manejo de proposiciones mediante conectivos lógicos.
 
Una \textbf{proposición} es un enunciado que puede calificarse de verdadero o 
falso, por ejemplo:
\bi
 \item El es bribón
 \item Tu eres caballero
 \item Hoy es jueves.
 \item Hay vida en Marte.
\ei
 
\textbf{NO} son proposiciones: ?`Llueve?, el perro azul, !`Viva Pancho Villa!, 
el cerro de la silla, juan perez, etc.


\subsection{El lenguaje \propo}
\noindent
Definimos ahora un lenguaje formal para la l\'ogica de proposiciones.
El alfabeto de este lenguaje consta de:
\bi
 \item Símbolos o variables proposicionales (un número infinito): 
  $p_1,\ldots,p_n,\ldots$
 \item Constantes l\'ogicas: $\bot,\top$
 \item Conectivos u operadores lógicos: $\neg,\land,\lor,\imp,\iff$
 \item Símbolos auxiliares: $(,\;)$
\ei

\noindent El conjunto de expresiones o fórmulas atómicas, denotado $\atom\;$ 
consta de:
\bi
\item Las variables proposicionales: $p_1,\ldots,p_n,\ldots$.
\item Las constantes $\bot,\top$.
\ei

Las expresiones que formarán nuestro lenguaje $\propo$, llamadas usualmente 
fórmulas, se definen recursivamente como sigue: 
\begin{enumerate}
 \item Si $\varphi\in\atom$ entonces $\varphi\in\propo$. 
 Es decir, toda fórmula atómica es una fórmula.
 \item Si $\varphi\in\propo$ entonces $(\neg\varphi)\in\propo$
 \item Si $\varphi,\psi\in\propo$ entonces
  
$(\varphi\land\psi),(\varphi\lor\psi),(\varphi\imp\psi),
(\varphi\iff\psi)\in\propo$.
 \item Son todas.
\end{enumerate}
La última cláusula de la definición garantiza que el conjunto $\propo$ es el
mínimo conjunto cerrado bajo las tres primeras reglas.

\espc


La definición anterior puede darse en la llamada forma de Backus-Naur para
definir gramáticas como sigue:
\[
 \begin{array}{rcl}
  \varphi,\psi & ::= & VarP \mid \bot \mid \top \mid 
(\neg\varphi) \mid (\varphi\land\psi) \mid (\varphi\lor\psi) \mid 
(\varphi\imp\psi) \mid (\varphi\iff\psi) \\
VarP & ::= & p_1 \mid p_2 \mid \ldots \mid p_n \mid \ldots
 \end{array}
\]

\subsection{Precedencia y Asociatividad de los Operadores Lógicos}

Para evitar el uso de paréntesis al máximo definimos la siguiente precedencia
de operadores de mayor a menor:
\bi
\item $\neg$
\item $\land,\lor$
\item $\imp$
\item $\iff$
\ei

De manera que la operación a realizar primero es la dada por el conectivo de
mayor precedencia. Veamos algunos ejemplos:
\[
\begin{array}{rll}
\neg p\imp r  & \text{ es } &  (\neg p)\imp r \\
p\imp q\land r& \text{ es } & p\imp (q\land r)\\ 
\neg p\imp q \iff r & \text{ es } & \big((\neg p)\imp q\big) \iff r\\
p\imp q \land r\iff s\lor \neg t & \text{ es } & 
\big(p\imp q \land r\big)\iff \big(s\lor (\neg t)\big)\\
p\land q \imp \neg q\lor s  & \text{ es } & (p\land q)\imp ((\neg q)\lor s) \\
\end{array}
\]


Obsérvese que los operadores $\land,\lor$ tienen la misma precedencia de
manera que expresiones como $p\land q\lor r$ son ambiguas y no pueden ni
deben ser utilizadas.\\

Acerca de la asociatividad, los operadores $\land,\lor,\iff$ son asociativos
de manera que expresiones como $p\lor q\lor r$ o $\neg p\iff p\lor s\iff t$
están libres de ambigüedades.\\
Por otra parte el operador $\imp$ no es asociativo pero adoptaremos la
convención usual de asociarlo a la derecha, es decir expresiones del estilo
$\vphi_1\imp\vphi_2\imp\vphi_3$ se entenderán como 
$\vphi_1\imp(\vphi_2\imp\vphi_3)$


\section{Definiciones Recursivas y el Principio de Inducción}

Las definiciones recursivas son omnipresentes en ciencias de la computación, 
la definición del lenguaje {\propo} es un ejemplo de esta clase de definiciones
que consisten en definir propiedades o funciones en una estructura de datos
mediante un análisis de casos de las distintas formas sintácticas que definen a 
la estructura de datos correspondiente.\\

Veamos un ejemplo de definición recursiva, la función $np:\propo\imp\N$ que 
devuelve el número de paréntesis de una fórmula dada. Se define como sigue:
\[
\begin{array}{rll}
np(\varphi) & = & 0\quad \text{ si }\varphi \text{ es atómica }\\
np((\neg\varphi)) & = & np(\varphi)+2\\
np((\varphi\star\psi)) & = & np(\varphi)+np(\psi)+2
\end{array}
\]

Si se desean probar propiedades acerca de estructuras o funciones definidas
recursivamente como nuestro lenguaje {\propo}, como por ejemplo que 
$np(\varphi)$ siempre es un n\'umero par, lo adecuado es usar el llamado
principio de inducción estructural que enunciamos a continuación.

\defin{[Principio de Inducción Estructural para {\propo}]
Sea $\Pe$ una propiedad acerca de fórmulas del lenguaje {\propo}. 
Para probar que toda fórmula $\varphi\in\propo$ tiene la propiedad $\Pe$ basta 
demostrar lo siguiente:
\bi
\item Caso base: toda variable proposicional tiene la propiedad $\Pe$.
\item Hipótesis de inducción: suponer que se cumple la propiedad $\Pe$ para
  $\varphi$ y $\psi$.
\item Paso inductivo: mostrar, usando la hipótesis de inducción, que 
\begin{enumerate}
\item $(\neg\varphi)$ cumple $\Pe$.
\item $(\varphi\star\psi)$ cumple
con $\Pe$, donde $\star\in\{\imp,\land,\lor,\iff\}$.
\end{enumerate}
\ei
}

\subsection{Algunas funciones de importancia}

Las siguientes funciones serán de utilidad más adelante. 
Su definición recursiva se deja como ejercicio:
\bi
 \item Profundidad de una fórmula: 
  $depth(\vp)$ devuelve la profundidad o altura del árbol de análisis 
  sintáctico de $\vp$.
 \item Número de conectivos de una fórmula: 
  $con(\vp)$ devuelve el número de conectivos de $\vp$.
 \item Variables de una fórmula: 
  $vars(\vp)$ devuelve el conjunto o lista de variables que figuran en $\vp$ 
  (sin repeticiones).
 \item Atómicas en una fórmula: 
  $atom(\vp)$ devuelve el número de presencias de fórmulas atómicas en $\vp$.
\ei

Las siguientes relaciones entre algunas de las funciones recién especificadas 
deben verificarse mediante inducción estructural:
\bi
\item $con(\vp) < 2^{depth(\vp)}$
\item $depth(\vp) \leq con(\vp)$.
\item $atom(\vp) \leq 2 con(\vp)+1$.
\ei

\section{Sustitución}
Una operación sintáctica fundamental de las f\'ormulas proposicionales es
la sustitución: en una fórmula dada $\varphi$, una variable proposicional 
$p$ cambia por una fórmula $\psi$.
Así se genera una nueva fórmula denotada $\varphi[p:=\psi]$ obtenida
al sustituir todas las presencias de $p$ en $\varphi$  por $\psi$. 
Esta operación se conoce como sustitución textual y se define recursivamente 
como sigue:
\[
\begin{array}{rll}
p[p:=\psi] & = & \psi \\
q[p:=\psi] & = & q, \text{ si } p\neq q \\
\top[p:=\psi] & = & \top \\
\bot[p:=\psi] & = & \bot \\
(\neg \varphi)[p:=\psi] & = & \neg (\varphi[p:=\psi])\\
(\varphi\land\chi)[p:=\psi] & = &
(\varphi[p:=\psi]\land\chi[p:=\psi])\\
(\varphi\lor\chi)[p:=\psi] & = &
(\varphi[p:=\psi]\lor\chi[p:=\psi])\\
(\varphi\imp\chi)[p:=\psi] & = &
(\varphi[p:=\psi]\imp\chi[p:=\psi])\\
(\varphi\iff\chi)[p:=\psi] & = &
(\varphi[p:=\psi]\iff\chi[p:=\psi])\\
\end{array}
\]
An\'alogamente se define la sustitución simultánea de $n$ variables
proposicionales por $n$ fórmulas 
$\varphi[p_1,\ldots,p_n:=\psi_1,\ldots,\psi_n]$, también
denotada como $\vp[\vec{p}:= \vec{\psi}\,]$.

La operación de sustitución tiene las siguientes propiedades que deben 
mostrarse usando inducción estructural:
\bi
 \item Si $p$ no figura en $\varphi$ entonces $\varphi[p:=\psi]=\varphi$.

 \item Si $p\neq q$ y $p$ no figura en $\chi$ entonces
  $$ \varphi[p:=\psi][q:=\chi]=\varphi[q:=\chi][p:=\psi[q:=\chi]] $$

 \item Si $p\neq q_1,\ldots, q_n$ y $p$ no figura en $\psi_1,\ldots.\psi_n$
  entonces  
$$\varphi[\vec{q},p:=\vec{\psi},\chi]=\varphi[\vec{q}:=\vec{\psi}\;][p:=\chi]$$
\ei

\noindent Algunos ejemplos correspondientes a las propiedades anteriores son:
\bi
\item $(p\imp q\land r)[s:=t\lor p]=p\imp q\land r$, pues $s$ no figura en
  $p\imp q\land r$.
\item Ejemplificamos ahora la segunda propiedad 
\[
\begin{array}{rll}
(q\lor r\imp t\iff s)[r:=p\land t][t:= p\land q] & = & \\
(q\lor (p\land t)\imp t\iff s)[t:= p\land q] & = & \\ 
q\lor (p\land (p\land q))\imp (p\land q)\iff s
\end{array}
\]

Claramente $r \neq t$ y $r\notin Vars(p\land q)$ donde $Vars$ es la 
función que obtiene el conjunto de variables proposicionales de una fórmula, 
en particular $Vars(p\land q)=\{p,q\}$, y tenemos
\[
\begin{array}{rll}
(q\lor r\imp t\iff s)[t:= p\land q][r:=(p\land t)[t:=p\land q]] & = & \\
(q\lor r\imp t\iff s)[t:= p\land q][r:=p\land (p\land q)] & = & \\
(q\lor r\imp (p\land q)\iff s)[r:=p\land (p\land q)] & = & \\
q\lor (p\land (p\land q))\imp (p\land q)\iff s 
\end{array}
\]

\item Un ejemplo de la tercera propiedad es:
\[
\begin{array}{rll}
(r\land t \iff (q\land p\imp s))[r,p,s:=p\land s, t, r][q:= p\lor r] & = & \\
((p\land s)\land t \iff (q\land t\imp r))[q:= p\lor r] & = & \\
((p\land s)\land t \iff ((p\lor r)\land t\imp r)) & = & \\
(r\land t \iff (q\land p\imp s))[r,p,s,q:=p\land s, t, r, p\lor r]
\end{array}
\]
\ei


\end{document}
