	\documentclass[11pt,letterpaper]{article}
\usepackage{../packageslc}
\usepackage{../optionslc}

%%%% Macros para las notas de lenguajes de programacion


%%%% math

\newcommand{\vphi}{\varphi}
\newcommand{\vp}{\varphi}

% \newcommand{\dn}{\mathsf{DN}}
% \newcommand{\dnC}{\mathsf{DN_C}}
% \newcommand{\dnM}{\mathsf{DN_M}}
% \newcommand{\dnp}{\mathsf{DN_p}}
% \newcommand{\dnm}{\mathsf{DN_p^M}}
% \newcommand{\dnc}{\mathsf{DN_p^C}}

%\newcommand{\case}{\mathsf{case}}
%\renewcommand\labelitemi{$\circ$}

\newcommand{\imp}{\rightarrow}
\newcommand{\Imp}{\Rightarrow}
\renewcommand{\iff}{\leftrightarrow}
\newcommand{\Iff}{\Leftrightarrow}
\newcommand{\G}{\Gamma}
\newcommand{\D}{\Delta}


\newcommand{\De}{\mathcal{D}}
\newcommand{\F}{\mathcal{F}}
\newcommand{\Ge}{\mathcal{G}}
\newcommand{\Pe}{\mathcal{P}}
\newcommand{\I}{\mathcal{I}}
\newcommand{\C}{\mathcal{C}}
\newcommand{\K}{\mathcal{K}}
\renewcommand{\L}{\mathcal{L}}
\newcommand{\M}{\mathcal{M}}
\newcommand{\Nc}{\mathcal{N}}
%\newcommand{\E}{\mathcal{E}}
%\newcommand{\R}{\mathcal{R}}
%\newcommand{\Q}{\mathcal{Q}}
\newcommand{\Sc}{\mathcal{S}}
\newcommand{\Te}{\mathcal{T}}
\newcommand{\W}{\mathcal{W}}

\newcommand{\Db}{\mathbb{D}}
\newcommand{\Fb}{\mathbb{F}}
\newcommand{\Kb}{\mathbb{K}}
\newcommand{\Eb}{\mathbb{E}}
\newcommand{\Ebs}{\mathbb{E}^\star}
\newcommand{\Ob}{\mathbb{O}}
\newcommand{\Ib}{\mathbb{I}}
\newcommand{\Rb}{\mathbb{R}}
\newcommand{\Qb}{\mathbb{Q}}
\newcommand{\Kbb}{\mathbb{K}}
\newcommand{\T}{\mathbb{\Theta}}


\newcommand{\kb}{\bbkappa}

\newcommand{\Sf}{\mathsf{\Sigma}}

\newcommand{\fa}{\forall}
\newcommand{\ex}{\exists}

\newcommand{\inc}{\subseteq}

\newcommand{\Lb}{\Lambda}
\newcommand{\Om}{\Omega}
\newcommand{\lb}{\lambda}
\newcommand{\al}{\alpha}
\newcommand{\ga}{\gamma}


\newcommand{\mg}{\mathbb{m}}

\newcommand{\cg}{\mathbb{C}}
\newcommand{\dg}{\mathbb{D}}
\newcommand{\jg}{\mathbb{J}}
\newcommand{\Ha}{\mathcal{H}}
%\newcommand{\A}{\mathcal{A}}
\newcommand{\sg}{\mathbb{S}}

\newcommand{\Bc}{\mathcal{B}}
\newcommand{\Df}{\mathfrak{D}}
\newcommand{\Dc}{\mathcal{D}}
%\newcommand{\Tc}{\mathcal{T}}
\newcommand{\Mf}{\mathfrak{M}}

\newcommand{\Sg}{\mathbb{S}}

\newcommand{\Q}{\ensuremath{\mathbb{Q}}}
\newcommand{\Z}{\ensuremath{\mathbb{Z}}}
\newcommand{\N}{\ensuremath{\mathbb{N}}}
\newcommand{\R}{\ensuremath{\mathbb{R}}}
\renewcommand{\S}{\mathbb{\Sigma}}
\newcommand{\A}{\mathcal{A}}
\newcommand{\E}{\ensuremath{\exists}}
\newcommand{\iso}{\ensuremath{\cong}}
\newcommand{\union}{\ensuremath{\cup}}
\newcommand{\morinyec}{\ensuremath{\precapprox}}

\newcommand{\nin}{\ensuremath{\notin}}
\newcommand{\tog}{\makebox[7mm][l]}
\newcommand{\toge}{\makebox[11mm][l]}
\newcommand{\toget}{\makebox[13mm][l]}
\newcommand{\togeth}{\makebox[14mm][l]}
\newcommand{\togethe}{\makebox[15mm][l]}
\newcommand{\together}{\makebox[17mm][l]}
\newcommand{\niso}{\ensuremath{\not \cong}}


\newcommand{\Mg}{\mathbb{M}}
\newcommand{\Bg}{\mathbb{B}}
\newcommand{\Lg}{\mathbb{L}}
\newcommand{\Tg}{\mathbb{T}}

\newcommand{\sketch}{\Red{{\sc sketch}}}

\newcommand{\restr}[2]{#1\!\!\boldsymbol{\restriction}\!#2}

\newcommand{\vacio}{\varnothing}
\newcommand{\done}{\ensuremath{\checkmark}}

\newcommand{\ida}{$\Rightarrow \; )$ }
\newcommand{\regr}{$\Leftarrow \; )$ }

\newcommand{\ol}[1]{\overline{#1}}

\newcommand{\Tsf}{\mathsf{T}}

\newcommand{\inds}[1]{\index[simb]{#1}}

\newcommand{\B}{\mathbb{B}}
%\newcommand{\N}{\mathbb{N}}

\newcommand{\vx}{\vec{x}}
\newcommand{\vy}{\vec{y}}
\newcommand{\vz}{\vec{z}}
\newcommand{\vt}{\vec{t}}
\newcommand{\vf}{\vec{f}}


% \newcommand{\propo}{\ensuremath{\mathsf{PROP}}}
% \newcommand{\atom}{\ensuremath{\mathsf{ATOM}}}
\newcommand{\term}{\ensuremath{\mathsf{TERM}}}
\newcommand{\form}{\mathsf{FORM}}

\newcommand{\true}{\mathop{\mathsf{true}}}

%\newcommand{\id}{\mathsf{Id}}

%\newcommand{\uc}{\mathcal{U}}
%\newcommand{\Ic}{\mathcal{I}}
%\newcommand{\pc}{\mathcal{P}}
%\newcommand{\qc}{\mathcal{Q}}
%\newcommand{\mc}{\mathcal{M}}
\newcommand{\supc}{\supseteq}
\newcommand{\limo}{\mathop{\mathpzc{Lim}}}
\newcommand{\ord}{\mathsf{OR}}

\newcommand{\pt}[1]{\langle #1 \rangle}


%%%% frames
\newcommand{\titulos}[2]{\frametitle{#1}\framesubtitle{#2}}
\newcommand{\fot}[1]{\footnote{\scriptsize{#1}}}


%%%% ambientes

\newcommand{\cb}[2]{\colorbox{#1}{#2}}

\newcommand{\bc}{\begin{center}}
\newcommand{\ec}{\end{center}}
\newcommand{\be}{\begin{enumerate}}
\newcommand{\ee}{\end{enumerate}}
\newcommand{\bi}{\begin{itemize}}
\newcommand{\ei}{\end{itemize}}
\newcommand{\beq}{\begin{equation}}
\newcommand{\eeq}{\end{equation}}
\newcommand{\beqs}{\begin{equation*}}
\newcommand{\eeqs}{\end{equation*}}
\newcommand{\ba}{\begin{array}}
\newcommand{\ea}{\end{array}}


% \newtheorem{theorem}{Teorema}
% \newcommand{\teo}[1]{\begin{theorem} #1 \end{theorem}}
% \newtheorem{proposition}{Proposici\'on}
% \newcommand{\prop}[1]{\begin{proposition} #1 \end{proposition}}
% \newtheorem{definition}{Definici\'on}
% \newcommand{\defin}[1]{\begin{definition} #1 \end{definition}}
% \newtheorem{corollary}{Corolario}
% \newcommand{\cor}[1]{\begin{corollary} #1 \end{corollary}}
% \newtheorem{lemma}{Lema}
% \newcommand{\lema}[1]{\begin{lemma} #1 \end{lemma}}
% \newcommand{\dem}[1]{\begin{proof} #1 \end{proof}}

%\renewcommand{\qed}{\qedsymbol{$\mathbf{\dashv}$}}

%\newcommand{\proof}{\hfill\\\noindent\textbf{\textit{Demostraci\'on. }}}

\newcommand{\hint}{\emph{Sugerencia: }}


\newcounter{EjempCtr}[section]
\newenvironment{enumrom}{\renewcommand{\theenumi}{\roman{enumi}}%
\renewcommand{\theenumii}{\roman{enumii}}
\renewcommand{\theenumiii}{\roman{enumiii}}
\renewcommand{\theenumiv}{\roman{enumiv}}
\begin{enumerate}}{\end{enumerate}}
\newenvironment{Ejemplo}
        {\stepcounter{EjempCtr}%
        \begin{description}\item[Ejemplo \thesection.\arabic{EjempCtr}]}%
        {\end{description}}
\newenvironment{demostr}{%
             {\em Demostración:}
                \begin{quotation}}{\end{quotation}}

   \newcommand{\beje}{\begin{Ejemplo}}
\newcommand{\eeje}{\end{Ejemplo}}


\newtheorem{eje}{Ejemplo}[section]
\newcommand{\ejem}[1]{\begin{eje}\normalfont #1 \end{eje}}

% \renewcommand\contentsname{\'Indice}
%\renewcommand\chaptername{Cap\'itulo}
% \renewcommand\indexname{\'Indice}

%%\newcommand{\qed}{\hfill$\mathbb{Qed}$}
%\newcommand{\qed}{\hfill$\mathsf{\boldsymbol{\dashv}}$}
%\renewcommand{\qed}{\hfill$\boldsymbol{\dashv}$}


\newenvironment{prueba}{\vspace{-5mm}\noindent\textbf{Demostraci\'on}\\}{
\noindent$\blacksquare$\\}

\newcommand{\Ejercicios}{\section*{Ejercicios}}

 \newenvironment{manitas}{%
      \renewcommand{\labelitemi}{\ding{44}}%
      \vspace{-0.5cm}%
      \begin{itemize}%
      \setlength{\itemsep}{0pt}\setlength{\parsep}{0pt}\setlength{\topsep}{0pt}%
      }{\end{itemize}}
\newenvironment{malitos}{%
      \renewcommand{\labelitemi}%
            {\raisebox{1.5ex}{\makebox[0.3cm][l]{\begin{rotate}{-90}%
            \ding{43}\end{rotate}}}}%
      \vspace{-0.5cm}%
      \begin{itemize}%
      \setlength{\itemsep}{0pt}\setlength{\parsep}{0pt}\setlength{\topsep}{0pt}%
      }{\end{itemize}}
\newenvironment{ejercs}{
     \renewcommand{\labelenumi}{\thesection.\theenumi.-}
     \renewcommand{\labelenumii}{\theenumii)}
     \begin{enumerate}}
     {\end{enumerate}}

   \newcommand{\bej}{\begin{ejercs}}
\newcommand{\eej}{\end{ejercs}}


%\newenvironment{leterize}{%
%        \renewcommand{\theenumi}{\alph{enumi}}
%        \begin{enumerate}}{\end{enumerate}}

%\newenvironment{manitas}{%
%      \renewcommand{\labelitemi}{\ding{44}}%
%      \vspace{-0.5cm}%
%      \begin{itemize}%
%      
% \setlength{\itemsep}{0pt}\setlength{\parsep}{0pt}\setlength{\topsep}{0pt}%
%      }{\end{itemize}}



%%=============================================================================

\def\stackunder#1#2{\mathrel{\mathop{#2}\limits_{#1}}}


%%%% notas

\newcommand{\doubt}{\Red{{\LARGE {\sf ??}}}}

\newcommand{\coment}[1]{\hfill\\ \Big[{\bf Comentario Privado:} #1\Big]}
\newcommand{\preg}[1]{\hfill\\ \BrickRed{{\bf Pregunta:} #1}}
\newcommand{\conjet}[1]{\hfill\\ \OliveGreen{{\bf Conjecura:} #1}}

\newcommand{\pendiente}{\BrickRed{{\sc Pendiente}}}
\newcommand{\verifpendiente}{\BrickRed{{\sc Verificación pendiente}}}


%--------------------------------------------------------------------------

\DeclareMathAlphabet{\mathpzc}{OT1}{pzc}{m}{it}



\title{Especificación formal con lógica de predicados. \\ 
Lógica Computacional 2018-2, Nota de clase 6}
\author{Favio Ezequiel Miranda Perea \and Araceli Liliana Reyes Cabello \and
Lourdes Del Carmen Gonz\'alez Huesca \and Pilar Selene Linares Ar\'evalo}
\date{ Facultad de Ciencias UNAM \\ 13 de marzo de 2018 \\
Material desarrollado bajo el proyecto UNAM-PAPIME PE102117}

\begin{document}
\maketitle

El proceso de especificación o traducción de un lenguaje natural, en 
nuestro caso del español, a la lógica formal no es siempre sencillo. Algunas 
frases del español no se pueden traducir o especificar de una manera 
completamente fiel a la lógica de predicados, como veremos en algunos ejemplos. 
Sin embargo, el proceso es de gran importancia pues es la base de muchos métodos 
de especificación formal utilizados en inteligencia artificial o ingenieria de 
software, como son el lenguaje de especificación \textsc{Z} que es un lenguaje 
formal basado en la teoría de conjuntos de Zermelo-Fraenkel y en la lógica de 
predicados que se usa en la industria, especialmente en sistemas de alta 
integridad, como parte del proceso de desarrollo de software o hardware
tanto en Europa como en Estados Unidos. % o el analizador de modelos Alloy

\section{Algunos consejos}

A continuación presentamos algunos consejos y observaciones que pretenden 
facilitar el proceso de especificación del español por medio de la lógica.
\begin{itemize}
 \item Únicamente podemos especificar afirmaciones o proposiciones, no es 
  posible traducir preguntas, exclamaciones, órdenes, invitaciones, etcétera.

 \item La idea básica es extraer predicados a partir de los enunciados dados 
  en español de manera que el enunciado completo se construya al combinar 
  dichos predicados mediante conectivos y cuantificadores.\\
  Por ejemplo la frase \textit{me gustan los tacos y las pizzas} debe entenderse 
  como \textit{me gustan los tacos y me gustan las pizzas}.\\
  Análogamente \textit{iré de vacaciones a la playa o a la montaña} significa 
  \textit{iré de vacaciones a la playa ó iré de vacaciones a la montaña}.

 \item La conjunción \enquote{y} se traduce como $\land$. La palabra 
  \enquote{pero} también corresponde a una conjunción aunque el sentido 
  original del español se pierde. \\ 
  Por ejemplo \textit{te doy dulces pero haces la tarea} sólo puede 
  especificarse como \textit{te doy dulces y haces la tarea}. Lo cual es 
  diferente en el español.

 \item La disyunción es incluyente \textit{comeremos pollo o vegetales} es 
  decir, se incluye el caso en que se coman ambos.

 \item Con la implicación hay que ser cautelosos, sobre todo en el caso de 
  frases de la forma  \textit{A sólo si B}, lo cual es equivalente con la frase
  \textit{Si no B entonces no A}, que a su vez es equivalente con        
  \textit{Si A entonces~B}. 
  Es un error común intentar especificar dicha frase inicial mediante~$B\imp A$.

 \item Si en el español aparecen frases como \textit{para todos, para 
cualquier,   todos, cualquiera,los, las etc.} debe usarse el cuantificador 
universal~$\fa$

 \item Si en el español frases como \textit{para alg\'un, existe un, 
  alguno,alguna, uno, una etc.} debe usarse el cuantificador 
  existencial~$\ex$.\\ 
  \textbf{Importante:} En ciertas ocasiones, frases en español que
  involucran \textit{alguien, algo} deben especificarse con un cuantificador 
  universal y no un existencial. \\
  Por ejemplo, el enunciado \textit{si hay alguien demasiado alto entonces se 
  pegará con el marco de la puerta} se puede reescribir en español como
  \textit{cualquiera demasiado alto se pegará con el marco de la puerta},
  lo cual nos lleva a la f\'ormula~$\fa x(A(x)\imp C(x))$.\\
  El lector debe convencerse de que no es posible especificar esta oración 
  usando un cuantificador existencial. 
% a veces alguna presencia de \textit{alguien} en
% español puede generar un cuantificador universal y no un existencial. 
% % debido a la equivalencia
% % lógica $\ex x\vp \imp \psi\equiv \fa x(\vp\imp\psi)$ cuando $x\notin
% % Vl(\psi)$.
% Por ejemplo el enunciado 
% \textit{si hay alguien demasiado alto entonces chocará con el marco de la 
% puerta} se puede reescribir en español 
% como \textit{cualquier persona muy alta chocará con el marco de la puerta}, 
% lo cual nos lleva a $\fa x(Ax\imp Cx)$

 \item Cualquier especificación compuesta que involucre cuantificadores puede
  formarse identificando en ella alguno de los cuatro juicios aristotélicos:
  \bi
   \item Universal afirmativo \textit{Todo S es P} traducido mediante
    $\fa x(S(x)\imp P(x))$.
%  Est\'a proposici\'on corresponde al juicio universal afirmativo y
%se traduce con la f\'or\-mu\-la  
   \item Existencial afirmativo. \textit{Alg\'un S es P} traducido mediante
    $\ex x(S(x)\land P(x)).$
%Corresponde al juicio existencial afirmativo y se traduce
%mediante la f\'or\-mu\-la
   \item Universal negativo: \textit{Ning\'un S es P} traducido mediante 
    $\fa x(S(x)\imp \neg P(x))$
%Corresponde al juicio universal negativo, la palabra \enquote{ning\'un} es 
%una contracci\'on de \enquote{no alg\'un} y se traduce con
 %o equivalentemente con $\neg\ex x\;(S(x)\land P(x))$. 
   \item Existencial negativo:  \textit{Alg\'un S no es P} traducido mediante
    $\ex x(S(x)\land \neg P(x))$
  %Esta proposici\'on corresponde al juicio existencial afirmativo y
%se traduce con la f\'ormula
  \ei
  Por supuesto que en especificaciones complicadas $S$ y $P$ no serán predicados
  sino fórmulas compuestas.

 \item Pronombres como \textit{él, ella, eso} no se refieren a un individuo
  particular sino que se usan como referencia a algo o alguien mencionado 
  previamente, por lo que obtienen significado del contexto particular.
  Cuando un pronombre aparezca en un enunciado debe uno averiguar primero a 
  quién o qué se refiere. \\
  Por ejemplo, en el enunciado \textit{Martha es amiga de Lupita pero ella no 
es   amiga de Karla} debe traducirse como \textit{Martha es amiga de Lupita y 
  Lupita no es amiga de Karla}. Similarmente, cuando necesitamos de 
  variables, como en \textit{hay un perro con manchas y él ladra en las 
  mañanas} es un error tratar de traducir por separado \textit{hay un perro con
  manchas} y \textit{él ladra en las mañanas} puesto que lo que existe está
  ligado con lo que ladra por la conjunción, de manera que debe
  utilizarse una variable que modele esta conexión. 

 \item Las variables no se mencionan en español, son sólo un formalismo
  para representar individuos, por ejemplo la fórmula~$\fa x(M(x)\imp T(x))$ 
  puede escribirse en español como \textit{Cualquier minotauro es troyano} 
  y no como \textit{para cualquier x, si x es minotauro entonces x es 
  troyano}. \\
  Enunciados de esta forma sólo sirven como pasos intermedios en el proceso de 
  traducción pues no forman parte del español correcto ni de la lógica formal. 
  A este mismo respecto es prácticamente imposible que en una traducción del 
  español figuren variables libres.

 \item Los esquemas~$\forall x(A\to B)$ y~$\exists x(A\land B)$ son de gran
  utilidad y bastante comunes. \\
  Menos comunes, aunque también adecuados, son los esquemas 
  \makebox[4em]{$\forall x(A\!\land\!B)$}, 
  \makebox[4em]{$\forall x(A\!\lor\!B)$} y 
  \makebox[4em]{$\exists x(A\!\lor\!B)$}\;.

 \item El esquema \makebox[4.5em]{$\exists x(A\!\to\!B)$}, si bien es una 
  fórmula sintácticamente correcta, es extremadamente raro que figure en una
  traducción del español.
  
 \item El hecho de que se usen dos o más variables distintas no implica
  que éstas representen a elementos distintos del universo, de manera
  que para especificar dos individuos distintos no es suficiente contar
  simplemente con variables distintas.\\
  Las fórmulas~$\exists x P(x)$ y~$\exists x\exists y(P(x)\land P(y))$ expresan 
  ambas lo mismo, a saber que un objeto del universo cumple~$P$. Se debe 
  agregar explícitamente que $x$ e $y$ tienen la propiedad de ser distintos, es 
  decir~$x\ne y$.

% \item El cuantificador existencial no restringe a que el objeto del
%   que se habla sea único. Tanto la afirmación \textit{hay una luna} como {\em
%     hay lunas} se traducen como $\ex x L(x)$. Esta fórmula es equivalente a 
% $\ex x L(x)\land \ex y L(y)$
%  y a $\ex x\ex y (L(x)\land L(y))$ de manera que el hecho de que se usen dos
%  variables no implica la existencia de dos objetos puesto que ambas variables
%  pueden denotar al mismo individuo.
%  Si se debe enfatizar por ejemplo \textit{hay dos lunas} entonces debe usarse
%   la igualdad como en la siguiente fórmula $\ex x\ex y(L(x)\land L(y)\land 
% x\neq y)$.

% \item La frase \textit{Todo gato maulla} debe traducirse como
% $\fa x(G(x)\imp M(x))$ y no como $\fa x(G(x)\land M(x))$
% ya que esta última se traduce como \textit{Todos son gatos y maullan}.

% \item La frase \textit{Si el dios Baco bebe vino entonces todos beben vino} 
% se traduciría como $\ex x (B(x)\imp \fa y V(y))$ pero esta frase
% es falsa, se ha generado una fórmula válida que no lo es ya que 
% no es cierto que el dios Baco exista y que no es cierto que todos
% bebamos.
\end{itemize}


En las siguientes secciones damos diversos ejemplos de especificación formal
con lógica de predicados.

\section{Especificación de listas}

En lenguajes de programación dado un tipo de datos $A$ se define el tipo de
datos de las listas con elementos de tipo $A$, denotado $L_A$ recursivamente
como sigue:
\be
\item La lista vacía, denotada $nil$ es un elemento de $L_A$
\item Si $a$ es un elemento de $A$ y $\ell$ es una lista entonces
  $cons(a,\ell)$ es un elemento de $L_A$.
\item Son todas.
\ee

A continuación presentamos la especificación formal en lógica de predicados 
para este tipo de datos y para algunas de sus operaciones: \\
Sean $nil$ un símbolo de constante para la lista vacía, $cons^{(2)}$ un
símbolo de función para la construcción de listas y $A^{(1)},L_A^{(1)}$  
símbolos de predicado para el tipo $A$ y listas de $A$ respectivamente.\\
La definición del tipo de datos de listas con elementos en $A$ se formaliza 
como sigue:
\bi
 \item $L_A(nil)$
 \item $\fa x\fa y\big(A(x)\land L_A(y)\to L_A(cons(x,y)\big)$
 \item $\fa x\Big(L_A(x)\to x=nil \lor \exists y\exists z\big(A(y) \land 
  L_A(z) \land x = cons(y,z)\big)\Big)$
\ei
La \'ultima f\'ormula indica que cualquier lista s\'olo puede tener una de dos 
formas.
  
\noindent Otras especificaciones acerca de listas son:
\bi
 \item La lista vacía no se puede construir con la función~$cons$. 
 $$ \fa x \fa y (cons(x,y)\neq nil)$$

 \item La concatenación de dos listas es una lista.
  \bi
   \item Funcionalmente: sea $app^{(2)}$ un símbolo de función que
    representa a la concatenación de dos listas. La especificación es
    $$ \fa x\fa y \big(L_A(x)\land L_A(y)\to L_A(app(x,y))\big)$$ 
   \item Relacionalmente: sea $App^{(3)}$ un predicado ternario
    que simbolice a la relación 
    \enquote{$z$ es la concatenación de $x$ con $y$}. La especificación es
    $$\fa x \fa y \fa z\big(L_A(x)\land L_A(y)\land App(x,y,z)\to L_A(z)\big)$$
   \ei
De este ejemplo se observa que una especificación formal puede darse usando 
funciones o bien predicados o relaciones. Cúal opción elegir dependerá de los 
objetivos particulares del problema en cuestión. 

 \item La concatenación de la lista vacía con cualquier otra lista $\ell$ 
  es~$\ell$.
  \bi
   \item Funcionalmente: $ \fa x (app(nil,x)=x) $
   \item Relacionalmente: $ \fa x App(nil,x,x)$
  \ei

 \item La concatenación de una lista no vacía~$\ell_1$ con una lista
  cualquiera~$\ell_2$ es la lista cuya cabeza es la cabeza de~$\ell_1$ y cuya 
  cola es la concatenación de la cola de $\ell_1$ con $\ell_2$.
 \bi
  \item Funcionalmente utilizando las funciones $cons$ y $app$:
    $$ \fa x\fa y\fa z \big(app(cons(x,y),z) = cons(x,app(y,z))\big)$$
  \item Funcionalmente utilizando funciones $hd^{(1)}$ y $tl^{(1)}$ que 
   calculan la cabeza y cola de una lista:
    $$\fa x\fa y\big(x\neq nil \to app(x,y)=cons(hd(x),app(tl(x),y))\big)$$
  \item Relacionalmente usando el constructor de listas~$cons$
    $$\fa x\fa y\fa w\fa z\big(App(x,y,w)\to App(cons(z,x),y,cons(z,w))\big)$$
  \item Relacionalmente usando cabeza y cola:
    $$\fa x\fa y\fa w\fa z\big(App(tl(x),y,w)\to App(x,y,cons(hd(x),w))\big)$$
 \ei

 \item La reversa de una lista es una lista.
 \bi
  \item Funcionalmente: sea $rev$ un símbolo de función de índice uno
  que representa a la reversa de una lista. La especificación es:
    $$ \fa x \big(L_A(x)\to L_A(rev(x))\big) $$
  \item Relacionalmente: sea $Rev(x,y)$ un predicado binario que represente
  a la relación \enquote{y es la reversa de x}. La especificación es:
    $$\fa x \fa y\big(L_A(x)\land Rev(x,y)\to L_A(y)\big)$$
 \ei

 \item La reversa de la lista vacía es la lista vacía.
 \bi
  \item Funcionalmente: $ rev\,nil=nil $
  \item Relacionalmente: $ Rev(nil,nil) $
 \ei
 
 \item La reversa de una lista no vacía es la concatenación de la
  reversa de su cola con la lista que contiene a su cabeza.
 \bi
  \item Funcionalmente: 
    $\fa x\fa y \big(rev(cons(x,y))=app(rev(y),cons(x,nil))\big)$
  \item Funcionalmente usando cabeza y cola:
    $\fa x\big(x\neq nil \to rev(x)=app(rev(tl(x)),cons(hd(x),nil))\big)$
  \item Relacionalmente usando~$cons$:
    $ \fa x Rev\big(cons(x,y),app(rev(y),cons(x,nil))\big) $
  \item Relacionalmente usando cabeza y cola:
    $\fa x \big(x\neq nil\to Rev(x,app(rev(tl(x)),cons(hd(x),nil)))\big)$
 \ei
\ei

  
\section{El juego de Risk}

Risk es un juego de mesa que modela estrategias de guerra en el cual pueden 
participar de 2 a 6 jugadores. Aquí vamos a representar algunos aspectos de la
versión Risk II para computadora. \\
En el mundo hay jugadores, pa\'ises, tarjetas y n\'umeros:
\bi
 \item Los jugadores pueden ser humanos o virtuales manejados por la
  computadora. Aquí no haremos distinción. Algunos jugadores son
  \textit{Barbacena, Bonaparte, Wellington, Solignac, Maransin, Marmont,
  Spencer, Mackenzie, Pach, Menelao, Mega, Charal, Polain, Ardilio, Taupin, 
  Vauban}
 \item Los pa\'ises son algunos de los pa\'ises actuales y otros que no 
  necesariamente corresponden a la división política actual.
 \item Las tarjetas se otorgan si un jugador hizo al menos una conquista en 
  su turno. Al tener un jugador 5 tarjetas está obligado a cambiarlas. De 
  acuerdo a la figura en la tarjeta (tropa, cañ\'on, caballo) se reciben 
  refuerzos por cada 3 tarjetas.
 \item Los números se usarán como auxiliar para contar.
\ei

Introducimos un lenguaje de predicados para representar el juego
mediante ejemplos, usaremos nombres completos para constantes y
nombres cortos para predicados para facilitar la correspondencia con
el español.
\bi
 \item Solignac ha ocupado China   $\qquad  Oc(solignac,china)$

 \item Barbacena ha defendido con éxito Peru  $\qquad  DefEn(barbacena,peru)$

 \item Bonaparte ha ganado una tarjeta $\qquad  Gt(bonaparte,1)$

 \item Vauban vence a Mackenzie en Quebec  
    $\qquad Ven(vauban,mackenzie,quebec)$

 \item Taupin se retira de Japón  $\qquad   Ret(taupin,japon)$

 \item Spencer ha intercambiado tarjetas $\qquad  Int(spencer)$

 \item Mackenzie ataca a D'erlon en Argentina desde Brasil 
    $\qquad At(mackenzie,d'erlon,argentina,brasil)$

 \item Menelao recibe 10 refuerzos  $\qquad  Ref(menelao,10)$

 \item Vauban hace un movimiento táctico de Alaska a Ucrania 
    $\qquad Tac(vauban,alaska,ucrania)$

 \item Spencer ha sido aniquilado por Bonaparte  $\qquad Ani(bonaparte,spencer)$

 \item $3$ tropas se colocan en Francia $\qquad Tro(3,francia)$

 \item $5$ tropas se mueven de Quebec a Groenlandia 
    $\qquad Mov(5,quebec,groenlandia)$
                 
 \item Maransin tiene 8 refuerzos $\qquad Tref(maransin,8)$
 
 \item Marmont tiene 5 tarjetas $\qquad Ttar(marmont,5)$
 
 \item Wellington tiene 23 pa\'ises $\qquad Tpais(wellington,23)$
\ei

\newpage

A partir de estos predicados básicos podemos representar más propiedades del 
juego o partida en particular.
Obsérvese que algunos predicados requieren más información de la que a veces se 
tiene, analicemos por ejemplo el caso de un ataque:
\bi
\item \textit{Wellington ataca a Menelao.}\\ En este caso no se tiene
  información acerca de los lugares de ataque. Con el predicado
  definido anteriormente tendr\'iamos 
  \[
  \ex x\ex y At(wellington,menelao,x,y)
  \]
Si se prefiere evitar el uso de cuantificadores se pueden definir predicados 
mas simples, por ejemplo:
\bi
 \item $x$ ataca.  $At1(x)$.
 \item $x$ ataca a $y$.  $At2(x,y)$
 \item $x$ ataca a $y$ en $z$ $At2e(x,y,z)$
 \item Atacar desde el pais $x$ hacia el pais $y$.  $AtD(x,y)$
\ei

 \item \textit{Hay un ataque.} Usando distintos predicados de ataque:
     \[
     \ba{l}
     \ex x At1(x)\\
     \ex x\ex y At2(x,y)\\
     \ex x \ex y\ex z At2e(x,y,z)\\
     \ex x\ex y\ex z\ex w At(x,y,z,w)\\
     \ea
     \]
     
 \item \textit{Alguien ataca a Freire (Freire es atacado.)}
     \[
     \ba{l}
     \ex x At2(x,freire)\\
     \ex y\ex y At2e(x,freire,y)\\
     \ex x\ex z\ex w At(x,Freire,z,w)
     \ea
     \]
 \item \textit{Aubert ataca}
      \[
      \ba{l}
        \ex x At1(aubert)\\
       \ex x At2(aubert,x)\\
       \ex x\ex y At2e(aubert,x,y)\\
      \ex y\ex z\ex w At(Aubert,y,z,w)
      \ea
      \]
 \item \textit{Hay un ataque en Argentina}
  \[
  \ba{l}
  \ex x AtD(x,argentina)\\
  \ex x \ex y\ex z At2e(x,y,argentina)\\
  \ex x\ex y\ex w At(x,y,argentina,w)
  \ea
  \]

 \item \textit{Hay un ataque desde Madagascar}
    \[
    \ba{l}
    \ex x AtD(madagascar,x)\\   
    \ex x,y,z At(x,y,z,madagascar)
    \ea
    \]

 \item \textit{Todos los pa\'ises están ocupados.}
  \[
  \fa x\ex y Oc(y,x)
  \]
    
 \item \textit{Si un jugador tiene 3 tarjetas y las intercambia entonces recibe
  refuerzos.}
  \[
  \fa x (Ttar(x,3)\land Int(x)\imp \ex y Ref(x,y))
  \]
  
 \item \textit{Si un jugador tiene 5 o mas tarjetas debe intercambiarlas.}
  \[
  \fa x,y(Ttar(x,y) \land y>5\imp Int(x))
  \]

 \item \textit{Si un jugador tiene 8 pa\'ises o menos  entonces recibe $3$
  refuerzos.}
  \[
  \fa x\fa n(Tpais(x,n)\land n\leq 8 \imp Ref(x,3))
  \]
 
 \item Si un jugador ocupa $n$  pa\'ises con $n \geq 9$ entonces recibe
  $(n\;mod\;3)$ refuerzos
    \[
    \fa x\fa n(Tpais(x,n)\land n\geq 9\imp Ref(x,n\;mod\;3))
    \]
 Obsérvese en este ejemplo el uso de una función ($mod$) como
 argumento de un predicado. 

 \item \textit{Un pais puede ser ocupado por un único jugador.}
  \[
  \fa x\fa y\fa z (Oc(x,z)\land Oc(y,z)\imp x=y)
  \]
  Cuando hablamos de cosas únicas es necesario usar la igualdad.
  
 \item \textit{Un jugador puede hacer movimiento táctico de un pais a otro
   sólo si posee ambos pa\'ises y estos están contiguos.}
   \[
   \fa x\fa y\fa z\Big( Tac(x,y,z)\imp Oc(x,y)\land Oc(x,z)\land Cont(y,z)\Big)
   \]
Aquí introducimos el nuevo predicado binario $Cont(x,y)$. 
  
 \item \textit{Solignac fue aniquilado}  \[ \ex y Ani(y,solignac)\]

 \item \textit{Si Pach ataca a Mega en Siam y lo vence entonces lo aniquilará}
  \[
  \ex y (At(pach,mega,siam,y))\land Ven(pach,mega,siam)\imp Ani(pach,mega)
  \]

 \item \textit{Polain ataca desde Ontario a Menelao quien se defiende con
   éxito.}
  \[
   \ex x\big(  At(polain,menelao,x,Ontario)\land DefEn(menelao,x))
  \]

 \item \textit{Si todos atacan a Ardilio en Rusia entonces  él intercambiará
  tarjetas y colocará 50 tropas ahí.}
  \[
   \fa x \ex y At(x,ardilio,rusia,y)\imp Int(ardilio)\land Trop(50,rusia)
  \]

 \item \textit{Si Mega ataca a Polain en Venezuela y gana entonces Polain
   es aniquilada, Mega moverá 7 tropas ahí y recibirá 4 tarjetas.}
        \[
        \ba{c}
        At2e(mega,polain,venezuela) \land Ven(mega,polain,venezuela)\imp
        \\
        Ani(mega,polain)\land MovA(mega,7,venezuela)\land Gt(mega,4)\Big)
        \ea
        \]
 Aquí introducimos un nuevo predicado $MovA(x,y,z)$ para la relación 
 \enquote{x mueve y tropas a z}.
 
 \item Menelao recibe 10 refuerzos y los coloca en Gibraltar mientras que
  Mega se retira de Egipto moviendo 7 tropas hacia el Congo.
  \[
  Ref(menelao,10)\land Tro(10,gibraltar)\land Ret(mega,egipto)\land 
  Mov(7,egipto,congo)
  \]
Obsérvese que en la formalización no se captura el hecho de que las 10
tropas de refuerzo son las mismas 10 que se colocan en Gibraltar. Una
mejor solución sería utilizar un nuevo predicado ternario
$RefEn(x,y,z)$ para la relación 
\enquote{x recibe y refuerzos y los coloca en z}.
  
 \item Todos los que ataquen a menelao serán derrotados y aniquilados por él.
  \[
  \fa x \fa y\fa z\Big(At(x,menelao,y,z)\imp Ven(menelao,x,y)\land 
  Ani(menelao,x)\Big)
  \]

 \item Si Mega recibe 15 refuerzos y los coloca en Filipinas entonces
  Charal no lo atacará en Japón y se retirará de Kamchatka a Alaska.
  \[
  \ba{c}
  Ref(mega,15)\land Tro(15,filipinas)\imp \\
  \neg
  At2e(charal,mega,japon)\land RetH(charal,kamchatka,alaska)
  \ea
  \]
 Aquí introducimos el nuevo predicado~$RetH(x,y,z)$ para la relación
 \enquote{x se retira de y hacia z}.
\ei


\subsection{Predicados para tipos}

Obsérvese que aún cuando el mundo es heterog\'eneo en las especificaciones 
acerca del juego de Risk no hemos tenido el cuidado de distinguir entre los
distintos tipos de objetos, con lo cual podemos construir fórmulas con
argumentos incorrectos desde el punto de vista semántico. Por ejemplo
\[
Tro(filipinas,mega), Ret(venezuela,peru), At2e(15,polain, mega)
\]
Si bien éstas son fórmulas bien formadas, no tienen sentido en la situación que
hemos fijado. Hemos preferido no especificar los tipos para simplificar las
traducciones, sin embargo éstos pueden recuperarse introduciendo predicados
unitarios por ejemplo $J(x), N(x), P(x)$ para las propiedades de ser jugador,
número y predicado respectivamente. En tal caso por ejemplo la frase
\textit{alguién ataca a Mega} traducida antes como~$\ex x At2(x,mega)$
debe traducirse ahora como~$\ex x(J(x)\land At2(x,mega))$ de esta manera se
evitan fórmulas sin sentido como las de arriba. 
Por ejemplo \textit{Cualquier entero es más pequeño que un natural} se puede 
traducir como~$\fa x\ex y (x < y)$ sin embargo se pierde mucha información por
lo que debe preferirse la traducción con tipos
$\fa x\big(Int(x)\imp \ex y(Nat(y)\land x<y)\big)$. 

Para ahorrarnos algunos conectivos podemos introducir abreviaturas al
estilo de los lenguajes de programación:
\[
\fa x\!:\!A.\, P(x) \mbox{ significa }  \fa x (A(x)\imp P(x))
\]
\[
\ex x\!:\!A.\, P(x) \mbox{ significa }  \ex x (A(x)\land P(x))
\]
Por ejemplo la traducción anterior se expresa como 
$\fa x:Int.\,\ex y:Nat.(x<y)$.
Veamos un último ejemplo considerando un tipo~$Prog(x)$ para programas
así como los predicados $F(x)$ para funcionar y $T(x)$ para terminar.
\bc
\textit{Todos los programas que funcionan terminan.}
\ec
\bi
 \item Sin tipos: $\fa x (F(x)\imp T(x))$
 \item Con tipos: $\fa x (P(x)\imp F(x)\imp T(x))$
 \item Con tipos abreviados: $\fa x:P. (F(x)\imp T(x))$
\ei

\section{Libros y Autores}
Considérense las siguientes declaraciones de predicados y constantes: \\
%$\Si=\{L^{(2)},\;E^{(2)},\;A^{(1)},\;F^{(1)},\allowbreak
%M^{(1)},\;g,q,c,a, h, d\}$, con el siguiente significado:\\
\begin{tabular}{ll}
 $L(x,y)$  & \enquote{x es m\'as largo que y} \\
 $E(x,y)$ & \enquote{x es escrito por y}\\
 $A(x)$  &  \enquote{x est\'a escrito en alem\'an}\\
 $M(x)$ & \enquote{x es matem\'atico}\\
 $F(x)$ & \enquote{x es fil\'osofo}\\
 $B(x)$ & \enquote{x es libro}\\
 $g$ & \enquote{Grund\-leh\-ren der Mathematik}\\
 $q$ & \enquote{El Quijote de la Mancha}\\
 $c$ & \enquote{La construcci\'on l\'ogica del mundo}\\
 $a$ & \enquote{Alicia en el pa\'{\i}s de las maravillas}\\
 $h$ & David Hilbert \\
 $d$ & Charles Dogdson.
\end{tabular}\\

Usaremos especificaciones con tipos para libros, matemáticos y filósofos en 
algunos casos:
\bi
 \item \textit{\enquote{Grundlehren der Mathematik} es un libro escrito en
  alem\'an.}
  $$ B(g)\land A(g)$$
 \item \textit{Hay un libro m\'as largo que \enquote{El Quijote de la Mancha}.}
  $$ \ex x:B.\,L(x,q) $$
 \item \textit{\enquote{La construcci\'on l\'ogica del mundo} es un libro 
  escrito por un fil\'osofo y no es mas largo que 
  \enquote{Alicia en el pa\'{\i}s de las maravillas}}
  $$ (\ex x:F.\,E(c,x)) \land \neg L(c,a) $$
 \item \textit{\enquote{Alicia en el pa\'{\i}s de las maravillas} es un libro
  escrito por un matem\'atico}.
  \[
  \ex x:M.(B(a)\land E(a,x))\; \mbox{equivalente a }\;B(a)\land \ex x:M.E(a,x)
  \]
 \item \textit{Hay un libro escrito en alem\'an por un matem\'atico que es
  m\'as largo que \enquote{Grundlehren der Mathematik}}.
  $$ \ex y:B.\ex x:M.\,(A(y)\land E(y,x)\land L(y,g)) $$
 \item \textit{Si \enquote{Alicia en el pa\'{\i}s de las maravillas} es un libro
  escrito por un matemático entonces es un libro escrito por un fil\'osofo.}
  $$ \ex x:M.\,(B(a)\land E(a,x))\imp\ex y:F\,(B(a)\land E(a,y)) $$
 \item \textit{No hay un libro escrito en alem\'an por un fil\'osofo.}
  $$ \neg\ex x:B.\,(A(x)\land \ex y:F.\,E(x,y)) $$
 \item \textit{No es el caso que cualquier libro es escrito por un
  fil\'osofo.}
  $$ \neg\fa x\ex y (F(y)\land E(x,y)) $$
 \item \textit{Para cualquier libro escrito por Dogdson, 
  \enquote{Grundlehren der Mathematik} es un libro m\'as largo escrito por 
  Hilbert.}
  $$ \fa x:B.\, (E(x,d)\imp E(g,h)\land L(g,x)) $$
 \item \textit{Hay alguien que es fil\'osofo y matem\'atico y ha escrito un
  libro en alem\'an.}
  $$ \ex x(F(x)\land M(x)\land \ex y:B (E(y,x)\land A(y))) $$
\ei


\section{El Mundo de Cubos}
Tenemos seis cubos de color amarillo, azul o verde. Un cubo puede estar
sobre otro o en el piso. Consid\'erense los predicados $S(x,y)$
representando que $x$ est\'a sobre $y$, $P(x)$ para $x$ está en el piso,  
$A(x),Az(x),V(x)$ representan los colores, $L(x)$ significa que $x$ está
libre, es decir que ning\'un cubo est\'a sobre $x$. Obsérvese que en
el mundo hay cubos únicamente por lo que un predicado para 
\enquote{ser cubo} es redundante.\\ Simbolizar los siguiente:
\bi
 \item Hay un cubo azul en el piso con un cubo amarillo sobre \'el y un
  cubo verde sobre el amarillo. 
  \[
  \ex x\ex y\ex w
  \Big(Az(x)\land A(y)\land V(w)\land P(x)\land S(y,x)\land S(w,y)\Big)
  \]
 \item Ning\'un cubo amarillo est\'a libre.
  \[
  \fa x(A(x)\imp \neg L(x))
  \]
 \item Hay un cubo azul libre y un cubo verde libre.
  \[
  \ex x\ex y\big(Az(x)\land L(x)\land V(y)\land L(y)\big)
  \]
 \item Cualquier cubo amarillo tiene un cubo sobre \'el.
  \[
  \fa x\big(A(x)\imp \ex y S(y,x)\big)
  \]
 \item No todos los cubos azules est\'an libres.
  \[
  \ex x(Az(x)\land\neg L(x))
  \]
 \item Cualquier cubo verde est\'a libre.
  \[
  \fa x\big(V(x)\imp L(x)\big)
  \]
 \item Todos los cubos en el piso son azules.
  \[
  \fa x\big(P(x)\imp Az(x)\big)
  \]
 \item Cualquier cubo que est\'e sobre un cubo amarillo es verde o azul.
  \[
  \fa x\Big(\ex y(A(y)\land S(x,y))\imp V(x)\lor Az(x)\Big)
  \]
 \item Hay un cubo verde sobre un cubo verde.
  \[
  \ex x \ex y\big(V(x)\land V(y)\land S(x,y)\big)
  \]
 \item Hay un cubo amarillo libre en el piso.
  \[
  \ex x\big(A(x)\land L(x)\land P(x)\big)
  \]
 \item Ning\'un cubo est\'a en el piso.
  \[
  \fa x\;\neg P(x)
  \]
 \item Hay un cubo amarillo que est\'a sobre uno azul y hay un cubo azul
  sobre \'el.
  \[
 \ex x \ex y\Big(A(x)\land Az(y)\land S(x,y) \land \ex w(Az(w)\land S(w,x))\Big)
  \]
 \item Todos los cubos est\'an sobre algo.
  \[
  \fa x\ex y S(x,y)
  \]
\ei


\section{Tri\'angulos, Círculos, Cuadrados}

El mundo consta de una cuadricula de cualquier tamaño. Los objetos son
c\'irculos, cuadrados o tri\'angulos y pueden ser pequeños, medianos o grandes.
También se tienen las relaciones dadas por la posición, sur, norte, este,
oeste y las relaciones dadas por la misma columna y el mismo renglón. Los
predicados para las figuras son: $T(x), C(x), S(x)$ para tri\'angulo, círculo y
cuadrado respectivamente; para tamaño tenemos $P(x),M(x),G(x)$ para pequeño,
mediano y grande. Para la posición tenemos $Su(x,y), N(x,y), E(x,y), O(x,y)$
por ejemplo $N(x,y)$ significa $x$ está al norte de $y$.
Finalmente tenemos $Co(x,y), R(x,y)$ para $x$ está en la misma columna o
renglón que $y$. 
\bi
 \item Hay c\'irculos medianos y cuadrados grandes
  \[
  \ex x(C(x)\land M(x))\land \ex y(S(y)\land G(y))
  \]
 \item No hay cuadrados pequeños.
  \[
  \fa x(S(x)\imp\neg P(x))
  \]
 \item Hay un tri\'angulo al sur de todos los c\'irculos
  \[
  \ex x\Big(T(x)\land \fa y(C(y)\imp Su(x,y))\Big)
  \]
 \item No hay dos tri\'angulos en el mismo renglón
  \[
  \neg\ex y\ex x\big(T(x)\land T(y)\land R(x,y)\big)
  \]
 \item Hay un círculo tal que todos los círculos al oeste de él son grandes.
  \[
  \ex x\Big(C(x)\land \fa y(C(y)\land O(y,x)\imp G(y))\Big)
  \]
 \item Ningún cuadrado está al norte de un círculo grande.
  \[
  \neg\ex x(S(x)\land \ex y\Big(C(y)\land G(y)\land N(x,y))\Big)
  \]
 \item Todos los c\'irculos medianos están al oeste de un mismo tri\'angulo 
grande
  \[
  \ex x(T(x)\land G(x)\land \fa y(C(y)\land M(y)\imp O(y,x)))
  \]
 \item Todos los cuadrados pequeños están al sur de cualquier tríangulo.
  \[
  \fa x\Big(S(x)\land P(x)\imp \fa y(T(y)\imp Su(x,y))\Big)
  \]
 \item Si dos cuadrados están en el mismo renglón entonces cualquier tri\'angulo
  al sur de ambos es mediano.
  \[
  \fa x\fa y\Big(S(x)\land S(y)\land R(x,y)\imp \fa z\big(T(z)\land 
  Su(z,x)\land Su(z,y)\imp M(z)\big)\Big)
  \]
 \item No hay dos tri\'angulos medianos en la misma columna y si un
  tri\'angulo es grande entonces hay un c\'irculo pequeño al este de él.
  \[
  \neg \ex x\ex y\big(T(x)\land T(y)\land Co(x,y)\big)\land
  \fa z\Big(T(z)\land G(z)\imp \ex w(C(w)\land P(w)\land E(w,z))\Big)
  \]
\ei

\begin{comment}
\section{Ejercicios}

\be
\item Realizar la especificación formal acerca de las siguientes propiedades de 
la función que verifica la pertenencia de un elemento a una lista dada. En cada 
caso dar dos especificaciones, una funcional y otra relacional. Definir primero 
el lenguaje a utilizar.
\be
 \item Ningún elemento pertenece a la lista vacía.
 \item Si cierto elemento es la cabeza de una lista, entonces dicho elemento 
  pertenece a esa lista.
 \item La cabeza de una lista pertenece a dicha lista.
 \item Si un elemento pertenece a una lista entonces o es la cabeza de esa 
  lista o pertenece a la cola.
 \item Si un elemento pertenece a una lista entonces pertenece a cualquier otra 
  lista cuya cola es la lista anterior.
 \item Si un elemento pertenece a la concatenación de dos listas entonces 
  pertenece a alguna de estas dos listas.
 \item Si un elemento pertenece a una lista entonces dicha lista es la 
  concatenación de otras dos donde la segunda tiene como cabeza a dicho 
  elemento.
 \item Que un elemento pertenezca a la reversa de una lista equivale a que 
  pertenezca a la lista.
\ee

\item Realizar la especificación formal de las siguientes propiedades de las 
funciones reversa de una lista y concatenación de dos listas. En cada caso dar 
dos especificaciones, una funcional y otra relacional. Definir primero el 
lenguaje a utilizar.
\be
 \item La operación de concatenación es asociativa.
 \item Si la concatenación de dos listas es vacía entonces ambas listas son 
  vacías.
 \item Si la concatenación de dos listas es una lista unitaria entonces una de 
  dichas listas es vacía y la otra es la misma lista unitaria.
 \item La reversa de la concatenación de dos listas es la concatenación de la 
  reversa de la segunda lista con la reversa de la primera lista.
 \item La reversa de la reversa de una lista es la misma lista.
 \item Si la reversa de una lista es la misma lista entonces dicha lista es 
  vacía o unitaria.
\ee

\item En el lenguaje de programación {\sc LISP} una expresión simbólica o 
S-expresión es alguna de las siguientes
\be
\item Un símbolo.
\item Una S-lista, donde una S-lista es una sucesión, tal vez vacía, de 
S-expresiones, encerrada entre paréntesis.
\ee
Por ejemplo, las siguientes son cinco S-expresiones:
\bc
{\tt e,\;\;\;(),\;\;(e z),\;\;(b (a m)),\;\;(f\; (a\; ()\; )\; v\; (i\; o))}
\ec
%\be
Dé una especificación formal de las S-expresiones.
% Defina a las S-expresiones mediante un juicio $x\;{\tt SE}$. %Puede suponer 
% \textit{únicamente} la existencia de un juicio $x\;S$ para símbolos. 
% \item Muestre usando su definición y mediante una derivación formal que la 
% siguiente es una S-expresión:
% \bc
% {\tt ((s) (a p) o)}
% \ec 
%\ee

\item Realizar la especificación formal acerca del tipo abstracto de datos pila 
considerando homogeneidad, es decir, suponiendo que todos los elementos en una 
pila son del mismo tipo digamos~$A$.  En cada caso dar dos especificaciones, 
una funcional y otra relacional. Definir primero el lenguaje a utilizar.
\bi
 \item Definición del tipo de datos pila: 
 \be
  \item La vacía es una pila (de elementos de~$A$).
  \item El resultado de agregar un elemento de~$A$ en el tope de una pila es 
    una pila.
  \item Son todos.
 \ee
 \item Si una pila no es vacía entonces el elemento en el tope es un elemento 
  de~$A$ 
 \item Si una pila no es vacía entonces la pila que resulta al eliminar 
  el elemento en el tope es una pila de elementos de $A$.
 \item Si una pila no es vacía entonces la pila que resulta al agregar el tope 
  de la pila dada a la pila obtenida al eliminar el tope de la pila dada es la 
  pila dada.
 \item La pila vacía no tiene elementos.
 \item La pila que resulta de agregar un elemento a una pila dada tiene un 
  elemento más que la pila dada.
 \item Si la pila resultante de eliminar el tope de una pila dada es vacía 
  entonces la pila dada tiene sólo un elemento.
\ei

\item Realizar una especificación formal de alguna estructura o tema de interés 
particular. Definir primero el lenguaje a utilizar. Algunas sugerencias son:
\be
\item Sistemas numéricos y sus propiedades:
\be
\item Enteros, racionales, etc..
\item Binarios, ternarios, etc..
\item Propiedades de conversión entre sistemas.
\ee
\item Estructuras de datos y sus propiedades:
\be
\item Listas con el constructor {\tt snoc}
\item Listas heterogéneas con elementos de dos tipos dados.
\item Árboles binarios
\item Colas, tablas, arreglos, etc.
\item Árboles rojinegros, árboles AVL, etc..
\item Conjuntos.
\ee
\item Micromundos:
\be
\item Extensión del micromundo de figuras a tres dimensiones.
\item Mundo de bloques agregando un brazo de robot (reglas para que el brazo 
mueva los bloques ( planeación en inteligencia artificial))
\item Sistema de correo electrónico.
\item Sistema de funcionamiento de un elevador.
\ee
\item Reglas de algún juego de su interés (juegos de mesa, juegos de rol, 
ajedrez, sudoku, etc)
\ee
\ee
\end{comment}

%\section{Otros Ejemplos}

%%\ejem{\textit{Existe un \'unico objeto con la propiedad P}. En este caso la
%%signatura es $\Si=\{P\}$ y usaremos un lenguaje con igualdad. A veces es
%%conveniente transformar la oraci\'on en espa\~nol en una otra que diga lo 
%%mismo
%%pero que tenga una forma m\'as cercana a los conectivos l\'ogicos que puede
%%incluir variables. En
%%nuestro caso la oraci\'on original se transforma en \textit{Existe un objeto
%%x con la propiedad P y cualquier objeto con la propiedad P es x}. La 
%%traducci\'on es:
%%$$
%%\ex x ( P(x)\land \fa y(P(y)\imp y=x))
%%$$
%%}
%\bi
%\item\textit{Una condici\'on necesaria y suficiente para que el sult\'an sea
%fel\'{\i}z es que tenga vino, mujeres y m\'usica}.% Hacemos
%%$\Si=\{F^{(1)},V^{(1)},M^{(1)}, Mu^{(1)},s\}$ con el significado obvio.
%%La traducci\'on es:
%$$
%F(s)\iff V(s)\land M(s)\land Mu(s)
%$$ 
%Como se ve esta es una f\'ormula proposicional, es decir, no se
%necesitaron cuantificadores.

%\item\textit{No hay un barbero que rasura exactamente a aquellos hombres que
%no se rasuran a si mismos.} Esto es equivalente a \textit{no existe un
%barbero que rasura a un hombre $z$ si y s\'olo si $z$ no se rasura a si
%mismo}. %Los predicados son $B(x),R(x,y)$ y la traducci\'on es:
%$$
%\neg\ex x(B(x) \land \fa z(R(x,z)\iff \neg R(z,z))) 
%$$
%\item Cualquiera que sea persistente puede aprender l\'ogica.
%\[
%\fa x(P(x)\imp L(x))
%\]
%\item Ning\'un guajolote vuela.
%\[
%\neg\ex x(G(x)\land V(x))
%\]
%\item No todos los pachucos bailan.
%\[
%\ex y(P(y)\land \neg B(y))
%\]
%\item Alguna rumbera baila con algun pachuco.
%\[
%\ex y(R(y)\land \ex z(P(z)\land B(x,y)))
%\]
%\ei



%\item Hacer las siguientes traducciones, dando previamente una signatura 
%adecuada.
%\be

%\item Los perros muerden a los carteros.
%\item Existe un perro que muerde a los carteros.
%\item Existe un cartero que es mordido por todos los perros.
%\item Hay un perro que no muerde carteros.
%\item Hay un cartero que no es mordido por perros.
%\item Hay un perro que es cartero y se muerde a si mismo.
%\ee
%\item Tenemos seis cubos de color amarillo, azul o verde. Un cubo puede estar
%uno sobre otro o en el piso. Considerese la signatura $\Si=\{S^{(2)},
%A^{(1)},Az^{(1)},\allowbreak V^{(1)}, L^{(1)}, p\}$ donde $S(x,y)$ significa 
%\enquote{x est\'a sobre y}, $A(x),Az(x),V(x)$ representan los colores, $L(x)$ 
%significa que \enquote{x esta libre} es decir que ning\'un cubo est\'a sobre 
%x y $p$ representa al piso. \\ Simbolizar los siguiente:
%\be
%\item Hay un cubo azul sobre el piso con un cubo amarillo sobre \'el y un
%cubo verde sobre el amarillo. 
%\item Ning\'un cubo amarillo est\'a libre.
%\item Hay un cubo azul libre y un cubo verde y libre.
%\item Cualquier cubo amarillo tiene un cubo sobre \'el.
%\item No todos los cubos azules est\'an libres.
%\item Cualquier cubo verde est\'a libre.
%\item Todos los cubos sobre el piso son azules.
%\item Cualquier cubo que est\'e sobre un cubo amarillo es verde o azul.
%\item Hay un cubo verde sobre un cubo verde.
%\item Hay un cubo amarillo libre sobre el piso.
%\item Ning\'un cubo est\'a sobre el piso.
%\item Hay un cubo amarillo que est\'a sobre uno azul y hay un cubo azul
%sobre \'el.
%\item Todos los cubos est\'an sobre algo.
%\ee   


\end{document}
