\documentclass[11pt,letterpaper]{article}
\usepackage{../packageslc}
\usepackage{../optionslc}

%%%% Macros para las notas de lenguajes de programacion


%%%% math

\newcommand{\vphi}{\varphi}
\newcommand{\vp}{\varphi}

% \newcommand{\dn}{\mathsf{DN}}
% \newcommand{\dnC}{\mathsf{DN_C}}
% \newcommand{\dnM}{\mathsf{DN_M}}
% \newcommand{\dnp}{\mathsf{DN_p}}
% \newcommand{\dnm}{\mathsf{DN_p^M}}
% \newcommand{\dnc}{\mathsf{DN_p^C}}

%\newcommand{\case}{\mathsf{case}}
%\renewcommand\labelitemi{$\circ$}

\newcommand{\imp}{\rightarrow}
\newcommand{\Imp}{\Rightarrow}
\renewcommand{\iff}{\leftrightarrow}
\newcommand{\Iff}{\Leftrightarrow}
\newcommand{\G}{\Gamma}
\newcommand{\D}{\Delta}


\newcommand{\De}{\mathcal{D}}
\newcommand{\F}{\mathcal{F}}
\newcommand{\Ge}{\mathcal{G}}
\newcommand{\Pe}{\mathcal{P}}
\newcommand{\I}{\mathcal{I}}
\newcommand{\C}{\mathcal{C}}
\newcommand{\K}{\mathcal{K}}
\renewcommand{\L}{\mathcal{L}}
\newcommand{\M}{\mathcal{M}}
\newcommand{\Nc}{\mathcal{N}}
%\newcommand{\E}{\mathcal{E}}
%\newcommand{\R}{\mathcal{R}}
%\newcommand{\Q}{\mathcal{Q}}
\newcommand{\Sc}{\mathcal{S}}
\newcommand{\Te}{\mathcal{T}}
\newcommand{\W}{\mathcal{W}}

\newcommand{\Db}{\mathbb{D}}
\newcommand{\Fb}{\mathbb{F}}
\newcommand{\Kb}{\mathbb{K}}
\newcommand{\Eb}{\mathbb{E}}
\newcommand{\Ebs}{\mathbb{E}^\star}
\newcommand{\Ob}{\mathbb{O}}
\newcommand{\Ib}{\mathbb{I}}
\newcommand{\Rb}{\mathbb{R}}
\newcommand{\Qb}{\mathbb{Q}}
\newcommand{\Kbb}{\mathbb{K}}
\newcommand{\T}{\mathbb{\Theta}}


\newcommand{\kb}{\bbkappa}

\newcommand{\Sf}{\mathsf{\Sigma}}

\newcommand{\fa}{\forall}
\newcommand{\ex}{\exists}

\newcommand{\inc}{\subseteq}

\newcommand{\Lb}{\Lambda}
\newcommand{\Om}{\Omega}
\newcommand{\lb}{\lambda}
\newcommand{\al}{\alpha}
\newcommand{\ga}{\gamma}


\newcommand{\mg}{\mathbb{m}}

\newcommand{\cg}{\mathbb{C}}
\newcommand{\dg}{\mathbb{D}}
\newcommand{\jg}{\mathbb{J}}
\newcommand{\Ha}{\mathcal{H}}
%\newcommand{\A}{\mathcal{A}}
\newcommand{\sg}{\mathbb{S}}

\newcommand{\Bc}{\mathcal{B}}
\newcommand{\Df}{\mathfrak{D}}
\newcommand{\Dc}{\mathcal{D}}
%\newcommand{\Tc}{\mathcal{T}}
\newcommand{\Mf}{\mathfrak{M}}

\newcommand{\Sg}{\mathbb{S}}

\newcommand{\Q}{\ensuremath{\mathbb{Q}}}
\newcommand{\Z}{\ensuremath{\mathbb{Z}}}
\newcommand{\N}{\ensuremath{\mathbb{N}}}
\newcommand{\R}{\ensuremath{\mathbb{R}}}
\renewcommand{\S}{\mathbb{\Sigma}}
\newcommand{\A}{\mathcal{A}}
\newcommand{\E}{\ensuremath{\exists}}
\newcommand{\iso}{\ensuremath{\cong}}
\newcommand{\union}{\ensuremath{\cup}}
\newcommand{\morinyec}{\ensuremath{\precapprox}}

\newcommand{\nin}{\ensuremath{\notin}}
\newcommand{\tog}{\makebox[7mm][l]}
\newcommand{\toge}{\makebox[11mm][l]}
\newcommand{\toget}{\makebox[13mm][l]}
\newcommand{\togeth}{\makebox[14mm][l]}
\newcommand{\togethe}{\makebox[15mm][l]}
\newcommand{\together}{\makebox[17mm][l]}
\newcommand{\niso}{\ensuremath{\not \cong}}


\newcommand{\Mg}{\mathbb{M}}
\newcommand{\Bg}{\mathbb{B}}
\newcommand{\Lg}{\mathbb{L}}
\newcommand{\Tg}{\mathbb{T}}

\newcommand{\sketch}{\Red{{\sc sketch}}}

\newcommand{\restr}[2]{#1\!\!\boldsymbol{\restriction}\!#2}

\newcommand{\vacio}{\varnothing}
\newcommand{\done}{\ensuremath{\checkmark}}

\newcommand{\ida}{$\Rightarrow \; )$ }
\newcommand{\regr}{$\Leftarrow \; )$ }

\newcommand{\ol}[1]{\overline{#1}}

\newcommand{\Tsf}{\mathsf{T}}

\newcommand{\inds}[1]{\index[simb]{#1}}

\newcommand{\B}{\mathbb{B}}
%\newcommand{\N}{\mathbb{N}}

\newcommand{\vx}{\vec{x}}
\newcommand{\vy}{\vec{y}}
\newcommand{\vz}{\vec{z}}
\newcommand{\vt}{\vec{t}}
\newcommand{\vf}{\vec{f}}


% \newcommand{\propo}{\ensuremath{\mathsf{PROP}}}
% \newcommand{\atom}{\ensuremath{\mathsf{ATOM}}}
\newcommand{\term}{\ensuremath{\mathsf{TERM}}}
\newcommand{\form}{\mathsf{FORM}}

\newcommand{\true}{\mathop{\mathsf{true}}}

%\newcommand{\id}{\mathsf{Id}}

%\newcommand{\uc}{\mathcal{U}}
%\newcommand{\Ic}{\mathcal{I}}
%\newcommand{\pc}{\mathcal{P}}
%\newcommand{\qc}{\mathcal{Q}}
%\newcommand{\mc}{\mathcal{M}}
\newcommand{\supc}{\supseteq}
\newcommand{\limo}{\mathop{\mathpzc{Lim}}}
\newcommand{\ord}{\mathsf{OR}}

\newcommand{\pt}[1]{\langle #1 \rangle}


%%%% frames
\newcommand{\titulos}[2]{\frametitle{#1}\framesubtitle{#2}}
\newcommand{\fot}[1]{\footnote{\scriptsize{#1}}}


%%%% ambientes

\newcommand{\cb}[2]{\colorbox{#1}{#2}}

\newcommand{\bc}{\begin{center}}
\newcommand{\ec}{\end{center}}
\newcommand{\be}{\begin{enumerate}}
\newcommand{\ee}{\end{enumerate}}
\newcommand{\bi}{\begin{itemize}}
\newcommand{\ei}{\end{itemize}}
\newcommand{\beq}{\begin{equation}}
\newcommand{\eeq}{\end{equation}}
\newcommand{\beqs}{\begin{equation*}}
\newcommand{\eeqs}{\end{equation*}}
\newcommand{\ba}{\begin{array}}
\newcommand{\ea}{\end{array}}


% \newtheorem{theorem}{Teorema}
% \newcommand{\teo}[1]{\begin{theorem} #1 \end{theorem}}
% \newtheorem{proposition}{Proposici\'on}
% \newcommand{\prop}[1]{\begin{proposition} #1 \end{proposition}}
% \newtheorem{definition}{Definici\'on}
% \newcommand{\defin}[1]{\begin{definition} #1 \end{definition}}
% \newtheorem{corollary}{Corolario}
% \newcommand{\cor}[1]{\begin{corollary} #1 \end{corollary}}
% \newtheorem{lemma}{Lema}
% \newcommand{\lema}[1]{\begin{lemma} #1 \end{lemma}}
% \newcommand{\dem}[1]{\begin{proof} #1 \end{proof}}

%\renewcommand{\qed}{\qedsymbol{$\mathbf{\dashv}$}}

%\newcommand{\proof}{\hfill\\\noindent\textbf{\textit{Demostraci\'on. }}}

\newcommand{\hint}{\emph{Sugerencia: }}


\newcounter{EjempCtr}[section]
\newenvironment{enumrom}{\renewcommand{\theenumi}{\roman{enumi}}%
\renewcommand{\theenumii}{\roman{enumii}}
\renewcommand{\theenumiii}{\roman{enumiii}}
\renewcommand{\theenumiv}{\roman{enumiv}}
\begin{enumerate}}{\end{enumerate}}
\newenvironment{Ejemplo}
        {\stepcounter{EjempCtr}%
        \begin{description}\item[Ejemplo \thesection.\arabic{EjempCtr}]}%
        {\end{description}}
\newenvironment{demostr}{%
             {\em Demostración:}
                \begin{quotation}}{\end{quotation}}

   \newcommand{\beje}{\begin{Ejemplo}}
\newcommand{\eeje}{\end{Ejemplo}}


\newtheorem{eje}{Ejemplo}[section]
\newcommand{\ejem}[1]{\begin{eje}\normalfont #1 \end{eje}}

% \renewcommand\contentsname{\'Indice}
%\renewcommand\chaptername{Cap\'itulo}
% \renewcommand\indexname{\'Indice}

%%\newcommand{\qed}{\hfill$\mathbb{Qed}$}
%\newcommand{\qed}{\hfill$\mathsf{\boldsymbol{\dashv}}$}
%\renewcommand{\qed}{\hfill$\boldsymbol{\dashv}$}


\newenvironment{prueba}{\vspace{-5mm}\noindent\textbf{Demostraci\'on}\\}{
\noindent$\blacksquare$\\}

\newcommand{\Ejercicios}{\section*{Ejercicios}}

 \newenvironment{manitas}{%
      \renewcommand{\labelitemi}{\ding{44}}%
      \vspace{-0.5cm}%
      \begin{itemize}%
      \setlength{\itemsep}{0pt}\setlength{\parsep}{0pt}\setlength{\topsep}{0pt}%
      }{\end{itemize}}
\newenvironment{malitos}{%
      \renewcommand{\labelitemi}%
            {\raisebox{1.5ex}{\makebox[0.3cm][l]{\begin{rotate}{-90}%
            \ding{43}\end{rotate}}}}%
      \vspace{-0.5cm}%
      \begin{itemize}%
      \setlength{\itemsep}{0pt}\setlength{\parsep}{0pt}\setlength{\topsep}{0pt}%
      }{\end{itemize}}
\newenvironment{ejercs}{
     \renewcommand{\labelenumi}{\thesection.\theenumi.-}
     \renewcommand{\labelenumii}{\theenumii)}
     \begin{enumerate}}
     {\end{enumerate}}

   \newcommand{\bej}{\begin{ejercs}}
\newcommand{\eej}{\end{ejercs}}


%\newenvironment{leterize}{%
%        \renewcommand{\theenumi}{\alph{enumi}}
%        \begin{enumerate}}{\end{enumerate}}

%\newenvironment{manitas}{%
%      \renewcommand{\labelitemi}{\ding{44}}%
%      \vspace{-0.5cm}%
%      \begin{itemize}%
%      
% \setlength{\itemsep}{0pt}\setlength{\parsep}{0pt}\setlength{\topsep}{0pt}%
%      }{\end{itemize}}



%%=============================================================================

\def\stackunder#1#2{\mathrel{\mathop{#2}\limits_{#1}}}


%%%% notas

\newcommand{\doubt}{\Red{{\LARGE {\sf ??}}}}

\newcommand{\coment}[1]{\hfill\\ \Big[{\bf Comentario Privado:} #1\Big]}
\newcommand{\preg}[1]{\hfill\\ \BrickRed{{\bf Pregunta:} #1}}
\newcommand{\conjet}[1]{\hfill\\ \OliveGreen{{\bf Conjecura:} #1}}

\newcommand{\pendiente}{\BrickRed{{\sc Pendiente}}}
\newcommand{\verifpendiente}{\BrickRed{{\sc Verificación pendiente}}}


%--------------------------------------------------------------------------

\DeclareMathAlphabet{\mathpzc}{OT1}{pzc}{m}{it}



\title{L\'ogica Clausular Proposicional \\ 
L\'ogica Computacional 2018-2, Nota de clase 3}
\author{Favio Ezequiel Miranda Perea\and Araceli Liliana Reyes Cabello\and
Lourdes Del Carmen Gonz\'alez Huesca \and Pilar Selene Linares Ar\'evalo}
\date{Facultad de Ciencias UNAM \\ 2 de marzo de 2018 \\
Material desarrollado bajo el proyecto UNAM-PAPIME PE102117}


\begin{document}

\maketitle

Las cláusulas son una clase especial de fórmulas relevantes en distintas 
aplicaciones dentro y fuera de la lógica. En esta nota discutimos la 
transformación de cualquier fórmula a la llamada forma clausular así como el 
famoso problema de satisfacibilidad {\sat} de gran importancia en la teoría 
de la complejidad computacional y la regla de resolución binaria de 
Robinson, piedra angular de la programación lógica.

% La regla de resolución binaria de Robinson proporciona un método más para 
% decidir la correctud de un argumento. 
% En esta nota veremos las ideas básicas de este método
% para la lógica proposicional, el cual requiere que las fórmulas se
% encuentren en una forma sintáctica especial, llamada forma clausular.

Para llegar a la forma clausular de una fórmula  se utilizan transformaciones 
de fórmulas mediante equivalencias lógicas llamadas formas normales, que 
estudiamos a continuación.


\section{Forma Normal Negativa}

El objetivo de esta forma normal es obtener una f\'ormula equivalente a una
f\'ormula dada sin implicaciones ni bicondicionales, donde además los símbolos 
de negación sólo afectan a f\'ormulas at\'omicas.

\defin{Una f\'ormula~$\vp$ est\'a en forma normal negativa si y s\'olo si se 
cumplen las siguientes:
\be
\item $\vp$ no contiene ni equivalencias ni implicaciones;
\item las negaciones que figuran en~$\vp$ afectan s\'olo a f\'ormulas 
at\'omicas. 
\ee
\index{forma normal!negativa}
}

\noindent La transformación a forma normal negativa se sirve de las siguientes 
equivalencias.

\prop{[Leyes de la Negaci\'on]
Se cumplen las siguientes equivalencias l\'ogicas.
\bi
\item Doble Negaci\'on: $\neg\neg\vp\equiv\vp$.
\item De Morgan: $\neg(\vp\lor\psi)\equiv \neg\vp\land\neg\psi$.
\item De Morgan: $\neg(\vp\land\psi)\equiv \neg\vp\lor\neg\psi$.
\item $\neg(\vp\imp\psi)\equiv\vp\land\neg\psi$
\item $\neg(\vp\iff\psi)\equiv\neg\vp\iff\psi\equiv\vp\iff\neg\psi$.
% \item $\neg\fa x \vp\equiv\ex x\neg \vp$
% \item $\neg\ex x \vp\equiv\fa x\neg \vp$
\ei
\index{leyes!de la negaci\'on}
\index{leyes!de De Morgan}
\label{leyneg}
}
\vspace*{-10pt}
\begin{proof} 
Se deja como ejercicio.
\qed
\end{proof}

\prop{Sea~$\vp$ una f\'ormula. Podemos encontrar de manera algorítmica una
f\'ormula~$\psi$ lógicamente equivalente a~$\vp$ y tal que~$\psi$ est\'a en
forma normal negativa. En tal caso $\psi$ se denota con $fnn(\psi)$. 
}
\begin{proof}
Dada~$\vp$, eliminar los símbolos $\imp,\iff$ mediante las equivalencias 
usuales, es decir:
\bi
 \item Eliminar bicondicionales usando que 
  $\vp\iff\psi \equiv (\vp\imp\psi)\land(\psi\imp\vp)$.
 \item Eliminar implicaciones usando que 
  $\vp\imp\psi \equiv \lnot\vp\lor\psi$
\ei
Posteriormente 
\bi
 \item Introducir las negaciones de manera que s\'olo afecten a literales 
  mediante: \\
  $\lnot(\vp\land\psi)\equiv\lnot\vp\lor\lnot\psi \qquad 
  \lnot(\vp\lor\psi)\equiv\lnot\vp\land\lnot\psi$
 \item Eliminar las dobles negaciones mediante 
  $\lnot\lnot\vp\equiv\vp$.
\ei
Es decir se aplicarán exhaustivamente las leyes de negaci\'on de 
manera adecuada. Así la f\'ormula~$\psi$ obtenida es~$fnn(\vp)$.
\qed
\end{proof}


\subsection{Implementación}

Se deja como ejercicio implementar la función \verb=fnn: Prop -> Prop= que 
transforma una fórmula~$\vp$ en su forma normal negativa. 
Las siguientes observaciones son de importancia:
\be
 \item Podemos suponer que la fórmula de entrada no tiene implicaciones ni 
  equivalencias (recordar las funciones que eliminan estos conectivos en la 
  nota 2). 
  Esto deja las siguientes clases de fórmulas: atómicas, conjunciones, 
  disyunciones y las negaciones de cada una de estas.
 \item Si la fórmula no es una negación entonces: si es atómica ya está en 
  \verb'fnn' y en otro caso la \verb'fnn' se calcula \textit{recursivamente} en 
  los operandos respetando el conectivo principal.
 \item Si la fórmula es una negación debe hacerse un análisis de casos: 
  \be 
   \item si es negación de atómica ya está en \verb'fnn'; 
   \item si es negación de negación, entonces ambas negaciones se eliminan y se 
    llama recursivamente a la función~\verb'fnn'; 
   \item si es negación de conjunción o de disyunción se aplican las leyes de 
    De Morgan haciendo las llamadas recursivas donde corresponda.
  \ee
\ee

\section{Forma Normal Conjuntiva}

La llamada \emph{forma normal conjuntiva} permite expresar cualquier f\'ormula 
proposicional como una conjunci\'on de disyunciones llamadas 
\textbf{cláusulas}. 
% Esta transformación es la que nos permitir emplear el método de resolución 
% binaria.

\defin{Una literal~$\ell$ es una fórmula atómica (variable proposicional $p$,
  $\bot$ o $\top$) o la negaci\'on de una fórmula atómica.}
Una literal es \textbf{negativa} si es una negación, en otro caso decimos
que es \textbf{positiva}.

\defin{Dada una literal $\ell$ definimos su literal contraria, 
denotada~$\ell^c$, como sigue:   
\[ 
\ell^c =
\left\{
\ba{rl}
 a & \text{ si } \;\;\;\ell=\neg a \\
\neg a  & \text{ si } \;\;\;\ell=a
\ea
\right.
\]
Donde~$a$ es una fórmula atómica, es decir, una variable proposicional, $\top$ 
o $\bot$. \\
El par $\{\ell,\ell^c\}$ se llama un par de literales complementarias.
}

Las literales constituyen los objetos b\'asicos del lenguaje clausular a
partir de los cuales se construyen las cl\'ausulas.

\defin{Una cl\'ausula~$\C$ es una literal o una disyunci\'on de literales. 
% Recursivamente:
% \bi
% \item $\cv$ es una cl\'ausula, llamada \emph{cl\'ausula vacía}, que 
% representa a una disyunción degenerada con cero componentes.
% \item Si $\ell$ es una literal entonces $\ell$ es una cl\'ausula. Es decir,
%   las literales son cláusulas
% \item Si $\C_1,\C_2$ son cl\'ausulas entonces $\C_1\lor \C_2$ es una 
% cl\'ausula.
% \item Son todas. %Una expresi\'on es una cl\'ausula s\'olo si es de alguna de 
% las formas
% %descritas anteriormente.
% \ei
% Adem\'as se observan las siguientes convenciones: 
% \beqs
% \C_1\lor\cv\lor \C_2= \C_1\lor\C_2\;\;\;
% \mbox{y}\;\;\; \cv\lor\cv=\cv
% \eeqs
}

\defin{Una f\'ormula~$\vp$ est\'a en \textbf{forma normal conjuntiva} 
\verb"fnc" si y s\'olo si es de la forma $\C_1\land \C_2\land\ldots\land\C_n$ 
donde para cada $\C_i$ es una cláusula.}

En particular se sigue que cualquier literal y cualquier cláusula están en
forma normal conjuntiva.
Los conceptos anteriores de literal, cláusula y forma normal conjuntiva se 
definen de manera breve mediante la siguientes gramática:
\[
 \ba{rcll}
  a & ::= & \bot \mid \top \mid p & \;\;\;\;\text{-- atómicas}\\
  \ell & ::= & a \mid \neg\,a & \;\;\;\;\text{-- literales}\\ 
  \C & ::= & \ell \mid \ell\lor\C & \;\;\;\;\text{-- cláusulas}\\ 
  F & ::= & \C \mid  \C\land F & \;\;\;\;\text{-- formas normales conjuntivas}
 \ea
\]

\prop{Cualquier f\'ormula proposicional~$\vp$ puede transformarse 
algorítmicamente en una f\'ormula~$\psi$ tal que~$\vp\equiv\psi$ y $\psi$ 
est\'a en forma normal conjuntiva. En este caso~$\psi$ se denota con $fnc(\vp)$.
}
\begin{proof}
Basta aplicar las siguientes transformaciones en orden:
\bi
 \item Obtener la forma normal negativa correspondiente de~$\vp$.
 \item Reescribir las conjunciones y disyunciones mediante las leyes 
  distributivas:
  \[
   \ba{c}
    (\vp\land\psi)\lor(\vp\land\chi)\equiv  \vp\land(\psi\lor\chi) \\
    (\vp\lor\psi)\land(\vp\lor\chi)\equiv\vp\lor(\psi\land\chi) 
  \ea 
  \]
 Cuyo caso general es
%  se puede aplicar la siguiente ley de distributividad generalizada:
 \[
  \ba{rcl}
   (\alpha_1\lor\ldots\lor\alpha_n)\land (\beta_1\lor\ldots\lor\beta_m) 
   \equiv &
   (\alpha_1\land\beta_1)\lor(\alpha_1\land\beta_2)\lor 
      \ldots\lor(\alpha_1\land\beta_m) & \lor\\
   &(\alpha_2\land\beta_1)\lor(\alpha_2\land\beta_2)\lor
      \ldots\lor(\alpha_3\land\beta_m) & \lor\\ 
   &  \vdots  &\\
   & (\alpha_n\land\beta_1)\lor(\alpha_n\land\beta_2)\lor
      \ldots\lor(\alpha_n\land\beta_m) & 
  \ea
 \]
 \item {\bf Opcionalmente} se puede simplificar usando
  \[
  \ba{ccc}
   \lnot\vp\lor\vp\equiv\top & \qquad \top\lor\vp\equiv\top &\qquad
    \top\land\vp\equiv\vp  \\
   \bot\lor\vp\equiv\vp & \qquad \bot\land\vp\equiv\bot & \qquad 
   \vp\lor\vp\equiv\vp
  \ea
  \]
\ei
%Utilizando equivalencias lógicas conocidas podemos suponer que $\vp$ ya no 
%contiene implicaciones ni equivalencias. Si $\vp$ es una literal, es una 
% cláusula y ya est\'a en fnc. 
%En cualquier otro caso mediante distributividades o mediante la siguiente
%equivalencia, obtenida con distributividad, se puede obtener la fnc deseada:\\
\qed
\end{proof}

\ejem{Obtener la forma normal conjuntiva de 
$\vp=(\neg p\imp q)\imp (\neg r\imp s)$. \\
Aplicando el método recién descrito obtenemos: 
\[
 \ba{rcl}
  (\neg p\imp q)\imp (\neg r\imp s) & \equiv &
  \neg (\neg\neg p \lor q) \lor (\neg\neg r\lor s) \\
  &\equiv& (\neg p\land\neg q)\lor (r\lor s)  \\
  & \equiv & (\neg p\lor r\lor s)\land(\neg q\lor r\lor s) \\
  & = &  fnc(\vp)
 \ea
\]
}

\subsection{Implementación de la transformación a la forma normal conjuntiva}

Suponemos que las fórmulas de entrada ya están en forma normal negativa por lo 
que las únicas fórmulas a considerar son literales, conjunciones y 
disyunciones.

La función \verb'fnc:: Prop -> Prop' que lleva a cabo la transformación 
deseada debe definirse \textit{recursivamente} de acuerdo a las siguientes 
observaciones:
\bi
 \item Si la fórmula de entrada es una literal ya está en forma normal 
  conjuntiva y simplemente se devuelve como salida.
 \item Si la fórmula es una conjunción $\vp_1\land\vp_2$ entonces se llama 
  recursivamente a la función \verb'fnc' en los operandos de la conjunción, 
  devolviendo \verb'fnc'$(\vp_1)\land$\verb'fnc'$(\vp_2)$
 \item Si la fórmula es una disyunción $\vp_1\lor\vp_2$ entonces se llama 
  recursivamente a la función \verb'fnc' en los operandos de la disyunción pero 
  no podemos devolver \verb'fnc'$(\vp_1)\lor$\verb'fnc'$(\vp_2)$ pues esta 
  fórmula no está en forma normal conjuntiva (por ser una disyunción).
  Por lo que tenemos que transformar esta fórmula en una conjunción mediante 
las   leyes de distributividad. \\
  Para esto tenemos que definir una función auxiliar
  \verb'distr:: Prop -> Prop -> Prop' que aplique adecuadamente las leyes 
  distributivas a las formas normales conjuntivas de $\vp_1$ y $\vp_2$. 
  De esta manera la fórmula que \verb'fnc' devuelve en el caso de la disyunción 
  será \verb'distr(fnc('$\vp_1$\verb'),fnc('$\vp_2$\verb'))'
\ei

Para implementar la función auxiliar \verb'distr:: Prop -> Prop -> Prop' son 
útiles las siguientes observaciones:
\bi
 \item Podemos suponer que las fórmulas de entrada $\vp_1$ y $\vp_2$ son formas 
  conjuntivas.
 \item Si ambas $\vp_1,\vp_2$ son literales entonces devolvemos 
  $\vp_1\lor\vp_2$ pues en este caso tal disyunción es una cláusula y por lo 
  tanto una forma conjuntiva.
 \item Si $\vp_1=\vp_{11}\land\vp_{12}$ entonces aplicamos distributividad:
  \[
  \vp_1\lor\vp_2=(\vp_{11}\land\vp_{12})\lor\vp_2\equiv 
  (\vp_{11}\lor\vp_2)\land(\vp_{12}\lor\vp_2)
  \]
  pero para el resultado debemos también aplicar recursivamente la función 
  \verb'distr' a los pares $\vp_{11},\vp_2$ y $\vp_{12},\vp_2$.
 \item Si $\vp_2=\vp_{21}\land\vp_{22}$ la definición es análoga al caso 
  anterior.
\ei
Obsérvese que la definición es \textit{no determinista} para el caso en que 
ambas fórmulas de entrada sean conjunciones.


\subsection{Validez y satisfacibilidad mediante formas normales conjuntivas}

El empleo de las formas normales conjuntivas simplifica el procedimiento para
decidir si una fórmula dada es válida, es decir es tautología. 

\prop{Una cláusula~$\C=\ell_1\lor\ell_2\lor\ldots\lor\ell_n$ es tautología 
$(\models\vp)$  si y s\'olo si existen $1\leq i,j\leq n$ tales que
$\ell_i^c=\ell_j$. 
Es decir, $\models\C$ si y s\'olo si $\C$ contiene un par de literales 
complementarias.
}
\begin{proof}
 Ejercicio \qed
\end{proof}

\newpage

La proposición anterior permite tener un algoritmo para verificar 
si~$\models\vp$, cuando $\vp$ está en forma normal conjuntiva, 
digamos~$\vp=\C_1\land\ldots\land\C_n$:
\be
\item Para cada $1\leq i\leq n$, buscar en $\C_i$ un par de literales
  complementarias.
\item Si tal par existe para cada cláusula $\C_i$ entonces $\models\vp$.
\item En otro caso $\not\models\vp$, es decir $\vp$ no es tautología.
\ee

\noindent Ahora bien, recordando el principio de refutación 
$$\models \vp \text{ si y s\'olo si } \neg\vp \text{ es no satisfacible}$$
podemos también usar el algoritmo anterior para decidir la satisfacibilidad de 
una fórmula puesto que, de acuerdo al principio anterior, $\vp$ es satisfacible 
si y s\'olo si~$\not\models\neg \vp$, es decir, si y s\'olo si $\neg \vp$ no
es tautología. \\
Obsérvese que este algoritmo es s\'olo para decidir satisfacibilidad y por si 
solo no proporciona un estado $\I$ que satisfaga a $\vp$. 

\espc

\noindent Se deja como ejercicio mostrar la correctud del siguiente argumento 
mediante el algoritmo anterior:
\bc
\textit{Si el dragón se enfada, el caballero Lancelot quedará paralizado del 
susto,\\ pero si se queda paralizado del susto, entonces \\no puede sino apelar 
a la bondad del dragón y así no ser engullido. \\Luego entonces, si el dragón 
se enfada, \\el caballero Lancelot tendrá que apelar a su bondad o será 
engullido}
\ec
% \beqs
% d\to c,\; c\to b\land \neg e\;/\;\therefore d\to b\lor e
% \eeqs

\section{El problema {\sat}}

El problema de satisfacibilidad para la lógica proposicional, denotado 
usualmente como {\sat} es de suma importancia en la teoría de la complejidad 
computacional y en general en Ciencia de la Computación. Este problema tiene 
aplicaciones tanto teóricas como prácticas. En particular resultó ser el primer 
problema NP-completo, lo cual fue probado por Cook en 1971. De hecho antes del 
Teorema de Cook, el concepto de NP completud ni siquiera existía.

\noindent En su forma más común, el problema {\sat} es el siguiente:
\bc
\textit{Dado un conjunto~$P=\{p_1,\ldots,p_n\}$ de variables proposicionales 
y\\ un conjunto $C$ de cláusulas con variables en $P$\\
 ?`Existe una interpretación $\I$ que satisfaga a $C$? }
\ec

El área de la lógica conocida como razonamiento automatizado se encarga, entre 
muchas otras cosas, del desarrollo de algoritmos para resolver el 
problema~{\sat} sobre todo con respecto a ciertas familias de cláusulas para 
los que sí existen algoritmos eficientes, como son las llamadas cláusulas de 
Horn. Hoy en día existen programas sumamente eficientes para resolver el 
problema {\sat}, los cuales se conocen como solucionadores 
{\sat}~\footnote{En inglés {\em {\sat} solvers}.} y tienen muchas 
aplicaciones relevantes fuera de la lógica.

Es de interés observar que si se restringe el problema de forma que cada 
cláusula dada tenga a lo más $k$-variables (el llamado problema $k$-{\sat}) 
sigue siendo NP-completo para~$k\geq 3$. Por otra parte el caso~$k=2$ se 
resuelve en tiempo polinomial.

% Algunos problemas de interés que se muestran NP-completos mediante una 
% reducción directa al problema {\sat} son:


\section{Resolución Proposicional}

Para terminar nuestro panorama de la lógica clausular presentamos la regla de 
resolución binaria de Robinson y en particular un nuevo método para decidir la 
correctud de un argumento lógico. Más adelante estudiaremos a profundidad la 
versión de esta regla para la lógica de predicados y su relevancia en la 
programación lógica. Las teorías de prueba basadas en resolución han 
demostrado ser completas tanto para lógica proposicional como para lógica de 
predicados.
La aportación de Robinson~\footnote{Robinson, J. Alan (1965). \enquote{A 
Machine-Oriented Logic Based on the Resolution Principle}. Journal of the ACM 
vol. 12 doi:10.1145/321250.321253.}, es dar una versión generalizada de la 
regla de inferencia de resolución sobre cláusulas.

\defin{Sean $\C_1,\;\C_2$ cláusulas y~$\ell$ una literal. La regla de
  inferencia conocida como resolución binaria proposicional se define como 
  sigue:
  \begin{mathpar}
    \inferrule*[right=Res]{
    \C_1\lor \ell \and  \ell^c \lor\C_2
    }{
    \C_1\lor \C_2
    }
  \end{mathpar}
 donde $\ell^c$ es la literal contraria de $\ell$. En tal situación decimos que 
 se \textbf{resuelven} las dos premisas con respecto a la literal~$\ell$ y a la 
 cláusula resultante $\C_1\lor \C_2$ se le llama resolvente o resolvente 
 binario.
}
El adjetivo \textit{binaria} se refiere al hecho de que tenemos dos premisas,
la regla toma dos cláusulas tales que existe una literal~$\ell$ que figura en 
una de ellas mientras que la literal contraria~$\ell^c$ figura en la otra. Si 
bien en la definición de la regla las literales~$\ell$ y $\ell^c$ aparecen al 
final y principio de las cláusulas respectivamente, el orden no importa dado 
que la disyunción es conmutativa. 


\noindent Veamos un par de ejemplos:
\begin{mathpar}
 \inferrule[]{
 \neg p\lor q\lor r \and s\lor \neg r\lor \neg t
 }{
 \neg p\lor q\lor s\lor \neg t
 } 
 
 \inferrule[]{
 t\lor\neg s\lor q \and \neg q\lor w\lor s\lor u
 }{
 t\lor\neg s\lor w\lor s\lor u
 }
\end{mathpar}
Obsérvese que la regla es \textbf{no determinista}, en el segundo ejemplo otro
resolvente posible es $t\lor q\lor \neg q\lor w\lor u$. \\

Dado que las literales también son cláusulas la aplicación de resolución a~$p$ 
y $\neg p$ devuelve como resultado la llamada \textbf{cláusula vacía}, 
denotada~$\cv$, es decir la siguiente es una instancia válida de resolución
\begin{mathpar}
 \inferrule[]{
  p \and \neg p
  }{
  \cv
  }
 \end{mathpar}
 
Es importante observar que la regla opera sólamente sobre un par de literales 
complementarias a la vez y nunca con más, por ejemplo la siguiente es
una aplicación \textbf{incorrecta} de resolución:
\begin{mathpar}
 \inferrule[]{
 p\lor\neg q \and q\lor\neg p
 }{
 \cv
 }
\end{mathpar}

Aplicaciones correctas dan como resultado los resolventes totalmente diferentes 
a saber $p\lor\neg p$ y $\neg q\lor q$. En particular, a partir de las premisas 
dadas jamás se podrá generar la cláusula vacía.

\espc

La resolución binaria proporciona un método de decisión para la lógica que, al 
igual que en el caso de los tableaux semánticos, utiliza el principio de 
\textbf{refutación} para decidir la consecuencia lógica:
\bc
\textit{para decidir si $\G\models\vp$ basta demostrar que $\G\cup\{\neg\vp\}$ 
es insatisfacible}
\ec
En resolución binaria basta con obtener la cláusula vacía usando la regla de 
resolución binaria repetidas veces, partir del conjunto de cláusulas de la 
formas normales conjuntivas del conjunto~$\G\cup\{\neg\vp\}$. 
Este proceso se conoce como una \textit{refutación} del conjunto de cláusulas.
%El teorema de resoluci\'on proporciona un procedimiento de refutaci\'on
%para derivar la cl\'ausula vac\'{\i}a a partir de un conjunto no
%satisfacible de cl\'ausulas.
%En un \emph{procedimiento por refutaci\'on} una prueba que las f\'ormulas
%$\vp_1,\ldots,\allowbreak\vp_n,\neg\psi$ no pueden ser todas verdaderas; en 
% nuestro caso lo primero que se hace es transformas las f\'ormulas a 
% cl\'ausulas y luego tratar de generar la cl\'ausula vac\'{\i}a a partir del 
% conjunto original de cl\'ausulas.

\ejem{Veamos un ejemplo sencillo, queremos decidir si 
$$\{p\imp (q\imp r),\lnot (q\imp r)\}\models \lnot p $$
El conjunto de formas normales conjuntivas para $\G\cup\{\lnot\lnot p\}$ es:
$$ \{\lnot p\lor\lnot q\lor r, q\land\lnot r, p\} $$
a partir de este se obtiene el conjunto de cláusulas correspondiente al separar
las cláusulas de cada forma normal conjuntiva 
$$ \{\lnot p\lor\lnot q\lor r, q,\lnot r,p\} $$
Finalmente de este último conjunto obtenemos la siguiente \textbf{derivación} 
de~$\cv$:
\[
\ba{rll}
1. & \lnot p\lor\lnot q\lor r & Hip.\\
2. & q & Hip. \\
3. & \lnot r & Hip. \\
4. & p & Hip. \\
5. & \lnot p\lor r & Res(1,2)\\
6. & r & Res(5,4)\\
7. & \cv & Res(6,3)
\ea
\]
}

\ejem{Veamos otro ejemplo, queremos decidir si el argumento
$$ p\imp q,\; r\imp s,\; p\lor r\;/\;\therefore q\lor s $$
es correcto.
El conjunto a refutar es:
$$ \{\lnot p\lor q,\;\lnot r\lor s,\;p\lor r,\;\neg q,\;\neg s\} $$
Obtenemos la siguiente derivación de~$\cv$:
\[
\ba{rll}
1.&  \lnot p\lor q & Hip.\\
2.& \lnot r \lor s & Hip. \\
3.& p\lor r & Hip. \\
4.& \neg q & Hip. \\
5.& \neg s & Hip. \\
6.& \lnot p & Res(1,4)\\
7.& r & Res(6,3)\\
8.& s & Res(2,7)\\
9.& \cv & Res(5,8)
\ea
\]
De manera que el argumento es correcto.
}

\section{Algoritmos de saturación}

\defin{Si~$\Sg$ es cualquier conjunto de cl\'ausulas entonces la resolución de 
$\Sg$, denotada $\mathcal{R}(\Sg)$ es el conjunto que consiste de~$\Sg$ junto 
con \textbf{todos} los resolventes de cl\'ausulas de~$\Sg$, es decir
$$ \mathcal{R}(\Sg)=\Sg\cup \{E \mid \text{ existen } C,D\in\Sg 
\text{ tales que } E \text{ es un resolvente de } C \text{ y } D\} $$
La $n$-\'esima resoluci\'on de~$\Sg$ se define recursivamente como sigue:
\[
\ba{rl}
Res_0(\Sg) & =\Sg \\
Res_{n+1}(\Sg) & =\mathcal{R}(Res_n(\Sg))
\ea
\]
}

La importancia de la $n$-\'esima resoluci\'on se da en la siguiente proposición
que nos da una herramienta muy \'util para la inferencia l\'ogica:
\begin{proposition}
Sea $\Sg$ es un conjunto finito de cl\'ausulas. Entonces $\Sg$ es no 
satisfacible si y s\'olo si \\$\cv\in Res_n(\Sg)$ para alguna $n\in\mathbb{N}$.
\end{proposition}
%\proof
%La demostraci\'on queda fuera de los objetivos de este libro, pero puede
%consultarse en \cite{mir}.
%\qed


%\definition{
%Sean $\mathcal{R}$ el c\'alculo que consta \'unicamente de la regla de
%resoluci\'on binaria y que no tiene axiomas y $\Sg$ un conjunto de
%cl\'ausulas. Una {refutaci\'on} de $\Sg$ es una derivaci\'on de la
%cl\'ausula vac\'{\i}a a partir de $\Sg$. 
%}

El resultado anterior proporciona, en teoría, un método para verificar
si un conjunto de cláusulas~$\Sg$ es insatisfacible, basta construir
los conjuntos~$Res_n(\Sg)$ hasta hallar~$\cv$. Esto se hace mediante
los llamados algoritmos de saturación que se encargan de generar todas las
derivaciones con resolución a partir de un conjunto dado~$\Sg$.\\ 
Por supuesto, el método por si s\'olo es demasiado ineficiente pero combinado
con algoritmos de poda del espacio de búsqueda y de eliminación de 
cláusulas redundantes se obtiene un mecanismo poderoso de inferencia.

\noindent Analicemos los escenarios posibles,en teoría hay tres:
\be
\item En algún momento~$\cv$ es generada, es decir $\cv\in Res_n(\Sg)$
  para algún $n\in\N$.\\ En este caso el conjunto~$\Gamma$ de entrada es 
  insatisfacible.
 \item El algoritmo termina sin generar~$\cv$ jamás, es decir, en algún 
  momento se tiene $Res_n(\Sg)=Res_{n+1}(\Sg)$ por lo que no hay más 
  resolventes posibles, pero $\cv\notin Res_n(\Sg)$.\\
  En este caso $\Gamma$ es satisfacible. 
\item La saturación no termina nunca pero tampoco se genera la cláusula
  vacía, es decir, se tiene la certeza de que para cualquier $n\in\N$,
  $\cv\notin Res_n(\Sg)$. \\En este caso $\Gamma$ es satisfacible.
\ee

\noindent En la práctica el tercer escenario es poco factible y en su lugar
tenemos
\be
%\item En algún momento $\cv$ es generada, en este caso el conjunto
%  $\Gamma$ de entrada es insatisfacible.\espc
% \item El algoritmo termina sin generar $\cv$ jamás, en este caso $\Gamma$
%   es satisfacible. \espc
\item[3'.] La saturación no termina, el algoritmo se ejecuta {hasta
    agotar los recursos computacionales} pero no se genera~$\cv$.\\
    En este caso se \textbf{desconoce} si $\Gamma$ es satisfacible.
\ee


\ejem{\label{Ej:res3}
Sea $\Sg$ el conjunto que consta de las siguientes cl\'ausulas:
$$
\ba{rrl}
1. &  p \lor r  % Px,Rfa & \pmi.  &
\\
2. &  \neg p \lor \neg r %      & \pmi   & Px, Rgx.
\\
3. & p %3. & Pb     & \pmi.  &
\\
4. & r % & Rga    & \pmi.  &     
\ea
$$
Empezamos con $Res_0(\Sg)=\Sg$. \\
Entonces, $Res_1(\Sg)=\mathcal{R}(Res_0(\Sg))=Res_0(\Sg)\;\cup$
$$
\ba{rrlll}
5. & \neg r \lor r % Rfa      & \pmi  & Rga.
& res(1,2)\;\; \\ %(\mbox{renombre en 2. con $[x:=y]$}). \\
6. & \neg r %         & \pmi. & Rgb.
& res(2,3). \\
7. &  \neg p %        & \pmi. & Pa.
& res(2,4). 
\ea
$$ 
Y finalmente:
$Res_2(\Sg)=\mathcal{R}(Res_1(\Sg))=Res_1(\Sg)\;\cup$
$$
\ba{rrlll}
8. & p\lor r % Rfa     & \pmi. &
& res(1,5). \\
9. & p % Rfa     & \pmi. &
& res(1,6). \\
10. & r % Rfa     & \pmi. &
& res(1,7). \\
11. & \neg p\lor \neg r % Rfa     & \pmi. &
& res(2,5). \\
12. & \cv % Rfa     & \pmi. &
& res(3,7).\\ 
\ea
$$
%$Res_3(\Sg)=Res(\Sg)\;\cup\varnothing=Res_2(\Sg)$\\
Por lo tanto $\cv\in Res_2(\Sg)$ y $\Sg$ no es satisfacible.
}

\ejem{\label{Ej:res1}
Sea $\Sg$ el conjunto que consta de las siguientes cl\'ausulas:
$$
\ba{rrl}
1. & \neg p  \\%&        & \pmi  & Pa. \\
2. & \neg q \lor p \\ %& Px     & \pmi  & Qx. \\
3. & \neg r \lor p  \\ %& Px     & \pmi  & Rfx. \\
4. & q \lor r \\ %& Qa,Rfa & \pmi. & 
\ea
$$
Comenzamos con $Res_0(\Sg)=\Sg$.\\
La primera iteración genera: 
$Res_1(\Sg)=\mathcal{R}(Res_0(\Sg))=Res_0(\Sg)\;\cup$
$$
\ba{rrl}
5. & \neg q  % &        & \pmi  & Qa.
& res(1,2). \\
6. & \neg r % &        & \pmi  & Rfa.
& res(1,3). \\
7. & p \lor r %Pa,Rfa & \pmi. &
& res(2,4). \\
8. & p \lor q % Pa, Qa & \pmi. &
& res(3,4). 
\ea
$$ 
La siguiente iteración
$Res_2(\Sg)=\mathcal{R}(Res_1(\Sg))=\;Res_1(\Sg)\;\cup$
$$
\ba{rrl}
9.  & r %& \pmi. &
& res(1,7). \\
10. & q  %& \pmi. &
& res(1,8). \\
11. & p\lor p  %& \pmi. &
& res(2,8). \\
12. & p\lor p  %& \pmi. &
& res(3,7). \\
%13. & Pa  %& \pmi. &
%& res(3,7). \\
13. & r %& \pmi. &
& res(4,5). \\
14. & q  %& \pmi. &
& res(4,6). \\
15. & p  %& \pmi. &
& res(5,8). \\
16. & p  %& \pmi. &
& res(6,7). 
\ea
$$
Y finalmente 
$Res_3(\Sg)=\mathcal{R}(Res_2(\Sg))=Res_2(\Sg)\;\cup$
$$
\ba{rrl}
17. & p %& \pmi. &
& res(1,11).\\
18. & p %& \pmi. &
& res(1,12).\\
19. & \cv %& \pmi. &
& res(1,15).\\
% 20. & P %& \pmi. &
% & res(2,14).\\
% 21. & P %& \pmi. &
% & res(3,9).\\
% 22. & P %& \pmi. &
% & res(3,13).
\ea
$$
Por lo tanto $\cv$ pertenece a la tercera
resoluci\'on de $\Sg$, as\'{\i} que $\Sg$ es no satisfacible.
}

%\ejem{\label{Ej:res2}
%Sea $\Sg$ el conjunto que consta de las siguientes cl\'ausulas:
%$$
%\ba{rrll}
%1. & Px,Qx  & \pmi.  & \\
%2. &        & \pmi   & Px, Qx. \\
%3. & Qx     & \pmi   & Px. \\
%4. &        & \pmi   & Qa. 
%\ea
%$$
%$Res(\Sg)=\Sg\;\cup$
%$$
%\ba{rrlll}
%5. & Qx       & \pmi  & Qx.  & res(1,2). \\
%6. & Qx       & \pmi. &      & res(1,3). \\
%7. & Pa       & \pmi. &      & res(1,4). \\
%8. &          & \pmi  & Px.  & res(2,3). \\
%9. &          & \pmi  & Pa.  & res(3,4). 
%\ea
%$$ 
%$Res_2(\Sg)=Res(\Sg)\;\cup$
%$$
%\ba{rrlll}
%10. & Px, Qx & \pmi. &         & res(1,5). \\
%11. & Qx     & \pmi. &         & res(1,8). \\
%12. & Qa     & \pmi. &         & res(1,9). \\
%13. &        & \pmi  & Px, Qx. & res(2,5). \\
%14. &        & \pmi  & Px.     & res(2,6). \\
%15. &        & \pmi  & Qa.     & res(2,7). \\
%16. & Qx     & \pmi  & Px.     & res(3,5). \\
%17. & Qa     & \pmi. &         & res(3,7). \\
%18. &        & \pmi  & Qa      & res(4,5). \\
%19. &        & \pmi. &         & res(4,6).
%\ea
%$$

%Por lo tanto $\cv$  pertenece a la segunda
%resoluci\'on de $\Sg$, as\'{\i} que $\Sg$ es no satisfacible.
%}



\ejem{Sea $\Sg$ el conjunto que consta de las siguientes cl\'ausulas:
$$
\ba{rrl}
1. & r\lor p\lor q  \\%&        & \pmi  & Pa. \\
2. & r \\ %& Px     & \pmi  & Qx. \\
3. & \neg q  \\ %& Px     & \pmi  & Rfx. \\
\ea
$$
Comenzamos con $Res_0(\Sg)=\Sg$.\\
Después $Res_1(\Sg)=\mathcal{R}(Res_0(\Sg))=Res_0(\Sg)\;\cup$
$$
\ba{rrl}
4. & r \lor p & res(1,3) \\%& Qa,Rfa & \pmi. & 
\ea
$$
Pero al calcular la siguiente iteración nos encontramos con
$Res_2(\Sg)=\mathcal{R}(Res_1(\Sg))=Res_1(\Sg)\;\cup$
$\vacio$
% \ba{rrl}
% 4. & r \lor p & res(1,3) \\%& Qa,Rfa & \pmi. & 
% \ea

\noindent Por lo tanto $Res_n(\Sg)=Res_1(\Sg)$ para toda $n\geq 1$ y como 
$\cv\notin
Res_1(\Sg)$ entonces $\Sg$ es satisfacible.

}


Los siguientes ejercicios para decidir la correctud de argumentos dejan ver 
varios problemas en la búsqueda de la cláusula vacía, los cuales complican el 
proceso de mecanización discutido arriba. Los sistemas que implementan el 
cálculo de resolventes binarios deben servirse de diversas \textit{heurísticas} 
para podar el espacio de búsqueda. El área de la lógica que se dedica a estos 
menesteres es, nuevamente, el razonamiento automatizado.

\bi
\item $p\lor (q\land r),\; p\imp q,\;q\iff s,\;/\therefore\; q\land s$
\item $\neg (\neg p\imp q),\;\neg (r\iff p),\;p\lor r,\;\neg (r\imp 
q)\;/\therefore\; \neg (p\imp q)$
\item $p\to q,\;r\to s,\;p\lor r,\;\neg (q\land s)\;/\therefore\;(q\to 
p)\land(s\to r)$
\ei


\end{document}
