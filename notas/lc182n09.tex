\documentclass[11pt,letterpaper]{article}
\usepackage{../packageslc}
\usepackage{../optionslc}

%%%% Macros para las notas de lenguajes de programacion


%%%% math

\newcommand{\vphi}{\varphi}
\newcommand{\vp}{\varphi}

% \newcommand{\dn}{\mathsf{DN}}
% \newcommand{\dnC}{\mathsf{DN_C}}
% \newcommand{\dnM}{\mathsf{DN_M}}
% \newcommand{\dnp}{\mathsf{DN_p}}
% \newcommand{\dnm}{\mathsf{DN_p^M}}
% \newcommand{\dnc}{\mathsf{DN_p^C}}

%\newcommand{\case}{\mathsf{case}}
%\renewcommand\labelitemi{$\circ$}

\newcommand{\imp}{\rightarrow}
\newcommand{\Imp}{\Rightarrow}
\renewcommand{\iff}{\leftrightarrow}
\newcommand{\Iff}{\Leftrightarrow}
\newcommand{\G}{\Gamma}
\newcommand{\D}{\Delta}


\newcommand{\De}{\mathcal{D}}
\newcommand{\F}{\mathcal{F}}
\newcommand{\Ge}{\mathcal{G}}
\newcommand{\Pe}{\mathcal{P}}
\newcommand{\I}{\mathcal{I}}
\newcommand{\C}{\mathcal{C}}
\newcommand{\K}{\mathcal{K}}
\renewcommand{\L}{\mathcal{L}}
\newcommand{\M}{\mathcal{M}}
\newcommand{\Nc}{\mathcal{N}}
%\newcommand{\E}{\mathcal{E}}
%\newcommand{\R}{\mathcal{R}}
%\newcommand{\Q}{\mathcal{Q}}
\newcommand{\Sc}{\mathcal{S}}
\newcommand{\Te}{\mathcal{T}}
\newcommand{\W}{\mathcal{W}}

\newcommand{\Db}{\mathbb{D}}
\newcommand{\Fb}{\mathbb{F}}
\newcommand{\Kb}{\mathbb{K}}
\newcommand{\Eb}{\mathbb{E}}
\newcommand{\Ebs}{\mathbb{E}^\star}
\newcommand{\Ob}{\mathbb{O}}
\newcommand{\Ib}{\mathbb{I}}
\newcommand{\Rb}{\mathbb{R}}
\newcommand{\Qb}{\mathbb{Q}}
\newcommand{\Kbb}{\mathbb{K}}
\newcommand{\T}{\mathbb{\Theta}}


\newcommand{\kb}{\bbkappa}

\newcommand{\Sf}{\mathsf{\Sigma}}

\newcommand{\fa}{\forall}
\newcommand{\ex}{\exists}

\newcommand{\inc}{\subseteq}

\newcommand{\Lb}{\Lambda}
\newcommand{\Om}{\Omega}
\newcommand{\lb}{\lambda}
\newcommand{\al}{\alpha}
\newcommand{\ga}{\gamma}


\newcommand{\mg}{\mathbb{m}}

\newcommand{\cg}{\mathbb{C}}
\newcommand{\dg}{\mathbb{D}}
\newcommand{\jg}{\mathbb{J}}
\newcommand{\Ha}{\mathcal{H}}
%\newcommand{\A}{\mathcal{A}}
\newcommand{\sg}{\mathbb{S}}

\newcommand{\Bc}{\mathcal{B}}
\newcommand{\Df}{\mathfrak{D}}
\newcommand{\Dc}{\mathcal{D}}
%\newcommand{\Tc}{\mathcal{T}}
\newcommand{\Mf}{\mathfrak{M}}

\newcommand{\Sg}{\mathbb{S}}

\newcommand{\Q}{\ensuremath{\mathbb{Q}}}
\newcommand{\Z}{\ensuremath{\mathbb{Z}}}
\newcommand{\N}{\ensuremath{\mathbb{N}}}
\newcommand{\R}{\ensuremath{\mathbb{R}}}
\renewcommand{\S}{\mathbb{\Sigma}}
\newcommand{\A}{\mathcal{A}}
\newcommand{\E}{\ensuremath{\exists}}
\newcommand{\iso}{\ensuremath{\cong}}
\newcommand{\union}{\ensuremath{\cup}}
\newcommand{\morinyec}{\ensuremath{\precapprox}}

\newcommand{\nin}{\ensuremath{\notin}}
\newcommand{\tog}{\makebox[7mm][l]}
\newcommand{\toge}{\makebox[11mm][l]}
\newcommand{\toget}{\makebox[13mm][l]}
\newcommand{\togeth}{\makebox[14mm][l]}
\newcommand{\togethe}{\makebox[15mm][l]}
\newcommand{\together}{\makebox[17mm][l]}
\newcommand{\niso}{\ensuremath{\not \cong}}


\newcommand{\Mg}{\mathbb{M}}
\newcommand{\Bg}{\mathbb{B}}
\newcommand{\Lg}{\mathbb{L}}
\newcommand{\Tg}{\mathbb{T}}

\newcommand{\sketch}{\Red{{\sc sketch}}}

\newcommand{\restr}[2]{#1\!\!\boldsymbol{\restriction}\!#2}

\newcommand{\vacio}{\varnothing}
\newcommand{\done}{\ensuremath{\checkmark}}

\newcommand{\ida}{$\Rightarrow \; )$ }
\newcommand{\regr}{$\Leftarrow \; )$ }

\newcommand{\ol}[1]{\overline{#1}}

\newcommand{\Tsf}{\mathsf{T}}

\newcommand{\inds}[1]{\index[simb]{#1}}

\newcommand{\B}{\mathbb{B}}
%\newcommand{\N}{\mathbb{N}}

\newcommand{\vx}{\vec{x}}
\newcommand{\vy}{\vec{y}}
\newcommand{\vz}{\vec{z}}
\newcommand{\vt}{\vec{t}}
\newcommand{\vf}{\vec{f}}


% \newcommand{\propo}{\ensuremath{\mathsf{PROP}}}
% \newcommand{\atom}{\ensuremath{\mathsf{ATOM}}}
\newcommand{\term}{\ensuremath{\mathsf{TERM}}}
\newcommand{\form}{\mathsf{FORM}}

\newcommand{\true}{\mathop{\mathsf{true}}}

%\newcommand{\id}{\mathsf{Id}}

%\newcommand{\uc}{\mathcal{U}}
%\newcommand{\Ic}{\mathcal{I}}
%\newcommand{\pc}{\mathcal{P}}
%\newcommand{\qc}{\mathcal{Q}}
%\newcommand{\mc}{\mathcal{M}}
\newcommand{\supc}{\supseteq}
\newcommand{\limo}{\mathop{\mathpzc{Lim}}}
\newcommand{\ord}{\mathsf{OR}}

\newcommand{\pt}[1]{\langle #1 \rangle}


%%%% frames
\newcommand{\titulos}[2]{\frametitle{#1}\framesubtitle{#2}}
\newcommand{\fot}[1]{\footnote{\scriptsize{#1}}}


%%%% ambientes

\newcommand{\cb}[2]{\colorbox{#1}{#2}}

\newcommand{\bc}{\begin{center}}
\newcommand{\ec}{\end{center}}
\newcommand{\be}{\begin{enumerate}}
\newcommand{\ee}{\end{enumerate}}
\newcommand{\bi}{\begin{itemize}}
\newcommand{\ei}{\end{itemize}}
\newcommand{\beq}{\begin{equation}}
\newcommand{\eeq}{\end{equation}}
\newcommand{\beqs}{\begin{equation*}}
\newcommand{\eeqs}{\end{equation*}}
\newcommand{\ba}{\begin{array}}
\newcommand{\ea}{\end{array}}


% \newtheorem{theorem}{Teorema}
% \newcommand{\teo}[1]{\begin{theorem} #1 \end{theorem}}
% \newtheorem{proposition}{Proposici\'on}
% \newcommand{\prop}[1]{\begin{proposition} #1 \end{proposition}}
% \newtheorem{definition}{Definici\'on}
% \newcommand{\defin}[1]{\begin{definition} #1 \end{definition}}
% \newtheorem{corollary}{Corolario}
% \newcommand{\cor}[1]{\begin{corollary} #1 \end{corollary}}
% \newtheorem{lemma}{Lema}
% \newcommand{\lema}[1]{\begin{lemma} #1 \end{lemma}}
% \newcommand{\dem}[1]{\begin{proof} #1 \end{proof}}

%\renewcommand{\qed}{\qedsymbol{$\mathbf{\dashv}$}}

%\newcommand{\proof}{\hfill\\\noindent\textbf{\textit{Demostraci\'on. }}}

\newcommand{\hint}{\emph{Sugerencia: }}


\newcounter{EjempCtr}[section]
\newenvironment{enumrom}{\renewcommand{\theenumi}{\roman{enumi}}%
\renewcommand{\theenumii}{\roman{enumii}}
\renewcommand{\theenumiii}{\roman{enumiii}}
\renewcommand{\theenumiv}{\roman{enumiv}}
\begin{enumerate}}{\end{enumerate}}
\newenvironment{Ejemplo}
        {\stepcounter{EjempCtr}%
        \begin{description}\item[Ejemplo \thesection.\arabic{EjempCtr}]}%
        {\end{description}}
\newenvironment{demostr}{%
             {\em Demostración:}
                \begin{quotation}}{\end{quotation}}

   \newcommand{\beje}{\begin{Ejemplo}}
\newcommand{\eeje}{\end{Ejemplo}}


\newtheorem{eje}{Ejemplo}[section]
\newcommand{\ejem}[1]{\begin{eje}\normalfont #1 \end{eje}}

% \renewcommand\contentsname{\'Indice}
%\renewcommand\chaptername{Cap\'itulo}
% \renewcommand\indexname{\'Indice}

%%\newcommand{\qed}{\hfill$\mathbb{Qed}$}
%\newcommand{\qed}{\hfill$\mathsf{\boldsymbol{\dashv}}$}
%\renewcommand{\qed}{\hfill$\boldsymbol{\dashv}$}


\newenvironment{prueba}{\vspace{-5mm}\noindent\textbf{Demostraci\'on}\\}{
\noindent$\blacksquare$\\}

\newcommand{\Ejercicios}{\section*{Ejercicios}}

 \newenvironment{manitas}{%
      \renewcommand{\labelitemi}{\ding{44}}%
      \vspace{-0.5cm}%
      \begin{itemize}%
      \setlength{\itemsep}{0pt}\setlength{\parsep}{0pt}\setlength{\topsep}{0pt}%
      }{\end{itemize}}
\newenvironment{malitos}{%
      \renewcommand{\labelitemi}%
            {\raisebox{1.5ex}{\makebox[0.3cm][l]{\begin{rotate}{-90}%
            \ding{43}\end{rotate}}}}%
      \vspace{-0.5cm}%
      \begin{itemize}%
      \setlength{\itemsep}{0pt}\setlength{\parsep}{0pt}\setlength{\topsep}{0pt}%
      }{\end{itemize}}
\newenvironment{ejercs}{
     \renewcommand{\labelenumi}{\thesection.\theenumi.-}
     \renewcommand{\labelenumii}{\theenumii)}
     \begin{enumerate}}
     {\end{enumerate}}

   \newcommand{\bej}{\begin{ejercs}}
\newcommand{\eej}{\end{ejercs}}


%\newenvironment{leterize}{%
%        \renewcommand{\theenumi}{\alph{enumi}}
%        \begin{enumerate}}{\end{enumerate}}

%\newenvironment{manitas}{%
%      \renewcommand{\labelitemi}{\ding{44}}%
%      \vspace{-0.5cm}%
%      \begin{itemize}%
%      
% \setlength{\itemsep}{0pt}\setlength{\parsep}{0pt}\setlength{\topsep}{0pt}%
%      }{\end{itemize}}



%%=============================================================================

\def\stackunder#1#2{\mathrel{\mathop{#2}\limits_{#1}}}


%%%% notas

\newcommand{\doubt}{\Red{{\LARGE {\sf ??}}}}

\newcommand{\coment}[1]{\hfill\\ \Big[{\bf Comentario Privado:} #1\Big]}
\newcommand{\preg}[1]{\hfill\\ \BrickRed{{\bf Pregunta:} #1}}
\newcommand{\conjet}[1]{\hfill\\ \OliveGreen{{\bf Conjecura:} #1}}

\newcommand{\pendiente}{\BrickRed{{\sc Pendiente}}}
\newcommand{\verifpendiente}{\BrickRed{{\sc Verificación pendiente}}}


%--------------------------------------------------------------------------

\DeclareMathAlphabet{\mathpzc}{OT1}{pzc}{m}{it}



\title{Unificaci\'on de t\'erminos \\
L\'ogica Computacional 2018-2, Nota de clase 9}
\author{Favio Ezequiel Miranda Perea\and Araceli Liliana Reyes Cabello\and
Lourdes Del Carmen Gonz\'alez Huesca \and Pilar Selene Linares Ar\'evalo}
\date{\today\\ Facultad de Ciencias UNAM}

\begin{document}

\maketitle
La regla de resolución binaria de Robinson, que veremos más adelante, 
proporciona un método más para decidir la correctud de un argumento formalizado 
en lógica de predicados. 
% En esta nota veremos los conceptos básicos del método
% de resolución binaria en lógica de predicados.

Si bien la idea del método de prueba es la misma que en la lógica 
proposicional, es decir llegar a la cláusula vacía para mostrar que un argumento 
es correcto o que una fórmula es válida, en lógica de predicados el método se 
complica debido a la presencia de variables, por ejemplo: a diferencia de las 
literales $p,\neg p$ en lógica proposicional, las literales $Pax, \neg Pzb$ no 
pueden resolverse aunque tengan el mismo símbolo de predicado, pues no son 
contrarias estrictamente. 
Sin embargo, como veremos después,  podemos aplicar la sustitución $\sigma=[x,z:=b,a]$ a las fórmulas $ Pax$ y $\neg Pzb$, 
obteniendo como resultado $Pab$ y $\neg Pab$ que sí son literales contrarias. 
Por lo tanto la búsqueda y uso de sustituciones que permitan obtener literales 
contrarias será de primordial importancia para usar resolución en lógica de 
predicados. Esta clase de sustituciones se conocen como \textbf{unificadores} y 
en esta nota veremos c\'omo es posible hallarlos algorítmicamente.


\section{Sustituciones}

El concepto de sustitución usado hasta ahora ha sido informal, en adelante 
necesitamos una definición formal discutida a continuación.

\defin{Una sustitución en un lenguaje de predicados es una función recursiva
$\sigma: Term \imp Term$ con la propiedad de que $\sigma(x)=x$ para casi
todas las variables. Es decir $\sigma(x_i)\neq x_i$ únicamente para un número
finito de variables~$x_i$ con $1\leq i\leq n$. 
Esto se conoce en matemáticas como la propiedad de que una función tenga 
soporte finito, siendo el soporte de $\sigma$ el conjunto de variables~$x_i$ 
con $\sigma(x_i)\neq x_i$.
}
Obsérvese que una sustitución~$\sigma$ queda completamente determinada mediante 
su soporte (finito), si éste es $x_1,\ldots,x_n$ con $\sigma(x_i)=t_i$ entonces
la sustitución se denotará con
$$ \sigma = [x_1:= t_1,\;x_2:=t_2,\;\ldots,\;x_n:=t_n] $$
%[x_1,x_2,\ldots,x_n:=t_1,t_2,\ldots,t_n]
o de manera más breve $\sigma=[\vx:=\vt\,]$. \\


La composición de sustituciones se define como una composición de funciones de 
donde se ha tomado la siguiente caracterización por motivos de implementación:
\prop{Sean $\sigma=[x_1:= t_1,\;\ldots,\;x_n:=t_n]$  y 
%[x_1,\ldots,x_n:=t_1,\ldots,t_n]$, 
$\tau=[y_1:=s_1,\;\ldots,\;y_m:=s_m]$ dos sustituciones.
La composición, denotada $\sigma\circ\tau$ o simplemente $\sigma\tau$, es una 
sustitución definida por:
\[
\sigma\tau = [z_1:=t_1\tau,\;\ldots,\;z_k:=t_k\tau,\,w_1:=r_1,\;\ldots,w_l:=r_l]
% [z_1,\ldots,z_k,w_1,\ldots,w_l  :=  t_1\tau,\ldots,t_k\tau,\,r_1,\ldots,r_l]
\]
donde $k\leq n,l\leq m$, las variables~$z_i$ son exactamente aquellas $x_p$ 
tales que $x_p\neq t_p\tau$, y los pares $w_j:=r_j$ son aquellos pares 
$y_q:=s_q$ tales que $y_q\notin\{x_1,\ldots,x_n\}$
% no incluidas en $x_1,\ldots,x_n$. 
}
\proof
Ejercicio
\qed

\prop{La composición de sustituciones es asociativa, es decir, si
  $\sigma,\;\rho$ y $\tau$ son tres sustituciones entonces 
  $(\sigma\rho)\tau=\sigma(\rho\tau)$.
}
\proof
Ejercicio
\qed

\ejem{Sean  % $\sigma=[x,y:=f(y),w],\;\rho=[x,z:=g(w),b],
% \;\tau=[y,w,v:=b,f(c),w]$
\[
\sigma=[x:=f(y),\;y:=w]\qquad \rho=[x:=g(w),\;z:=b]\qquad  
\tau=[y:=b,\;w:=f(c),\;v:=w]
\]
Entonces
\[
\ba{rll}
\sigma\rho & = & [x:=f(y),\;y:=w,\;z:=b]\\ \\ %[x,y,z:=f(y),w,b]. \\ \\
\rho\tau & = & [x:=g(f(c)),\;z:=b,\;y:=b,\;w:=f(c),\;v:=w]\\ \\ 
%[x,y,z:=f(y),w,b]. 
\\ 
(\sigma\rho)\tau & = & [x:=f(b),\;y:=f(c),\;z:=b,\;w:=f(c),\;v:=w] \\ \\ 
%[x,y,z,w,v:=f(b),f(c),b,f(c),w].
\sigma(\rho\tau) & = & [x:=f(b),\;y:=f(c),\;z:=b,\;w:=f(c),\;v:=w]
\ea
\]
}


\section{Unificaci\'on}

El proceso de unificaci\'on consiste en encontrar, dado un conjunto de 
literales o t\'erminos $W$, una sustitución~$\sigma$ de tal forma que el 
conjunto resultante $W\sigma$ conste de un solo elemento. \\
En adici\'on a sus aplicaciones en programaci\'on l\'ogica, la
unificaci\'on tambi\'en es importante para los sistemas de reescritura de
t\'erminos, el razonamiento autom\'atizado y los sistemas de tipos.
A continuación estudiamos el caso general así como un algoritmo de unificación.

\defin{Sea $W$ un conjunto no vac\'{\i}o de términos. Un unificador
de $W$ es una sustitución~$\sigma$ tal que $|W\sigma|=1$, es decir
tal que el conjunto $W\sigma$ resultante de aplicar $\sigma$ a todos los 
elementos de $W$ consta de un mismo elemento. Si $W$ tiene un
unificador decimos que $W$ es unificable.
\index{unificador}\index{conjunto!unificable}
}


\ejem{Sea $W=\{g(x,f(y)),g(x,f(x)),g(u,v)\}$ un conjunto de términos, entonces 
la sustituci\'on~$\sigma=[x:=a,\,\allowbreak y:=a,\,u:=a,\,v:=f(a)\}$ es un 
unificador de $W$ ya que $W\sigma=\{g(a,f(a))\}$
}

Un conjunto de términos puede tener una infinidad de unificadores o
ninguno, dado un conjunto finito de f\'ormulas $W$, es decidible mediante
un algoritmo si $W$ es unificable o no; si $W$ es
unificable el algoritmo proporciona un unificador llamado
\textbf{unificador m\'as general}.\index{unificador!m\'as general}

\defin{Un unificador~$\sigma$ de un conjunto de términos~$W$, se
llama unificador m\'as general (\textbf{umg}) si para cada unificador $\tau$ de 
$W$, existe una sustituci\'on~$\vartheta$, tal que $\sigma\vartheta=\tau$.
\label{defiumg}
}
La importancia de los unificadores m\'as generales es que nos permiten
representar de manera finita un n\'umero infinito de sustituciones.
% de manera que no estamos restringidos a razonar solamente sobre dominios
% finitos. 

\ejem{Sean $W=\{fgaxgyb,fzguv\}$ con $f^{(2)},\;g^{(2)}$ y
$$ \tau=[x:=a, z:=gaa, y:=u,v:=b] \qquad \sigma=[z:=gax,y:=u,v:=b].$$
Entonces $\tau$ y $\sigma$ son unificadores de $W$ y es fácil ver que $\sigma$ 
resulta ser el \textbf{umg} y en particular si $\vartheta=[x:=a]$ entonces 
$\sigma\vartheta=\tau$. 
}

\vspace*{20pt}

Antes de discutir un algoritmo de unificación es conveniente hacer un 
an\'alisis intuitivo del problema. Basta analizar un conjunto de dos
términos digamos $W=\{t_1,t_2\}$, queremos ver si el
conjunto $W$ es unificable. 
Hay que analizar varios casos:
\be
 \item Los términos $t_1$ y $t_2$ son constantes. \\
  En este caso $t_i\sigma=t_i$ para cualquier sustituci\'on $\sigma$, de manera 
  que $W$ ser\'a unificable si y sólo si $t_1=t_2$.
 \item Alguno de los términos es una variable. \\
  Supongamos que $t_1=x$, entonces si $x$ figura en $t_2$ entonces $W$ no es 
  unificable, en caso contrario $\sigma=[x:=t_2]$ unifica a $W$. 
 \item $t_1$ y $t_2$ son términos funcionales. \\
  En este caso $W$ es unificable si y sólo si se cumplen las siguientes dos 
  condiciones:
  \bi
   \item Los símbolos principales (es decir los primeros de izquierda a 
    derecha), de $t_1$ y $t_2$ son el mismo.  
   \item Cada par correspondiente de subexpresiones de $t_1$ y $t_2$ deben ser
    unificables. 
  \ei
\ee

\ejem{Considerando los mismos s\'imbolos de funci\'on que el ejemplo anterior, 
tenemos los siguientes ejemplos:
\begin{itemize}
 \item El conjunto $\{c,d\}$ no es unificable pues consta de dos 
constantes diferentes. 
 \item El conjunto $\{x,fy\}$ es unificable mediante $\sigma=[x:=fy]$.
 \item El conjunto $\{gxw, hya\}$ no es unificable pues $g$ y $h$ son símbolos 
distintos.
 \item El conjunto $\{fxgyw, fagbhw\}$ no es unificable pues los subtérminos 
$w$ y $hw$ no son unificables.
\end{itemize}
}

% La siguiente definición es indispensable para el algoritmo de unificación.

% \defin{Sea $W\neq\varnothing$ un conjunto de expresiones. Definimos el
% conjunto discorde de\index{conjunto!discorde} $W$ como sigue:
% localizar la posici\'on del primer s\'{\i}mbolo, es decir, el s\'{\i}mbolo
% m\'as a la izquierda en el que al menos dos expresiones de $W$ difieren y
% extraer de cada expresi\'on de $W$ la subexpresi\'on que inicia en esta
% posici\'on; el conjunto $D_W$ de dichas subexpresiones es el conjunto
% discorde de $W$.
% }
% \ejem{Si $W=\{f(x,y),g(a)\}$ entonces $D_W=W$.\\
% Si $W=\{P(x,f(y)),P(x,f(a)),P(x,f(h(y)))\}$ entonces $D_W=\{y,a,h(y)\}$.
% }
% A continuación presentamos un algoritmo de unificaci\'on \emph{AU} y algunos 
% ejemplos
% utiliz\'andolo:
% %\newpage
% %\begin{algorithmic}
% %\REQUIRE $k=0,\;\sigma_0=\emptyset$
% %\WHILE{$W\sigma_k$ no sea unitario}
% %\STATE Hallar el conjunto discorde $D_k$ de $W\sigma_k$
% %\IF{existen $v,t\in D_k$ tales que $v$ no figura en $t$}
% %\STATE $\sigma_{k+1}:=\sigma_k\circ[v:=t]$
% %\STATE $k := k+1$ 
% %\ELSE
% %\STATE write('$W$ no es unificable')
% %\STATE \textbf{halt}
% %\ENDIF
% %\ENDWHILE
% %\ENSURE $\sigma_k$ es umg de $W$ \'o $W$ no es unificable. 
% %\end{algorithmic}

% \begin{alltt}
% Precondicion: k=0, \(\sigma\sb{0}\)=[]

% Mientras \(W\sigma\sb{k}\) no sea unitario hacer:
%    Hallar el conjunto discorde \(D\sb{k}\) de \(W\sigma\sb{k}\)
%    Si existen \(v,t\in D\sb{k}\) tales que \(v\) no figura en \(t\)
%    Entonces \(\sigma\sb{k+1}:=\sigma\sb{k}\circ[v:=t]\)
%             \(k := k+1\)
%    Sino ('\(W\) no es unificable')

% Postcondicion: \(\sigma\sb{k}\) es umg de \(W\) o \(W\) no es unificable. 
% \end{alltt}

% \ejem{Sea $W=\{P(f(a),g(x)),\;P(y,y)\}$. Tomamos el primer conjunto
% discorde $D_0=\{f(a),y\}$. Como $y$ no figura en $f(a)$ hacemos
% $\sigma_1=[y:=f(a)]$. $W\sigma_1=\{P(f(a),g(x),\;P(f(a),f(a))\}$. Como 
% $W\sigma_1$
% no es unitario obtenemos $D_1=\{g(x),f(a)\}$ y dado que en $D_1$ no hay
% variables el algoritmo termina reportanto que $W$ no es unificable.
% }
% \ejem{Sea $W=\{f(a,x,h(g(z))),\; f(z,h(y),h(y))\}$. Veamos las iteraciones
% del algoritmo en la siguiente tabla:\begin{center}
% \begin{tabular}{ll}
% Sustituci\'on & Conjunto Discorde \\
% $\sigma_0=\varnothing$ & $D_0=\{a,z\}$  \\
% $\sigma_1=[z:=a]$ & $D_1=\{x,h(y)\}$\\
% $\sigma_2=[x:=h(y),z:=a]$ & $D_2=\{g(a),y\}$\\
% $\sigma_3=[x:=h(g(a)),y:=g(a),z:=a]$ & \\
% \end{tabular}\end{center}
% $\sigma_3$ es un umg de $W$.  
% }
% Veamos por \'ultimo un ejemplo con tres expresiones.
% \ejem{Sea $W=\{Q(x,a,z),Q(y,a,h(y)),Q(x,a,h(g(b)))\}$. Las iteraciones son
% las siguientes:\begin{center}
% \begin{tabular}{ll}
% Sustituci\'on & Conjunto Discorde \\
% $\sigma_0=\varnothing$ & $D_0=\{x,y\}$  \\
% $\sigma_1=[y:=x]$ & $D_1=\{z,h(x)\}$\\
% $\sigma_2=[y:=x, z:=h(x)]$ & $D_2=\{x,g(b)\}$\\
% $\sigma_3=[x:=g(b),y:=g(b),z:=h(g(b))]$ & \\
% \end{tabular}\end{center}
% $\sigma_3$ es un umg de $W$.Este ejemplo tambi\'en muestra el caracter no
% determinista del algoritmo, ya que la sustituci\'on $x/y$ tambi\'en pudo
% ser usada para $\sigma_1$. 
% }

% Para terminar enunciamos el teorema de correctud total del algoritmo $AU$.

% \teo{[Correctud total de $AU$]
% Dado un conjunto finito de expresiones $W$, el algoritmo $AU$ termina,
% dando como resultado el mensaje '$W$ no es unificable', en el caso en que
% $W$ no sea unificable, y un umg de $W$ en el caso en que $W$ sea unificable. 
% }


Para el caso general en que $W=\{t_1,\ldots,t_n\}$ el unificador se obtiene 
aplicando recursivamente el caso para dos términos como sigue:
\bi
\item Hallar $\mu$ \textbf{umg} de $t_1,t_2$.
\item Hallar $\nu$ unificador de $\{t_2\mu,t_3\mu_1,\ldots,t_n\mu\}$.
\item El unificador de $W$ es la composición $\nu\mu$.
\ei

\subsection{El algoritmo de unificación de Martelli-Montanari}

A continuación se describe el algoritmo de unificación de 
Martelli-Montanari~\footnote{``An efficient unification algorithm'', ACM Trans. 
Prog. Lang. and Systems vol. 4, 1982, p.p. 258-282}:

\bi
\item[] \textbf{Entrada}: un conjunto de ecuaciones 
$\{s_1=r_1,\ldots,s_k=r_k\}$ tales que 
se desea unificar simult\'aneamente $s_i$ con $r_i$ para $1\leq i\leq k$.
\item[] \textbf{Salida}: un unificador más general $\mu$ tal que 
$s_i\mu=r_i\mu$ para toda $1\leq i\leq n$.
\ei

Para unificar un conjunto de términos:
\bi
\item se colocan como ecuaciones las expresiones a unificar
\item escoger una de ellas de manera \textit{no determin\'istica} que tenga la 
forma de alguna de las siguientes opciones 
y realizar la acci\'on correspondiente:
\bc
\begin{tabular}{lc|c|c}
& Nombre de la regla &  $t_1 = t_2 $ &  Acci\'on\\\hline
{\sc [Desc]} & Descomposición &  $fs_1\dots s_n = ft_1\dots t_n$ 
  & \small{sustituir por  $\{s_i = t_i\}$} \\
{\sc [DFalla]} & Desc. fallida & 
  $fs_1\dots s_n=gt_1\dots t_n\text{ donde } g\neq f$  
  & \small{falla}\\
{\sc [Elim]} & Eliminación &  $x = x$ & \small{eliminar} \\
{\sc [Swap]} & Intercambio & 
  $t = x \text{ donde } t \text{ no es una variable }$
  & \small{sustituir por $x=t$}\\
%x = t$ donde $x$ no figura en $t$ pero  &  aplicar la sustituci\'on $[x:=t]$\\
{\sc [Sust]} & Sustitución &  $x = t \text{ donde } x \text{ no figura en } t$  
  &  \small{eliminar $x=t$ y }\\ 
  & & & \small{aplicar la sustituci\'on $[x:=t]$}\\ %x$ aparece en las dem\'as 
  & & & \small{a las ecuaciones restantes }\\
{\sc [SFalla]} & Sust. fallida & 
  $x = t \text{ donde } x \text{ figura en } t \text{ y } x\neq t$ 
  & \small{falla}
\end{tabular}
\ec

\item el algoritmo termina cuando no se puede llevar a cabo alguna acci\'on 
o cuando falla. \\
En caso de \'exito se obtiene el conjunto vacío de ecuaciones 
y el unificador m\'as general se obtiene al componer todas las sustituciones 
usadas por la regla de sustitución en el orden en que se generaron.
\ei

\paragraph{Observaciones:} El caso en que se deba unificar a dos constantes 
iguales $a$, se genera la ecuación $a=a$ que por la regla de descomposición se 
sustituye por el conjunto vacío de ecuaciones dado que constantes se 
consideran funciones sin argumentos, as\'i cualquier ecuación de la forma 
$a=a$ se elimina. 
En el caso de una ecuación~$a=b$ (con dos constantes distintas) falla por la 
regla de descomposición fallida.

\begin{Ejemplo}
  Sean $f^{(2)},g^{(1)},h^{(2)}$ y $W=\{fgxhxu,\;fzhfyyz\}$. Mostramos el 
proceso de ejecución del algoritmo de manera similar al razonamiento de 
verificación de correctud de argumentos por interpretaciones.\\
\begin{tabular}{rcl}
 1.& $\{fgxhxu=fzhfyyz\}$ &  Entrada \\
 2.& $\{gx=z,\;hxu=hfyyz\}$ & {\sc Desc},1 \\
 3.& $\{z=gx,\;hxu=hfyyz\}$ & {\sc Swap},2 \\
 4.& $\{hxu=hfyygx\}$ & {\sc Sust},3, $[z:=gx]$ \\
 5.& $\{\;x=fyy,\;u=gx\}$ & {\sc Desc},4  \\
 6.& $\{u=gfyy\}$& {\sc Desc},5, $[x:=fyy]$  \\
 7.& $\vacio$ &{\sc Desc},6, $[u:=gfyy]$ 
\end{tabular}

El unificador se obtiene al componer las sustituciones utilizadas desde el 
inicio:
\[
\ba{rll}
\mu & = & [z:=gx][x:=fyy][u:=gfyy]\\
    & = & [z:=gfyy,x:=fyy][u:=gfyy]\\
    & = & [z:=gfyy,x:=fyy,u:=gfyy]
\ea
\]
\end{Ejemplo}

\noindent Veamos ahora un ejemplo de un conjunto no unificable:
\begin{Ejemplo}
  Sean $f^{(3)},g^{(1)}$ y $W=\{fxyx,\;fygxx\}$.
  
\begin{tabular}{rcl}
 1.& $\{fxyx=fygxx\}$ & Entrada\\ 
 2.& $\{x=y,\;y=gx,\;x=x\}$ & {\sc Desc},1 \\
 3.& $\{x=y,\;y=gx\}$  & {\sc Elim},2\\
 4.& $\{y=gy\}$ &{\sc Sust},3, $[x:=y]$ \\
 5.& $\verb! X!$ &{\sc SFalla},4 
\end{tabular}
\end{Ejemplo}

Para terminar enunciamos el teorema de correctud total del algoritmo.

\teo{[Correctud total del algoritmo de Martelli-Montanari]
Dado un conjunto finito de expresiones~$W$, el algoritmo MM termina,
dando como resultado el mensaje ``$W$ no es unificable'', en el caso en que
$W$ no sea unificable, y un unificador más general~$\mu$ de $W$ en el caso en 
que $W$ sea unificable. 
}


\subsection{Unificación de literales}

El algoritmo de unificación puede generalizarse para unificar literales de 
manera sencilla simplemente adaptando las reglas de descomposición y 
descomposición fallida considerando los casos para símbolos de predicado de 
manera análoga a los símbolos de función.

Como ejemplo se deja al lector unificar el siguiente conjunto

\[
W=\{Qxaz,Qyahy,Qxahgb\}
\]


\section{Implementación del algoritmo de unificación}

La implementación del algoritmo se sirve de la implementación previa de 
términos y sustituciones dada en la nota de clase 4. 
Implementamos primero el caso particular en que la entrada del algoritmo es una 
única ecuación $t_1=t_2$ mediante la función \verb! unifica! descrita abajo.\\
Las funciones a implementar son:
\bi
%\item \verb! subId:: Subst!, la sustitución identidad o vacía
\item \verb! simpSus:: Subst -> Subst! que dada una sustitución elimina de ella 
los pares con componentes iguales correspondientes a sustituciones de la forma 
$x:=x$
\item \verb! compSus:: Subst -> Subst ->Subst! que dadas dos sustituciones 
devuelve su composición
\item \verb! unifica :: Term -> Term -> [Subst]! que dados dos términos 
devuelva 
una lista de sustituciones de tal forma que
\bi
\item Si $t_1,t_2$ no son unificables la lista es vacía
\item Si sí lo son la lista contiene como único elemento al unificador 
correspondiente.
\ei
Para el caso en que $t_1,t_2 $  sean términos funcionales se necesitará una 
función \\
\verb! unificaListas :: [Term] -> [Term] -> [Subst]!
que unifique dos listas de términos de la misma longitud componente a 
componente. Es decir, 
\verb!unificaListas [s1,..,sk] [r1,..,rk]! devuelve la lista $[\mu]$ con un 
unificador $\mu$ tal que $s_j\mu=r_j\mu$ para toda $1\leq j\leq k$. En 
cualquier otro caso esta función devuelve la lista vacía. 
\item \verb! unificaConj :: [Term] -> [Subst]! que implementa el caso general 
para unificar un conjunto (lista) $W=\{t_1,\ldots,t_n\}$.%  se implementa con 
una función 
\item \verb! unificaLit :: Form -> Form -> [Subst]! que unifique dos literales.
% que devuelve la lista unitaria con un unificador de $W$ o bien la lista 
% vacía. La idea es iterar la función \verb! unifica} adecuadamente.

% \item Tipo para variables: usamos simplemente un alias para términos por 
% cuestión pragmática
% \bi
% \item[] \verb! type Var = Term!
% \ei
%  teniendo en cuenta que en caso de que se espere algo de tipo \verb! 
% Variable!
% debe ser un término de la forma \verb! V x!
% \item Tipo para sustituciones: son listas de pares de variables y términos
% \bi
% \item[] \verb! type Sust = [(Var,Term)]!
% \ei
\ei
\end{document}
