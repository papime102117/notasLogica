	\documentclass[11pt,letterpaper]{article}
\usepackage{../packageslc}
\usepackage{../optionslc}

%%%% Macros para las notas de lenguajes de programacion


%%%% math

\newcommand{\vphi}{\varphi}
\newcommand{\vp}{\varphi}

% \newcommand{\dn}{\mathsf{DN}}
% \newcommand{\dnC}{\mathsf{DN_C}}
% \newcommand{\dnM}{\mathsf{DN_M}}
% \newcommand{\dnp}{\mathsf{DN_p}}
% \newcommand{\dnm}{\mathsf{DN_p^M}}
% \newcommand{\dnc}{\mathsf{DN_p^C}}

%\newcommand{\case}{\mathsf{case}}
%\renewcommand\labelitemi{$\circ$}

\newcommand{\imp}{\rightarrow}
\newcommand{\Imp}{\Rightarrow}
\renewcommand{\iff}{\leftrightarrow}
\newcommand{\Iff}{\Leftrightarrow}
\newcommand{\G}{\Gamma}
\newcommand{\D}{\Delta}


\newcommand{\De}{\mathcal{D}}
\newcommand{\F}{\mathcal{F}}
\newcommand{\Ge}{\mathcal{G}}
\newcommand{\Pe}{\mathcal{P}}
\newcommand{\I}{\mathcal{I}}
\newcommand{\C}{\mathcal{C}}
\newcommand{\K}{\mathcal{K}}
\renewcommand{\L}{\mathcal{L}}
\newcommand{\M}{\mathcal{M}}
\newcommand{\Nc}{\mathcal{N}}
%\newcommand{\E}{\mathcal{E}}
%\newcommand{\R}{\mathcal{R}}
%\newcommand{\Q}{\mathcal{Q}}
\newcommand{\Sc}{\mathcal{S}}
\newcommand{\Te}{\mathcal{T}}
\newcommand{\W}{\mathcal{W}}

\newcommand{\Db}{\mathbb{D}}
\newcommand{\Fb}{\mathbb{F}}
\newcommand{\Kb}{\mathbb{K}}
\newcommand{\Eb}{\mathbb{E}}
\newcommand{\Ebs}{\mathbb{E}^\star}
\newcommand{\Ob}{\mathbb{O}}
\newcommand{\Ib}{\mathbb{I}}
\newcommand{\Rb}{\mathbb{R}}
\newcommand{\Qb}{\mathbb{Q}}
\newcommand{\Kbb}{\mathbb{K}}
\newcommand{\T}{\mathbb{\Theta}}


\newcommand{\kb}{\bbkappa}

\newcommand{\Sf}{\mathsf{\Sigma}}

\newcommand{\fa}{\forall}
\newcommand{\ex}{\exists}

\newcommand{\inc}{\subseteq}

\newcommand{\Lb}{\Lambda}
\newcommand{\Om}{\Omega}
\newcommand{\lb}{\lambda}
\newcommand{\al}{\alpha}
\newcommand{\ga}{\gamma}


\newcommand{\mg}{\mathbb{m}}

\newcommand{\cg}{\mathbb{C}}
\newcommand{\dg}{\mathbb{D}}
\newcommand{\jg}{\mathbb{J}}
\newcommand{\Ha}{\mathcal{H}}
%\newcommand{\A}{\mathcal{A}}
\newcommand{\sg}{\mathbb{S}}

\newcommand{\Bc}{\mathcal{B}}
\newcommand{\Df}{\mathfrak{D}}
\newcommand{\Dc}{\mathcal{D}}
%\newcommand{\Tc}{\mathcal{T}}
\newcommand{\Mf}{\mathfrak{M}}

\newcommand{\Sg}{\mathbb{S}}

\newcommand{\Q}{\ensuremath{\mathbb{Q}}}
\newcommand{\Z}{\ensuremath{\mathbb{Z}}}
\newcommand{\N}{\ensuremath{\mathbb{N}}}
\newcommand{\R}{\ensuremath{\mathbb{R}}}
\renewcommand{\S}{\mathbb{\Sigma}}
\newcommand{\A}{\mathcal{A}}
\newcommand{\E}{\ensuremath{\exists}}
\newcommand{\iso}{\ensuremath{\cong}}
\newcommand{\union}{\ensuremath{\cup}}
\newcommand{\morinyec}{\ensuremath{\precapprox}}

\newcommand{\nin}{\ensuremath{\notin}}
\newcommand{\tog}{\makebox[7mm][l]}
\newcommand{\toge}{\makebox[11mm][l]}
\newcommand{\toget}{\makebox[13mm][l]}
\newcommand{\togeth}{\makebox[14mm][l]}
\newcommand{\togethe}{\makebox[15mm][l]}
\newcommand{\together}{\makebox[17mm][l]}
\newcommand{\niso}{\ensuremath{\not \cong}}


\newcommand{\Mg}{\mathbb{M}}
\newcommand{\Bg}{\mathbb{B}}
\newcommand{\Lg}{\mathbb{L}}
\newcommand{\Tg}{\mathbb{T}}

\newcommand{\sketch}{\Red{{\sc sketch}}}

\newcommand{\restr}[2]{#1\!\!\boldsymbol{\restriction}\!#2}

\newcommand{\vacio}{\varnothing}
\newcommand{\done}{\ensuremath{\checkmark}}

\newcommand{\ida}{$\Rightarrow \; )$ }
\newcommand{\regr}{$\Leftarrow \; )$ }

\newcommand{\ol}[1]{\overline{#1}}

\newcommand{\Tsf}{\mathsf{T}}

\newcommand{\inds}[1]{\index[simb]{#1}}

\newcommand{\B}{\mathbb{B}}
%\newcommand{\N}{\mathbb{N}}

\newcommand{\vx}{\vec{x}}
\newcommand{\vy}{\vec{y}}
\newcommand{\vz}{\vec{z}}
\newcommand{\vt}{\vec{t}}
\newcommand{\vf}{\vec{f}}


% \newcommand{\propo}{\ensuremath{\mathsf{PROP}}}
% \newcommand{\atom}{\ensuremath{\mathsf{ATOM}}}
\newcommand{\term}{\ensuremath{\mathsf{TERM}}}
\newcommand{\form}{\mathsf{FORM}}

\newcommand{\true}{\mathop{\mathsf{true}}}

%\newcommand{\id}{\mathsf{Id}}

%\newcommand{\uc}{\mathcal{U}}
%\newcommand{\Ic}{\mathcal{I}}
%\newcommand{\pc}{\mathcal{P}}
%\newcommand{\qc}{\mathcal{Q}}
%\newcommand{\mc}{\mathcal{M}}
\newcommand{\supc}{\supseteq}
\newcommand{\limo}{\mathop{\mathpzc{Lim}}}
\newcommand{\ord}{\mathsf{OR}}

\newcommand{\pt}[1]{\langle #1 \rangle}


%%%% frames
\newcommand{\titulos}[2]{\frametitle{#1}\framesubtitle{#2}}
\newcommand{\fot}[1]{\footnote{\scriptsize{#1}}}


%%%% ambientes

\newcommand{\cb}[2]{\colorbox{#1}{#2}}

\newcommand{\bc}{\begin{center}}
\newcommand{\ec}{\end{center}}
\newcommand{\be}{\begin{enumerate}}
\newcommand{\ee}{\end{enumerate}}
\newcommand{\bi}{\begin{itemize}}
\newcommand{\ei}{\end{itemize}}
\newcommand{\beq}{\begin{equation}}
\newcommand{\eeq}{\end{equation}}
\newcommand{\beqs}{\begin{equation*}}
\newcommand{\eeqs}{\end{equation*}}
\newcommand{\ba}{\begin{array}}
\newcommand{\ea}{\end{array}}


% \newtheorem{theorem}{Teorema}
% \newcommand{\teo}[1]{\begin{theorem} #1 \end{theorem}}
% \newtheorem{proposition}{Proposici\'on}
% \newcommand{\prop}[1]{\begin{proposition} #1 \end{proposition}}
% \newtheorem{definition}{Definici\'on}
% \newcommand{\defin}[1]{\begin{definition} #1 \end{definition}}
% \newtheorem{corollary}{Corolario}
% \newcommand{\cor}[1]{\begin{corollary} #1 \end{corollary}}
% \newtheorem{lemma}{Lema}
% \newcommand{\lema}[1]{\begin{lemma} #1 \end{lemma}}
% \newcommand{\dem}[1]{\begin{proof} #1 \end{proof}}

%\renewcommand{\qed}{\qedsymbol{$\mathbf{\dashv}$}}

%\newcommand{\proof}{\hfill\\\noindent\textbf{\textit{Demostraci\'on. }}}

\newcommand{\hint}{\emph{Sugerencia: }}


\newcounter{EjempCtr}[section]
\newenvironment{enumrom}{\renewcommand{\theenumi}{\roman{enumi}}%
\renewcommand{\theenumii}{\roman{enumii}}
\renewcommand{\theenumiii}{\roman{enumiii}}
\renewcommand{\theenumiv}{\roman{enumiv}}
\begin{enumerate}}{\end{enumerate}}
\newenvironment{Ejemplo}
        {\stepcounter{EjempCtr}%
        \begin{description}\item[Ejemplo \thesection.\arabic{EjempCtr}]}%
        {\end{description}}
\newenvironment{demostr}{%
             {\em Demostración:}
                \begin{quotation}}{\end{quotation}}

   \newcommand{\beje}{\begin{Ejemplo}}
\newcommand{\eeje}{\end{Ejemplo}}


\newtheorem{eje}{Ejemplo}[section]
\newcommand{\ejem}[1]{\begin{eje}\normalfont #1 \end{eje}}

% \renewcommand\contentsname{\'Indice}
%\renewcommand\chaptername{Cap\'itulo}
% \renewcommand\indexname{\'Indice}

%%\newcommand{\qed}{\hfill$\mathbb{Qed}$}
%\newcommand{\qed}{\hfill$\mathsf{\boldsymbol{\dashv}}$}
%\renewcommand{\qed}{\hfill$\boldsymbol{\dashv}$}


\newenvironment{prueba}{\vspace{-5mm}\noindent\textbf{Demostraci\'on}\\}{
\noindent$\blacksquare$\\}

\newcommand{\Ejercicios}{\section*{Ejercicios}}

 \newenvironment{manitas}{%
      \renewcommand{\labelitemi}{\ding{44}}%
      \vspace{-0.5cm}%
      \begin{itemize}%
      \setlength{\itemsep}{0pt}\setlength{\parsep}{0pt}\setlength{\topsep}{0pt}%
      }{\end{itemize}}
\newenvironment{malitos}{%
      \renewcommand{\labelitemi}%
            {\raisebox{1.5ex}{\makebox[0.3cm][l]{\begin{rotate}{-90}%
            \ding{43}\end{rotate}}}}%
      \vspace{-0.5cm}%
      \begin{itemize}%
      \setlength{\itemsep}{0pt}\setlength{\parsep}{0pt}\setlength{\topsep}{0pt}%
      }{\end{itemize}}
\newenvironment{ejercs}{
     \renewcommand{\labelenumi}{\thesection.\theenumi.-}
     \renewcommand{\labelenumii}{\theenumii)}
     \begin{enumerate}}
     {\end{enumerate}}

   \newcommand{\bej}{\begin{ejercs}}
\newcommand{\eej}{\end{ejercs}}


%\newenvironment{leterize}{%
%        \renewcommand{\theenumi}{\alph{enumi}}
%        \begin{enumerate}}{\end{enumerate}}

%\newenvironment{manitas}{%
%      \renewcommand{\labelitemi}{\ding{44}}%
%      \vspace{-0.5cm}%
%      \begin{itemize}%
%      
% \setlength{\itemsep}{0pt}\setlength{\parsep}{0pt}\setlength{\topsep}{0pt}%
%      }{\end{itemize}}



%%=============================================================================

\def\stackunder#1#2{\mathrel{\mathop{#2}\limits_{#1}}}


%%%% notas

\newcommand{\doubt}{\Red{{\LARGE {\sf ??}}}}

\newcommand{\coment}[1]{\hfill\\ \Big[{\bf Comentario Privado:} #1\Big]}
\newcommand{\preg}[1]{\hfill\\ \BrickRed{{\bf Pregunta:} #1}}
\newcommand{\conjet}[1]{\hfill\\ \OliveGreen{{\bf Conjecura:} #1}}

\newcommand{\pendiente}{\BrickRed{{\sc Pendiente}}}
\newcommand{\verifpendiente}{\BrickRed{{\sc Verificación pendiente}}}


%--------------------------------------------------------------------------

\DeclareMathAlphabet{\mathpzc}{OT1}{pzc}{m}{it}



\title{Lógica de Primer Orden: Conceptos Sintácticos Importantes \\ 
L\'ogica Computacional 2018-2, Nota de clase 5}
\author{Favio Ezequiel Miranda Perea\and Araceli Liliana Reyes Cabello\and
Lourdes Del Carmen Gonz\'alez Huesca \and Pilar Selene Linares Ar\'evalo}
\date{Facultad de Ciencias UNAM \\ 9 de marzo de 2018 \\
Material desarrollado bajo el proyecto UNAM-PAPIME PE102117}


\begin{document}

\maketitle

Al tener cuantificadores en la l\'ogica de primer orden se deben revisar con 
detenimiento algunos conceptos sint\'acticos como la importancia de un 
cuantificador en una f\'ormula as\'i como los diferentes casos a estudiar al 
tratar de realizar una sustituci\'on. 

\section{Ligado y alcance}

En lógica de predicados los cuantificadores~$\fa x$ y~$\ex x$ son ejemplos del
fenómeno de \textit{ligado} que también surge en la mayoría de lenguajes de
programación.
La idea general es que ciertas presencias de variables son presencias
\textit{ligadas} o simplemente \textit{ligas}, cada una de las cuales se asocia 
con una expresión llamada su \textit{alcance}.

% A continuación damos las definiciones para el caso de los cuantificadores.

\defin{Dada una cuantificaci\'on $\fa x\vp$ o $\ex x\vp$, la presencia de~$x$
  en~$\fa x$ o~$\ex x$ es la variable (artificial) que liga el cuantificador
  correspondiente, mientras que la f\'ormula $\vp$ se llama el alcance, ámbito 
  o radio de acción de este cuantificador.
}

\defin{Una presencia de la variable~$x$ en la f\'ormula~$\vp$ est\'a
ligada o acotada si es la variable que liga a un cuantificador de~$\vp$ o si
figura en el alcance de un cuantificador $\fa x$ o $\ex x$ de~$\vp$.\\ Si una 
presencia de la variable~$x$ en la f\'ormula~$\vp$ no est\'a ligada, decimos 
que est\'a libre en~$\vp$. 
}

\ejem{Sea $\vp=\fa x\ex z\big(Q(y,z)\lor R(z,x,y)\big)\land P(z,x)$. 
\bi
 \item El alcance del cuantificador $\fa$ es la f\'ormula 
  $\ex z (Q(y,z)\lor R(z,x,y))$.
 \item El alcance del cuantificador $\ex$ es la f\'ormula 
  $Q(y,z)\lor R(z,x,y)$. 
 \item En $\vp$ hay tres presencias de~$x$, las dos primeras ligadas y la 
  \'ultima libre.
 \item Las presencias de $z$ son cuatro, ligadas las tres primeras
  y libre la última.
 \item Finalmente las dos presencias de $y$ son libres.  
\ei
}


\defin{Sea $\vp$ una f\'ormula. El conjunto de variables libres de~$\vp$, se
denota~$FV(\vp)$.\\
Es decir, $FV(\vp)=\{x\in\Var \mid x \text{ figura libre en }\vp\}$. \\
La notaci\'on~$\vp(x_1,\ldots,x_n)$ quiere decir 
que~$\{x_1,\ldots,x_n\}\inc FV(\vp)$.
}
Obsérvese que una misma variable~$x$ puede figurar tanto libre como ligada en
una fórmula.

\defin{Una f\'ormula~$\vp$ es \emph{cerrada} si no tiene variables
libres, es decir, si $FV(\vp)=\vacio$. Una f\'ormula cerrada tambi\'en se
conoce como enunciado o sentencia. 
% El conjunto de~$\Si$-f\'ormulas cerradas se denota con $FORM_0(\Si)$.
}

\defin{Sea $\vp(x_1,\ldots,x_n)$ una f\'ormula 
con~$FV(\vp)=\{x_1,\ldots,x_n\}$. 
La cerradura universal de~$\vp$, denotada $\fa\vp$, es la 
f\'ormula~$\fa x_1\ldots\fa x_n\;\vp$. 
La cerradura existencial de $\vp$, denotada $\ex\vp$ es la 
f\'ormula~$\ex x_1\ldots\ex x_n\;\vp$.
}
Obs\'ervese que una cerradura se obtiene cuantificando todas las variables
libres de una f\'ormula. 


\subsection{Implementación}
\noindent Se deja como ejercicio implementar todos los conceptos de esta 
sección mediante las siguientes funciones:
\bi
%\item Alcance: \verb-Alcance :: Form -> Form}
\item Lista de variables libres: \verb=fv :: Form -> [Nombre]=
\item Lista de variables ligadas: \verb=bv :: Form -> [Nombre]=
\item Cerradura universal: \verb=aCl :: Form -> Form=
\item Cerradura existencial: \verb=eCl :: Form -> Form=
\item Alcances en una fórmula : \verb=alcF :: Form -> [(Form,Form)]=, al tomar 
como entrada $P$ esta función devuelve una lista de pares de la forma $(A,B)$ 
donde $A$ es una cuantificación que es subfórmula de $P$ y $B$ es el alcance de 
$A$.
\ei


\section{Sustitución}
La noción de sustitución en la lógica de predicados es más complicada que en
la lógica proposicional debido a la existencia de términos y de variables
ligadas. A diferencia de lo que sucede en lógica proposicional, donde una 
sustitución es sólo una operación textual sobre las fórmulas, en la lógica de 
predicados las sustituciones son operaciones sobre los términos y sobre las 
fórmulas y en este último caso deben respetar el ligado de variables, por lo 
tanto \textbf{no} son operaciones textuales.
% tienen interés por sí solas, como lo veremos al tratar el tema
% de unificación en programación lógica, de manera que empezamos por definir
% que es una sustitución.

\defin{Una sustitución en un lenguaje de predicados $\L$ es una tupla
  de variables y términos denotada como
  $[x_1,x_2,\ldots,x_n:=t_1,\ldots,x_n]$
  donde 
  \bi
  \item $x_1,\ldots,x_n$ son variables distintas.
  \item $t_1,\ldots,t_n$ son términos de $\L$.
  \item $x_i\neq t_i$ para cada $1\leq i\leq n$
  \ei
  Por lo general denotaremos a una sustitución con $[\vx:=\vt\,]$.
}
% \defin{Una sustitución en un lenguaje de predicados $\L$ es una función 
% $\sigma:\Var\imp\term_\L$ con la propiedad de que $\sigma(x_i)=x_i$ para casi
% toda $i\in\N$. Es decir $\sigma(x_i)\neq x_i$ únicamente para un número
% finito de variables. Esto se conoce en matemáticas en general como la
% propiedad de que una función dada tenga soporte finito, siendo el soporte de 
% $\sigma$ el conjunto de variables $x_i$ con $\sigma(x_i)\neq x_i$.
% }
% Obsérvese que una sustitución $\sigma$ queda completamente determinada 
% mediante su soporte (finito), si éste es $x_1,\ldots,x_n$ con 
% $\sigma(x_i)=t_i$ entonces la sustitución se denotará con
% $\sigma = [x_1,x_2,\ldots,x_n:=t_1,t_2,\ldots,t_n] $
% o de manera más breve $\sigma=[\vx:=\vt\,]$
% La aplicación de una sustitución $\sigma$ a un término $t$ puede verse
% formalmente como una extensión $\sigma^\star:\term_\L\imp\term_\L$ de
% $\sigma$, que se denota igual que $\sigma$ sobrecargando la operación.


\subsection{Implementación}
Representaremos la sustitución $[\vx:=\vt\,]$ mediante listas de pares
$ [(x_1,t_1),\ldots,(x_n,t_n)] $
Para esto definimos el tipo de sustituciones:
\begin{verbatim}
type Subst = [(Nombre,Term)]
\end{verbatim}
Obsérvese que esto no garantiza que si \verb=s:Subst= entonces \verb-s- sea una 
sustitución legal. Por lo tanto debe escribirse una función 
\verb=verifSus:: Subst -> Bool= que verifique si una lista dada es una 
sustitución.


\subsection{Aplicación de una sustitución a un término}

La aplicación de una sustitución~$[\vx:=\vt]$ a un término~$r$,
denotada~$r[\vx:=\vt]$, se define como el término obtenido al reemplazar 
\textbf{simult\'aneamente} todas las presencias de~$x_i$ en~$r$ por~$t_i$. 
Este proceso de define  recursivamente como sigue:
\[
\ba{rll}
x_i[\vx:=\vt\,] & = & t_i \qquad  1\leq i\leq n \\ \\ 
z[\vx:=\vt\,] & = & z \qquad  \text{ si } z\neq x_i\;1\leq i\leq n  \\ \\
c[\vx:=\vt\,] & = & c \qquad 
  \text{ si } c\in\C\text{, es decir, }c\text{ constante } \\ \\
f(t_1,\ldots,t_m)[\vx:=\vt\,] & = & 
f(t_1[\vx:=\vt\,],\ldots,t_m[\vx:=\vt\,]) \qquad \text{ con }f^{(m)}\in\F.
\ea
\]

Obsérvese entonces que la aplicación de una sustitución a término es
simplemente una sustitución textual tal y como sucede en lógica de 
proposiciones.

\subsection{Implementación}
La aplicación de una sustitución~$[\vec{x}:=\vec{t}]$ a un término~$t$ se 
implementa mediante una función
\begin{verbatim}
apsubT :: Term -> Subst -> Term
\end{verbatim}
tal que si \verb-s- implementa a la sustitución~$[\vx:=\vt\,]$ entonces 
\verb-apsubT r s- debe devolver el término~$r[\vx:=\vt\,]$.
% Las siguientes propiedades de las sustituciones serán de
% utilidad más adelante.
% \prop{Sean $\sigma=[x_1,\ldots,x_n:=t_1,\ldots,t_n]$, 
% $\tau=[y_1,\ldots,y_m:=s_1,\ldots,s_m]$ sustituciones.
% La composición $\sigma\tau$ es una sustitución definida por:
% \[
% \sigma\tau =
% %[z_1:=t_1\tau,\ldots,z_k:=t_k\tau,\,w_1:=w_1\tau,\ldots,w_l:=w_l\tau]
% [z_1,\ldots,z_k,w_1,\ldots,w_l  :=  
% t_1\tau,\ldots,t_k\tau,\,r_1,\ldots,r_l]
% \]
% donde $k\leq n,l\leq m$, las $z_i$ son exactamente aquellas $x_p$ tales que
% $x_p\neq t_p\tau$ y las $w_j$ son aquellas $y_q$ no incluidas en 
% $x_1,\ldots,x_n$. }

% \prop{La composición de sustituciones es asociativa, es decir si
% $\sigma,\rho,\tau$ son sustituciones entonces 
% $(\sigma\rho)\tau=\sigma(\rho\tau)$.
% }

% \ejem{Sean  $\sigma=[x,y:=f(y),w],\;\rho=[x,z:=g(w),b],
% \;\tau=[y,w,v:=b,f(c),w]$. Entonces
% \[
% \ba{rll}
% \sigma\rho & = & [x,y,z:=f(y),w,b]. \\ \\
% (\sigma\rho)\tau & = & [x,y,z,w,v:=f(b),f(c),b,f(c),w].
% \ea
% \]
% }

\subsection{Sustitución en Fórmulas}

La aplicación de sustituciones a fórmulas, necesita de ciertos cuidados debido 
a la presencia de variables ligadas mediante cuantificadores. La aplicación de 
una sustitución textual a una fórmula puede llevar a situaciones problemáticas
con respecto a la sintaxis y a la semántica de la lógica. En particular deben 
evitarse los siguientes problemas:

\paragraph{Sustitución de variables ligadas}
  La definición de sustitución textual 
  \[
  (\fa y\vp)[\vx:=\vt\,] =_{def} \fa (y[\vx:=\vt\,])(\vp[\vx:=\vt\,])
  \]
  obliga a sustituir todas las variables de una fórmula, pero tal definición 
  genera expresiones que ni siquiera son fórmulas, por ejemplo tendríamos que
  \[
  \big(\fa x P(y,f(x))\big)[x,y:=g(y),z] = \fa g(y) P\big(z,f(g(y))\big)
  \]
  La expresión de la derecha no es una fórmula puesto que la cuantificación 
  sobre términos que no sean variables no está permitida. 
  Lo mismo sucederá al definir la sustitución de esta manera para una fórmula
  existencial. 
  En conclusión, la sustitución en fórmulas \textbf{NO} debe ser una 
  sustitución textual puesto que las presencias ligadas de variables no pueden 
  sustituirse.\\ \\
  Una solución inmediata al problema anterior consiste en definir la sustitución
  sólamente sobre variables libres, como sigue 
% dada una sustitución $[\vx:=\vt]$ definir~$x$ como la sustitución que 
% coincide con $\sigma$ en todo su soporte, excepto en $x$, y donde 
% $\sigma_x(x)=x$. De esta manera definiríamos 
  \[
  (\fa x\vp)[\vx:=\vt\,] =_{def} \fa x(\vp[\vx:=\vt\;]_x)
  \]
  donde $[\vx:=\vt\,]_x$ denota a la sustitución que elimina al par $x:=t$
  de $[\vx:=\vt\,]$.  \\\\  
  Por ejemplo $[z,y,x:=y,f(x),c]_x=[z,y:=y,f(x)]$. \\
  Y así la aplicación de sustitución para la fórmula del primer ejemplo se 
  ejecuta como sigue:
  \[
  \ba{rll}
  \big(\fa x P(y,f(x))\big)[x,y:=g(y),z] & = &  
  \fa x \big(P(y,f(x)[x,y:=g(y),z]_x\big) \\ & = & 
  \fa x \big(P(y,f(x))[y:=z]\big) \\ & = &
  \fa x P\big(z,f(x)\big)
  \ea
  \]
  De esta manera se garantiza que el resultado de aplicar una sustitución a una
  cuantificación siempre será una fórmula. Análogamente podemos definir la
  sustitución para una fórmula existencial. Pasemos ahora a ver el segundo
  problema a evitar.

 \paragraph{Captura de variables libres}
  Si bien la solución dada al problema anterior es suficiente desde el punto 
  de vista sintáctico, no lo es al entrar en juego la semántica o el 
  significado de las fórmulas.\\ \\
  Consid\'erese la f\'ormula~$\fa x\vp$, la semántica intuitiva nos dice que 
  esta fórmula será cierta siempre y cuando~$\vp(x)$ sea cierta para cualquier 
  valor posible de~$x$, de manera que nos gustar\'ia poder obtener, a partir de 
  la de verdad~$\fa x\vp$, la verdad de cualquier aplicación de 
  sustitución~$\vp[x:=t]$. \\
  Veamos ahora que que sucede si consideramos la f\'ormula 
  \[\fa x\big(\ex   y(\,x\neq y)\big)\]  
  Est\'a f\'ormula expresa el hecho de que para cualquier individuo existe otro 
  distinto de \'el y es cierta si el universo tiene al menos dos elementos. \\
  Ahora bien, al momento de calcular el caso particular 
  $\big(\ex y(x\neq y)\big)[x:=y]$ con la definición anterior de aplicación de 
  sustituciones obtenemos:
  \[ 
  \big(\ex y\,(x\neq y)\big)[x:=y] \;= \;
  \ex y\big(\,(x\neq y)[x:=y]_y\big) \; =  \;
  \ex y\big(\,(x\neq y)[x:=y]\big) \; =\;
  \ex y\,(y\neq y)
  \]  
  Pero en esta última fórmula se ha cambiado el sentido: se está diciendo que 
  hay un objeto~$y$ que es distinto de si mismo lo cual es absurdo. \\
  De manera que con la definición dada no podemos garantizar la verdad de un 
  caso particular de~$\vp$ aun suponiendo la verdad de la cuantificación 
  universal~$\fa x\vp$.\\\\
  ?` Qu\'e es lo que anda mal? obs\'ervese que la presencia libre de~$x$ en la 
  f\'ormula~$\ex y(x\neq y)$ se ligó al aplicar la sustituci\'on~$[x:=y]$. 
  Así que una variable libre que representa a un objeto particular se sustituyó 
  por una variable que representa al objeto cuya existencia se está asegurando. 
  Esta clase de sustituciones son inadmisibles, las posiciones que corresponden 
  a una presencia libre de una variable representan a objetos particulares y el
  permitir que se liguen después de una sustitución modificará el significado 
  intensional de la fórmula. \\\\
  Existen dos maneras de solucionar el problema, la primera es aceptando la 
  definición dada arriba mediante el uso de sustituciones de la forma
  $[\vx:=\vt\,]_x$ pero prohibiendo la aplicación de
  sustituciones que liguen posiciones libres de variables. Para esto
  debe darse una una definición sustitución admisible para una fórmula, este 
  método es el más usado en textos de lógica matemática.\\
  Nosotros preferimos el segundo método, utilizado generalmente en la teoría de 
  lenguajes de programación: \textit{la aplicación de una sustitución a una 
  fórmula se  define renombrando variables ligadas de manera que siempre 
  podremos obtener  una sustitución admisible.}
  
\newpage

Despu\'es de analizar los problemas anteriores, podemos dar una definición 
correcta para la sustitución:
\defin{La aplicación de una sustitución~$[\vx:=\vt]$ a una fórmula~$\vp$,
denotada~$\vp[\vx:=\vt]$ se define como la fórmula obtenida al reemplazar 
\textit{simultáneamente} todas las presencias libres de~$x_i$ en~$\vp$ 
por~$t_i$, verificando que este proceso no capture posiciones de variables 
libres.
}

La aplicación de una sustitución a una fórmula $\vp[\vx:=\vt\,]$ se define  
\textbf{recursivamente} como sigue
% Sean $\vp$ una fórmula y $[\vx:=\vt\;]$ una sustitución. Definimos 
% recursivamente la aplicación de $[\vx:=\vt\,]$ a $\vp$, 
% denotada $\vp[\vx:=\vt\;]$ como sigue:
\[
 \ba{rll}
  \bot[\vx:=\vt\;] & = & \bot \\
  \top[\vx:=\vt\;] & = & \top \\
  P(t_1,\ldots,t_m)[\vx:=\vt\;] & = &
    P\big(t_1[\vx:=\vt\;], \ldots, t_m[\vx:=\vt\;]\big) \\ 
  (t_1 = t_2)[\vx:=\vt\;] & = & t_1[\vx:=\vt\;]=t_2[\vx:=\vt\;]\\ \\
  (\neg \vp)[\vx:=\vt\;] & = & \neg \big(\vp[\vx:=\vt\;]\big)\\
  (\vp\land\psi)[\vx:=\vt\;] & = &
  \big(\vp[\vx:=\vt\;]\land\psi[\vx:=\vt\;]\big)\\
  (\vp\lor\psi)[\vx:=\vt\;] & = &
  \big(\vp[\vx:=\vt\;]\lor\psi[\vx:=\vt\;]\big)\\
  (\vp\imp\psi)[\vx:=\vt\;] & = &
  \big(\vp[\vx:=\vt\;]\imp\psi[\vx:=\vt\;]\big)\\
  (\vp\iff\psi)[\vx:=\vt\;] & = &
  \big(\vp[\vx:=\vt\;]\iff\psi[\vx:=\vt\;]\big)\\ \\
  (\fa y\vp)[\vx:=\vt\;] & = & \fa y\big(\vp[\vx:=\vt\;]\big)\; 
\text{ \textbf{si} } y\notin
  \vx\cup Var(\vt) \\
(\ex y\vp)[\vx:=\vt\;] & = & \ex y\big(\vp[\vx:=\vt\;]\big)\; \text{ 
\textbf{si} } y\notin  \vx\cup Var(\vt) \\
\ea
\]

% Análogamente a la sustitución en términos, la aplicación de una
% sustitución $\sigma$ a una fórmula  $\vp$ puede verse
% formalmente como una extensión $\sigma^\#:\form_\L\imp\form_\L$ de
% $\sigma$, denotada igual que $\sigma$, sobrecargando nuevamente la
% operación $\sigma$.\\

Mencionamos ahora algunas propiedades de la sustitución que pueden
verificarse mediante inducción sobre las fórmulas.

\prop{La operación de sustitución tiene las siguientes propiedades, 
  considerando~$x\notin \vx$.
  \bi
   \item Si $x\notin FV(\vp)$ entonces 
    $\vp[\vec{x},x:=\vec{t},r\,] = \vp[\vec{x}:=\vec{t}\,]$.
   \item Si $\vec{x}\cap FV(\vp)=\vacio$ entonces $\vp[\vx:=\vt\,]=\vp$.
   \item $FV(\vp[\vx:=\vt\,])\inc (FV(\vp)-\vx)\cup Var(\vt)$.
   \item Si $x\notin \vx\cup Var(\vt\,)$ entonces $x\in FV(\vp)$
    si y sólo si $x\in FV(\vp[\vx:=\vt\,])$.
   \item Si $x\notin \vx\cup Var(\vt\,)$ entonces            
    $\vp[x:=t][\vx:=\vt\,]=\vp\big[\vx:=\vt\,\big]\big[x:=t[\vx:=\vt\,]\big]$.
   \item Si $x\notin \vx\cup Var(\vt\,)$ entonces 
    $\vp[\vx,x := \vt,t]=\vp[\vx:=\vt\,][x:=t]$.
  \ei
}

\subsection{Implementación}
La aplicación de una sustitución $[\vec{x}:=\vec{t}]$ a una fórmula $A$ se 
implementa mediante una función
\begin{verbatim}
apsubF :: Form -> Subst -> Form
\end{verbatim}
donde si \verb-s- implementa a la sustitución~$[\vx:=\vt\,]$ entonces 
\verb-apsubF A s- debe devolver la fórmula~$A[\vx:=\vt\,]$. 

\vspace*{10pt}

\noindent Hasta este momento debemos hacer la siguiente simplificación en la 
implementación: \\
en el caso de una sustitución donde la fórmula tiene como conectivo 
principal una cuantificación, primero se debe verificar la condición de las 
variables en la sustitución: \textit{las variables ligadas \textbf{no} 
aparacen en ninguno de los términos de la sustitución.}\\
De no cumplirse convenimos en devolver la fórmula de entrada tal cual, por 
ejemplo el resultado de $(\fa x P(x,y))[y:=f(x)]$ debe ser~$\fa x P(x,y)$.

% \begin{verbatim}
% apsubF fu@(All x p) s = if not (elem x (dom s)++(map VarT (ran s))) 
%                           then  All x (apsubF p s) 
%                            else fu
% \end{verbatim}


\subsection{La relación de $\al$-equivalencia}

La definición de sustitución en fórmulas cuenta con una restricción aparente
en el caso de los cuantificadores, por ejemplo, la sustitución
\[
\fa x (Q(x)\to R(z,x))[z:=f(x)]
\] 
no está definida, puesto que~$x$ figura en~$f(x)$ es decir~$x\in Var(f(x))$, 
con lo que no se cumple la condición necesaria para aplicar la sustitución, a 
saber que las variables en la sustitución son ajenas a las de la fórmula.
Por lo tanto, la aplicación de cualquier sustitución a una fórmula hasta 
ahora es una función parcial.

\vspace*{10pt}

Esta aparente restricción desaparece al notar que los nombres de las 
variables ligadas no importan: por ejemplo, las fórmulas $\fa x P(x)$ y 
$\fa y P(y)$ significan exactamente lo mismo, a saber que todos los 
elementos del universo dado cumplen la propiedad~$P$.\\
% Por ahora podemos argumentar intuitivamente que el
% nombre de las variables ligadas no importa, pues no denotan objetos en
% particular sino que hacen una gene\-ralización existencia o universal. La
% fórmula $\fa xP(x)$ representa el hecho de que la propiedad $P$ se cumple
% para todos los objetos, propiedad representada igualmente por $\fa yP(y)$.\\

Por lo tanto convenimos en identificar fórmulas que sólo difieren en sus
variables ligadas, esto se hace formalmente mediante la llamada relación de
$\al$-equivalencia definida como sigue:
\defin{Decimos que dos fórmulas~$\vp_1,\;\vp_2$ son $\al$-equivalentes lo cual
escribimos $\vp_1\sim_\al \vp_2$ si y sólo si $\vp_1$ y $\vp_2$ difieren a lo 
más en los nombres de sus variables ligadas.
}

\ejem{Las siguientes expresiones son $\al$-equivalentes.
\[
\fa x P(x,y)\to \ex y R(x,y,z)\;\;\sim_\al\;\; 
\fa w P(w,y)\to \ex v R(x,v,z) \;\; \sim_\al \;\; 
\fa z P(z,y)\to \ex u R(x,u,z)
\]
}


Más tarde demostraremos que las fórmulas $\al$-equivalentes también son 
lógicamente equivalentes y por lo tanto son intercambiables en cualquier 
contexto o bajo cualquier operación.

\vspace*{10pt}

Usando la $\al$-equivalencia, la operación de sustitución en fórmulas se
vuelve una función \textit{total} por lo que siempre está definida. 
\ejem{Por ejemplo se tiene que
\[
\fa x (Q(x)\to R(z,x))[z:=f(x)] \;=\; \fa y(Q(y)\to R(z,y))[z:=f(x)]
\;=\; \fa y(Q(y)\to R(f(x),y))
\]
}
Veamos otros ejemplos de sustituciones:

\ejem{Las siguientes sustituciones se sirven, en caso necesario, de la 
$\al$-equivalencia
\[
\ba{rll}
\fa x R(x,y,f(z))[y,z:=d,g(w)] & = & \fa x R(x,d,f(g(w))) \\ \\
\fa x P(x,y)[y,z:=f(c),w] & = & \fa x P(x,f(c)) \\ \\
Q(y,z,y)[y,z:=g(d),f(y)] & = & Q(g(d),f(y),g(d)) \\\\
\ex w Q(y,z,y)[y,z:=g(d),f(y)] & = & \ex w Q(g(d),f(y),g(d)) \\ \\
\ex y Q(y,z,y)[y,z:=g(d),f(y)] & = & \ex w Q(w,f(y),w) \\ \\
\ex z Q(y,z,y)[y,z:=g(d),f(y)] & = & \ex w Q(g(d),w,g(d)) \\\\
\fa x\big( Q(z,y,x)\land \ex z T(f(z),w,y)\big)[x,y,z:=a,z,g(w)] & = &
\fa u\big( Q(g(w),z,u)\land \ex v T(f(v),w,z)\big) \\ \\
\fa x\big( Q(z,y,x)\land \ex z T(f(z),w,y)\big)[z,w,y:=g(x),h(z),w] & = &
\fa u\big( Q(g(x),w,u)\land \ex v T(f(v),h(z),w)\big) 
\ea
\]
}
%Veamos porque es necesario el concepto de instancia admisible. En el
%siguiente cap\'{\i}tulo daremos un significado formal a todas las
%nociones sint\'acticas que hemos introducido hasta ahora, sin embargo, ya
%les hemos dado un significado intuitivo que usaremos ahora. Considerese la
%f\'ormula $\fa x\vp$, con la cual representamos que la f\'ormula se cumple
%para todo valor posible de la variable $x$, de manera que nos gustar\'{\i}a
%poder concluir, a partir de la f\'ormula $\fa x\vp$, la instancia
%$\vp\{x/t\}$ para cualquier t\'ermino $t$. Veamos que pasa si consideramos
%la f\'ormula $\fa x\ex y\neg(x=y)$, est\'a f\'ormula expresa el hecho de
%que para cualquier individuo existe otro distinto de \'el y es cierta si
%nuestro universo de discurso tiene al menos dos elementos, 
%pero al momento de obtener el caso particular para $y$  obtenemos la
%f\'ormula $\ex y\neg (y=y)$ que claramente es falsa. `? Qu\'e es lo est\'a
%mal?, obs\'ervese que la presencia libre de $x$ en la f\'ormula $\ex y\neg
%(x=y)$ se acot\'o al aplicar la sustituci\'on $\{x/y\}$, es decir la
%instancia $(\ex y\neg(x=y))\{x/y\}$ no es admisible. 
%La condici\'on de admisibilidad prohibe que sucedan situaciones erroneas
%como la anterior.

%\ejem{Considerense las siguientes sustituciones, 
% $\si=\{x/f(x),y/c\},\;\allowbreak
%\tau=\{x/f(y),y/z\},\;\rho=\{x/c, z/g(x,y)\}$, entonces:
%\bi
%\item[] $(\fa x P(x,y))\si=\fa x P(x,c)$ es admisible.
%\item[] $(\ex y R(x,z,y))\tau=\ex y R(f(y),z,y)$ no es admisible pues la
%presencia de $y$ en el t\'ermino $f(y)$ se vuelve ligada.
%\item[] $(\fa x\ex y Q(z,x,y)\lor P(c,x))\rho=\fa x\ex y Q(g(x,y),x,y)\lor
%P(c,c)$ no es admisible pues las presencias de $x,y$ en $g(x,y)$ se vuelven 
% ligadas.
%\ei
%}



%Definimos la sustitución de la letra proposicional $p$ por la fórmula
%$\psi$ en una fórmula dada $\vp$, denotada $\vp[p:=\psi]$
%recursivamente como sigue:

%\[
%\ba{rll}
%p[p:=\psi] & = & \psi \\
%q[p:=\psi] & = & q \\
%\top[p:=\psi] & = & \top \\
%\bot[p:=\psi] & = & \bot \\
%(\neg \vp)[p:=\psi] & = & \neg (\vp[p:=\psi])\\
%(\vp\land\chi)[p:=\psi] & = &
%(\vp[p:=\psi]\land\chi[p:=\psi])\\
%(\vp\lor\chi)[p:=\psi] & = &
%(\vp[p:=\psi]\lor\chi[p:=\psi])\\
%(\vp\imp\chi)[p:=\psi] & = &
%(\vp[p:=\psi]\imp\chi[p:=\psi])\\
%(\vp\iff\chi)[p:=\psi] & = &
%(\vp[p:=\psi]\iff\chi[p:=\psi])\\
%\ea
%\]

% Analogamente se define la sustitución simultanea 
% $\vp[p_1,\ldots,p_n:=\psi_1,\ldots,\psi_n]$
%La operación sustitución tiene las siguientes propiedades:
%\bi
%\item Si $p$ no figura en $\vp$ entonces
%  $\vp[p:=\psi]=\vp$
%  \item Si $p\neq q$ y $p$ no figura en $\chi$ entonces
%$$\vp[p:=\psi][q:=\chi]=\vp[q:=\chi][p:=\psi[q:=\chi]]$$
%\item Si $p\neq q_1,\ldots, q_n$ y $p$ no figura en $\psi_1,\ldots.\psi_n$
%  entonces 
% $$\vp[\vec{q},p:=\vec{\psi},\chi]=\vp[\vec{q}:=\vec{\psi}\;][p:=\chi]$$
% 
%\ei


\subsection{Implementación}

Se deja como ejercicio implementar un test para la $\al$-equivalencia así como 
redefinir la función de sustitución usando el renombre de variables ligadas. 
Estos ejercicios son más complicados que los anteriores.

\bi
\item \verb=vAlfaEq:: Form -> Form -> Bool= verifica si dos fórmulas son 
$\al$-equivalentes.
\item \verb=renVL:: Form -> Form= renombra las variables ligadas de una fórmula 
de manera que las listas de variables libres y ligadas sean ajenas. Esto es un 
caso particular de la siguiente función.
\item \verb=renVLconj :: Form -> [Nombre] -> Form= renombra las variables 
ligadas 
de una fórmula de forma que sus nombres sean ajenos a los de una lista dada. 
\item \verb=apSubF2 :: Form -> Subst -> Form= que implemente la sustitución en 
fórmulas usando la $\al$-equivalencia.
\ei


\end{document}
