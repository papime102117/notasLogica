\documentclass[11pt,letterpaper]{article}
\usepackage{../packageslc}
\usepackage{../optionslc}

%%%% Macros para las notas de lenguajes de programacion


%%%% math

\newcommand{\vphi}{\varphi}
\newcommand{\vp}{\varphi}

% \newcommand{\dn}{\mathsf{DN}}
% \newcommand{\dnC}{\mathsf{DN_C}}
% \newcommand{\dnM}{\mathsf{DN_M}}
% \newcommand{\dnp}{\mathsf{DN_p}}
% \newcommand{\dnm}{\mathsf{DN_p^M}}
% \newcommand{\dnc}{\mathsf{DN_p^C}}

%\newcommand{\case}{\mathsf{case}}
%\renewcommand\labelitemi{$\circ$}

\newcommand{\imp}{\rightarrow}
\newcommand{\Imp}{\Rightarrow}
\renewcommand{\iff}{\leftrightarrow}
\newcommand{\Iff}{\Leftrightarrow}
\newcommand{\G}{\Gamma}
\newcommand{\D}{\Delta}


\newcommand{\De}{\mathcal{D}}
\newcommand{\F}{\mathcal{F}}
\newcommand{\Ge}{\mathcal{G}}
\newcommand{\Pe}{\mathcal{P}}
\newcommand{\I}{\mathcal{I}}
\newcommand{\C}{\mathcal{C}}
\newcommand{\K}{\mathcal{K}}
\renewcommand{\L}{\mathcal{L}}
\newcommand{\M}{\mathcal{M}}
\newcommand{\Nc}{\mathcal{N}}
%\newcommand{\E}{\mathcal{E}}
%\newcommand{\R}{\mathcal{R}}
%\newcommand{\Q}{\mathcal{Q}}
\newcommand{\Sc}{\mathcal{S}}
\newcommand{\Te}{\mathcal{T}}
\newcommand{\W}{\mathcal{W}}

\newcommand{\Db}{\mathbb{D}}
\newcommand{\Fb}{\mathbb{F}}
\newcommand{\Kb}{\mathbb{K}}
\newcommand{\Eb}{\mathbb{E}}
\newcommand{\Ebs}{\mathbb{E}^\star}
\newcommand{\Ob}{\mathbb{O}}
\newcommand{\Ib}{\mathbb{I}}
\newcommand{\Rb}{\mathbb{R}}
\newcommand{\Qb}{\mathbb{Q}}
\newcommand{\Kbb}{\mathbb{K}}
\newcommand{\T}{\mathbb{\Theta}}


\newcommand{\kb}{\bbkappa}

\newcommand{\Sf}{\mathsf{\Sigma}}

\newcommand{\fa}{\forall}
\newcommand{\ex}{\exists}

\newcommand{\inc}{\subseteq}

\newcommand{\Lb}{\Lambda}
\newcommand{\Om}{\Omega}
\newcommand{\lb}{\lambda}
\newcommand{\al}{\alpha}
\newcommand{\ga}{\gamma}


\newcommand{\mg}{\mathbb{m}}

\newcommand{\cg}{\mathbb{C}}
\newcommand{\dg}{\mathbb{D}}
\newcommand{\jg}{\mathbb{J}}
\newcommand{\Ha}{\mathcal{H}}
%\newcommand{\A}{\mathcal{A}}
\newcommand{\sg}{\mathbb{S}}

\newcommand{\Bc}{\mathcal{B}}
\newcommand{\Df}{\mathfrak{D}}
\newcommand{\Dc}{\mathcal{D}}
%\newcommand{\Tc}{\mathcal{T}}
\newcommand{\Mf}{\mathfrak{M}}

\newcommand{\Sg}{\mathbb{S}}

\newcommand{\Q}{\ensuremath{\mathbb{Q}}}
\newcommand{\Z}{\ensuremath{\mathbb{Z}}}
\newcommand{\N}{\ensuremath{\mathbb{N}}}
\newcommand{\R}{\ensuremath{\mathbb{R}}}
\renewcommand{\S}{\mathbb{\Sigma}}
\newcommand{\A}{\mathcal{A}}
\newcommand{\E}{\ensuremath{\exists}}
\newcommand{\iso}{\ensuremath{\cong}}
\newcommand{\union}{\ensuremath{\cup}}
\newcommand{\morinyec}{\ensuremath{\precapprox}}

\newcommand{\nin}{\ensuremath{\notin}}
\newcommand{\tog}{\makebox[7mm][l]}
\newcommand{\toge}{\makebox[11mm][l]}
\newcommand{\toget}{\makebox[13mm][l]}
\newcommand{\togeth}{\makebox[14mm][l]}
\newcommand{\togethe}{\makebox[15mm][l]}
\newcommand{\together}{\makebox[17mm][l]}
\newcommand{\niso}{\ensuremath{\not \cong}}


\newcommand{\Mg}{\mathbb{M}}
\newcommand{\Bg}{\mathbb{B}}
\newcommand{\Lg}{\mathbb{L}}
\newcommand{\Tg}{\mathbb{T}}

\newcommand{\sketch}{\Red{{\sc sketch}}}

\newcommand{\restr}[2]{#1\!\!\boldsymbol{\restriction}\!#2}

\newcommand{\vacio}{\varnothing}
\newcommand{\done}{\ensuremath{\checkmark}}

\newcommand{\ida}{$\Rightarrow \; )$ }
\newcommand{\regr}{$\Leftarrow \; )$ }

\newcommand{\ol}[1]{\overline{#1}}

\newcommand{\Tsf}{\mathsf{T}}

\newcommand{\inds}[1]{\index[simb]{#1}}

\newcommand{\B}{\mathbb{B}}
%\newcommand{\N}{\mathbb{N}}

\newcommand{\vx}{\vec{x}}
\newcommand{\vy}{\vec{y}}
\newcommand{\vz}{\vec{z}}
\newcommand{\vt}{\vec{t}}
\newcommand{\vf}{\vec{f}}


% \newcommand{\propo}{\ensuremath{\mathsf{PROP}}}
% \newcommand{\atom}{\ensuremath{\mathsf{ATOM}}}
\newcommand{\term}{\ensuremath{\mathsf{TERM}}}
\newcommand{\form}{\mathsf{FORM}}

\newcommand{\true}{\mathop{\mathsf{true}}}

%\newcommand{\id}{\mathsf{Id}}

%\newcommand{\uc}{\mathcal{U}}
%\newcommand{\Ic}{\mathcal{I}}
%\newcommand{\pc}{\mathcal{P}}
%\newcommand{\qc}{\mathcal{Q}}
%\newcommand{\mc}{\mathcal{M}}
\newcommand{\supc}{\supseteq}
\newcommand{\limo}{\mathop{\mathpzc{Lim}}}
\newcommand{\ord}{\mathsf{OR}}

\newcommand{\pt}[1]{\langle #1 \rangle}


%%%% frames
\newcommand{\titulos}[2]{\frametitle{#1}\framesubtitle{#2}}
\newcommand{\fot}[1]{\footnote{\scriptsize{#1}}}


%%%% ambientes

\newcommand{\cb}[2]{\colorbox{#1}{#2}}

\newcommand{\bc}{\begin{center}}
\newcommand{\ec}{\end{center}}
\newcommand{\be}{\begin{enumerate}}
\newcommand{\ee}{\end{enumerate}}
\newcommand{\bi}{\begin{itemize}}
\newcommand{\ei}{\end{itemize}}
\newcommand{\beq}{\begin{equation}}
\newcommand{\eeq}{\end{equation}}
\newcommand{\beqs}{\begin{equation*}}
\newcommand{\eeqs}{\end{equation*}}
\newcommand{\ba}{\begin{array}}
\newcommand{\ea}{\end{array}}


% \newtheorem{theorem}{Teorema}
% \newcommand{\teo}[1]{\begin{theorem} #1 \end{theorem}}
% \newtheorem{proposition}{Proposici\'on}
% \newcommand{\prop}[1]{\begin{proposition} #1 \end{proposition}}
% \newtheorem{definition}{Definici\'on}
% \newcommand{\defin}[1]{\begin{definition} #1 \end{definition}}
% \newtheorem{corollary}{Corolario}
% \newcommand{\cor}[1]{\begin{corollary} #1 \end{corollary}}
% \newtheorem{lemma}{Lema}
% \newcommand{\lema}[1]{\begin{lemma} #1 \end{lemma}}
% \newcommand{\dem}[1]{\begin{proof} #1 \end{proof}}

%\renewcommand{\qed}{\qedsymbol{$\mathbf{\dashv}$}}

%\newcommand{\proof}{\hfill\\\noindent\textbf{\textit{Demostraci\'on. }}}

\newcommand{\hint}{\emph{Sugerencia: }}


\newcounter{EjempCtr}[section]
\newenvironment{enumrom}{\renewcommand{\theenumi}{\roman{enumi}}%
\renewcommand{\theenumii}{\roman{enumii}}
\renewcommand{\theenumiii}{\roman{enumiii}}
\renewcommand{\theenumiv}{\roman{enumiv}}
\begin{enumerate}}{\end{enumerate}}
\newenvironment{Ejemplo}
        {\stepcounter{EjempCtr}%
        \begin{description}\item[Ejemplo \thesection.\arabic{EjempCtr}]}%
        {\end{description}}
\newenvironment{demostr}{%
             {\em Demostración:}
                \begin{quotation}}{\end{quotation}}

   \newcommand{\beje}{\begin{Ejemplo}}
\newcommand{\eeje}{\end{Ejemplo}}


\newtheorem{eje}{Ejemplo}[section]
\newcommand{\ejem}[1]{\begin{eje}\normalfont #1 \end{eje}}

% \renewcommand\contentsname{\'Indice}
%\renewcommand\chaptername{Cap\'itulo}
% \renewcommand\indexname{\'Indice}

%%\newcommand{\qed}{\hfill$\mathbb{Qed}$}
%\newcommand{\qed}{\hfill$\mathsf{\boldsymbol{\dashv}}$}
%\renewcommand{\qed}{\hfill$\boldsymbol{\dashv}$}


\newenvironment{prueba}{\vspace{-5mm}\noindent\textbf{Demostraci\'on}\\}{
\noindent$\blacksquare$\\}

\newcommand{\Ejercicios}{\section*{Ejercicios}}

 \newenvironment{manitas}{%
      \renewcommand{\labelitemi}{\ding{44}}%
      \vspace{-0.5cm}%
      \begin{itemize}%
      \setlength{\itemsep}{0pt}\setlength{\parsep}{0pt}\setlength{\topsep}{0pt}%
      }{\end{itemize}}
\newenvironment{malitos}{%
      \renewcommand{\labelitemi}%
            {\raisebox{1.5ex}{\makebox[0.3cm][l]{\begin{rotate}{-90}%
            \ding{43}\end{rotate}}}}%
      \vspace{-0.5cm}%
      \begin{itemize}%
      \setlength{\itemsep}{0pt}\setlength{\parsep}{0pt}\setlength{\topsep}{0pt}%
      }{\end{itemize}}
\newenvironment{ejercs}{
     \renewcommand{\labelenumi}{\thesection.\theenumi.-}
     \renewcommand{\labelenumii}{\theenumii)}
     \begin{enumerate}}
     {\end{enumerate}}

   \newcommand{\bej}{\begin{ejercs}}
\newcommand{\eej}{\end{ejercs}}


%\newenvironment{leterize}{%
%        \renewcommand{\theenumi}{\alph{enumi}}
%        \begin{enumerate}}{\end{enumerate}}

%\newenvironment{manitas}{%
%      \renewcommand{\labelitemi}{\ding{44}}%
%      \vspace{-0.5cm}%
%      \begin{itemize}%
%      
% \setlength{\itemsep}{0pt}\setlength{\parsep}{0pt}\setlength{\topsep}{0pt}%
%      }{\end{itemize}}



%%=============================================================================

\def\stackunder#1#2{\mathrel{\mathop{#2}\limits_{#1}}}


%%%% notas

\newcommand{\doubt}{\Red{{\LARGE {\sf ??}}}}

\newcommand{\coment}[1]{\hfill\\ \Big[{\bf Comentario Privado:} #1\Big]}
\newcommand{\preg}[1]{\hfill\\ \BrickRed{{\bf Pregunta:} #1}}
\newcommand{\conjet}[1]{\hfill\\ \OliveGreen{{\bf Conjecura:} #1}}

\newcommand{\pendiente}{\BrickRed{{\sc Pendiente}}}
\newcommand{\verifpendiente}{\BrickRed{{\sc Verificación pendiente}}}


%--------------------------------------------------------------------------

\DeclareMathAlphabet{\mathpzc}{OT1}{pzc}{m}{it}



\title{Sem\'antica de la L\'ogica de Proposiciones \\ 
L\'ogica Computacional 2018-2, Nota de clase 2}
\author{Favio Ezequiel Miranda Perea\and Araceli Liliana Reyes Cabello\and
Lourdes Del Carmen Gonz\'alez Huesca \and Pilar Selene Linares Ar\'evalo}
\date{28 de febrero de 2018 \\
Material desarrollado bajo el proyecto UNAM-PAPIME PE102117}


\begin{document}

\maketitle

En esta nota revisaremos brevemente los conceptos sem\'anticos m\'as relevantes 
de la l\'ogica proposicional. Todos los conceptos involucrados en esta nota 
fueron estudiados a detalle en el curso de Estructuras Discretas.

\section{Significado de los conectivos l\'ogicos}

\noindent Veamos el significado de cada conectivo l\'ogico:

\paragraph{La Negaci\'on}
La \emph{negaci\'on} de la f\'ormula $\vp$ es la f\'ormula $\lnot \vp$.

\begin{itemize}
 \item S\'imbolo utilizado:  $\lnot$ 
 \item Correspondencia con el espa\~nol: No, no es cierto que, es falso que, 
  etc.
 \item Otros s\'imbolos: $\sim\!\vp,\; \overline{\vp}$. 
 \item Sem\'antica (tabla de verdad):
  \[ 
   \begin{array}{c|c}
    \vp & \lnot \vp \\ \hline 
    1 & 0\\ 
    0 & 1 
   \end{array}
  \]
\end{itemize}


\paragraph{La Disyunci\'on}
La \emph{disyunci\'on} de las f\'ormulas $\vp,\psi$ es la f\'ormula
$\vp\lor \psi$. 
Las f\'ormulas $\vp,\psi$ se llaman \emph{disyuntos}. \index{disyuntos}

\begin{itemize}
 \item S\'imbolo utilizado: $\lor$
 \item Correspondencia con el espa\~nol: o. 
 \item Otros s\'imbolos: $\vp+\psi,\;\vp\,|\,\psi$. \index{disyunci\'on}
 \item Sem\'antica (tabla de verdad):  
  \[
   \begin{array}{c|c|c}
    \vp & \psi & \vp\lor \psi\\ \hline 
    1 & 1 & 1\\
    1 & 0 & 1\\
    0 & 1 & 1\\
    0 & 0 & 0\\
   \end{array}
  \]
\end{itemize}

\paragraph{La Conjunci\'on}\index{conjunci\'on}
La \emph{conjunci\'on} de las f\'ormulas $\vp,\psi$ es la f\'ormula
$\vp\land \psi$. Las f\'ormulas $\vp,\psi$ se llaman 
\emph{conyuntos}\index{conyuntos}. 

\begin{itemize}
 \item S\'imbolo utilizado: $\land$
 \item Correspondencia con el espa\~nol: y, pero. 
 \item Otros s\'imbolos: $\vp\,\&\, \psi,\;\vp\cdot \psi$ o tambi\'en $\vp\psi$ 
  sin usar un operador expl\'icito.
 \item Sem\'antica (tabla de verdad): 
  \[
   \begin{array}{c|c|c}
    \vp & \psi & \vp\land \psi\\ \hline 
    1 & 1 & 1\\
    1 & 0 & 0\\
    0 & 1 & 0\\
    0 & 0 & 0\\
   \end{array}
  \]
\end{itemize}


\paragraph{La Implicaci\'on}\index{implicaci\'on}\index{condicional}
La \emph{implicaci\'on} o \emph{condicional} de las f\'ormulas $\vp,\psi$ es la 
f\'ormula $\vp\imp \psi$. 
La f\'ormula $\vp$ es el \emph{antecedente}\index{antecedente} y la
f\'ormula $\psi$ es el \emph{consecuente} de la implicaci\'on. 

\begin{itemize}
 \item S\'imbolo utilizado: $\imp$
 \item Correspondencia con el espa\~nol: $\vp\imp \psi$ significa: si $\vp$
  entonces $\psi$; $\psi$, si $\vp$; $\vp$ s\'olo si $\psi$; $\vp$ es 
  condici\'on suficiente para $\psi$; $\psi$ es condici\'on necesaria para 
  $\vp$.
 \item Otros s\'imbolos: $\vp\Imp \psi,\;\vp\supset \psi$.
 \item Sem\'antica (tabla de verdad): 
  \[
   \begin{array}{c|c|c}
    \vp & \psi & \vp\imp \psi\\\hline 
    1 & 1 & 1\\
    1 & 0 & 0\\
    0 & 1 & 1\\
    0 & 0 & 1\\
   \end{array}
  \]
\end{itemize}

\paragraph{La Equivalencia}
La equivalencia o bicondicional de las f\'ormulas $\vp,\psi$ es la f\'ormula
$\vp\iff \psi$. 

\begin{itemize}
 \item S\'imbolo utilizado: $\iff$
 \item Correspondencia con el espa\~nol: $\vp$ es equivalente a $\psi$; 
  $\vp$ si y s\'olo si $\psi$; $\vp$ es condici\'on necesaria y suficiente 
  para $\psi$.
 \item Otros s\'imbolos: $\vp\Iff \psi,\;\vp\equiv \psi$.
 \item Sem\'antica (tabla de verdad): 
  \[
   \begin{array}{c|c|c}
    \vp & \psi & \vp\iff \psi\\ \hline 
    1 & 1 & 1\\
    1 & 0 & 0\\
    0 & 1 & 0\\
    0 & 0 & 1\\
   \end{array}
  \]
\end{itemize}

\section{Sem\'antica formal de los conectivos l\'ogicos}

\defin{El tipo de valores booleanos denotado $\bool$ se define como 
$\bool=\{0,1\}$ }

\defin{Un estado o asignaci\'on de las variables (proposicionales) es
  una funci\'on $$\I:VarP\imp\bool$$
  }

\noindent Dadas $n$ variables proposicionales existen $2^n$ estados distintos
para estas variables.

\defin{Dado un estado de las variables $\I:VarP\imp\bool$, definimos la
 interpretaci\'on de las f\'ormulas con respecto a $\I$ como la funci\'on
 $\I^\star:\propo\imp\bool$ tal que:
 \bi
  \item $\I^\star(p)=\I(p)$ para $p\in VarP$, es decir
    $\restr{\I^\star}{_{VarP}}=\I$.
  \item $\I^\star(\top)=1$
  \item $\I^\star(\bot)=0$
  \item $\I^\star(\lnot\vphi)=1$ syss $\I^\star(\vphi)=0$.
  \item $\I^\star(\vphi\land\psi)=1$ syss $\I^\star(\vphi)=\I^\star(\psi)=1$.
  \item $\I^\star(\vphi\lor\psi)=0$ syss $\I^\star(\vphi)=\I^\star(\psi)=0$.
  \item $\I^\star(\vphi\imp\psi)=0$ syss $\I^\star(\vphi)=1$ \textbf{e}
    $\I^\star(\psi)=0$.
  \item $\I^\star(\vphi\iff\psi)=1$ syss $\I^\star(\vphi)=\I^\star(\psi)$. 
 \ei
}


Obs\'ervese que dado un estado de las variables~$\I$ (o tambi\'en~$\I_v$), la 
interpretaci\'on~$\I^\star$ generada por~$\I$ est\'a determinada de manera 
\'unica por lo que de ahora en adelante escribiremos simplemente $\I$ en lugar 
de~$\I^\star$. Este abuso notacional es pr\'actica com\'un en ciencias de la
computaci\'on y se conoce como \textit{sobrecarga} de operadores. \\

\noindent El siguiente lema es de importancia para restringir las 
interpretaciones de inter\'es al analizar una f\'ormula.

\lema{[Coincidencia]
Sean $\I_1,\I_2:\propo\imp\bool$ dos estados que coinciden en las variables 
proposicionales de la f\'ormula $\vphi$, es decir $\I_1(p)=\I_2(p)$ para toda 
$p\in vars(\vp)$.\\
Entonces $\I_1(\vphi)=\I_2(\vphi)$.}
\proof
Inducci\'on estructural sobre $\vp$
\qed

\espc

El lema anterior implica que a\'un cuando existen una infinidad de estados, dada 
una f\'ormula $\vphi$ basta considerar \'unicamente aquellos que difieren en 
las variables proposicionales de $\vphi$, a saber $2^n$ estados distintos si 
$\vphi$ tiene $n$ variables proposicionales.


\defin{[Estado modificado o actualizado] 
Sean $\I:\vars \imp \bool$ un estado de las variables, $p$ una variable 
proposicional y $v\in\bool$. Definimos la actualizaci\'on de $\I$ en $p$ por 
$v$, denotado $\I[p/v]$ como sigue:
\[
\I[p/v\;](q) =\left\{\ba{rl}
v & \text{ si } q=p \\ \\
\I(q) & \text{ si } q \neq p
\ea
\right.
\]
%Es decir, $\sigma[\vx/\vec{m}\;]$ coincide con $\sigma$ en todas las variables
%excepto en $x_1,\ldots,x_n$ siendo $m_i$ su valor en $x_i$.\\
El estado $\I[p/v]$ se conoce como un estado modificado o una actualizaci\'on 
de $\I$.
}

El concepto de estado modificado se relaciona con el concepto de sustituci\'on 
textual de una manera importante que permite eficientar el proceso de 
interpretaci\'on de una f\'ormula. Dicho proceso corresponde al siguiente

\lema{[Sustituci\'on]
Sean $\I$ una interpretaci\'on, $p$ una variable 
proposicional y $\psi$ una f\'ormula tal que $\I^\star(\psi)=v$. Entonces
% $r\in\term$, $\sigma$ un estado de
%   las variables, $[\vx:=\vt\;]$ una sustituci\'on y $m_1,\ldots,m_n\in
%   M$ tales que $\I(t_i)=m_i\;\;1\leq i\leq n$. Entonces
$$\I\big(\vp[p:=\psi\;]\big) = \I[p/v\;](\vp)$$
}
\proof Inducci\'on sobre $\vp$. %Se vio en clase. 
\qed


\subsection{Observaciones sobre la implementaci\'on funcional}

\bi
\item Estados: un estado $\I$ puede implementarse directamente como una 
funci\'on 
\begin{verbatim}
 i :: VarP -> Bool 
\end{verbatim}
o bien, dado que las listas son una estructura de datos muy conveniente en {\sc 
Haskell} podemos definir el  tipo de Estados como
\begin{verbatim}
type Estado = [VarP] 
\end{verbatim}
con lo cual representamos a un estado mediante la lista de variables 
proposicionales que est\'an definidas como verdaderas. \\
Por ejemplo si definimos $\I_1(p)=\I_1(q)=1$ e $\I_1(r)=0$ entonces 
implementamos a dicho estado como \verb~i1=[p,q]~. 
Similarmente el estado \verb~i2=[r,s,t]~ representa al estado 
$\I_2(r)=\I_2(s)=\I_2(t)=1$. 

\item Funci\'on de interpretaci\'on: la funci\'on de interpretaci\'on 
$\I^\star$ se implementa mediante una funci\'on 
\begin{verbatim}
interp :: Estado -> Prop -> Bool 
\end{verbatim}
especificada por: $$ {\tt interp}\;i\;\vp = \I^\star(\vp) $$ 
donde el estado $i$ representa al estado $\I$ de la teor\'ia.

\item Estados posibles: dada una f\'ormula $\vp$ con $n$-variables 
proposicionales, la funci\'on 

\verb~estados :: Prop -> [Estado]~
devuelve la lista con los $2^n$ estados distintos para $\vp$. 
Su definici\'on es  %dada {\tt phi::\propo}
\begin{verbatim}
estados phi = subconj (vars phi)
\end{verbatim}
donde \verb~vars:: Prop -> [VarP]~ es la funci\'on que 
devuelve la lista de variables proposicionales que figuran en $\vp$ (sin 
repetici\'on) y \verb= subconj :: [a] -> [[a]]= es una funci\'on que dada una 
lista $\ell$, devuelve la lista con las sublistas de $\ell$. Por ejemplo:
\begin{verbatim}
subconj [1,2] = [[1,2],[1],[2],[]] 
\end{verbatim}
\ei

\section{Conceptos Sem\'anticos B\'asicos}

Dada una f\'ormula $\vp$ podemos preguntarnos ?`Cu\'antas interpretaciones hacen
verdadera a $\vp$? las posibles respuestas llevan a las siguientes
definiciones:

\defin{\label{def3}Sea $\vp$ una f\'ormula. Entonces
\bi
\item Si $\I(\vp)=1$ para toda interpretaci\'on $\I$ decimos que $\vp$ es una
  tautolog\'ia o f\'ormula v\'alida y escribimos $\models\vp$.
\item Si $\I(\vp)=1$ para alguna interpretaci\'on $\I$ decimos que $\vp$ es
  satisfacible, que $\vp$ es verdadera en $\I$ o que $\I$ es modelo de
  $\vp$ y escribimos $\I\models\vp$
\item Si $\I(\vp)=0$ para alguna interpretaci\'on $\I$ decimos que $\vp$ es
  falsa o insatisfacible en $\I$ o que $\I$ no es modelo de $\vp$ y
  escribimos $\I\not\models\vp$
\item Si $\I(\vp)=0$ para toda interpretaci\'on $\I$ decimos que $\vp$ es una
  contradicci\'on o f\'ormula no satisfacible.   
\ei

Similarmente si $\Gamma$ es un conjunto de f\'ormulas decimos que:
\bi
\item $\Gamma$ es satisfacible si tiene un modelo, es decir, si existe una 
interpretaci\'on $\I$ tal que
$\I(\vp)=1$ para toda $\vp\in\Gamma$. Lo cual denotamos a veces,
abusando de la notaci\'on, con $\I(\Gamma)=1$.
\item $\Gamma$ es insatisfacible o no satisfacible si no tiene un
  modelo, es decir, si no existe una
  interpretaci\'on $\I$ tal que $\I(\vp)=1$ para toda $\vp\in\Gamma$.
\ei
}
Con respecto a las tablas de verdad tenemos las siguientes observaciones:
\bi
\item Una f\'ormula $\vp$ es satisfacible si en alguna linea de la tabla de
  verdad $\vp$ toma el valor $1$. En caso contrario, es decir si en {\bf
    todas} las
  lineas toma el valor $0$ es insatisfacible (contradicci\'on).
\item Un conjunto de f\'ormulas $\Gamma$ es satisfacible si existe alguna linea
  de la tabla de verdad en la que {\bf todas} las f\'ormulas de $\Gamma$ toman 
el valor $1$.
\ei



\prop{Sea $\Gamma$ un conjunto de f\'ormulas, $\vp\in\Gamma$, $\tau$ una 
tautolog\'ia y $\chi$ una contradicci\'on. 
\bi
\item[] Si $\Gamma$ es satisfacible entonces
\bi
\item $\Gamma\setminus\{\vp\}$ es satisfacible.
\item $\Gamma\cup\{\tau\}$ es satisfacible.
\item $\Gamma\cup\{\chi\}$ es insatisfacible.
\ei
\item[] Si $\Gamma$ es insatisfacible entonces
\bi
\item $\Gamma\cup\{\psi\}$ es insatisfacible, para cualquier $\psi\in\propo$.
\item $\Gamma\setminus\{\tau\}$ es insatisfacible.
\ei
\ei
}
\proof
Ejercicio
\qed

\espc

Si $\Gamma$ es satisfacible y $\psi$ tambi\'en nada se puede decir acerca de
$\Gamma\cup\{\psi\}$. Analogamente si $\Gamma$ es insatisfacible y $\psi$ no es
tautolog\'ia nada se puede decir de $\Gamma\setminus\{\psi\}$.

\espc

\prop{Sea $\Gamma=\{\vp_1,\ldots,\vp_n\}$ un conjunto de f\'ormulas. 
\bi
\item $\Gamma$ es satisfacible syss $\vp_1\land\ldots\land\vp_n$ es
  satisfacible.
\item $\Gamma$ es insatisfacible syss $\vp_1\land\ldots\land\vp_n$ es
  una contradicci\'on.
\ei
}
\proof
Ejercicio
\qed

\espc
\subsection{Observaciones sobre la implementaci\'on funcional}

\noindent Para implementar los conceptos de la definici\'on~\ref{def3} nos 
serviremos de la funci\'on 
\begin{verbatim}
modelos :: Prop -> [Estados]                 
\end{verbatim}
que dada una f\'ormula $\vp$ devuelve la lista de todos sus modelos. \\
Esta funci\'on se especifica como sigue:
\begin{alltt}
modelos(\(\vp\)) = [i | i \(\in\) estados(\(\vp\)),\;interp\;i\;\(\vp\) = True]
\end{alltt}
Es decir \verb=modelos(=$\vp$\verb=)= devuelve la lista 
con todos aquellos estados $\I$ tales que $\I(\vp)=1$.
Los conceptos de la definici\'on \ref{def3} se implementan comparando las 
listas \verb=estados(=$\vp$\verb=)=  y \verb=modelos(=$\vp$\verb=)=.

\bi
\item Tautolog\'ia: \\
\verb=tautologia:: Prop -> Bool= de la definici\'on $\vp$ es tautolog\'ia syss 
todos sus estados (interpretaciones) son modelos:
\begin{alltt} 
 tautologia(\(\vp\)) = estados(\(\vp\)) == modelos(\(\vp\))
\end{alltt}

\item Satisfacible en una interpretaci\'on: \\
\verb=satisfen:: Estado -> Prop -> Bool= donde $\vp$ es satisfacible en $\I$ 
($\I\models\vp$) syss $\I(\vp)=1$. 
%alg\'un estado (interpretaci\'on) es modelo, es decir, si la lista de modelos 
% no es vac\'ia.
\begin{alltt}
satisfen i \(\vp\) = interp i \(\vp\) == True 
\end{alltt}

\item Satisfacible: \\
\verb~satisf:: Prop -> Bool~ es decir $\vp$ es satisfacible syss 
alg\'un estado (interpretaci\'on) es modelo, es decir, si la lista de modelos 
no es vac\'ia:
\begin{alltt}
 satisf(\(\vp\)) = modelos(\(\vp\)) /= []
\end{alltt}

\item Insatisfacible en una interpretaci\'on:\\
\verb~insatisfen:: Estado -> Prop -> Bool~ donde $\vp$ es insatisfacible en 
$\I$ i.e. $\I\not\models\vp$  syss $\I(\vp)=0$. 
%alg\'un estado (interpretaci\'on) es modelo, es decir, si la 
% lista de modelos no es vac\'ia.
\begin{alltt}
insatisfen i \(\vp\) = interp i \(\vp\) == False 
\end{alltt}

\item Insatisfacible o contradicci\'on: \\
\verb~contrad:: Prop -> Bool~ donde $\vp$ es insatisfacible o contradicci\'on 
syss ning\'un estado (interpretaci\'on) es modelo, es decir, si la lista de 
modelos es vac\'ia:
\begin{alltt}
contrad(\(\vp\)) = modelos(\(\vp\)) = [] 
\end{alltt}

\item Extensi\'on a conjuntos: las funciones que generan los estados y modelos 
para un conjunto (lista) de f\'ormulas $\G$ y aquellas que verifican si 
$\G$ es satisfacible o insatisfacible se denotan:
\bi
\item \verb~estadosConj :: [Prop] -> [Estado]~
\item \verb~modelosConj :: [Prop] -> [Estado]~
\item \verb~satisfenConj:: Estado -> [Prop] -> Bool~
\item \verb~satisfConj:: [Prop] -> Bool~
\item \verb~insatisfenConj:: Estado -> [Prop] -> Bool~
\item \verb~insatisfConj:: [Prop] -> Bool~
\ei
la implementaci\'on de las mismas es similar al caso de las funciones para 
f\'ormulas y se deja como ejercicio.
\ei


\section{Equivalencia de F\'ormulas}

\defin{Dos f\'ormulas $\vp,\psi$ son equivalentes si $\I(\vp)=\I(\psi)$ para
  toda interpretaci\'on $\I$. En tal caso escribimos $\vp\equiv\psi$.
}
Es f\'acil ver que la relaci\'on $\equiv$ es una relaci\'on de equivalencia.


\prop{Sean $\vp,\psi$ f\'ormulas. Entonces $\vp\equiv\psi$ syss
  $\models\vp\iff\psi$}

\proof
Ejercicio
\qed

\espc
Dado que ya implementamos la funci\'on {\tt tautolog\'ia} podemos definir 
f\'acilmente una funci\'on \\
\verb~equiv :: Prop -> Prop -> Bool~ que 
decida si dos f\'ormulas son equivalentes usando la proposici\'on anterior:
\begin{alltt}
equiv \(\vp\; \psi\) = tautolog\'ia (\(\vp\) <-> \(\psi\))
\end{alltt}


La siguiente propiedad es fundamental para el razonamiento ecuacional con 
equivalencias l\'ogicas.

\prop{[Regla de Leibniz] Si $\vp\equiv\psi$ y $p\in\atom$ entonces
  $\chi[p:=\vp]\equiv\chi[p:=\psi]$}

\proof
Inducci\'on estructural sobre $\chi$.
\qed

\espc

Las f\'ormulas equivalentes son de gran importancia pues nos permiten definir
algunos conectivos en t\'erminos de otros, as\'i como simplificar f\'ormulas
compuestas.


\subsection{Algunas equivalencias l\'ogicas}

\bi
\item Conmutatividad:
$$ \vp \lor \psi \equiv \psi \lor \vp 
\qquad \qquad 
\vp \land \psi \equiv \psi \land \vp $$

\item Asociatividad:
$$ \vp \lor( \psi \lor \chi )\equiv(\vp \lor \psi )\lor \chi  
\qquad \qquad 
\vp \land( \psi \land \chi)\equiv(\vp \land \psi )\land \chi $$

\item Distributividad:
$$\vp \lor( \psi \land \chi )\equiv(\vp \lor \psi )\land(\vp \lor \chi )
\qquad \qquad 
\vp \land( \psi \lor \chi )\equiv(\vp \land \psi )\lor(\vp \land \chi ) $$

\item Leyes de la negaci\'on:
\bi
\item Doble Negaci\'on: $\lnot\lnot \vp \equiv \vp $
\item De Morgan: $\lnot(\vp \lor \psi )\equiv \lnot \vp \land\lnot \psi \qquad 
\lnot(\vp \land \psi )\equiv \lnot \vp \lor\lnot \psi $ 
\item $\lnot(\vp \imp \psi )\equiv \vp \land\lnot \psi $ 
\item $\lnot(\vp \iff \psi )\equiv\lnot \vp \iff \psi \equiv \vp \iff\lnot \psi 
$. 
%\item $\lnot\fa x \vp \equiv\ex x\lnot \vp $
%\item $\lnot\ex x \vp \equiv\fa x\lnot \vp $
\ei

\item Propiedades de la implicaci\'on:
\bi
\item Contrapositiva: $\vp \imp \psi \equiv\lnot \psi \imp\lnot \vp $

\item Importaci\'on/Exportaci\'on:
$\vp \land \psi \imp \chi \equiv \vp \imp( \psi \imp \chi )$

\ei

\item Eliminaci\'on de conectivos:
$$\vp \imp \psi \equiv\lnot \vp \lor \psi 
\qquad \qquad 
\vp \imp \psi \equiv\lnot(\vp \land\lnot \psi ) $$
$$ \vp \iff \psi \equiv (\vp \imp \psi )\land ( \psi \imp \vp )
\qquad \qquad 
\vp \iff \psi \equiv (\vp \land  \psi )\lor (\lnot \vp \land\lnot \psi )$$
$$ \vp \lor \psi \equiv \lnot \vp \imp \psi 
\qquad \qquad 
\vp \land \psi \equiv \lnot(\vp \imp\lnot \psi )$$

\item Leyes simplificativas:
\bi
\item Idempotencia: $\vp \lor \vp \equiv \vp \qquad \vp \land \vp \equiv \vp $

\item Absorci\'on:
$\vp \lor(\vp \land \psi )\equiv \vp \qquad \vp \land(\vp \lor \psi)\equiv \vp$ 
\ei
% En lo que sigue $\top$ denota a cualquier tautolog\'ia y
% $\bot$ a cualquier contradicci\'on.
\bi
\item Neutros:
$\vp \lor\bot\equiv \vp \qquad \vp \land\top\equiv \vp  $

\item Inversos:
\bi
 \item Tercero Excluido: $\vp \lor\lnot \vp \equiv\top$ 
 \item Contradicci\'on: $\vp \land\lnot \vp \equiv\bot$ 
\ei

\item Dominancia:
$ \vp \lor\top\equiv\top \qquad \vp \land\bot\equiv\bot $
\ei
\ei

Veamos un ejemplo de simplificaci\'on de una f\'ormula mediante el uso de 
equivalencias l\'ogicas y la regla de Leibniz:
\[
\ba{rll}
(p\lor q)\land\lnot(\lnot p\land q)&\equiv &
(p\lor q)\land (\lnot\lnot p\lor\lnot q) \\
& \equiv & (p\lor q)\land (p\lor\lnot q) \\
& \equiv & p\lor (q\land\lnot q) \\
& \equiv & p\lor \bot \\ 
& \equiv & p
\ea
\]

\subsection{Eliminaci\'on de implicaciones y equivalencias}

Las equivalencias l\'ogicas que permiten eliminar los conectivos $\to,\iff$ son 
de gran importancia para la llamada l\'ogica clausular que estudiaremos m\'as 
delante. Se deja como ejercicio definir recursivamente e implementar las 
siguientes funciones:
\bi
\item \verb~elimEquiv:: Prop -> Prop~ tal que
\begin{center}
\verb~elimEquiv~ $\vp = \psi$ syss  $\psi\equiv\vp$ y $\psi$ no 
contiene el s\'imbolo $\iff$.
\end{center}

\item \verb~elimImp:: Prop -> Prop~ tal que
\bc
\verb~elimImp~ $\vp = \psi$ syss $\psi\equiv\vp$ y $\psi$ no contiene el 
s\'imbolo $\to$.
\ec
\ei
Obs\'ervese que estas funciones pueden definirse de diversas formas pero las 
m\'as adecuadas son usando las equivalencias usuales:
$$ \vp\iff\psi\equiv (\vp \imp \psi) \land (\psi \imp \vp)\qquad 
\vp\imp\psi\equiv \lnot\vp\lor\psi $$



\section{Consecuencia L\'ogica}

\defin{Sean $\Gamma$ un conjunto de f\'ormulas y $\vp$ una f\'ormula. Decimos 
que $\vp$ es consecuencia l\'ogica de $\Gamma$ si para toda interpretaci\'on 
$\I$ que sastisface a $\Gamma$ se tiene $\I(\vp)=1$. 
Es decir si se cumple que siempre que $\I$ satisface a $\Gamma$ entonces 
necesariamente $\I$ satisface a $\vp$.
En tal caso escribimos $\Gamma\models\vp$.}
\noindent
N\'otese que la relaci\'on de consecuencia l\'ogica esta dada por una 
implicaci\'on de la forma 
$$ \I(\Gamma)=1\Imp\I(\vp)=1 $$
lo cual informalmente significa que 
\emph{todo modelo de $\Gamma$ es modelo de $\vp$}.\\
De manera que no se afirma nada acerca de la satisfacibilidad del conjunto
$\Gamma$, simplemente se asume que es satisfacible y en tal caso se prueba que
la f\'ormula $\vp$ tambi\'en lo es con la misma interpretaci\'on.\\
  
Obs\'ervese tambi\'en la sobrecarga del s\'imbolo $\models$ que previamente 
utilizamos para denotar satisfacibilidad  $\I\models\vp$ y tautolog\'ias 
$\models\vp$.

\prop{\label{prop:conslog}La relaci\'on de consecuencia l\'ogica cumple las 
siguientes propiedades:
\bi
\item Si $\vp\in\Gamma$ entonces $\Gamma\models\vp$.
\item Principio de refutaci\'on: $\Gamma\models\vp$ syss 
$\Gamma\cup\{\lnot\vp\}$
  es insatisfacible.
\item $\Gamma\models\vp\imp\psi$ syss $\Gamma\cup\{\vp\}\models\psi$.
\item Insatisfacibilidad implica trivialidad: Si $\Gamma$ es insatisfacible
entonces $\Gamma\models\vp$ para toda $\vp\in\propo$.
\item Si $\Gamma\models\bot$ entonces $\Gamma$ es insatisfacible.
\item $\vp\equiv\psi$ syss $\vp\models\psi$ y $\psi\models\vp$.
\item $\models\vp$ (es decir si $\vp$ es tautolog\'ia) syss
  $\vacio\models\vp$ (es decir $\vp$ es consecuencia l\'ogica del conjunto 
vac\'io).
\ei 
}
\proof
Ejercicio
\qed


\subsection{Observaciones sobre la implementaci\'on funcional}

\noindent La funci\'on \verb~consecuencia: [Prop] -> Prop -> Bool~ se 
especifica como:
\bc
\verb~consecuencia~ $\G\;\vp =$ \verb~True~ syss $\G\models\vp$.
\ec
La implementaci\'on se sirve del principio de refutaci\'on dado en la 
proposici\'on \ref{prop:conslog}:
\bc
\verb~consecuencia gamma phi = insatisfConj ((neg phi):gamma)~
\ec
donde \verb~neg: Prop -> Prop~ es una funci\'on que construye la negaci\'on 
de una f\'ormula.


Otra posibilidad es verificar que no existen modelos de $\G$ que no sean 
modelos de $\vp$, es decir, que la lista de modelos de $\G$ que no son modelos 
de $\vp$ es vac\'ia. Esto se puede hacer directamente usando 
el mecanismo de listas por comprensi\'on de {\sc Haskell}:

%{\tt consecuencia gamma phi = null [i\;|i <- estadosConj 
% (phi:gamma),\;satisfenConj i gamma,\;not (satisfen i phi)]}
\begin{verbatim}
consecuencia gamma phi = null [i | i <- estadosConj (phi:gamma),
                                   satisfenConj i gamma,
                                   not (satisfen i phi)
                              ]
\end{verbatim}


\subsection{Correctud de Argumentos L\'ogicos}

\defin{Un argumento l\'ogico es una sucesi\'on de f\'ormulas 
$\vp_1,\ldots,\vp_n$ llamadas premisas y una f\'ormula $\psi$ llamada 
conclusi\'on. Dicha sucesi\'on se escribe usualmente como
\begin{mathpar}
 \inferrule*[]{
  \vp_1  \\\\
  \vdots \\\\
  \vp_n 
 }{
 \therefore \psi
 } 
\end{mathpar}
}
Un problema central de la l\'ogica consiste en decidir si un argumento dado es
correcto o no, para lo cual la herramienta principal es la noci\'on de
consecuencia l\'ogica.

\defin{Un argumento con premisas $\vp_1,\ldots,\vp_n$ y conclusi\'on $\psi$ es
  l\'ogicamente correcto si la conclusi\'on se sigue l\'ogicamente de las 
premisas, es decir si $\{\vp_1,\ldots,\vp_n\}\models\psi$
}

\noindent Esta definici\'on deja ver de inmediato una implementaci\'on 
\verb~ argcorrecto :: [Prop] -> Prop -> Bool~
dado que ya tenemos un programa (funci\'on) para decidir la consecuencia 
l\'ogica.
\begin{verbatim}
argcorrecto [phi1,...,phin] psi = consecuencia [phi1,...,phin] psi
\end{verbatim}

Es importante tener un m\'etodo formal para decidir la correctud de argumentos
del lenguaje natural los cuales pueden parecer correctos y no serlo. Veamos
un ejemplo sencillo considerando el siguiente argumento: 
\bc
\emph{Si hoy es viernes entonces ma\~nana es s\'abado, ma\~nana es s\'abado, 
por lo tanto hoy es viernes.}
\ec
\noindent
Palabras como \enquote{por lo tanto, as\'i que, luego entonces, de forma que, 
etc.} se\~nalan la conclusi\'on del argumento.
La representaci\'onn l\'ogica formal del argumento anterior es:
\begin{mathpar}
 \inferrule*[]{
 v\imp s \\\\
 s
 }{
 \therefore v
 }
\end{mathpar}
\noindent
De manera que el argumento es correcto si y s\'olo si $\{v\imp s,s\}\models v$.
Pero la interpretaci\'on~$\I(v)=0, \;\I(s)=1$ confirma que el argumento 
es incorrecto.\\

\newpage

Y m\'as importante resaltar que al analizar argumentos lo \'unico que interesa 
es su forma l\'ogica y no su significado, como lo muestra el siguiente ejemplo:
\bc
\emph{Si hoy tirila y Chubaka es kismi entonces Chubaka es borogrove.\\
  Si hoy no tirila entonces hay fefos.\\ M\'as a\'un sabemos que no hay fefos y 
  que\\ Chubaka es kismi, luego entonces Chubaka es borogrove.
}
\ec
Lo cual nos lleva a
\begin{mathpar}
 \inferrule*[]{
  t\land k\imp b \\\\
  \lnot t\imp f \\\\
  \lnot f\land k 
 }{
 \therefore b
 }
\end{mathpar}
Veamos que $\{t\land k\imp b, \lnot t\imp f,\lnot f\land k\}\models b$
es un argumento correcto: 

\begin{proof}
\locallabelreset

Mostraremos que $b$ se sigue l\'ogicamente de $\G=\{t\land k\imp b, \lnot 
t\imp f,\lnot f\land k\}$, es decir que para toda interpretaci\'on $\I$ tal que 
$\I(\G)=1$ entonces $\I(b)=1$. \\

Suponer $\I$ tal que $\I(\G)=1$, entonces $\I(t\land k\imp b)=1$~\llabel{p1}, 
$\I(\lnot t\imp f)=1$~\llabel{p2} y $\I(\lnot f\land k)=1$~\llabel{p3}. 

\noindent Analizando las suposiciones anteriores, de~\lref{p3} podemos saber 
que $\I(\lnot f) =1$ es decir $\I(f)=0$~\llabel{p4} y que 
$\I(k)=1$~\llabel{p5}.\\
Utilizando~\lref{p4} y siguiendo la definici\'on de interpretaci\'on 
para~\lref{p2} podemos concluir que $\I(\lnot t) = 0$ forzosamente, as\'i que 
$\I(t)=1$~\llabel{p6}. \\
Por lo tanto, $\I(t\land k) =1$ ya que tenemos~\lref{p6} y~\lref{p5} y entonces 
podemos concluir que $\I(b)=1$ de~\lref{p1}.\\
De lo anterior hemos \textbf{construido una interpretaci\'on} tal que $b$ es 
consecuencia l\'ogica de $\G$ y por tanto el argumento es correcto. 
\qed
\end{proof}


\subsection{M\'etodos sem\'anticos para demostrar la correctud de argumentos}

Para mostrar la correctud del argumento l\'ogico 
$\vp_1,\ldots,\vp_n\;/\therefore\; \psi$
mediante interpretaciones, se puede proceder de alguna de las siguientes dos 
formas:

\bi
\item M\'etodo directo: probar la consecuencia $\vp_1,\ldots,\vp_n\models \psi$.

\item M\'etodo indirecto (refutaci\'on): probar que el conjunto
$\{\vp_1,\ldots, \vp_n,\lnot \psi\}$ es insatisfacible.
\ei

Por supuesto que existen otros m\'etodos para mostrar la correctud de un
argumento l\'ogico, que resultan m\'as adecuados para la
automatizaci\'on, como son el c\'alculo exhaustivo de una tabla de verdad 
(sumamente ineficiente),
el uso de tableaux o el m\'etodo de resoluci\'on binaria que veremos m\'as 
adelante.


\subsection{Algunos argumentos correctos relevantes}
\begin{mathpar}
 \inferrule*[Right=Modus Ponens]{
 \vp \\\\ \vp\imp \psi 
}{
\therefore\;\psi
}

 \inferrule*[Right=Modus Tollens]{
 \lnot \psi \\\\ \vp\imp \psi
}{
\therefore\;\lnot \vp
} \\\\

 \inferrule*[Right=Silogismo hip\'otetico]{
 \vp\imp \psi \\\\ \psi\imp \chi
 }{
 \therefore\;\vp\imp \chi
 } \hspace*{2cm}
 
 \inferrule*[Right=Prueba por casos]{
 \vp\imp \chi \\\\ \psi\imp \chi 
 }{
 \therefore\;\vp\lor \psi\imp \chi
 } \\\\
 
 \inferrule*[Right=Introd. $\land$]{
 \vp\\\\ \psi
 }{
 \therefore\;\vp\land \psi
 }
 
 \inferrule*[Right=Elim. $\land$]{
 \vp\land \psi
 }{
 \therefore\;\vp
 }
 
 \inferrule*[Right=Elim. $\land$]{
 \vp\land \psi
 }{
 \therefore\;\psi
 } \\\\
 
 \inferrule*[Right=Introd. $\lor$]{
 \vp
 }{
 \therefore\;\vp\lor \psi
 }
 
 \inferrule*[Right=Introd. $\lor$]{
 \psi
 }{
 \therefore\;\vp\lor \psi
 }\\\\
 
 \inferrule*[Right= Resoluci\'on binaria]{
 \vp\lor \psi \\\\ \lnot \psi\lor \chi
 }{
 \therefore\;\vp\lor \chi
 }\hspace*{2cm}
 
 \inferrule*[Right=Dilema\; constructivo]{
 \vp\imp \psi_1 \\\\ \chi\imp \psi_2 \\\\ \vp\lor \chi
 }{
 \therefore\;\psi_1\lor\psi_2
 }\\\\
 
 \inferrule*[Right=Dilema\; destructivo]{
 \vp\imp \psi_1 \\ \chi\imp \psi_2 \\ \lnot \psi_1 \lor \lnot \psi_2
 }{
 \therefore\;\lnot \vp\lor\lnot \chi
 }
\end{mathpar}



\end{document}