\documentclass[11pt,letterpaper]{article}
\usepackage{../packageslc}
\usepackage{../optionslc}

%%%% Macros para las notas de lenguajes de programacion


%%%% math

\newcommand{\vphi}{\varphi}
\newcommand{\vp}{\varphi}

% \newcommand{\dn}{\mathsf{DN}}
% \newcommand{\dnC}{\mathsf{DN_C}}
% \newcommand{\dnM}{\mathsf{DN_M}}
% \newcommand{\dnp}{\mathsf{DN_p}}
% \newcommand{\dnm}{\mathsf{DN_p^M}}
% \newcommand{\dnc}{\mathsf{DN_p^C}}

%\newcommand{\case}{\mathsf{case}}
%\renewcommand\labelitemi{$\circ$}

\newcommand{\imp}{\rightarrow}
\newcommand{\Imp}{\Rightarrow}
\renewcommand{\iff}{\leftrightarrow}
\newcommand{\Iff}{\Leftrightarrow}
\newcommand{\G}{\Gamma}
\newcommand{\D}{\Delta}


\newcommand{\De}{\mathcal{D}}
\newcommand{\F}{\mathcal{F}}
\newcommand{\Ge}{\mathcal{G}}
\newcommand{\Pe}{\mathcal{P}}
\newcommand{\I}{\mathcal{I}}
\newcommand{\C}{\mathcal{C}}
\newcommand{\K}{\mathcal{K}}
\renewcommand{\L}{\mathcal{L}}
\newcommand{\M}{\mathcal{M}}
\newcommand{\Nc}{\mathcal{N}}
%\newcommand{\E}{\mathcal{E}}
%\newcommand{\R}{\mathcal{R}}
%\newcommand{\Q}{\mathcal{Q}}
\newcommand{\Sc}{\mathcal{S}}
\newcommand{\Te}{\mathcal{T}}
\newcommand{\W}{\mathcal{W}}

\newcommand{\Db}{\mathbb{D}}
\newcommand{\Fb}{\mathbb{F}}
\newcommand{\Kb}{\mathbb{K}}
\newcommand{\Eb}{\mathbb{E}}
\newcommand{\Ebs}{\mathbb{E}^\star}
\newcommand{\Ob}{\mathbb{O}}
\newcommand{\Ib}{\mathbb{I}}
\newcommand{\Rb}{\mathbb{R}}
\newcommand{\Qb}{\mathbb{Q}}
\newcommand{\Kbb}{\mathbb{K}}
\newcommand{\T}{\mathbb{\Theta}}


\newcommand{\kb}{\bbkappa}

\newcommand{\Sf}{\mathsf{\Sigma}}

\newcommand{\fa}{\forall}
\newcommand{\ex}{\exists}

\newcommand{\inc}{\subseteq}

\newcommand{\Lb}{\Lambda}
\newcommand{\Om}{\Omega}
\newcommand{\lb}{\lambda}
\newcommand{\al}{\alpha}
\newcommand{\ga}{\gamma}


\newcommand{\mg}{\mathbb{m}}

\newcommand{\cg}{\mathbb{C}}
\newcommand{\dg}{\mathbb{D}}
\newcommand{\jg}{\mathbb{J}}
\newcommand{\Ha}{\mathcal{H}}
%\newcommand{\A}{\mathcal{A}}
\newcommand{\sg}{\mathbb{S}}

\newcommand{\Bc}{\mathcal{B}}
\newcommand{\Df}{\mathfrak{D}}
\newcommand{\Dc}{\mathcal{D}}
%\newcommand{\Tc}{\mathcal{T}}
\newcommand{\Mf}{\mathfrak{M}}

\newcommand{\Sg}{\mathbb{S}}

\newcommand{\Q}{\ensuremath{\mathbb{Q}}}
\newcommand{\Z}{\ensuremath{\mathbb{Z}}}
\newcommand{\N}{\ensuremath{\mathbb{N}}}
\newcommand{\R}{\ensuremath{\mathbb{R}}}
\renewcommand{\S}{\mathbb{\Sigma}}
\newcommand{\A}{\mathcal{A}}
\newcommand{\E}{\ensuremath{\exists}}
\newcommand{\iso}{\ensuremath{\cong}}
\newcommand{\union}{\ensuremath{\cup}}
\newcommand{\morinyec}{\ensuremath{\precapprox}}

\newcommand{\nin}{\ensuremath{\notin}}
\newcommand{\tog}{\makebox[7mm][l]}
\newcommand{\toge}{\makebox[11mm][l]}
\newcommand{\toget}{\makebox[13mm][l]}
\newcommand{\togeth}{\makebox[14mm][l]}
\newcommand{\togethe}{\makebox[15mm][l]}
\newcommand{\together}{\makebox[17mm][l]}
\newcommand{\niso}{\ensuremath{\not \cong}}


\newcommand{\Mg}{\mathbb{M}}
\newcommand{\Bg}{\mathbb{B}}
\newcommand{\Lg}{\mathbb{L}}
\newcommand{\Tg}{\mathbb{T}}

\newcommand{\sketch}{\Red{{\sc sketch}}}

\newcommand{\restr}[2]{#1\!\!\boldsymbol{\restriction}\!#2}

\newcommand{\vacio}{\varnothing}
\newcommand{\done}{\ensuremath{\checkmark}}

\newcommand{\ida}{$\Rightarrow \; )$ }
\newcommand{\regr}{$\Leftarrow \; )$ }

\newcommand{\ol}[1]{\overline{#1}}

\newcommand{\Tsf}{\mathsf{T}}

\newcommand{\inds}[1]{\index[simb]{#1}}

\newcommand{\B}{\mathbb{B}}
%\newcommand{\N}{\mathbb{N}}

\newcommand{\vx}{\vec{x}}
\newcommand{\vy}{\vec{y}}
\newcommand{\vz}{\vec{z}}
\newcommand{\vt}{\vec{t}}
\newcommand{\vf}{\vec{f}}


% \newcommand{\propo}{\ensuremath{\mathsf{PROP}}}
% \newcommand{\atom}{\ensuremath{\mathsf{ATOM}}}
\newcommand{\term}{\ensuremath{\mathsf{TERM}}}
\newcommand{\form}{\mathsf{FORM}}

\newcommand{\true}{\mathop{\mathsf{true}}}

%\newcommand{\id}{\mathsf{Id}}

%\newcommand{\uc}{\mathcal{U}}
%\newcommand{\Ic}{\mathcal{I}}
%\newcommand{\pc}{\mathcal{P}}
%\newcommand{\qc}{\mathcal{Q}}
%\newcommand{\mc}{\mathcal{M}}
\newcommand{\supc}{\supseteq}
\newcommand{\limo}{\mathop{\mathpzc{Lim}}}
\newcommand{\ord}{\mathsf{OR}}

\newcommand{\pt}[1]{\langle #1 \rangle}


%%%% frames
\newcommand{\titulos}[2]{\frametitle{#1}\framesubtitle{#2}}
\newcommand{\fot}[1]{\footnote{\scriptsize{#1}}}


%%%% ambientes

\newcommand{\cb}[2]{\colorbox{#1}{#2}}

\newcommand{\bc}{\begin{center}}
\newcommand{\ec}{\end{center}}
\newcommand{\be}{\begin{enumerate}}
\newcommand{\ee}{\end{enumerate}}
\newcommand{\bi}{\begin{itemize}}
\newcommand{\ei}{\end{itemize}}
\newcommand{\beq}{\begin{equation}}
\newcommand{\eeq}{\end{equation}}
\newcommand{\beqs}{\begin{equation*}}
\newcommand{\eeqs}{\end{equation*}}
\newcommand{\ba}{\begin{array}}
\newcommand{\ea}{\end{array}}


% \newtheorem{theorem}{Teorema}
% \newcommand{\teo}[1]{\begin{theorem} #1 \end{theorem}}
% \newtheorem{proposition}{Proposici\'on}
% \newcommand{\prop}[1]{\begin{proposition} #1 \end{proposition}}
% \newtheorem{definition}{Definici\'on}
% \newcommand{\defin}[1]{\begin{definition} #1 \end{definition}}
% \newtheorem{corollary}{Corolario}
% \newcommand{\cor}[1]{\begin{corollary} #1 \end{corollary}}
% \newtheorem{lemma}{Lema}
% \newcommand{\lema}[1]{\begin{lemma} #1 \end{lemma}}
% \newcommand{\dem}[1]{\begin{proof} #1 \end{proof}}

%\renewcommand{\qed}{\qedsymbol{$\mathbf{\dashv}$}}

%\newcommand{\proof}{\hfill\\\noindent\textbf{\textit{Demostraci\'on. }}}

\newcommand{\hint}{\emph{Sugerencia: }}


\newcounter{EjempCtr}[section]
\newenvironment{enumrom}{\renewcommand{\theenumi}{\roman{enumi}}%
\renewcommand{\theenumii}{\roman{enumii}}
\renewcommand{\theenumiii}{\roman{enumiii}}
\renewcommand{\theenumiv}{\roman{enumiv}}
\begin{enumerate}}{\end{enumerate}}
\newenvironment{Ejemplo}
        {\stepcounter{EjempCtr}%
        \begin{description}\item[Ejemplo \thesection.\arabic{EjempCtr}]}%
        {\end{description}}
\newenvironment{demostr}{%
             {\em Demostración:}
                \begin{quotation}}{\end{quotation}}

   \newcommand{\beje}{\begin{Ejemplo}}
\newcommand{\eeje}{\end{Ejemplo}}


\newtheorem{eje}{Ejemplo}[section]
\newcommand{\ejem}[1]{\begin{eje}\normalfont #1 \end{eje}}

% \renewcommand\contentsname{\'Indice}
%\renewcommand\chaptername{Cap\'itulo}
% \renewcommand\indexname{\'Indice}

%%\newcommand{\qed}{\hfill$\mathbb{Qed}$}
%\newcommand{\qed}{\hfill$\mathsf{\boldsymbol{\dashv}}$}
%\renewcommand{\qed}{\hfill$\boldsymbol{\dashv}$}


\newenvironment{prueba}{\vspace{-5mm}\noindent\textbf{Demostraci\'on}\\}{
\noindent$\blacksquare$\\}

\newcommand{\Ejercicios}{\section*{Ejercicios}}

 \newenvironment{manitas}{%
      \renewcommand{\labelitemi}{\ding{44}}%
      \vspace{-0.5cm}%
      \begin{itemize}%
      \setlength{\itemsep}{0pt}\setlength{\parsep}{0pt}\setlength{\topsep}{0pt}%
      }{\end{itemize}}
\newenvironment{malitos}{%
      \renewcommand{\labelitemi}%
            {\raisebox{1.5ex}{\makebox[0.3cm][l]{\begin{rotate}{-90}%
            \ding{43}\end{rotate}}}}%
      \vspace{-0.5cm}%
      \begin{itemize}%
      \setlength{\itemsep}{0pt}\setlength{\parsep}{0pt}\setlength{\topsep}{0pt}%
      }{\end{itemize}}
\newenvironment{ejercs}{
     \renewcommand{\labelenumi}{\thesection.\theenumi.-}
     \renewcommand{\labelenumii}{\theenumii)}
     \begin{enumerate}}
     {\end{enumerate}}

   \newcommand{\bej}{\begin{ejercs}}
\newcommand{\eej}{\end{ejercs}}


%\newenvironment{leterize}{%
%        \renewcommand{\theenumi}{\alph{enumi}}
%        \begin{enumerate}}{\end{enumerate}}

%\newenvironment{manitas}{%
%      \renewcommand{\labelitemi}{\ding{44}}%
%      \vspace{-0.5cm}%
%      \begin{itemize}%
%      
% \setlength{\itemsep}{0pt}\setlength{\parsep}{0pt}\setlength{\topsep}{0pt}%
%      }{\end{itemize}}



%%=============================================================================

\def\stackunder#1#2{\mathrel{\mathop{#2}\limits_{#1}}}


%%%% notas

\newcommand{\doubt}{\Red{{\LARGE {\sf ??}}}}

\newcommand{\coment}[1]{\hfill\\ \Big[{\bf Comentario Privado:} #1\Big]}
\newcommand{\preg}[1]{\hfill\\ \BrickRed{{\bf Pregunta:} #1}}
\newcommand{\conjet}[1]{\hfill\\ \OliveGreen{{\bf Conjecura:} #1}}

\newcommand{\pendiente}{\BrickRed{{\sc Pendiente}}}
\newcommand{\verifpendiente}{\BrickRed{{\sc Verificación pendiente}}}


%--------------------------------------------------------------------------

\DeclareMathAlphabet{\mathpzc}{OT1}{pzc}{m}{it}


\title{Lógica Computacional 2017-2, nota de clase 13 \\
Lógica Ecuacional
}
\author{Favio Ezequiel Miranda Perea \and Araceli Liliana Reyes Cabello\and
Lourdes Del Carmen Gonz\'alez Huesca \and Pilar Selene Linares Arévalo}
\date{8 de mayo de 2017 \\
Material desarrollado bajo el proyecto UNAM-PAPIME PE102117}

\begin{document}
\maketitle

\section{Introducción}
Durante este curso hemos enfocado nuestra atención en el estudio de la validez 
de enunciados lógicos y métodos para hacerlo. 
Consideremos las siguientes proposiciones que involucran una noci\'on de 
igualdad y analicemos su posible valor de verdad:
\begin{proposition} $p \land q \equiv q \land p$\end{proposition}\vspace*{-7pt}
¿Qué significa exactamete que esta proposición sea válida? ¿Es posible dar un 
contraejemplo a esta proposición?
Para mostrar que la sentencia es válida en lógica proposicional, podríamos 
proceder realizando la tabla de verdad (tabla con cuatro renglones)y verificar 
que no existe un contraejemplo, es decir, una asignación de valores para 
las variables involucradas de tal forma que $p \land q \iff q \land p$ se 
evalúe a falso.

% Consideremos ahora  la siguiente proposición:
\begin{proposition} $ x+y = y+x$  \end{proposition}\vspace*{-7pt}
Suponiendo un significado aritm\'etico a esta proposici\'on, ¿podemos dar un 
contrajemplo a la proposición? ¿Cómo demostramos que ésta válida? ¿Podríamos 
realizar algo similar al caso de lógica proposicional? Desafortunadamente, los 
valores que podemos asignarle a las variables es infinito, lo cual significa que 
no podríamos terminar de completar la tabla de asignaciones. Por lo tanto, 
necesitamos una forma diferente de demostrar que una proposición como la 
anterior es válida sin recurrir a su tabla de asignaciones. 

\bigskip

Aunado a lo anterior, hay una pregunta importante que nos debemos hacer: 
¿Cuándo una igualdad se puede inferir a partir de un conjunto dado de 
ecuaciones? Por ejemplo, a partir de las ecuaciones:
\[
\ba{rcl}
x + y& = & y + x\\ 
x + (y + z)& = & (x + y) + z
\ea
\]

\noindent podemos concluir que $w_{1} + (w_{2} + w_{3}) = (w_{3} + w_{2}) + w_{1}$ pero no podemos afirmar que $w_{1}+ w_{1} = w_{1}+ w_{2}$. \\

% Consideremos la siguiente proposición:
% \begin{proposition} $ p \land q \equiv q \land p$
% \end{proposition}
% 
% ¿Qué significa exactamete que la proposición 1 es válida? ¿Es posible dar un contraejemplo a esa proposición?
% Para mostrar que la sentencia es válida en lógica proposicional, podríamos proceder realizando la tabla de verdad (tabla con cuatro renglones)  y verificar que no existe un contraejemplo, es decir, una asignación de valores para las variables involucradas de tal  forma que $p \land q \iff q \land p$ se evalúe a falso.  \\
% 
% Consideremos ahora  la siguiente proposición:
% \begin{proposition} $ x+y = y+x$
% \end{proposition}
% 
% ¿Podemos dar un contrajemplo a la proposición? ¿Cómo demostramos que ésta es válida? ¿Podríamos realizar algo similar al caso de lógica proposicional? Desafortunadamente, los valores que podemos asignarle a las variables es infinito, lo cual significa que no podríamos terminar de completar la tabla de asignaciones. Por lo tanto, necesitamos una forma diferente de demostrar que una proposición como la anterior es válida sin recurrir a su tabla de asignaciones. 


Hasta ahora hemos trabajado con l\'ogica de proposiciones y de predicados
y en ellos hemos usado una noci\'on de equivalencia entre expresiones, 
$e_1 \equiv e_2$, que est\'a fuertemente basada en la sem\'antica formal de 
cada l\'ogica.


 
En esta nota, estudiaremos la Lógica Ecuacional, 
% un fragmento de la Lógica de Predicados 
una teoría que trabaja sobre expresiones que involucran al operador de 
igualdad, que reflejan la idea subyacente de objetos equiparables y no 
necesariamente de forma sint\'actica. 
La lógica ecuacional juega un papel importante en el razonamiento sobre 
equivalencias lógicas, igualdades algebraicas, definiciones de programas, 
propiedades de estructuras de datos, entre otros.  \\
Veamos algunos ejemplos en donde se ocupa un razonamiento puramente 
ecuacional para demostrar:


\paragraph{L\'ogica proposicional}
% Hasta ahora hemos trabajado con l\'ogica de proposiciones y de predicados
En esta lógica hemos usado una noci\'on de equivalencia entre 
expresiones~$e_1 \equiv e_2$, que est\'a  basada en la sem\'antica formal. 
Pero si se desean realizar diversas equivalencias resulta tedioso hacerlas 
formalmente y para ello hemos recurrido a utilizar s\'olo equivalencias 
demostradas previamente para justificar cada paso de alguna secuencia
del estilo $e_1 \equiv e_2 \equiv \ldots \equiv e_n$.\\

Las equivalencias de f\'ormulas proposicionales son usadas para simplificar o 
referirse a f\'ormulas cuyo valor booleano es el mismo, por ejemplo:
\[
\ba{rlll}
(p\lor q)\land\lnot(\lnot p\land q)&\equiv &
(p\lor q)\land (\lnot\lnot p\lor\lnot q) & \text{distributividad negaci\'on}\\
& \equiv & (p\lor q)\land (p\lor\lnot q) & \text{eliminaci\'on doble neg}\\
& \equiv & p\lor (q\land\lnot q) & \text{asociatividad}\\
& \equiv & p\lor \bot & \text{absorci\'on}\\ 
& \equiv & p & \text{neutro}
\ea
\]
Este razonamiento tambi\'en deber\'ia expresarse utilizando igualdad en lugar 
de equivalencia. 


\paragraph{Programa funcional}
%cap 13 G. Hutton 
% http://lya.fciencias.unam.mx/favio/pf/programming_in_Haskell.pdf
% Considera la definci\'on de listas en {\hsk}, una propiedad interesante a 
% demostrar es que la funci\'on \verb=reverse= no tiene ning\'un efecto sobre 
% de la funci\'on \verb-member-, es decir:
% \begin{center}
%  Si se cumple que \verb=member x l= para alg\'un elemento y una lista, \\
%  entonces tambi\'en se debe cumplir que \verb=member x (reverse l)=.
% \end{center}
% Veamos c\'omo demostrar esta propiedad:
% \begin{lstlisting}
%  reverse :: [a ] -> [a ]
%  reverse [] = []
%  reverse (x : xs) = reverse xs ++ [x]
% \end{lstlisting}

Consideremos la definici\'on de \'arboles binarios de n\'umeros naturales en 
{\hsk}:
\begin{lstlisting}
data Tree = Leaf Nat | Node Tree Tree
\end{lstlisting}
El aplanado de \'arboles es una funci\'on que dado un \'arbol devuelve 
una lista con todos sus elementos:
\begin{lstlisting}
flatten :: Tree -> [Nat]
flatten (Leaf n) = [n]
flatten (Node l r) = flatten l ++ flatten r
\end{lstlisting}

La funci\'on anterior es ineficiente, tiene complejidad cuadr\'atica: 
si el n\'umero de nodos del sub\'arbol izquierdo es $m_1$ entonces append y 
\verb-flatten- est\'an recorriendo esos $m_1$ elementos respectivamente,
adem\'as de que se visita cada hoja del sub\'arbol derecho $m_2$ para obtener 
la lista de elementos del árbol original. 

\bigskip

Si quisi\'eramos mejorar la definici\'on de \verb=flatten= debemos eliminar el 
uso de append, para esto usaremos una especificaci\'on m\'as general de 
aplanado de \'arboles utilizando una lista auxiliar:
\begin{center} \verb!flatten' t ns = flatten t ++ ns! \end{center}
Buscaremos una definici\'on que la cumpla tal especificaci\'on por medio de 
inducci\'on:
\bi
\item Caso Base: 
\begin{lstlisting}
flatten' (Leaf n) ns
        = { especificacion de flatten' }
flatten (Leaf n) ++ ns
        = { aplicando flatten }
[n] ++ ns
        = { aplicando ++ }
n:ns
\end{lstlisting}

% \newpage

\item Caso Inductivo:~\footnote{Donde la hip\'otesis de inducci\'on supone 
verdadera la propiedad para los sub\'arboles \texttt{l} y \texttt{r}.}
\begin{lstlisting}
flatten' (Node l r) ns
        = { especificacion de flatten' }
(flatten l ++ flatten r ) ++ ns
        = { asociatividad de ++ }
flatten l ++ (flatten r ++ ns)
        = { hipotesis induccion para l }
flatten' l (flatten r ++ ns)
        = { hipotesis induccion para r }
flatten' l (flatten' r ns)
\end{lstlisting}
\ei

Hemos obtenido una definici\'on mejorada que no utiliza la definici\'on de 
append y que usa una lista auxiliar que hace las funciones de 
\textbf{acumulador}:
\begin{lstlisting}
flatten' :: Tree -> [Nat] -> [Nat]
flatten' (Leaf n) ns = n : ns
flatten' (Node l r) ns = flatten' l (flatten' r ns)
\end{lstlisting}
De este ejemplo quisi\'eramos referirnos a la equivalencia en ambas 
definiciones, esto se puede describir como la igualdad de resultados al aplicar 
cualquiera de ellas al mismo árbol:
\begin{center}
\textit{para cualquier lista} \verb-l-, \textit{se cumple que} \; 
\verb-flatten l = flatten' l []-
\end{center}

\medskip

El ejemplo anterior es un razonamiento sobre la especificaci\'on de estructuras 
de datos (\'arboles, listas, n\'umeros naturales) y funciones de ellas 
(aplanado, concatenaci\'on, etc.). Una demostraci\'on utilizando el lenguaje de 
l\'ogica de predicados resultar\'ia en un ejercicio m\'as extenso y con 
notaci\'on l\'ogica que hemos usado hasta ahora: un lenguaje de predicados~$\L$ 
para las estructuras, funciones y relaciones y un modelo para \'este $\M_{\L}$.
As\'i la pregunta anterior se traduce en demostrar que:
\[
 \M_{\L} \models \forall x \big(Tree(x) \to flatten(x) = flatten'(x,nil) \big)
\]

% Veamos otro ejemplo de especificaci\'on formal:

\paragraph{Estructura de datos}
Utilizando la formalizaci\'on de listas polim\'orficas en l\'ogica de 
predicados, queremos demostrar la siguiente propiedad:
\begin{center}
 \textit{Cualquier lista no vac\'ia puede descomponerse en dos listas y un 
elemento\\ tal que ese elemento est\'a en una posici\'on intermedia en la lista 
original.}
\end{center}
Las siguientes f\'ormulas de primer orden corresponden a la formalizaci\'on de 
la propiedad, donde los elementos de la lista son de cierto tipo~$A$ 
\be
\item Funcionamente:
\[
\forall x \Big( L_{A}(x) \land x \neq nil \to 
    \big(\exists y,z,w \big(L_{A}(y) \land L_{A}(z) \land A(w) \land 
      x=app(y,cons(w,z))\big) \big) \Big)
\]
\item Relacionalmente:
\[
\forall x \Big( L_{A}(x) \land x \neq nil \to 
    \big(\exists y,z,w \big(L_{A}(y) \land L_{A}(z) \land A(w) \land 
      App(y,cons(w,z),x)\big) \big) \Big)
\]
\ee
Nuevamente para este ejemplo se requiere de un modelo dado por la 
formalizaci\'on vista en la nota de especificaci\'on formal.


\bigskip


Para aligerar estas demostraciones, se puede usar un razonamiento ecuacional 
para decidir si una ecuaci\'on en particular se sigue de un conjunto de `
ecuaciones dado como por ejemplo en programas funcionales.\\
Por tanto requerimos de una noci\'on formal para el manejo de igualdades que 
forme parte de cualquier lenguaje formal y que presentamos a continuación.
%En esta nota estudiaremos a fondo este concepto que tambi\'en es \'util en 
%otras \'areas de computaci\'on.

\subsection{Reglas y sistemas de deducci\'on}

Para describir las reglas que emplearemos en la Lógica ecuacional y que 
permiten 
\textit{encadenar} razonamientos respecto a ecuaciones, utilizaremos una 
representaci\'on que relaciona dos enunciados mediante una l\'inea horizontal.

Estos esquemas son usados en muchas \'areas matem\'aticas y computacionales ya 
pueden leerse en dos sentidos: de arriba hacia abajo y viceversa, dependiendo 
de 
un prop\'osito en el desarrollo de una demostraci\'on: generar, derivar o 
deducir informaci\'on.

Un sistema de deducci\'on tiene reglas b\'asicas o axiomas que son reglas que 
cuentan \'unicamente con la parte inferior para representar hechos o 
conclusiones sin premisas:
\begin{mathpar}
 \inferrule*[]{
 }{
 \quad B\quad 
 }
\end{mathpar}
Tambi\'en tienen reglas de deducci\'on que cuentan con ambas partes, premisas y 
conclusiones, \'utiles en la generaci\'on de informaci\'on:
\begin{mathpar}
 \inferrule*[]{
 \quad A_1\;A_2\; \ldots \;A_m \quad 
 }{
 \quad B  \quad 
 }
\end{mathpar}


En sentido de arriba hacia abajo, la regla de deducción anterior se puede leer 
como sigue: si sucede que se cumplen $A_{1}, A_{2}, ..., A_{m}$ entonces podemos 
concluir $B$. En sentido contrario (de abajo hacia arriba), la regla se puede 
interpretar como sigue: para demostrar $B$ basta con demostrar que se cumplen 
$A_{1}, A_{2}, ... , A_{n}$.


\defin{
Un sistema de deducci\'on para una teor\'ia~$\T$, es un sistema que consta de 
un n\'umero finito de reglas y axiomas que establecen la relaci\'on 
entre enunciados de $\T$.}

\defin{
Una deducci\'on o derivaci\'on de un enunciado $\J$ usando un sistema de 
deducci\'on, es una secuencia finita 
$\Pi = \langle \J_1,\dots,\J_k \rangle$ tal que $\J_k = \J$ y cada 
$\J_i$, $1\leq i \leq k$ es una instancia de una regla del sistema de 
deducci\'on. Tambi\'en puede usarse una representaci\'on de \'arbol de 
derivaci\'on donde la ra\'iz est\'a en la parte inferior y es el enunciado 
$\J$ y cada hoja del \'arbol es una instancia de un axioma.}

Los enunciados sobre ecuaciones que deseamos representar dependen de una serie 
de ecuaciones o de informaci\'on preestablecida, para ello usaremos de manera 
\textbf{impl\'icita} esta \enquote{base de datos} en cada derivaci\'on. 

\section{L\'ogica ecuacional}
% Del libro Logic: A foundation for computer science, Sperschnneider & 
% Antoniou

Un sistema de deducci\'on para razonar respecto a ecuaciones es el llamado 
C\'alculo de Birkhoff, el cual contiene cinco reglas:
\begin{mathpar}
   \inferrule*[right=Hyp]{
   }{
   \quad  {H} \quad 
   }
   
   \inferrule*[right=Refl]{
   }{
    \quad {t=t} \quad 
   }
  
   \inferrule*[right=Sym]{
    \quad {s=t} \quad 
   }{
    {t=s}
   }
  
   \inferrule*[right=Trans]{
    {t = r} \and {r = s}
   }{
    \quad {t=s}\quad 
   }\\
  
   \inferrule*[right=Inst]{
    {t=s}
   }{
    \quad {t\sigma=s\sigma} \quad 
   }
  
   \inferrule*[right=Congr]{
    {t_1=s_1} \quad \dots \quad {t_n=s_n}
   }{
    \quad  {f(t_1,\ldots,t_n) = f(s_1,\ldots,s_n)} \quad 
   }
 \end{mathpar}
\noindent 
El axioma b\'asico es el que permite obtener cierta informaci\'on conocida o 
\enquote{informaci\'on de la base de datos}. \\
El resto de las reglas hacen \'enfasis en ciertas ecuaciones y
\be
\item establecen las caracter\'isticas de una relaci\'on 
sim\'etrica y transitiva (\textsc{Refl}, \textsc{Sym}, \textsc{Trans})
\item abstraen patrones por medio de instanciaci\'on (\textsc{Inst})
\item construyen nuevos t\'erminos por medio de congruencia entre subtérminos 
(\textsc{Congr})
\ee

Estas reglas son suficientes para hacer demostraciones pero se pueden a\~nadir 
algunas otras reglas que factoricen aplicaciones de varias reglas asegurando 
una demostraci\'on o derivaci\'on con menos pasos, de hecho se pueden utilizar 
las siguientes en lugar de las reglas de \textsc{Inst} y \textsc{Congr}:
\begin{mathpar}
 \inferrule*[right=Rewrite]{
    {t=s} \and {E[x:=s\sigma]} 
  }{
    {t=s}{E[x:=t\sigma]}
  }
  
 \inferrule*[right=Rewrite <-]{
    {t=s} \and {E[x:=t\sigma]}   %\frac{s=t\;\;\;\;E[t]}{E[s]}
   }{
    {E[x:=s\sigma]}
   }
\end{mathpar}
En las reglas de reescritura $E[t]$ denota a una ecuaci\'on haciendo 
expl\'icita una presencia de algún término~$t$, por ejemplo si 
$E=_{def} x+2=f\,y$ podemos hacer expl\'icitas cualquiera de las subexpresiones 
involucradas en $E$, usando $E[x], E[x+2], E[f\,y]$. \\


% Veamos otros ejemplos:
% \beje[Ecuaciones de t\'erminos]
% Recordemos la gram\'atica para t\'erminos de cualquier lenguaje de primer 
% orden:
% $$ t ::=  x \mid c \mid f(t_1,t_2,\ldots,t_n) $$
% \eeje
% 
% \beje[otro ejemplo]
% Aquí me falta el ejemplo de las ecuaciones con funciones pero quiero ver si es 
% posible escribirlo de una manera menos fea que el que encontré.
% \eeje
% 

\section{Formalizaci\'on de teor\'ias}
%http://lya.fciencias.unam.mx/favio/pf/%5bRichard_Bird%5d_Thinking_Functionally_
% with_Haskell(BookZZ.org).pdf
El sistema de l\'ogica ecuacional presentado ayuda a realizar demostraciones 
de ecuaciones de forma clara. 
As\'i una derivaci\'on finita utilizar\'a unicamente estas reglas y si se desea 
realizar un \'arbol de derivaci\'on, entonces las hojas ser\'an instancias de 
los axiomas \textsc{Hyp} o \textsc{Refl}.
Una aplicaci\'on de esto es la formalizaci\'on de teor\'ias que requieren de 
ecuaciones para la demostraci\'on de propiedades o especificaciones de los 
elementos en la teor\'ia. 
Veamos dos ejemplos extendidos:

\beje[Estructura algebraica]
Considera el siguiente conjunto de ecuaciones~$\A$, que define la 
estructura de anillo~\footnote{Un anillo es un conjunto $R$ junto 
con dos operaciones $(R,+,\cdot)$, un elemento distinguido $0$ y una 
operaci\'on unaria $^{-1}$.}:
 \[
  \begin{array}{rclrcrclr}
  0 + x & = & x & A1 & \hspace{20pt} & x^{-1} + x & = & 0 & A2\\
  x + y & = & y + x & A3 & & (x + y) + z & = & x + (y + z) & A4\\
  x \cdot 1 & = & x & A5 & & (x\cdot y)\cdot z & = & x\cdot (y\cdot z) & A6\\ 
  x\cdot (y+z) & = & x\cdot y + x\cdot z & A7&& 
    (y+z) \cdot x & = & y\cdot x + z\cdot x & A8
  \end{array}
 \]
Demuestra que $x + (y + z) = y + (x + z)$. \\
% teniendo como hip\'otesis adicional que~$z\cdot z = z$ para cualquier 
% elemento~$x$ en $R$.
Recordemos que la base de datos es $\A$.
\locallabelreset
 \[
  \begin{array}{lll}
   \llabel{p1} & {(x + y) + z = x + (y + z)} 
      & \textsc{Hyp\; A5} \\
   \llabel{p2} & {x + (y + z) = (x + y) + z} 
      & \textsc{Sym}~\lref{p1} \\
   \llabel{p3} & {z = z} 
      & \textsc{Refl} \\
   \llabel{p4} & {x + y = y + x} 
      & \textsc{Hyp\; A4} \\
   \llabel{p5} & {(x + y) + z = (y + x) + z} 
      & \textsc{Congr}\; + ~\lref{p4}~\lref{p3}\\
   \llabel{p6} & {(y + x) + z = y + (x + z)} 
      & \textsc{Inst}~\lref{p1} \\
   \llabel{p7} & {x + (y + z) = (y + x) + z} 
      & \textsc{Trans}~\lref{p2}\lref{p5} \\
   \llabel{p8} & {x + (y + z) = y + (x + z)} 
      & \textsc{Trans}~\lref{p7}\lref{p6} 
  \end{array}
 \]
\eeje

\newpage

\beje[Estructura de datos]
Del ejemplo presentado al inicio:
\begin{center}
 \textit{Cualquier lista no vac\'ia puede descomponerse en dos listas y un 
elemento\\ tal que ese elemento est\'a en una posici\'on intermedia en la lista 
original.}
\end{center}
consideremos aqu\'i las listas de n\'umeros naturales. 
La base de datos correspondiente para la prueba tiene la definici\'on de 
listas as\'i como las definiciones de funciones sobre listas.

Esta demostraci\'on puede realizarse  considerando la hip\'otesis sobre la 
forma 
de la lista a saber $\ell\neq[\;]$ y donde el \'unico caso a analizar es $\ell= 
n::\ell'$. Exhibiremos la partici\'on trivial donde $x = [\;]$:
\bi
\item Demostrar que los elementos \verb!x = []!, \verb!z = n! y \verb!y = l'! 
tales que \verb!x++(z::y) = n::l'!, suponiendo que \verb!l = n:: l'!
\locallabelreset
 \[
 \begin{array}{lll}
  \llabel{p1} & x = [\;] & \textsc{Hyp} \\
  \llabel{p2} & z = n & \textsc{Hyp} \\
  \llabel{p3} & y = \ell' & \textsc{Hyp} \\
  \llabel{p4} & z::y = n::\ell' & \textsc{Congr}\; :: \;\lref{p2}\lref{p3}\\
  \llabel{p5} & z::y = z::y & \textsc{Refl} \\
  \llabel{p6} & x++(z::y) = [\;]++(z::y) & 
    \textsc{Congr}\;++\; \lref{p5} \lref{p1}\\
  \llabel{p7} & [\;]++(z::y) = z::y & \textsc{Ins}\; (\verb-nil ++ x = x- )\\
  \llabel{p8} & x++(z::y) = z::y & \textsc{Trans}\;\lref{p6}\lref{p7} \\
  \llabel{p9} & x++(z::y) = n::\ell' & \textsc{Trans}\;\lref{p4}\lref{p8}
 \end{array}
 \]
\ei
\eeje


\subsection{Guiar una demostraci\'on}
Las demostraciones de los ejemplos \textbf{3.1} y \textbf{3.2} pueden resultar 
elaboradas y con algunos pasos que resultan obvios, por ejemplo los pasos en 
donde se presentan las hip\'otesis o el uso de reflexividad.

Adem\'as buscamos explotar nuestra herramienta de trabajo, la computadora. Para 
ello utilizaremos un asistente de pruebas que facilitar\'a la formalizaci\'on y 
las demostraciones sobre alguna teor\'ia o sistema, por ejemplo en este caso 
nuestro sistema de razonamiento ecuacional.

Decimos que un asistente de pruebas es un sistema (software) que ayuda en el 
desarrollo de demostraciones formales mediante la interacci\'on con el usuario. 
El asistente o demostrador puede resolver algunos pasos de las demostraciones 
pero el usuario siempre est\'a ah\'i para guiar el proceso inclusive dar 
instrucciones para que se realice de manera casi autom\'atica.

El sistema de deducci\'on~$\mathcal{B}$ y algunos refinamientos del mismo 
resultan ser subconjuntos de t\'acticas del asistente de pruebas 
{\coq}~\footnote{\url{http://coq.inria.fr/}}.
Este asistente interact\'ua con el usuario mediante comandos y t\'acticas, 
estas \'ultimas son funciones que ayudan a descomponer un enunciado a demostrar 
en partes que resultan ser m\'as f\'aciles de demostrar y que una 
composici\'on final de todas las subdemostraciones asegura la prueba del 
enunuciado original.

Las t\'acticas correspondientes a las reglas del sistema~$\mathcal{B}$ son:
\begin{center}
\begin{tabular}{c|c}
 Regla & T\'actica \\\hline
 \textsc{Hyp} & \verb=assumption= \\
 \textsc{Refl} & \verb=reflexivity= \\
 \textsc{Sym} & \verb=symmetry= \\
 \textsc{Trans} & \verb=transitivity r= \\
 \textsc{Inst} & \verb=apply= \\
 \textsc{Congr} & \verb=rewrite=  ($n$ veces)\\
 \textsc{Rewrite} & \verb=rewrite= \\
 \textsc{Rewrite <-} & \verb=rewrite <-= \\
\end{tabular}
\end{center}

El uso de estas t\'acticas sigue el sentido inverso (\textit{backward 
reasoning}) en la construcci\'on de demostraciones de {\coq}, es decir que las 
reglas del sistema $\mathcal{B}$ ser\'an leidas y aplicadas de abajo hacia 
arriba.
Por esta raz\'on la regla de transitividad debe incluir el t\'ermino \verb-r- 
que permite pasar de \verb-s- a \verb-t-. 
Tambi\'en debe considerarse que no siempre es posible tener todas las 
ecuaciones de la base de datos en el contexto de prueba, as\'i la aplicaci\'on 
de la t\'actica \verb=assumption= ser\'a sustituida por las t\'acticas 
\verb=apply= o \verb=simpl=, la \'ultima realiza una simplificaci\'on de un 
t\'ermino. 
Veamos otros ejemplos:
\beje[Propiedades de la concatenaci\'on de listas]
Consideremos la implementaci\'on de listas en~{\coq} y algunas funciones:
\begin{lstlisting}
Require Import List.
Import ListNotations.

Section MyListSection.
Variable A : Type.

(* Recordatorio de las definiciones como aparecen en las bibliotecas de Coq
Inductive list (A : Type) : Type :=
 | nil : list A
 | cons : A -> list A -> list A.
 
Definition app (A : Type) : list A -> list A -> list A :=
  fix app l m :=
  match l with
   | nil => m
   | a :: l1 => a :: app l1 m
  end.
Infix "++" := app (right associativity, at level 60) : list_scope.
*)
\end{lstlisting}
Para demostrar que la concatenaci\'on asociativa, agregaremos dos axiomas a la 
base de datos que describen la conmutatividad de la concatenaci\'on respecto a 
la lista vac\'ia:
\begin{lstlisting}
Axiom axapp_nil_l : forall l:list A, [] ++ l = l.
Axiom axapp_nil_r : forall l:list A, l ++ [] = l.
\end{lstlisting}
Y un teorema auxiliar:
\begin{lstlisting}
Theorem aux_app_comm_cons : 
   forall (x y:list A) (a:A), a :: (x ++ y) = (a :: x) ++ y.
Proof.
induction x.
(* l = [] *)
+ intros. rewrite axapp_nil_l. simpl app. reflexivity.
(* l = a :: l *)
+ intros. rewrite <- IHx. simpl app.
reflexivity.
Qed.
\end{lstlisting}
As\'i el teorema se enuncia y demuestra como sigue:
\begin{lstlisting}
Theorem app_assoc : 
   forall l m n:list A, l ++ m ++ n = (l ++ m) ++ n.
Proof.
induction l.
(* l = [] *)
+ intros. apply app_nil_l.
(* l = a :: l *)
+ intros. rewrite <- aux_app_comm_cons. 
rewrite IHl. 
reflexivity.
Qed.
\end{lstlisting}
Hay que observar que la t\'actica \verb=intro= o \verb=intros= hace que {\coq} 
incluya todas las hip\'otesis de un enunciado. De esta forma se agrega m\'as 
informaci\'on a la base de datos.
\eeje

\beje
Una formalizaci\'on alternativa usando {\coq} del ejemplo \textbf{3.2} resulta 
en el siguiente c\'odigo:
\begin{lstlisting}
Theorem list_partition : 
   forall l:list nat, l <> nil -> exists (z:nat) (x y:list nat), l=x++(z::y).
\end{lstlisting}

La base de datos correspondiente para la prueba tiene la definici\'on de 
listas que aparece en el archivo \verb=Coq.Init.Datatypes= y las 
definiciones de funciones sobre listas dadas en \verb=Require Import List.=

Por otro lado, tambi\'en se puede considerar una implementaci\'on de una 
funci\'on que calcule \textit{todas} las particiones posibles para una lista 
dada:
\begin{lstlisting}
Fixpoint aux_partition (n: nat) (l:list nat) : list (list nat*nat*list nat) :=
   match n with 
   | 0 => nil
   | S n => ((firstn n l),(nth n l 0),(skipn (S n) l)) :: aux_partition n l
   end.

Definition partition (x:list nat) : list (list nat*nat*list nat) :=
   let l := length x in aux_partition l x.
\end{lstlisting}
Y despu\'es demostrar que esta implementaci\'on cumple con la 
especificaci\'on dada por la propiedad:
\begin{lstlisting}
Theorem func_partition : 
   forall (l:list nat) (z:nat) (x y:list nat), 
   In (x,z,y) (partition l) <-> l = x++(z::y).
\end{lstlisting}

Este \'ultimo ejemplo requiere de una regla con m\'as poder que las incluidas 
en el sistema de deducci\'on $\mathcal{B}$, es necesario esclarecer casos en 
los que la forma de un objeto es fundamental para la demostraci\'on es decir, 
el uso de inducci\'on es inminente como el caso de la funci\'on \verb-flatten-. 
Sin inducci\'on no es posible realizar dichas demostraciones.

Pero este esquema de demostraci\'on es manejado de manera sofisticada por 
{\coq} facilitando la construcci\'on interactiva de la prueba.
As\'i tambi\'en hacemos \'enfasis en que el sistema ecuacional estudiado en 
esta nota es un subsistema de t\'acticas que ofrece {\coq} y que el uso 
del asistente rebasa en gran medida al sistema $\mathcal{B}$ como lo muestra la 
demostraci\'on del teorema \verb=list_partition=:
\begin{lstlisting}
Proof.
intros.
case_eq l.
(* l=[] *)
+ intro. elim H. assumption.
(* l=n::l0 *)
+ intros. exists n. exists []. exists l0. simpl. reflexivity.
Qed.
\end{lstlisting}
\eeje
En los siguientes temas nos serviremos de este poder para realizar 
demostraciones asistidas por computadora. 
\end{document}
