
\documentclass[11pt,letterpaper]{article}
\usepackage{../packageslc}
\usepackage{../optionslc}

%%%% Macros para las notas de lenguajes de programacion


%%%% math

\newcommand{\vphi}{\varphi}
\newcommand{\vp}{\varphi}

% \newcommand{\dn}{\mathsf{DN}}
% \newcommand{\dnC}{\mathsf{DN_C}}
% \newcommand{\dnM}{\mathsf{DN_M}}
% \newcommand{\dnp}{\mathsf{DN_p}}
% \newcommand{\dnm}{\mathsf{DN_p^M}}
% \newcommand{\dnc}{\mathsf{DN_p^C}}

%\newcommand{\case}{\mathsf{case}}
%\renewcommand\labelitemi{$\circ$}

\newcommand{\imp}{\rightarrow}
\newcommand{\Imp}{\Rightarrow}
\renewcommand{\iff}{\leftrightarrow}
\newcommand{\Iff}{\Leftrightarrow}
\newcommand{\G}{\Gamma}
\newcommand{\D}{\Delta}


\newcommand{\De}{\mathcal{D}}
\newcommand{\F}{\mathcal{F}}
\newcommand{\Ge}{\mathcal{G}}
\newcommand{\Pe}{\mathcal{P}}
\newcommand{\I}{\mathcal{I}}
\newcommand{\C}{\mathcal{C}}
\newcommand{\K}{\mathcal{K}}
\renewcommand{\L}{\mathcal{L}}
\newcommand{\M}{\mathcal{M}}
\newcommand{\Nc}{\mathcal{N}}
%\newcommand{\E}{\mathcal{E}}
%\newcommand{\R}{\mathcal{R}}
%\newcommand{\Q}{\mathcal{Q}}
\newcommand{\Sc}{\mathcal{S}}
\newcommand{\Te}{\mathcal{T}}
\newcommand{\W}{\mathcal{W}}

\newcommand{\Db}{\mathbb{D}}
\newcommand{\Fb}{\mathbb{F}}
\newcommand{\Kb}{\mathbb{K}}
\newcommand{\Eb}{\mathbb{E}}
\newcommand{\Ebs}{\mathbb{E}^\star}
\newcommand{\Ob}{\mathbb{O}}
\newcommand{\Ib}{\mathbb{I}}
\newcommand{\Rb}{\mathbb{R}}
\newcommand{\Qb}{\mathbb{Q}}
\newcommand{\Kbb}{\mathbb{K}}
\newcommand{\T}{\mathbb{\Theta}}


\newcommand{\kb}{\bbkappa}

\newcommand{\Sf}{\mathsf{\Sigma}}

\newcommand{\fa}{\forall}
\newcommand{\ex}{\exists}

\newcommand{\inc}{\subseteq}

\newcommand{\Lb}{\Lambda}
\newcommand{\Om}{\Omega}
\newcommand{\lb}{\lambda}
\newcommand{\al}{\alpha}
\newcommand{\ga}{\gamma}


\newcommand{\mg}{\mathbb{m}}

\newcommand{\cg}{\mathbb{C}}
\newcommand{\dg}{\mathbb{D}}
\newcommand{\jg}{\mathbb{J}}
\newcommand{\Ha}{\mathcal{H}}
%\newcommand{\A}{\mathcal{A}}
\newcommand{\sg}{\mathbb{S}}

\newcommand{\Bc}{\mathcal{B}}
\newcommand{\Df}{\mathfrak{D}}
\newcommand{\Dc}{\mathcal{D}}
%\newcommand{\Tc}{\mathcal{T}}
\newcommand{\Mf}{\mathfrak{M}}

\newcommand{\Sg}{\mathbb{S}}

\newcommand{\Q}{\ensuremath{\mathbb{Q}}}
\newcommand{\Z}{\ensuremath{\mathbb{Z}}}
\newcommand{\N}{\ensuremath{\mathbb{N}}}
\newcommand{\R}{\ensuremath{\mathbb{R}}}
\renewcommand{\S}{\mathbb{\Sigma}}
\newcommand{\A}{\mathcal{A}}
\newcommand{\E}{\ensuremath{\exists}}
\newcommand{\iso}{\ensuremath{\cong}}
\newcommand{\union}{\ensuremath{\cup}}
\newcommand{\morinyec}{\ensuremath{\precapprox}}

\newcommand{\nin}{\ensuremath{\notin}}
\newcommand{\tog}{\makebox[7mm][l]}
\newcommand{\toge}{\makebox[11mm][l]}
\newcommand{\toget}{\makebox[13mm][l]}
\newcommand{\togeth}{\makebox[14mm][l]}
\newcommand{\togethe}{\makebox[15mm][l]}
\newcommand{\together}{\makebox[17mm][l]}
\newcommand{\niso}{\ensuremath{\not \cong}}


\newcommand{\Mg}{\mathbb{M}}
\newcommand{\Bg}{\mathbb{B}}
\newcommand{\Lg}{\mathbb{L}}
\newcommand{\Tg}{\mathbb{T}}

\newcommand{\sketch}{\Red{{\sc sketch}}}

\newcommand{\restr}[2]{#1\!\!\boldsymbol{\restriction}\!#2}

\newcommand{\vacio}{\varnothing}
\newcommand{\done}{\ensuremath{\checkmark}}

\newcommand{\ida}{$\Rightarrow \; )$ }
\newcommand{\regr}{$\Leftarrow \; )$ }

\newcommand{\ol}[1]{\overline{#1}}

\newcommand{\Tsf}{\mathsf{T}}

\newcommand{\inds}[1]{\index[simb]{#1}}

\newcommand{\B}{\mathbb{B}}
%\newcommand{\N}{\mathbb{N}}

\newcommand{\vx}{\vec{x}}
\newcommand{\vy}{\vec{y}}
\newcommand{\vz}{\vec{z}}
\newcommand{\vt}{\vec{t}}
\newcommand{\vf}{\vec{f}}


% \newcommand{\propo}{\ensuremath{\mathsf{PROP}}}
% \newcommand{\atom}{\ensuremath{\mathsf{ATOM}}}
\newcommand{\term}{\ensuremath{\mathsf{TERM}}}
\newcommand{\form}{\mathsf{FORM}}

\newcommand{\true}{\mathop{\mathsf{true}}}

%\newcommand{\id}{\mathsf{Id}}

%\newcommand{\uc}{\mathcal{U}}
%\newcommand{\Ic}{\mathcal{I}}
%\newcommand{\pc}{\mathcal{P}}
%\newcommand{\qc}{\mathcal{Q}}
%\newcommand{\mc}{\mathcal{M}}
\newcommand{\supc}{\supseteq}
\newcommand{\limo}{\mathop{\mathpzc{Lim}}}
\newcommand{\ord}{\mathsf{OR}}

\newcommand{\pt}[1]{\langle #1 \rangle}


%%%% frames
\newcommand{\titulos}[2]{\frametitle{#1}\framesubtitle{#2}}
\newcommand{\fot}[1]{\footnote{\scriptsize{#1}}}


%%%% ambientes

\newcommand{\cb}[2]{\colorbox{#1}{#2}}

\newcommand{\bc}{\begin{center}}
\newcommand{\ec}{\end{center}}
\newcommand{\be}{\begin{enumerate}}
\newcommand{\ee}{\end{enumerate}}
\newcommand{\bi}{\begin{itemize}}
\newcommand{\ei}{\end{itemize}}
\newcommand{\beq}{\begin{equation}}
\newcommand{\eeq}{\end{equation}}
\newcommand{\beqs}{\begin{equation*}}
\newcommand{\eeqs}{\end{equation*}}
\newcommand{\ba}{\begin{array}}
\newcommand{\ea}{\end{array}}


% \newtheorem{theorem}{Teorema}
% \newcommand{\teo}[1]{\begin{theorem} #1 \end{theorem}}
% \newtheorem{proposition}{Proposici\'on}
% \newcommand{\prop}[1]{\begin{proposition} #1 \end{proposition}}
% \newtheorem{definition}{Definici\'on}
% \newcommand{\defin}[1]{\begin{definition} #1 \end{definition}}
% \newtheorem{corollary}{Corolario}
% \newcommand{\cor}[1]{\begin{corollary} #1 \end{corollary}}
% \newtheorem{lemma}{Lema}
% \newcommand{\lema}[1]{\begin{lemma} #1 \end{lemma}}
% \newcommand{\dem}[1]{\begin{proof} #1 \end{proof}}

%\renewcommand{\qed}{\qedsymbol{$\mathbf{\dashv}$}}

%\newcommand{\proof}{\hfill\\\noindent\textbf{\textit{Demostraci\'on. }}}

\newcommand{\hint}{\emph{Sugerencia: }}


\newcounter{EjempCtr}[section]
\newenvironment{enumrom}{\renewcommand{\theenumi}{\roman{enumi}}%
\renewcommand{\theenumii}{\roman{enumii}}
\renewcommand{\theenumiii}{\roman{enumiii}}
\renewcommand{\theenumiv}{\roman{enumiv}}
\begin{enumerate}}{\end{enumerate}}
\newenvironment{Ejemplo}
        {\stepcounter{EjempCtr}%
        \begin{description}\item[Ejemplo \thesection.\arabic{EjempCtr}]}%
        {\end{description}}
\newenvironment{demostr}{%
             {\em Demostración:}
                \begin{quotation}}{\end{quotation}}

   \newcommand{\beje}{\begin{Ejemplo}}
\newcommand{\eeje}{\end{Ejemplo}}


\newtheorem{eje}{Ejemplo}[section]
\newcommand{\ejem}[1]{\begin{eje}\normalfont #1 \end{eje}}

% \renewcommand\contentsname{\'Indice}
%\renewcommand\chaptername{Cap\'itulo}
% \renewcommand\indexname{\'Indice}

%%\newcommand{\qed}{\hfill$\mathbb{Qed}$}
%\newcommand{\qed}{\hfill$\mathsf{\boldsymbol{\dashv}}$}
%\renewcommand{\qed}{\hfill$\boldsymbol{\dashv}$}


\newenvironment{prueba}{\vspace{-5mm}\noindent\textbf{Demostraci\'on}\\}{
\noindent$\blacksquare$\\}

\newcommand{\Ejercicios}{\section*{Ejercicios}}

 \newenvironment{manitas}{%
      \renewcommand{\labelitemi}{\ding{44}}%
      \vspace{-0.5cm}%
      \begin{itemize}%
      \setlength{\itemsep}{0pt}\setlength{\parsep}{0pt}\setlength{\topsep}{0pt}%
      }{\end{itemize}}
\newenvironment{malitos}{%
      \renewcommand{\labelitemi}%
            {\raisebox{1.5ex}{\makebox[0.3cm][l]{\begin{rotate}{-90}%
            \ding{43}\end{rotate}}}}%
      \vspace{-0.5cm}%
      \begin{itemize}%
      \setlength{\itemsep}{0pt}\setlength{\parsep}{0pt}\setlength{\topsep}{0pt}%
      }{\end{itemize}}
\newenvironment{ejercs}{
     \renewcommand{\labelenumi}{\thesection.\theenumi.-}
     \renewcommand{\labelenumii}{\theenumii)}
     \begin{enumerate}}
     {\end{enumerate}}

   \newcommand{\bej}{\begin{ejercs}}
\newcommand{\eej}{\end{ejercs}}


%\newenvironment{leterize}{%
%        \renewcommand{\theenumi}{\alph{enumi}}
%        \begin{enumerate}}{\end{enumerate}}

%\newenvironment{manitas}{%
%      \renewcommand{\labelitemi}{\ding{44}}%
%      \vspace{-0.5cm}%
%      \begin{itemize}%
%      
% \setlength{\itemsep}{0pt}\setlength{\parsep}{0pt}\setlength{\topsep}{0pt}%
%      }{\end{itemize}}



%%=============================================================================

\def\stackunder#1#2{\mathrel{\mathop{#2}\limits_{#1}}}


%%%% notas

\newcommand{\doubt}{\Red{{\LARGE {\sf ??}}}}

\newcommand{\coment}[1]{\hfill\\ \Big[{\bf Comentario Privado:} #1\Big]}
\newcommand{\preg}[1]{\hfill\\ \BrickRed{{\bf Pregunta:} #1}}
\newcommand{\conjet}[1]{\hfill\\ \OliveGreen{{\bf Conjecura:} #1}}

\newcommand{\pendiente}{\BrickRed{{\sc Pendiente}}}
\newcommand{\verifpendiente}{\BrickRed{{\sc Verificación pendiente}}}


%--------------------------------------------------------------------------

\DeclareMathAlphabet{\mathpzc}{OT1}{pzc}{m}{it}


\title{Lógica Computacional 2017-2, nota de clase 15\\
El Isomorfismo de Curry-Howard
}
\author{Favio Ezequiel Miranda Perea \and Araceli Liliana Reyes Cabello\and
Lourdes Del Carmen Gonz\'alez Huesca \and Pilar Selene Linares Arévalo}
\date{\today}

\begin{document}
\maketitle


\section{Un lenguaje para expresiones aritméticas y booleanas}

A continuación presentamos un lenguaje para expresiones aritméticas y
booleanas que incluye las operaciones de suma, producto,
orden, un test para cero y la negación booleana. \\
Las expresiones del lenguaje se definen como sigue: 
\[
\ba{rll} 
e & ::= & x\mid
n\mid\true \mid \false\mid \\ \\
  & & e + e \mid e * e \\ \\ %\mid\pred e \mid \\ \\
  & & e < e \mid\iszero e \mid\no e%\mid\lete{x=e}{e}
\ea
\]
Ejemplos de expresiones del lenguaje son:
\bi
\item $3+y$
\item $\no (\iszero 9)$
\item $\iszero 3<\no (2+4)$
\item $\no\false$
\item $\no (x+1)$
\item $\iszero\false$
\item $\no 7 * y$
\item $7<x+y$
\ei


Se observa que todas son expresiones sintácticamente correctas pero algunas de
ellas no tienen sentido desde el punto de vista semántico, es decir,
no es posible asignarles uno de los tipos posibles para el lenguaje
que son $\nat$ o $\bool$. Por ejemplo 
$\no(8+2)$ o $\iszero\false$. Además hay expresiones en donde no resulta
claro si son válidas o no, por ejemplo $x+2$ puede o no ser válida dependiendo 
de si $x$ representa a un número o a un booleano, lo mismo sucede con
$\no(\iszero x)$ o con $x<y+1$. 

\subsection{Intérprete}

Un intérprete para el lenguaje es simplemente una función que devuelve un 
natural o un booleano que resulta de evaluar una expresión $e$. La definición 
recursiva de esta función 
$$\mathtt{eval}: Exp\to \N\cup\B $$ 
donde $\N=\{0,1,\ldots\}$ y $\B=\{t,f\}$ es:
\begin{alltt}
  eval n = n
  eval true = t
  eval false = f
  eval (e1 + e2) = eval e1 + eval e2
  eval (e1 * e2) = eval e1 * eval e2
  eval (e1 < e2) = (eval e1) < (eval e2)
  eval (not e) = \(\neg\)\,(eval e) 
  eval (iszero e) = \(isz\)\,(eval e)
\end{alltt}
donde los símbolos del lado derecho $+,*,<$ denotan las operaciones usuales 
mientras que del lado izquierdo son únicamente símbolos que forman parte de las 
expresiones. Omitímos aquí el uso de variables por simplicidad en la 
presentación. Para agregar variables es necesario agregar un estado como 
argumento de entrada al intérprete, el cual se encargará de evaluar variables.\\

Se observa que la función de evaluación así definida resulta una función 
parcial puesto que la evaluación de expresiones como $(3*5)+\no\false$ no tiene 
sentido ya que no sabemos sumar $15+\true$.

  % eval (e1 < e2) = if (eval e1) < (eval e2) then true else false
  % eval (not e) = if eval e then false else true
  % eval (iszero e) = if eval e = 0 then true else false

\espc

Una manera de evitar estos errores es verificando previamente si la expresión 
es coherente para lo cual podemos usar la deducción natural.

\subsection{Eliminación de expresiones semánticamente incorrectas}

Sea $\G$ el conjunto de las siguientes fórmulas:
\[
\ba{l}
Nat(n)\\
Bool(true)\\
Bool(false)\\
\fa x \fa y(Nat(x)\land Nat(y)\to Nat(x+y))\\
\fa x \fa y(Nat(x)\land Nat(y)\to Nat(x*y))\\
\fa x \fa y(Nat(x)\land Nat(y)\to Bool(x<y))\\
\fa x \fa y(Bool(x)\to Bool(\no x))\\
\fa x \fa y(Nat(x)\to Bool(\iszero x))\\
\ea
\]

Podemos verificar si una expresión~$e$ es semánticamente correcta si es posible 
\textbf{clasificarla} como natural o booleano derivando el jucio 
$\G\vdash Nat(e)$ o $\G\vdash Bool(e)$ respectivamente. Por ejemplo
\[
\G\vdash \no (\iszero (2 + 3*4)) \qquad\qquad \G\not\vdash Bool(\iszero(3*7)+4)
\]

Además si hay variables debemos suponer de antemano si es que representan 
números o booleanos. Por ejemplo
\[
\G,Nat(x)\vdash Bool(x<5) \qquad \G,Bool(z)\not\vdash Nat(z+6)
\]
Esta clase de verificaciones son exactamente lo que hace un sistema de tipos en 
lenguajes de programación.


\section{Sistemas de tipos}

Un sistema de tipos es un formalismo lógico que impone ciertas restricciones en 
la formación de programas de un lenguaje de programación. Las expresiones del 
lenguaje se clasifican mediante tipos que dictan como pueden usarse en 
combinación con otras expresiones. \\
Intuitivamente el tipo de una expresión predice la forma de su valor, por 
ejemplo, la expresión~$e_1+e_2$ donde $e_1$ y $e_2$ son expresiones numéricas 
debe ser una expresión numérica nuevamente. Por otra parte si intentamos sumar 
una expresión numérica con una expresión booleana digamos $\true + 7$  dicha 
expresión debe ser prohibida ya que genera un error de 
tipos~\footnote{Type-checking error}.


% La semántica estática de un lenguaje se define con ayuda de un sistema
% de tipos, los cuales son formalismos lógicos simples
% y manejables cuyo objetivo es  descartar ciertos comportamientos
% particulares en la formación de frases de programa. Las frases de un
% lenguaje se clasifican mediante tipos los cuales rigen como éstas
% pueden usarse en combinación con otras frases. Intuitivamente el tipo
% de una expresión predice la forma de su valor final. Una frase está
% correctamente tipada si se construye de manera consistente con estas
% predicciones, por ejemplo la suma de dos expresiones numéricas debe ser
% nuevamente una expresión numérica. Por otra parte frases como $\true+7$ 
% producirán un error de tipos.\\

El uso de sistemas de tipos en el diseño de un lenguaje de
programación tiene las siguientes ventajas:
\bi
\item Permite descubrir errores de programación tempranamente.
\item Ofrece seguridad, un programa correctamente tipado no puede funcionar mal.
\item Soporta abstracción, importante para la descripción de interfaces.
\item Los tipos documentan un programa de manera mas simple y
  manejable que los comentarios.
\item Los lenguajes tipados pueden implementarse de manera más clara y
  eficiente.
\ei


% \subsection{Programas}
% Un programa es una lista de instrucciones
% \[
% \ba{l}
% P(nil)\\
% \fa x\fa y( I(x)\land P(y)\to P(x:y))
% \ea
% \]

% \subsection{Pilas de valores}
% \[
% \ba{l}
% V(n)\\
% V(\true)\\
% V(\false)\\
% S(snil)\\
% \fa x\fa y(V(x)\land S(y)\to S(x:y))
% \ea
% \]


% Esta relación se conoce como relación de tipado y se define
% recursivamente mediante las siguientes reglas de inferencia:
% % si y sólo si el valor de la expresión
% % $e$ es de tipo $\Tf$. Por ejemplo $5:\nat,\true:\bool$.

\espc

Para decidir si una expresión es semánticamente correcta o no, los lenguajes de
programación emplean contextos de declaración de variables que involucran a un 
predicado binario infijo denotado~$x:\Tf$, que relaciona a una variable~$x$ con 
un tipo~$\Tf$. Esta declaración significa que la variable~$x$ representa a un 
valor de tipo~$\Tf$. Un contexto es entonces un conjunto~$\G$ de la forma 
$\G=\{x_1:\Tf_1,\ldots, x_n:\Tf_n\}$ donde todas las variables~$x_i$ son 
distintas y les corresponde un tipo~$\Tf_i$.

Por ejemplo si $\G=\{x:\nat,y:\bool\}$ entonces las expresiones $x+2$
y $\no y$ son aceptadas, mientras que $y<2$ y $\no x$ son rechazadas
para su evaluación. Para formalizar la verificación de tipos correctos
se utiliza una relación ternaria que involucra a un contexto $\G$, a
una expresión cualquiera del lenguaje $e$ y a un tipo $\Tf$. \\
Dicha relación se escribe como $\G\vdash e:\Tf$ y se lee como \enquote{la
expresión $e$ es de tipo $\Tf$ en el contexto $\G$}.

Esta relación se conoce como relación de tipado y se define
recursivamente mediante las siguientes reglas de inferencia:
\begin{mathpar}
 \inferrule*[right=(Var)]{
 }{
 \G,x:\Tf\vdash x:\Tf
 }
 
 \inferrule*[right=(TNum)]{
 }{
 \G\vdash n:\nat
 }
 
 \inferrule*[right=(TSum)]{
  \G\vdash e_1:\nat\and \G\vdash e_2:\nat
  }{
  \G\vdash e_1+e_2:\nat
  }

 \inferrule*[right=(TProd)]{
 \G\vdash e_1:\nat\and \G\vdash e_2:\nat
 }{
 \G\vdash e_1*e_2:\nat
 }
% \inferrule*[right=]{\G\vdash e:\nat}{\G\vdash\suc 
% e:\nat}\;(TSuc)\;\;\;\;\;\;\;\;\;\;
% \inferrule*[right=]{\G\vdash e:\nat}{\G\vdash\pred 
% e:\nat}\;(TPred)\;\;\;\;\;\;\;\;\;\;
 
 \inferrule*[right=(Ttrue)]{
 }{
 \G\vdash \true:\bool
 }

 \inferrule*[right=(Tfalse)]{
 }{
 \G\vdash \false:\bool
 }
 
 \inferrule*[right=(Tmen)]{
 \G\vdash e_1:\nat \and \G\vdash e_2:\nat
 }{
 \G\vdash  e_1<e_2:\bool
 }

 \inferrule*[right=(Tisz)]{
 \G\vdash e:\nat
 }{
 \G\vdash\iszero e:\bool
 }

  \inferrule*[right=(Tnot)]{
  \G\vdash e:\bool
  }{
  \G\vdash\no e:\bool
  }
\end{mathpar}

Mediante estas reglas podemos cerciorarnos por ejemplo de que si
$x:\nat$ entonces $\iszero(x+2):\bool$ como sigue:
\be
\item $x:\nat\vdash x:\nat\;(Var)$
\item $x:\nat\vdash 2:\nat\;(TNum)$
\item $x:\nat\vdash x+2:\nat\;(TSum)\;1,2$
\item $x:\nat\vdash\iszero(x+2):\nat\;(Tisz)\;4$
\ee

Veamos otro ejemplo, esta vez tipando la expresión $\no (2+x<y)$
en el contexto $\G=\{x:\nat,y:\nat\}$
\be
\item $x:\nat,y:\nat\vdash x:\nat\;(Var)$
\item $x:\nat,y:\nat\vdash 2:\nat\;(TNum)$
\item $x:\nat,y:\nat\vdash 2+x:\nat\;(TSum)\;1,2$
\item $x:\nat,y:\nat\vdash y:\nat\;(Var)$
%\item $x:\nat,y:\nat\vdash \pred y:\nat\;(Tpred)$
\item $x:\nat,y:\nat\vdash 2+x< y:\bool\;(Tmen)\;3,4$
\item $x:\nat,y:\nat\vdash \no (2+x< y):\bool\;(Tneg)\;5$
\ee

% En estos casos para decidir si la
% expresión es semánticamente correcta o no, los lenguajes de
% programación emplean contextos de declaración de variables que
% involucran a un predicado binario infijo denotado $x:\Tf$, que relaciona a una
% variable $x$ con un tipo $\Tf$. Esta declaración significa que la
% variable $x$ representa a un valor de tipo $\Tf$. Un contexto es
% entonces un conjunto $\G$ de la forma $\G=\{x_1:\Tf_1,\ldots,
% x_n:\Tf_n\}$ donde todas las variables $x_i$ son distintas. \\

% Por ejemplo si $\G=\{x:\nat,y:\bool\}$ entonces las expresiones $x+2$
% y $\no y$ son aceptadas, mientras que $y<2$ y $\no x$ son rechazadas
% para su evaluación. Para formalizar la verificación de tipos correctos
% se utiliza una relación ternaria que involucra a un contexto $\G$, a
% una expresión cualquiera del lenguaje $e$ y a un tipo $\Tf$. \\
% Dicha relación se escribe como $\G\vdash e:\Tf$ y se lee como ``la
% expresión $e$ es de tipo $\Tf$ en el contexto $\G$''. \\


\section{Isomorfismo de Curry-Howard}

Como se observa los sistemas de tipos son muy similares a los sistemas
de deducción natural con contextos que ya hemos estudiado. De hecho
los sistemas de deducción natural pueden verse, desde el punto de
vista computacional, como un sistema de tipos para un \textit{prototipo de 
lenguaje de programación funcional} mediante la llamada correspondencia de 
Curry-Howard, también conocida como paradigma de derivaciones como programas o 
proposiciones como tipos. La idea a grandes rasgos es:
\bi
 \item Las fórmulas de la lógica corresponden a tipos para un lenguaje
  de programación.
 \item Las pruebas o derivaciones en la lógica corresponden a los
  pasos que el verificador de tipos realiza para mostrar que cierta
  expresión o programa del lenguaje de programación está bien tipada.
 \item Por lo tanto, verificar si una prueba es correcta en un contexto dado 
  corresponde a verificar si un programa tiene un tipo válido en el contexto 
  dado.
\ei

\section{Reglas de inferencia con codificación de pruebas}

La relación de inferencia $\Gamma \vdash A$ se modifica anotando a cada prueba
de un secuente $\G\vdash A$ mediante una expresión~$t$ que
que corresponde a un programa de un prototipo de lenguaje de programación,
esta expresión~$t$ resulta ser también un \textbf{código de la prueba} cuyo 
secuente final es $\G\vdash A$.\\

De esta manera obtenemos una relación $\G\vdash t:A$ donde
$\G=\{x_1:A_1,\ldots,x_n:A_n\}$ es un contexto de declaración de
variables $x_i:A_i$ consideradas locales. En este caso no sólo suponemos una
fórmula~$A_i$ sino que también suponemos que existe una prueba $x_i$ de
$A_i$. 
La nueva relación entre contextos $\G$, expresiones o códigos $t$ y fórmulas
$A$, puede entenderse desde dos puntos de vista distintos:
\bi
\item Lógicamente: $\G\vdash t:A$ significa que la fórmula $A$ es
  derivable a partir de las hipótesis $\G$ y $t$ es un código de
  prueba para la derivación $\G\vdash A$.
\item Computacionalmente: $\G\vdash t:A$ significa que $t$ es un
  programa con tipo $A$ cuyas variables locales están declaradas en $\G$.    
\ei

La nueva relación de inferencia se define recursivamente a partir de la regla
de inicio:
%It is important to remark that we make no syntactic distinction between object 
% variables and proof-term variables.\\
\begin{equation*}
\G,x:\vp\vdash x:\vp\;\;(Hip)\;\;\;\;
%\inferrule*[right=(start)]{s=t\in\Eb}{\G\vdash_\Eb s=t}
\end{equation*}
\noindent
mediante reglas de introducción y eliminación para cada conectivo 
\bi
\item Implicación: el conectivo $\to$ corresponde al tipo de funciones.
\begin{mathpar}
\inferrule*[right=($\imp$ I)]{
 \Gamma, x:\vp\vdash t:\psi
 }{
 \Gamma\vdash \fun(x:\vp.t):\vp\imp \psi
 }
 
 \inferrule*[right=($\imp$ E)]{
 \Gamma\vdash f: \vp\imp \psi\and \Gamma\vdash t:\vp
 }{
 \Gamma\vdash f\,t:\psi
 }
 \end{mathpar}
 
\item Conjunción: el conectivo $\land$ corresponde al tipo $\times$, es decir
  al producto cartesiano de dos tipos.
\begin{mathpar}
\inferrule*[right=($\land$ I)]{
 \G\vdash t: \vp\and \G\vdash s: \psi
 }{
 \G\vdash\pt{t,s}:\vp\land \psi
 }
 
 \inferrule*[right=($\land_{2}$ E)]{
 \G\vdash t: \vp\land\psi
 }{
 \G\vdash {\tt fst}\, t: \vp
 }

 \inferrule*[right=($\land_{1}$ E)]{
 \G\vdash t: \vp\land\psi
 }{
 \G\vdash {\tt snd}\, t: \psi
 }
\end{mathpar}

\item Disyunción: el conectivo $\lor$ corresponde al tipo $+$, es decir a la
  unión disjunta de dos tipos.
\begin{mathpar}
 \inferrule*[right=($\lor_{1}$ I)]{
  \G\vdash t: \vp
  }{
  \G\vdash {\tt inl}\, t: \vp\lor\psi
  }

 \inferrule*[right=($\lor_{2}$ I)]{
 \G\vdash t: \psi
 }{
 \G\vdash {\tt inr}\, t:\vp\lor\psi
 }

  \inferrule*[right=($\lor$ E)]{
  \G\vdash r: \vp\lor\psi \and 
  \G, x:\vp \vdash  s: \chi \and
  \G, y:\psi \vdash t:\chi
  }{
  \G\vdash {\tt case}\,r\,{\tt of }\;
  {\tt inl}\,x\Rightarrow s\mid{\tt inr}\,y\Rightarrow t\;:\chi
  }
\end{mathpar}

\item Falso: la constante de falsedad corresponde a un tipo vacío.
\begin{mathpar}
\inferrule*[right=($\bot$ E)]{
\G\vdash r: \bot
}{
\G\vdash {\tt abort}\, r:\vp
}
\end{mathpar}
\ei

\section{Ejemplos}

En los siguientes ejemplos se trata de hallar el programa~$t$ que
codifique la prueba correspondiente. Como ejercicios el lector debe
justificar cada paso y/o agregar algunos pasos faltantes.

\bi
\item $\vdash t: A\imp A$
  \be
  \item $x:A \vdash x:A$
    \item $\vdash \fun(x:A.x): A\imp A$
  \ee

\item $\vdash t: A\imp A\lor B$
  \be
  \item $x:A\vdash x:A$
    \item $x:A\vdash\inl x:A\lor B$
      \item $\vdash \fun(x:A.\inl x):A\imp A\lor B$
  \ee
  

\item $\vdash t: A\land B\imp B$
  \be
  \item $x:A\land B\vdash x:A\land B$
    \item $x:A\land B\vdash \Rp x:B$
      \item $\vdash\fun(x:A\land B.\Rp x):A\land B\imp B$
  \ee

\item $f:A\imp B,\;g:B\imp C\vdash t: A\imp C$
  \be
  \item $f:A\imp B,\;g:B\imp C, x:A \vdash f: A\imp B$
    \item $f:A\imp B,\;g:B\imp C, x:A\vdash g: B\imp C$
      \item $f:A\imp B,\;g:B\imp C, x:A\vdash x:A$
        \item $f:A\imp B,\;g:B\imp C, x:A\vdash fx:B$
          \item $f:A\imp B,\;g:B\imp C, x:A\vdash g(fx):C$
            \item $f:A\imp B,\;g:B\imp C\vdash \fun(x:A.g(fx)):A\imp C$
  \ee

\item $\vdash: t: B\imp (A\imp B)$
  \be
  \item $x:B,y:A\vdash x:B$
    \item $x:B\vdash\fun(y:A.x):A\imp B$
      \item $\vdash\fun(x:B.\fun(y:A.x)):B\imp (A\imp B)$
      
  \ee

\item $\vdash t: (A\land B\imp C)\imp (A\imp B\imp C)$
  \be
  \item $f:A\land B\imp C, x:A, y:B\vdash f:A\land B\imp C $
    \item $f:A\land B\imp C, x:A, y:B\vdash x:A$
      \item $f:A\land B\imp C, x:A, y:B\vdash y:B$
        \item $f:A\land B\imp C, x:A, y:B\vdash\pt{x,y}:A\land B$
         \item $f:A\land B\imp C, x:A, y:B\vdash f\pt{x,y}:C$
           \item $f:A\land B\imp C, x:A \vdash
             \fun(y:B.f\pt{x,y}):B\imp C$
              \item $f:A\land B\imp C \vdash
             \fun\big(x:A.\fun(y:B.f\pt{x,y})\big):A\imp B\imp C$
             \item $\vdash
             \fun\Big(f:A\land B\imp
             C.\fun\big(x:A.\fun(y:B.f\pt{x,y})\big)\Big): (A\land
             B\imp C)\imp A\imp B\imp C$ 
  \ee

  \item $f:A\imp B\imp C\vdash t: A\land B\imp C$
    \be
    \item $f:A\imp B\imp C, x: A\land B\vdash f:A\imp B\imp C$
      \item $f:A\imp B\imp C, x: A\land B\vdash f:A\imp B\imp C$
        \item $f:A\imp B\imp C, x: A\land B\vdash x:A\land B$
          \item $f:A\imp B\imp C, x: A\land B\vdash \Lp x:A$
            \item $f:A\imp B\imp C, x: A\land B\vdash \Rp x:B$
              \item $f:A\imp B\imp C, x: A\land B\vdash f(\Lp x):B\imp
                C$
                \item $f:A\imp B\imp C, x: A\land B\vdash f(\Lp x)(\Rp
                  x):
                C$
                \item $f:A\imp B\imp C\vdash \fun\big( x: A\land B.f(\Lp x)(\Rp
                  x)\big): A\land B\imp C$
              
    \ee
  
  \item $f:A\lor B\imp C\vdash t:A\imp C$
    \be
    \item $f:A\lor B\imp C, x:A\vdash f:A\lor B\imp C$
      \item $f:A\lor B\imp C, x:A\vdash x:A$
      \item $f:A\lor B\imp C, x:A\vdash \inl x:A\lor B$
        \item $f:A\lor B\imp C, x:A\vdash f(\inl x):C$
          \item $f:A\lor B\imp C \vdash \fun(x:A.f(\inl x)):A\imp C$
    \ee

\item $x:A\lor B, f:A\imp C, g:B\imp C\vdash t:C$
\be
 \item $x:A\lor B, f:A\imp C, g:B\imp C\vdash x: A\lor B $
 \item $x:A\lor B, f:A\imp C, g:B\imp C, y:A\vdash f y: C $
 \item $x:A\lor B, f:A\imp C, g:B\imp C, z:B\vdash g z:C $
 \item $x:A\lor B, f:A\imp C, g:B\imp C\vdash ($ {\tt case} $x$ {\tt of} 
 $\inl y\Imp fy\mid\inr z\Imp gz): C$
\ee
    
\item $f: A\imp B, g: C\imp D, x:A\lor C\vdash t:B\lor D$

  \be
  \item $f: A\imp B, g: C\imp D, x:A\lor C\vdash x:A\lor C$
    \item $f: A\imp B, g: C\imp D, x:A\lor C, y:A\vdash \inl
      (fy):B\lor D$
          \item $f: A\imp B, g: C\imp D, x:A\lor C, z:C\vdash\inr
            (gz):B\lor D$
                \item $f: A\imp B, g: C\imp D, x:A\lor C\vdash ($ {\tt case} 
$x$ {\tt
        of} $\inl y\Imp \inl (fy)\mid\inr z\Imp \inr (gz)): B\lor D$
  \ee


\item $x: (A\lor C)\land (B\imp C), f:(A\imp B)\vdash t:C$

  \be
  \item $x: (A\lor C)\land (B\imp C), f:(A\imp B)\vdash \Lp x:A\lor C$
    \item $x: (A\lor C)\land (B\imp C), f:(A\imp B)\vdash \Rp x:B\imp
      C$
      \item $x: (A\lor C)\land (B\imp C), f:(A\imp B), y:A\vdash fy: B$
        \item $x: (A\lor C)\land (B\imp C), f:(A\imp B), y:A\vdash
          (\Rp x)(fy): C$
          \item $x: (A\lor C)\land (B\imp C), f:(A\imp B), z:C\vdash
            z: C$
            \item $x: (A\lor C)\land (B\imp C), f:(A\imp B)\vdash ($ {\tt case} 
$x$ {\tt
        of} $\inl y\Imp \inl (\Rp x)(fy)\mid\inr z\Imp z): C$
  \ee
  
\item $x: A\land (B\lor C)\vdash t: (A\land B)\lor (A\land C)$
  \be
  \item $x: A\land (B\lor C)\vdash \Lp x: A$
    \item $x: A\land (B\lor C)\vdash \Rp x: B\lor C$
      \item $x: A\land (B\lor C), y:B \vdash \pt{\Lp x, y}:A\land B$
        \item $x: A\land (B\lor C), y:B \vdash \inl\pt{\Lp x, y}:(A\land
          B)\lor (A\land C)$
          \item $x: A\land (B\lor C), z:C \vdash \inr\pt{\Lp x, z}:(A\land
          B)\lor (A\land C)$
          \item $x: A\land (B\lor C)\vdash
               ($ {\tt case} $\Rp x$ {\tt
        of} $\inl y\Imp \inl \pt{\Lp x,y}\mid\inr z\Imp \inr\pt{\Lp x,z})
            :(A\land B)\lor (A\land C)$
          
  \ee


\item $x: (A\land B)\lor (A\land C)\vdash t: A\land (B\lor C)$
  \be
  \item $x: (A\land B)\lor (A\land C), y:A\land B\vdash \inl(\Rp
    y):B\lor C$
     \item $x: (A\land B)\lor (A\land C), y:A\land B\vdash \pt{\Lp y,\inl(\Rp
    y)}:A\land (B\lor C)$
     \item $x: (A\land B)\lor (A\land C), z:A\land C\vdash \pt{\Lp z,\inr(\Rp
    z)}:A\land (B\lor C)$
       \item $x: (A\land B)\lor (A\land C)\vdash
               ($ {\tt case} $x$ {\tt
        of} $\inl y\Imp \pt{\Lp y,\inl(\Rp
    y)}\mid\inr z\Imp \pt{\Lp z,\inr(\Rp
    z)}
            : A\land (B\lor C)$
  \ee
  

\item $x: A\lor B, f: B\lor C\imp D, g:A\imp C\vdash t:D$
  \be
  \item $x: A\lor B, f: B\lor C\imp D, g:A\imp C, y:A\vdash \inr (g
    y):B\lor C$
    \item $x: A\lor B, f: B\lor C\imp D, g:A\imp C, y:A\vdash f\inr (g
    y):D$
       \item $x: A\lor B, f: B\lor C\imp D, g:A\imp C, z:B\vdash
         f(\inl z):D$
               \item $x: A\lor B, f: B\lor C\imp D, g:A\imp C\vdash ($ {\tt 
case} $x$ {\tt
        of} $\inl y\Imp f\inr (gy)\mid\inr z\Imp f(\inl z):D$
  \ee
  
    
%    \item $x:(A\imp C)\land (B\imp C)\vdash t: A\lor B\imp C$
%\be
%\item $x:(A\imp C)\land (B\imp C), y:A\lor B\vdash x:(A\imp C)\land (B\imp C)$

%\item $x:(A\imp C)\land (B\imp C), y:A\lor B\vdash \Lp x:A\imp C$
%  \item $x:(A\imp C)\land (B\imp C), y:A\lor B\vdash \Rp x:B\imp C$
%    \item $x:(A\imp C)\land (B\imp C), y:A\lor B\vdash y:A\lor B$
%  \item $x:(A\imp C)\land (B\imp C), z:A\vdash z:A$
%    \item $x:(A\imp C)\land (B\imp C), z:A\vdash (\Lp x)z:C$
%        \item $x:(A\imp C)\land (B\imp C), w:B\vdash w:B$
%    \item $x:(A\imp C)\land (B\imp C), w:B\vdash (\Rp x)w:C$
  
%\ee

      



%\[
%\inferrule*[right=]{\Gamma\vdash_\Eb t:A}{\Gamma\vdash_\Eb t:\forall 
% xA}\;(\forall I)\;\;\inferrule*[right=]{\Gamma\vdash_\Eb t:\forall
%  xA}{\Gamma\vdash_\Eb t:A[x:=s]}\;(\forall E)
%\]

%\[
%\inferrule*[right=]{\Gamma\vdash_\Eb t:A}{\Gamma\vdash_\Eb t:\forall 
% XA}\;(\forall^2 I)\;\;\inferrule*[right=]{\Gamma\vdash_\Eb t:\forall
%  XA}{\Gamma\vdash_\Eb t:A[X:=\mathcal{F}]}\;(\forall^2 E)
%\]

%\[
%\inferrule*[right=]{\G\vdash_\Eb r:A[x:=s]\;\;\;\;\G\vdash_\Eb 
% s=t}{\G\vdash_\Eb r:A[x:=t]}\;\;(Eq)
%\]


%\section{Reglas de Inferencia para los Cuantificadores}

%Las reglas intuitivas discutidas en la sección anterior nos conducen a las
%siguientes reglas formales de deducción natural:
%\item Cuantificador Universal:
%\[
%\inferrule*[right=]{\G\vdash \vp\;\;\;\;\;x\notin Vl(\G)}{\G\vdash\fa 
% x\vp}\;(\fa I) \;\;\;\;\;\;\;\;\;
%\inferrule*[right=]{\G\vdash \fa x\vp}{\G\vdash \vp[x:=t]}\;(\fa E)
%\]
%\item Cuantificador Existencial:
%\[
%\inferrule*[right=]{\G\vdash \vp[x:=t]}{\G\vdash \ex x\vp}\;(\ex I) 
% \;\;\;\;\;\;\;\;\;
%\inferrule*[right=]{\G\vdash\ex x\vp\;\;\;\;\;\G,\vp\vdash 
% \psi\;\;\;\;\;x\notin
%  Vl(\G,\psi)}{\G\vdash\psi}\;(\ex E)
%\]

%La siguiente regla llamada \emph{monotonía} es derivable en el
%sistema, y será utilizada frecuentemente:
%\[
%\inferrule*[right=]{\G\vdash\vp}{\G,\psi\vdash\vp}\;(Mon)
%\]


%\item Verdadero (regla derivada):
%\[
%\inferrule*[right=]{}{\;\;\G\vdash \top\;\;}\;(\top I)
%\]


\section{Pruebas redundantes}

\bi
\item Dadas las derivaciones $\G\vdash\fun (x:A.e):A\to B$ y $\G\vdash r:A$ 
construir una derivación de $\G\vdash ?:B$.
\bi
\item La solución que salta a la vista es $\G\vdash\fun(x:A.e)\,r:B$
\item Sin embargo de las reglas y la prueba dada de $A\to B$, sabemos que
   previamente debio obtenerse la prueba $\G,x:A\vdash e:B$ puesto que la
   regla usada fue $(\to I)$. Ahora bien,
   obsérvese que como ya tenemos una prueba $\G\vdash r:A$, no es necesario 
   suponer $x:A$ puesto que en la prueba codificada por $e$ podemos sustituir 
   cada prueba $x$ de $A$ por la prueba dada $r$, obteniendo así una prueba 
   de $B$ sin necesidad de usar el modus ponens, a saber $\G\vdash e[x:=r]:B$. 
\ei

\item Dada la derivación $\G\vdash \pt{e_1,e_2}:A\land B$ construir una
    prueba $\G\vdash\,?:A$.
\bi
\item La respuesta más inmediata es $\G\vdash\Lp\pt{e_1,e_2}:A$.
\item Pero obsérvese que para construir la prueba codificada por
  $\Lp\pt{e_1,e_2}$ antes tuvo que construirse una prueba codificada por $e_1$
  que tiene que ser $\G\vdash e_1:A$ puesto que la regla utilizada fue 
  $(\land I)$.
\ei

% Sean $\G\vdash e_1:A,\;\G\vdash e_2:B$ dos pruebas dadas y supóngase que se 
% requiere dar una prueba de
%   $A$. Una respuesta posible es: 
% \be
% \item $\G\vdash e_1:A$ dada.
% \item $\G\vdash e_2:B$ dada.
% \item $\G\vdash \pt{e_1,e_2}:A\land B$
% \item $\G\vdash \Lp\pt{e_1,e_2}:A$
% \ee
% Por supuesto que existe una prueba más sencilla y directa, a saber $\G\vdash
% e_1:A$ que ya está dada. De manera que 
% requiera llamar a una prueba de $A$ 
\ei
\ei

De lo anterior se observa que las primeras pruebas son redundantes y podemos 
simplificarlas como sigue:
\bc
$\fun(x:A.e)\,r$ se simplifica a $e[x:=r]$
\ec
\bc
$\Lp\pt{e_1,e_2}$ se simplifica a $e_1$.
\ec

De la misma manera podemos justificar las siguientes simplificaciones

\bc
$\Rp\pt{e_1,e_2}$ se simplifica a $e_2$.
\ec
\bc
${\tt case}\,(\inl r)\,{\tt of }\;{\tt inl}\,x\Rightarrow s\mid{\tt
  inr}\,y\Rightarrow t$ se simplifica a $s[x:=r]$
\ec
\bc
${\tt case}\,(\inr r)\,{\tt of }\;{\tt inl}\,x\Rightarrow s\mid{\tt
  inr}\,y\Rightarrow t$ se simplifica a $t[x:=r]$
\ec

Estas reglas de simplificación pueden utilizarse como reglas de evaluación de
las expresiones de un lenguaje de programación. En conclusión las pruebas
redundantes de la lógica y su simplificación corresponden a un proceso de
evaluación de expresiones en un lenguaje de programación. 




% \subsection{Evaluación de expresiones}

% \begin{alltt}
%   eval n = n
%   eval true = t
%   eval false = f
%   eval (e1 + e2) = eval e1 + eval e2
%   eval (e1 * e2) = eval e1 * eval e2
%   eval (e1 < e2) = if (eval e1) < (eval e2) then true else false
%   eval (not e) = if eval e then false else true
%   eval (iszero e) = if eval e = 0 then true else false
% \end{alltt}


\subsection{Compilación de expresiones aritméticas y booleanas}

Presentamos un ejemplo de compilación del lenguaje de expresiones aritméticas y 
booleanas a una máquina abstracta de pila. Nos interesa en particular 
formalizar la prueba de la correctud del proceso de compilación mediante 
deducción natural.

\bi
\item Especificación del lenguaje objeto: se trata de un lenguaje de 
instrucciones primitivas
$$ i ::= n\mid t\mid f\mid+\mid*\mid<\mid\no\mid\iszero $$
de manera que una instrucción es un valor natural o booleano o bien un operador 
del lenguaje fuente de expresiones aritméticas y booleanas

\item Programas: un programa es una lista\footnote{En realidad es una pila y 
elegimos implementarla como una lista, lo mismo sucede con la memoria} de 
instrucciones
$p=[i_1,\ldots,i_n]$
% \[
% P (nil)\;\;\;\;\;\;
% \fa x\fa y(I(x)\land P(y)\to P(cons x y))
% \]

\item Memoria: la memoria es una lista de valores
$s=[v_1,\ldots,v_n]$ donde $v_i\in\{n,t,f\}$
% \[
% M(nil)\;\;\;\;;\
% \fa x \fa y((Nat(x)\lor Bool(x))\land M(y)\to S(push x y))
% \]

\item Ejecución de una instrucción: la función de ejecución de una instrucción 
$\ej$ recibe una instrucción {\tt i} y una memoria {\tt s} devolviendo la 
memoria resultante al ejecutar la instrucción.

  \begin{alltt}
    ej n s = (n:s)
    ej true s = (t:s)
    ej false s = (f:s)
    ej + (v1:v2:s) = (v1+v2) : s 
    ej * (v1:v2:s) = (v1*v2) : s
    ej < (v1:v2:s) = (v1<v2) : s
    ej \(\no\) (v:s) = (\(\neg\) v) : s
    ej \(\iszero\) (v:s) = (isz v) : s
  \end{alltt}

donde del lado derecho de las ecuaciones nos referimos a las operaciones 
binarias $+,*::\N\times\N\to\N,\;
<:\N\times\N\to\B$ y a las operaciones unarias $\neg:\B\to\B,\;isz:\N\to\B$ 

% \[
% \ba{rll}
% ej n s = push n s
% \ea
% \]


\item Ejecución de un programa: ejecutar un programa $p$ en la memoria dada $s$ 
devuelve la memoria obtenida al 
 ejecutar en orden todas instrucciones de $p$.

  \begin{alltt}
    ejp nil s = s
    ejp (i:p) s = ejp p (ej i s)
  \end{alltt}

\item Compilación de una expresión en un programa: el proceso de compilación 
convierte una expresión $e$ en un programa. 

  \begin{alltt}
    comp n = [n]
    comp true = [t]
    comp false = [f]
    comp (e1+e2) = comp e2 ++ comp e1 ++ [+]
    comp (e1*e2) = comp e2 ++ comp e1 ++ [*]
    comp (e1<e2) = comp e2 ++ comp e1 ++ [<]
    comp (not e) = comp e ++ [not] 
    comp (iszero e) = comp e ++ [iszero] 
  \end{alltt}

donde $\ap$ es la función de concatenación de listas. ?` Porqué en los casos de 
operadores binarios queda el programa resultado de compilar la segunda 
expresión $e_2$ al inicio de la lista?


\item Correctud del compilador: la memoria resultado de ejecutar el programa 
$p$ obtenido al compilar la expresión $e$ con la memoria vacía coincide con la 
memoria cuyo único valor es la evaluación de la expresión $e$

  \begin{alltt}
    \(\fa\) e (ejp (comp e) nil = [eval e])
  \end{alltt}
%\fa e (ejp (comp e) nil = [eval e])

Por supuesto en todas las definiciones anteriores estamos suponiendo que las 
expresiones de entrada son coherentes en el sentido de que cualquier operación 
estará bien definida. Es decir, estamos suponiendo que el proceso de 
verificación previa 
fue hecho por el sistema de tipos. Por lo que el cuantificador $\fa e$ se 
refiere únicamente a todas las expresiones semánticamente correctas.

\ei


\subsection{Prueba de la correctud (informal)}



Para probar la correctud del compilador  vamos a probar algo más general:

  \begin{alltt}
                  \(\fa\) e \(\fa\) p \(\fa\) s \big(ejp ( comp e \( \ap \) p) 
s =  ejp p (eval e:s)\big)
  \end{alltt}

de esta propiedad la correctud del compilador resulta un corolario tomando 
$p=nil,s=nil$.

\espc

La prueba es por inducción sobre las expresiones. Analizamos aquí un caso base 
y dos casos inductivos:

\bi
\item Base $e=n$: P.D. $\fa p\fa s(\ejp\; (\comp n\ap p)\; s =  \ejp\; p\; 
(\eval n:s))$.
\[
\ba{rll}
\ejp\;(\comp n\ap p)\; s & = & \ejp\; ( [n]\ap p)\; s\\
& = & \ejp\;(n:p)\;s\\
& = & \ejp\;p\;(\ej n\;s)\\
& = & \ejp\;p\;(n:s) \\
& = & \ejp\;p\;(\eval n:s) \\
\ea
\]
\item Paso inductivo $e=e_1*e_2$.
\bi
\item I.H.1.: $\fa p\fa s(\ejp\; (\comp e_1\ap p)\; s =  \ejp\; p\; (\eval 
e_1:s))$
\item I.H.2.: $\fa p\fa s(\ejp\; (\comp e_2\ap p)\; s =  \ejp\; p\; (\eval 
e_2:s))$
\ei

Queremos demostrar que:
\[
\fa p\fa s(\ejp\; (\comp (e_1*e_2)\ap p)\; s =  \ejp\; p\; (\eval (e_1*e_2):s))
\]

\[
\ba{rll}
\ejp\;(\comp (e_1*e_2)\ap p)\; s & = & 
\ejp\; ( (\comp e_2\ap\comp e_1\ap[*])\ap p)\; s\\
& = & \ejp\; ( (\comp e_2\ap(\comp e_1\ap[*]\ap p))\; s\\
& =_{Asoc\ap} & \ejp\; ( (\comp e_2\ap(\comp e_1\ap[*]\ap p))\; s\\
& =_{I.H.\text{2}} & \ejp\; (\comp e_1\ap[*]\ap p))\; (\eval e_2:s)\\
& =_{Asoc\ap} & \ejp\; (\comp e_1\ap([*]\ap p))\; (\eval e_2:s)\\
& =_{I.H.\text{1}} & \ejp\; ([*]\ap p)\; (\eval e_1:\eval e_2:s)\\
& = & \ejp\; (*:p)\; (\eval e_1:\eval e_2:s)\\
& = & \ejp\; p\; (\ej\; [*]\; (\eval e_1:\eval e_2:s))\\
& = & \ejp\; p\; (\eval e_1*\eval e_2:s)\\
& = & \ejp\; p\; (\eval (e_1*e_2):s)\\
% & & \ejp\;(n:p)\;s\\
% & & \ejp\;p\;(\ej n\;s)\\
% & = & \ejp\;p\;(n:s) \\
% & = & \ejp\;p\;(\eval n:s) \\
\ea
\]


\item Paso inductivo $e=\no e_1$.
\bi
\item I.H: $\fa p\fa s(\ejp\; (\comp e_1\ap p)\; s =  \ejp\; p\; (\eval e_1:s))$
%\item I.H.2.: $\fa p\fa s(\ejp\; (\comp e_2\ap p)\; s =  \ejp\; p\; (\eval 
% e_2:s))$
\ei

Queremos demostrar que:
\[
\fa p\fa s(\ejp\; (\comp (\no e_1)\ap p)\; s =  \ejp\; p\; (\eval (\no e_1):s))
\]

\[
\ba{rll}
\ejp\;(\comp (\no e_1)\ap p)\; s & = & 
\ejp\; ( (\comp e_1\ap[\neg])\ap p)\; s\\
%& = & \ejp\; ( (\comp e_2\ap(\comp e_1\ap[*]\ap p))\; s\\
& =_{Asoc\ap} & \ejp\; ( (\comp e_1\ap([\neg]\ap p))\; s\\
% & =_{I.H.\text{2}} & \ejp\; (\comp e_1\ap[*]\ap p))\; (\eval e_2:s)\\
% & =_{Asoc\ap} & \ejp\; (\comp e_1\ap([*]\ap p))\; (\eval e_2:s)\\
& =_{I.H} & \ejp\; ([\neg]\ap p)\; (\eval e_1:s)\\
& = & \ejp\; (\neg:p)\; (\eval e_1:s)\\
& = & \ejp\; p\; (\ej\; [\neg]\; (\eval e_1:s))\\
& = & \ejp\; p\; (\neg(\eval e_1):s)\\
& = & \ejp\; p\; (\eval (\no e_1):s)\\
\ea
\]

\ei

%de manera que por el principio de inducción estructural para expresiones 
% concluimos que

A continuación damos una idea de la formalización de esta prueba en el sistema 
de deducción natural.


\subsection{Bosquejo de la prueba formal (dentro del sistema de deducción 
natural)}

Queremos probar que $\G\vdash \fa e P(e)$ donde $\G$ es un conjunto de premisas 
adecuado y $P(e)$ es la fórmula que formaliza la propiedad requerida.

\bi
\item La fórmula a probar es $P(e)$ definida como
\[
P(e)=_{def} \fa p\fa s\big(\ejp\;(\comp e\ap p)\;s = \ejp\;p\;(\eval e : s)\big)
\]

\item $\G$ se define como la siguiente unión de conjuntos de fórmulas:
\[
\G=\{\I_P\}\cup\mathbb{F}\cup\mathbb{L}\cup\mathbb{E}
\]

donde

\item $\I_P$ es el principio de inducción para expresiones para el caso de $P$:
\[
\ba{rll}
\I_P & =_{def} & P(n) \land P(\true)\land P(\false) \land \\
     & & \fa e_1\fa e_2(P(e1)\land P(e_2)\to P(e_1+e_2))\;\land \\   
     & & \fa e_1\fa e_2(P(e1)\land P(e_2)\to P(e_1*e_2))\;\land \\    
     & & \fa e_1\fa e_2(P(e1)\land P(e_2)\to P(e_1<e_2))\;\land \\   
     & & \fa e(P(e)\to P(\no e))\land \\   
     & & \fa e(P(e)\to P(\iszero e)) \\   
     & & \to \fa eP(e) \\   
\ea
\]

\item $\mathbb{F}$ consta de las cerraduras universales de las ecuaciones que 
definen a las funciones $\eval,\ej,\ejp,\comp$. 
\[
\ba{rll}
\mathbb{F} & = \{ & \fa n (\eval n = n),\ldots,\\ 
 & & \fa e_1\fa e_2\big(\eval (e1 + e2) = \eval e1 + \eval e2\big),\ldots \\
 & & \fa v1\fa v2\fa s\big( \ej < (v1:v2:s) = (v1<v2) : s\big),\ldots\\
 & & \vdots \\
 & & \fa e\big(\comp (\iszero e) = \comp e \ap [\iszero]\big) 
\\ & \;\;\;\} &
\ea
\]


\item $\mathbb{L}$ consta de las propiedades requeridas de las operaciones de 
listas:
\bi
\item Concatenación de lista unitaria: $\fa x\fa \ell([x]\ap\ell=x:\ell)$.
\item Asociatividad de la concatenación:
\[
\fa x\fa y\fa z\big(x\ap(y\ap z)=(x\ap y)\ap z\big)
\]
\ei
\item $\Eb$ contiene los llamados axiomas de igualdad:
\bi
\item Reflexividad: $\fa x (x=x)$
\item Simetría $\fa x\fa y(x=y \to y = x)$
\item Transitividad: $\fa x\fa y\fa z(x=y\land y=z \to x=z)$
\item Sustitución (Leibniz): $\fa x\fa y (x=y\land A(x)\to A(y))$ donde $A$ es 
una fórmula cualquiera\footnote{Basta con agregar la instancia de esta fórmula 
para las fórmulas $A$ que necesitemos}
\ei
Alternativamente en vez de usar estas fórmulas directamente para derivar nuevas 
fórmulas o ecuaciones podemos utilizar la lógica ecuacional discutida antes, es 
decir, usar las siguientes reglas: 
% usarlas una vez para mostrar que las siguientes reglas de deducción natural 
para la igualdad son admisibles y en adelante usar directamente estas reglas 
adicionales para la igualdad: aquí $\G$ es cualquier contexto
\[
\inferrule*[right=]{}{\G\vdash 
t=t}(Refl)\;\;\;\;\;\;\inferrule*[right=]{\G\vdash s=t}{\G\vdash t =s}(Sim)
\;\;\;\;\;\;\;\;\;\;\inferrule*[right=]{\G\vdash s=t\;\;\;\;\;\G\vdash 
t=r}{\G\vdash s =r}(Trn)
\]


\[
\inferrule*[right=]{\G\vdash s=t\;\;\;\G\vdash A(s)}{\G\vdash A(t)}(Rewrite)
\]
\ei
donde $A$ es cualquier fórmula.

\espc


A continuación bosquejamos la prueba formal $\G\vdash \fa e P(e)$.

\bi
\item Para esto basta probar el antecedente del axioma de inducción $\I_P$ y 
usar modus ponens.
\item Como dicho antecedente es una conjunción basta probar cada parte por 
separado y usar la regla $(\land I)$.
\item Cada parte se prueba formalizando las pruebas informales, dado que el 
razonamiento ecuacional está permitido por las hipótesis de $\Eb\inc \G$ y por 
las propiedades de listas $\mathbb{L}\inc\G$. Como ejemplo formalizemos el caso 
base para números naturales:
\bi
\item Base P.D. $\G\vdash P(n)$, es decir, $\G\vdash \fa p\fa s(\ejp\; (\comp 
n\ap p)\; s =  \ejp\; p\; (\eval n:s))$.
\[
\ba{rll}
1.\;\;\G\vdash & \ejp\;(\comp n\ap p)\; s  =  \ejp\;(\comp n\ap p)\; s & (Refl) 
\\
2.\;\;\G\vdash & \ejp\;(\comp n\ap p)\; s  =  \ejp\; ( [n]\ap p)\; s & 
(Rewrite)\;1\; def. \comp n\\
3.\;\; \G\vdash & \ejp\;(\comp n\ap p)\; s  = \ejp\;(n:p)\;s & (Rewrite)\;2\; 
prop.\;de\; listas\\
4.\;\; \G\vdash & \ejp\;(\comp n\ap p)\; s  =  \ejp\;p\;(\ej n\;s) & 
(Rewrite)\;3\; def. \ejp \\
5.\;\;\G\vdash & \ejp\;(\comp n\ap p)\; s  =  \ejp\;p\;(n:s) & (Rewrite)\;4\; 
def. \ej \\
6.\;\;\G\vdash & \ejp\;(\comp n\ap p)\; s  =  \ejp\;p\;(\eval n:s) & 
(Rewrite)\;5\; def. \eval\\
7.\;\;\G\vdash & \fa s\big(\ejp\;(\comp n\ap p)\; s  =  \ejp\;p\;(\eval 
n:s)\big) & (\fa I)\;6,\;s\notin FV(\G)\\
8.\;\;\G\vdash & \fa p\fa s\big(\ejp\;(\comp n\ap p)\; s  =  \ejp\;p\;(\eval 
n:s)\big) & (\fa I)\;7,\;p\notin FV(\G)\\

\ea
\]
% \item Paso inductivo $e=e_1*e_2$.
% \bi
% \item I.H.1.: $\fa p\fa s(\ejp\; (\comp e_1\ap p)\; s =  \ejp\; p\; (\eval 
% e_1:s))$
% \item I.H.2.: $\fa p\fa s(\ejp\; (\comp e_2\ap p)\; s =  \ejp\; p\; (\eval 
% e_2:s))$
% \ei

% Queremos demostrar que:
% \[
% \fa p\fa s(\ejp\; (\comp (e_1*e_2)\ap p)\; s =  \ejp\; p\; (\eval 
% (e_1*e_2):s))
% \]

% \[
% \ba{rll}
% \ejp\;(\comp (e_1*e_2)\ap p)\; s & = & 
% \ejp\; ( (\comp e_2\ap\comp e_1\ap[*])\ap p)\; s\\
% & = & \ejp\; ( (\comp e_2\ap(\comp e_1\ap[*]\ap p))\; s\\
% & =_{Asoc\ap} & \ejp\; ( (\comp e_2\ap(\comp e_1\ap[*]\ap p))\; s\\
% & =_{I.H.\text{2}} & \ejp\; (\comp e_1\ap[*]\ap p))\; (\eval e_2:s)\\
% & =_{Asoc\ap} & \ejp\; (\comp e_1\ap([*]\ap p))\; (\eval e_2:s)\\
% & =_{I.H.\text{1}} & \ejp\; ([*]\ap p)\; (\eval e_1:\eval e_2:s)\\
% & = & \ejp\; (*:p)\; (\eval e_1:\eval e_2:s)\\
% & = & \ejp\; p\; (\ej\; [*]\; (\eval e_1:\eval e_2:s))\\
% & = & \ejp\; p\; (\eval e_1*\eval e_2:s)\\
% & = & \ejp\; p\; (\eval (e_1*e_2):s)\\
% % & & \ejp\;(n:p)\;s\\
% % & & \ejp\;p\;(\ej n\;s)\\
% % & = & \ejp\;p\;(n:s) \\
% % & = & \ejp\;p\;(\eval n:s) \\
% \ea
% \]


% \item Paso inductivo $e=\no e_1$.
% \bi
% \item I.H: $\fa p\fa s(\ejp\; (\comp e_1\ap p)\; s =  \ejp\; p\; (\eval 
% e_1:s))$
% %\item I.H.2.: $\fa p\fa s(\ejp\; (\comp e_2\ap p)\; s =  \ejp\; p\; (\eval 
% e_2:s))$
% \ei

% Queremos demostrar que:
% \[
% \fa p\fa s(\ejp\; (\comp (\no e_1)\ap p)\; s =  \ejp\; p\; (\eval (\no 
% e_1):s))
% \]

% \[
% \ba{rll}
% \ejp\;(\comp (\no e_1)\ap p)\; s & = & 
% \ejp\; ( (\comp e_1\ap[\neg])\ap p)\; s\\
% %& = & \ejp\; ( (\comp e_2\ap(\comp e_1\ap[*]\ap p))\; s\\
% & =_{Asoc\ap} & \ejp\; ( (\comp e_1\ap([\neg]\ap p))\; s\\
% % & =_{I.H.\text{2}} & \ejp\; (\comp e_1\ap[*]\ap p))\; (\eval e_2:s)\\
% % & =_{Asoc\ap} & \ejp\; (\comp e_1\ap([*]\ap p))\; (\eval e_2:s)\\
% & =_{I.H} & \ejp\; ([\neg]\ap p)\; (\eval e_1:s)\\
% & = & \ejp\; (\neg:p)\; (\eval e_1:s)\\
% & = & \ejp\; p\; (\ej\; [\neg]\; (\eval e_1:s))\\
% & = & \ejp\; p\; (\neg(\eval e_1):s)\\
% & = & \ejp\; p\; (\eval (\no e_1):s)\\
% \ea
% \]

\ei
\ei


% \[
% P(e,p,s)=_{def} \ejp (\comp e ++ p) s = \ejp p (\eval e : s)
% \]
% \item Principio de inducción para $P$:
% \[
% \ba{rll}
% \I_P & =_{def} & P(n,p,s) \land P(\true,p,s)\land P(\false,p,s) \\
%      & & \fa e_1\fa e_2(P(e1)   
% \ea
% \]


% \bi
% \item $\fa x\fa y()$
% \item $C(n,push n nil)$
% \item $C(\true,push \true nil)$
% \item $C(\false,push\false nil)$
% \item $\fa x\fa y\fa w\fa v\fa z(C(x,w)\land C(y,v)\to C(x+y, w++v++[+])$
% \item $\fa x\fa y\fa w\fa v\fa z(C(x,w)\land C(y,v)\to C(x*y, w++v++[*])$
% \item $\fa x\fa y\fa w\fa v\fa z(C(x,w)\land C(y,v)\to C(x<y, w++v++[<])$
% \item $\fa x\fa y(C(x,w)\to C(not x, w++[\neg])$
% \item $\fa x\fa y(C(x,w)\to C(\iszero x, w++[isz])$
% \ei







% Como se observa los sistemas de tipos son muy similares a los sistemas
% de deducción natural con contextos que ya hemos estudiado. De hecho
% los sistemas de deducción natural pueden verse, desde el punto de
% vista computacional, como un sistema de tipos para un prototipo de lenguaje de 
% programación
% funcional mediante la llamada correspondencia de Curry-Howard, también
% conocida como paradigma de derivaciones como programas o proposiciones
% como tipos. La idea a grandes rasgos es:

% \bi
% \item Las fórmulas de la lógica corresponden a tipos para un lenguaje
%   de programación.
%   \item Las pruebas o derivaciones en la lógica corresponden a los
%     pasos que el verificador de tipos realiza para mostrar que cierta
%     expresión o programa del lenguaje de programación está bien tipada.
%     \item Por lo tanto, verificar si una prueba es correcta en un contexto 
% dado corresponde a verificar
%       si un programa tiene un tipo válido en el contexto dado.
% \ei








% \section{Carácter Constructivo de Conectivos y Cuantificadores}

% Para poder poner en correspondencia el mundo de las pruebas formales con el
% mundo de los programas computacionales debemos darle un caracter constructivo
% a las primeras, de manera que puedan ser implementadas más adelante. Este
% caracter constructivo se obtiene al olvidarnos de la noción clásica 
% (platónica) de
% verdad y en su lugar entender al juicio $\vp\,\true$ mediante la existencia
% de una prueba constructiva de $\vp$, es decir la existencia de un método bien
% definido que nos lleve a construir a $\vp$. Las siguientes reglas de
% construcción se conocen como la \emph{interpretación de 
% Brouwer-Heyting-Kolmogorov (BHK)},
% y su sabor algorítmico nos llevará más tarde a la famosa correspondencia de
% Curry-Howard.\\ 



% \bi
% \item Una construcción de $\vp\land\psi$ consiste de una construcción de
%   $\vp$ y una construcción de $\psi$.
% \item Una construcción de $\vp\lor\psi$ consiste de una construcción de $\vp$
%   ó de una construcción de $\psi$.
% \item Una construcción de $\vp\imp\psi$ consiste de un método que transforma
%   cada construcción de $\vp$ en una construcción de $\psi$.
% \item Una construcción de $\fa x\vp$ es un método que transforma
%   cualquier objeto $t$ en una construcción de $\vp[x:=t]$.
%   \item Una construcción de $\ex x\vp$ es un par consistente de un
%     objeto $t$ y una construcción de $\vp[x:=t]$.
% \item No existe una construcción para $\bot$
%   \item $\top$ siempre es construible.

% \ei



\subsection{?`Para qué una prueba formal?}

Una prueba formal como la que acabamos de bosquejar no es una argumentación en 
algún lenguaje natural, como el español, o  bien semiformal como la mezcla de 
español y  matemáticas, la cual es susceptible de errores díficiles de encontrar 
en casos de aplicación real. Por ejemplo la construcción de un compilador 
correcto (altamente confiable) de un lenguaje real como {\sc C} (ver 
\url{http://compcert.inria.fr}). Las pruebas formales constan de un proceso de 
cálculo preciso que sigue reglas bien definidas, en nuestro caso las reglas de 
deducción natural. Una ventaja de construir esta clase de pruebas es que debido 
a su naturaleza precisa son rigurosas, libres de ambigüedades y susceptibles de 
construirse y verificarse mecánicamente por programas llamados asistentes de 
prueba como {\sc Coq} (ver \url{http://coq.inria.fr}). Esta verificación a la 
vez produce una certificación que garantiza que ciertas propiedades se cumplen, 
como por ejemplo la correctud de un compilador. Esto es deseable en sistemas 
cuyo funcionamiento incorrecto causa errores graves como sistemas de radiación 
para enfermos de cancer y sistemas de navegación de vuelo. \\
El uso de sistemas lógicos para desarrollar aplicaciones con propiedades 
certificadas mediante prueba formales ha generado una nueva área dentro de las 
ciencias de la computación conocida como Métodos Formales.

\end{document}
