\documentclass[11pt,letterpaper]{article}
\usepackage{../packageslc}
\usepackage{../optionslc}

%%%% Macros para las notas de lenguajes de programacion


%%%% math

\newcommand{\vphi}{\varphi}
\newcommand{\vp}{\varphi}

% \newcommand{\dn}{\mathsf{DN}}
% \newcommand{\dnC}{\mathsf{DN_C}}
% \newcommand{\dnM}{\mathsf{DN_M}}
% \newcommand{\dnp}{\mathsf{DN_p}}
% \newcommand{\dnm}{\mathsf{DN_p^M}}
% \newcommand{\dnc}{\mathsf{DN_p^C}}

%\newcommand{\case}{\mathsf{case}}
%\renewcommand\labelitemi{$\circ$}

\newcommand{\imp}{\rightarrow}
\newcommand{\Imp}{\Rightarrow}
\renewcommand{\iff}{\leftrightarrow}
\newcommand{\Iff}{\Leftrightarrow}
\newcommand{\G}{\Gamma}
\newcommand{\D}{\Delta}


\newcommand{\De}{\mathcal{D}}
\newcommand{\F}{\mathcal{F}}
\newcommand{\Ge}{\mathcal{G}}
\newcommand{\Pe}{\mathcal{P}}
\newcommand{\I}{\mathcal{I}}
\newcommand{\C}{\mathcal{C}}
\newcommand{\K}{\mathcal{K}}
\renewcommand{\L}{\mathcal{L}}
\newcommand{\M}{\mathcal{M}}
\newcommand{\Nc}{\mathcal{N}}
%\newcommand{\E}{\mathcal{E}}
%\newcommand{\R}{\mathcal{R}}
%\newcommand{\Q}{\mathcal{Q}}
\newcommand{\Sc}{\mathcal{S}}
\newcommand{\Te}{\mathcal{T}}
\newcommand{\W}{\mathcal{W}}

\newcommand{\Db}{\mathbb{D}}
\newcommand{\Fb}{\mathbb{F}}
\newcommand{\Kb}{\mathbb{K}}
\newcommand{\Eb}{\mathbb{E}}
\newcommand{\Ebs}{\mathbb{E}^\star}
\newcommand{\Ob}{\mathbb{O}}
\newcommand{\Ib}{\mathbb{I}}
\newcommand{\Rb}{\mathbb{R}}
\newcommand{\Qb}{\mathbb{Q}}
\newcommand{\Kbb}{\mathbb{K}}
\newcommand{\T}{\mathbb{\Theta}}


\newcommand{\kb}{\bbkappa}

\newcommand{\Sf}{\mathsf{\Sigma}}

\newcommand{\fa}{\forall}
\newcommand{\ex}{\exists}

\newcommand{\inc}{\subseteq}

\newcommand{\Lb}{\Lambda}
\newcommand{\Om}{\Omega}
\newcommand{\lb}{\lambda}
\newcommand{\al}{\alpha}
\newcommand{\ga}{\gamma}


\newcommand{\mg}{\mathbb{m}}

\newcommand{\cg}{\mathbb{C}}
\newcommand{\dg}{\mathbb{D}}
\newcommand{\jg}{\mathbb{J}}
\newcommand{\Ha}{\mathcal{H}}
%\newcommand{\A}{\mathcal{A}}
\newcommand{\sg}{\mathbb{S}}

\newcommand{\Bc}{\mathcal{B}}
\newcommand{\Df}{\mathfrak{D}}
\newcommand{\Dc}{\mathcal{D}}
%\newcommand{\Tc}{\mathcal{T}}
\newcommand{\Mf}{\mathfrak{M}}

\newcommand{\Sg}{\mathbb{S}}

\newcommand{\Q}{\ensuremath{\mathbb{Q}}}
\newcommand{\Z}{\ensuremath{\mathbb{Z}}}
\newcommand{\N}{\ensuremath{\mathbb{N}}}
\newcommand{\R}{\ensuremath{\mathbb{R}}}
\renewcommand{\S}{\mathbb{\Sigma}}
\newcommand{\A}{\mathcal{A}}
\newcommand{\E}{\ensuremath{\exists}}
\newcommand{\iso}{\ensuremath{\cong}}
\newcommand{\union}{\ensuremath{\cup}}
\newcommand{\morinyec}{\ensuremath{\precapprox}}

\newcommand{\nin}{\ensuremath{\notin}}
\newcommand{\tog}{\makebox[7mm][l]}
\newcommand{\toge}{\makebox[11mm][l]}
\newcommand{\toget}{\makebox[13mm][l]}
\newcommand{\togeth}{\makebox[14mm][l]}
\newcommand{\togethe}{\makebox[15mm][l]}
\newcommand{\together}{\makebox[17mm][l]}
\newcommand{\niso}{\ensuremath{\not \cong}}


\newcommand{\Mg}{\mathbb{M}}
\newcommand{\Bg}{\mathbb{B}}
\newcommand{\Lg}{\mathbb{L}}
\newcommand{\Tg}{\mathbb{T}}

\newcommand{\sketch}{\Red{{\sc sketch}}}

\newcommand{\restr}[2]{#1\!\!\boldsymbol{\restriction}\!#2}

\newcommand{\vacio}{\varnothing}
\newcommand{\done}{\ensuremath{\checkmark}}

\newcommand{\ida}{$\Rightarrow \; )$ }
\newcommand{\regr}{$\Leftarrow \; )$ }

\newcommand{\ol}[1]{\overline{#1}}

\newcommand{\Tsf}{\mathsf{T}}

\newcommand{\inds}[1]{\index[simb]{#1}}

\newcommand{\B}{\mathbb{B}}
%\newcommand{\N}{\mathbb{N}}

\newcommand{\vx}{\vec{x}}
\newcommand{\vy}{\vec{y}}
\newcommand{\vz}{\vec{z}}
\newcommand{\vt}{\vec{t}}
\newcommand{\vf}{\vec{f}}


% \newcommand{\propo}{\ensuremath{\mathsf{PROP}}}
% \newcommand{\atom}{\ensuremath{\mathsf{ATOM}}}
\newcommand{\term}{\ensuremath{\mathsf{TERM}}}
\newcommand{\form}{\mathsf{FORM}}

\newcommand{\true}{\mathop{\mathsf{true}}}

%\newcommand{\id}{\mathsf{Id}}

%\newcommand{\uc}{\mathcal{U}}
%\newcommand{\Ic}{\mathcal{I}}
%\newcommand{\pc}{\mathcal{P}}
%\newcommand{\qc}{\mathcal{Q}}
%\newcommand{\mc}{\mathcal{M}}
\newcommand{\supc}{\supseteq}
\newcommand{\limo}{\mathop{\mathpzc{Lim}}}
\newcommand{\ord}{\mathsf{OR}}

\newcommand{\pt}[1]{\langle #1 \rangle}


%%%% frames
\newcommand{\titulos}[2]{\frametitle{#1}\framesubtitle{#2}}
\newcommand{\fot}[1]{\footnote{\scriptsize{#1}}}


%%%% ambientes

\newcommand{\cb}[2]{\colorbox{#1}{#2}}

\newcommand{\bc}{\begin{center}}
\newcommand{\ec}{\end{center}}
\newcommand{\be}{\begin{enumerate}}
\newcommand{\ee}{\end{enumerate}}
\newcommand{\bi}{\begin{itemize}}
\newcommand{\ei}{\end{itemize}}
\newcommand{\beq}{\begin{equation}}
\newcommand{\eeq}{\end{equation}}
\newcommand{\beqs}{\begin{equation*}}
\newcommand{\eeqs}{\end{equation*}}
\newcommand{\ba}{\begin{array}}
\newcommand{\ea}{\end{array}}


% \newtheorem{theorem}{Teorema}
% \newcommand{\teo}[1]{\begin{theorem} #1 \end{theorem}}
% \newtheorem{proposition}{Proposici\'on}
% \newcommand{\prop}[1]{\begin{proposition} #1 \end{proposition}}
% \newtheorem{definition}{Definici\'on}
% \newcommand{\defin}[1]{\begin{definition} #1 \end{definition}}
% \newtheorem{corollary}{Corolario}
% \newcommand{\cor}[1]{\begin{corollary} #1 \end{corollary}}
% \newtheorem{lemma}{Lema}
% \newcommand{\lema}[1]{\begin{lemma} #1 \end{lemma}}
% \newcommand{\dem}[1]{\begin{proof} #1 \end{proof}}

%\renewcommand{\qed}{\qedsymbol{$\mathbf{\dashv}$}}

%\newcommand{\proof}{\hfill\\\noindent\textbf{\textit{Demostraci\'on. }}}

\newcommand{\hint}{\emph{Sugerencia: }}


\newcounter{EjempCtr}[section]
\newenvironment{enumrom}{\renewcommand{\theenumi}{\roman{enumi}}%
\renewcommand{\theenumii}{\roman{enumii}}
\renewcommand{\theenumiii}{\roman{enumiii}}
\renewcommand{\theenumiv}{\roman{enumiv}}
\begin{enumerate}}{\end{enumerate}}
\newenvironment{Ejemplo}
        {\stepcounter{EjempCtr}%
        \begin{description}\item[Ejemplo \thesection.\arabic{EjempCtr}]}%
        {\end{description}}
\newenvironment{demostr}{%
             {\em Demostración:}
                \begin{quotation}}{\end{quotation}}

   \newcommand{\beje}{\begin{Ejemplo}}
\newcommand{\eeje}{\end{Ejemplo}}


\newtheorem{eje}{Ejemplo}[section]
\newcommand{\ejem}[1]{\begin{eje}\normalfont #1 \end{eje}}

% \renewcommand\contentsname{\'Indice}
%\renewcommand\chaptername{Cap\'itulo}
% \renewcommand\indexname{\'Indice}

%%\newcommand{\qed}{\hfill$\mathbb{Qed}$}
%\newcommand{\qed}{\hfill$\mathsf{\boldsymbol{\dashv}}$}
%\renewcommand{\qed}{\hfill$\boldsymbol{\dashv}$}


\newenvironment{prueba}{\vspace{-5mm}\noindent\textbf{Demostraci\'on}\\}{
\noindent$\blacksquare$\\}

\newcommand{\Ejercicios}{\section*{Ejercicios}}

 \newenvironment{manitas}{%
      \renewcommand{\labelitemi}{\ding{44}}%
      \vspace{-0.5cm}%
      \begin{itemize}%
      \setlength{\itemsep}{0pt}\setlength{\parsep}{0pt}\setlength{\topsep}{0pt}%
      }{\end{itemize}}
\newenvironment{malitos}{%
      \renewcommand{\labelitemi}%
            {\raisebox{1.5ex}{\makebox[0.3cm][l]{\begin{rotate}{-90}%
            \ding{43}\end{rotate}}}}%
      \vspace{-0.5cm}%
      \begin{itemize}%
      \setlength{\itemsep}{0pt}\setlength{\parsep}{0pt}\setlength{\topsep}{0pt}%
      }{\end{itemize}}
\newenvironment{ejercs}{
     \renewcommand{\labelenumi}{\thesection.\theenumi.-}
     \renewcommand{\labelenumii}{\theenumii)}
     \begin{enumerate}}
     {\end{enumerate}}

   \newcommand{\bej}{\begin{ejercs}}
\newcommand{\eej}{\end{ejercs}}


%\newenvironment{leterize}{%
%        \renewcommand{\theenumi}{\alph{enumi}}
%        \begin{enumerate}}{\end{enumerate}}

%\newenvironment{manitas}{%
%      \renewcommand{\labelitemi}{\ding{44}}%
%      \vspace{-0.5cm}%
%      \begin{itemize}%
%      
% \setlength{\itemsep}{0pt}\setlength{\parsep}{0pt}\setlength{\topsep}{0pt}%
%      }{\end{itemize}}



%%=============================================================================

\def\stackunder#1#2{\mathrel{\mathop{#2}\limits_{#1}}}


%%%% notas

\newcommand{\doubt}{\Red{{\LARGE {\sf ??}}}}

\newcommand{\coment}[1]{\hfill\\ \Big[{\bf Comentario Privado:} #1\Big]}
\newcommand{\preg}[1]{\hfill\\ \BrickRed{{\bf Pregunta:} #1}}
\newcommand{\conjet}[1]{\hfill\\ \OliveGreen{{\bf Conjecura:} #1}}

\newcommand{\pendiente}{\BrickRed{{\sc Pendiente}}}
\newcommand{\verifpendiente}{\BrickRed{{\sc Verificación pendiente}}}


%--------------------------------------------------------------------------

\DeclareMathAlphabet{\mathpzc}{OT1}{pzc}{m}{it}


\title{Lógica Computacional 2017-2, nota de clase 12\\
Significado de programas l\'ogicos: Modelos de Herbrand}
\author{Favio Ezequiel Miranda Perea \and Araceli Liliana Reyes Cabello\and
Lourdes Del Carmen Gonz\'alez Huesca \and Pilar Selene Linares Arévalo}
\date{\today}

\begin{document}
\maketitle

\noindent 
Recordemos los dos significados que pueden tener los programas l\'ogicos:
\begin{itemize}
 \item Significado declarativo: 
  el significado de un programa l\'ogico $\P$ es el conjunto de todas 
  las consecuencias l\'ogicas del programa, noci\'on asociada a las respuestas 
  correctas.
 \item Significado operacional: 
  el significado de un programa l\'ogico $\P$ es el conjunto de todas 
  los \'exitos del programa, noci\'on asociada a la de respuesta computada.
\end{itemize}
Estudiaremos ahora la equivalencia entre estos dos significados utilizando los 
modelos de Herbrand para poder definir formalmente la sem\'antica de un programa 
l\'ogico.

\vspace*{-15pt}

\section{Universo y Modelos de Herbrand}

\defin{[Universo de Herbrand] Sea $\L$ un lenguaje de primer orden. El universo 
de Herbrand de $\L$ se define como el conjunto $\H_\L$ de los t\'erminos 
cerrados de $\L$. Es decir
$$ \H_\L=\{t\mid t\text{ es un } \L\text{-t\'ermino cerrado }\} $$
}
\ejem{\hfill
\be
\item Si $\L_1=\{a,b, P^{(1)},Q^{(1)},R^{(2)}\}$ entonces 
$\;\H_{\L_1}=\{a,b\}$
\item Si $\L_2=\{c,a,f^{(1)},g^{(1)}, P^{(1)}\}$ entonces\\
$ \H_{\L_2}=\{c,a,f(c),f(a),g(c),g(a),f(f(a)),f(g(a)),f(f(c)),f(g(c)),g(g(a)),
g(g(c)),\ldots\} $
\ee
}

Obs\'ervese que si el lenguaje NO contiene s\'imbolos de funci\'on entonces el
universo de Herbrand es finito y es simplemente el conjunto de s\'imbolos de 
constante. 
Por otro lado si el lenguaje contiene al menos un s\'imbolo de funci\'on y un 
s\'imbolo de constante el universo de Herbrand se vuelve infinito. Si en un
lenguaje no hubiera constantes se agrega una para poder formar el universo de 
Herbrand.

\defin{[Base de Herbrand]
Sean $\L$ un lenguaje y $\H_\L$ su universo de Herbrand. La base de Herbrand 
$\B_\L$ de $\L$ se define como el conjunto de f\'ormulas at\'omicas cerradas, 
es decir, el conjunto de f\'ormulas at\'omicas cuyos argumentos pertenecen al 
universo de Herbrand de $\L$.
}

\ejem{Con respecto a los lenguajes del ejemplo anterior se tiene
\be
\item $\B_{\L_1}=\{Pa,Pb, Qa,Qb,Raa,Rbb, Rab, Rba\}$
\item 
  $\B_{\L_2}=\{Pc,Pa,Pfc,Pfa,Pgc,Pga,Pffa,Pfga,Pffc,Pfgc,Pgga,Pggc,\ldots\}$
\ee
}

\smallskip

En particular nos interesar\'a construir el universo y base de Herbrand de un 
programa l\'ogico, definidos como sigue: 
\vspace*{-5pt}
\defin{Dado un programa l\'ogico $\P$ definimos el universo $\H_\P$  y la
base de Herbrand $\B_\P$ de $\P$ como:
\[
\H_\P=_{def}\H_{\L(\P)}\qquad \qquad \B_\P=_{def}\B_{\L(\P)}
\]
 donde $\L(\P)$ es el lenguaje de los s\'imbolos de constante,
 funci\'on y predicado que figuran en $\P$.
}



\beje
Si $\P=\{nat(0).,\;nat(s(X))\impp nat(X).\}$ entonces
$\L(\P)=\{nat^{(1)},0,s^{(1)}\}$ y
\[
\H_\P=\{0,s(0),s(s(0)),\ldots\}\;\;\;\;\;\B_\P=\{nat(0),nat(s(0)),nat(s(s(0))),
\ldots\}
\]
\eeje


\beje
Si $\P$ es el programa l\'ogico para caminos en una gr\'afica (como en la nota 
11) entonces el lenguaje de $\P$ es $\L(\P)=\{a,b,c,d,e,f, edge^{(2)}, 
path^{(2)}\}$ y
$ \H_\P=\{a,b,c,d,e,f\} $
\[
\B_\P=\{edge(a,a),edge(a,b)\ldots, edge(f,e), edge(f,f),
path(a,a),path(a,b)\ldots,path(f,e),path(f,f)\}
\]
\eeje


\defin{[Interpretaci\'on de Herbrand]Decimos que una interpretaci\'on
$\M=\pt{M,\I}$ de un lenguaje dado $\L$ es una interpretaci\'on o modelo
de Herbrand si y s\'olo si se cumple lo siguiente:
\be
\item $M =\H_\L$ es decir, el universo de $\M$ es el universo de Herbrand de
  $\L$.
\item Para todo s\'imbolo de constante $c$, se cumple $\I(c)=c$.
\item Para todo s\'imbolo de funci\'on $f^{(n)}$, se cumple
  $\I(f(t_1,\ldots,t_n))=f(t_1,\ldots,t_n)$.
\item Para todo s\'imbolo de predicado $P^{(n)}$ es un subconjunto de 
$P^\N\inc\H_\L^n$
\ee
}

Obs\'ervese que en una interpretaci\'on de Herbrand los s\'imbolos de constante 
y de funci\'on se interpretan como ellos mismos y que lo \'unico que no est\'a
determinado es la interpretaci\'on de los s\'imbolos de predicado por esta
raz\'on a los modelos de Herbrand tambien se les llama modelos sint\'acticos.


\subsection{Representaci\'on de modelos de Herbrand}

Una manera \'util de representar modelos de Herbrand es mediante subconjuntos de
la base de Herbrand del lenguaje en cuesti\'on. Dado un modelo de Herbrand 
$\M$, representamos a $\M$ mediante el subconjunto $B_\M\inc\B_\H$ de aquellas 
f\'ormulas at\'omicas que son verdaderas en $\M$. De esta manera, si la base es 
finita, es posible construir todos los modelos de Herbrand, y en el caso en que 
la base es infinita el n\'umero de modelos de Herbrand tambi\'en, pero pueden 
enumerarse efectivamente como subconjuntos de la base.

\beje
Si $\mathcal{L}=\{a, P^{(1)},Q^{(1)}\}$ entonces 
\[
\H_\L=\{a\}\;\;\;\;\B_\L=\{Pa,Qa\}
\]
y existen cuatro interpretaciones o modelos de Herbrand para $\L$:
\begin{enumerate}
\item $\M_1$ tal que $\M_1\models Pa$ y $\M_1\models Qa$
\item $\M_2$ tal que $\M_2\models Pa$ y $\M_2\not\models Qa$
\item $\M_3$ tal que $\M_3\not\models Pa$ y $\M_3\models Qa$
\item $\M_4$ tal que $\M_4\not\models Pa$ y $\M_4\not\models Qa$
\end{enumerate}
Cada modelo corresponde a un subconjunto $B_{\M_i}$ de la base como sigue:
\begin{enumerate}
\item $B_{\M_1}=\B_\L=\{Pa,Qa\}$ representa a $\M_1$
\item $B_{\M_2}=\{Pa\}$ representa a $\M_2$
\item $B_{\M_3}=\{Qa\}$ representa a $\M_3$
\item $B_{\M_4}=\varnothing$ representa a $\M_4$

\end{enumerate}
\eeje

Dado que cada subconjunto de $B\inc \B_\L$ determina de manera \'unica a un 
modelo de Herbrand $\M$,  identificamos a $\M$ con $B$. Por ejemplo, el modelo 
$\M_2$ del ejemplo anterior, se define como $\M_2=\{Pa\}$.
% \beje
% Consid\'erese el siguiente programa
% \begin{verbatim}
%   alumno(A,P) :- estudia(A,C), ensea(P,C).
%   estudia(ana,logica).
%   estudia(ana,algebra).
%   estudia(eva,algoritmos).
%   ensea(juan,logica).
%   ensea(juan,algebra)
%   ensea(carlos,algoritmos) 
% \end{verbatim}

Con esta idea en mente, si deseamos verificar que una f\'ormula at\'omica 
$A$ es verdadera en $\M$ basta ver que  $A\in B_\M$ 
% aunque usualmente abusamos de la notaci\'on usando $\M$ en vez de
% $B_\M$, es decir escribimos $A\in\M$. 

El siguiente lema nos dice c\'omo se evaluan los t\'erminos en una 
interpretaci\'on de Herbrand:

\lema{Sean $\L$ un lenguaje, $\M=\pt{M,\I}$ una interpretaci\'on de Herbrand 
para $\L$, $t$ un t\'ermino con variables $x_1,\ldots,x_n$ y $\nu$ un estado de 
las variables tal que $\nu(x_i)=r_i$ donde $r_i\in\H_\L$ es decir, $r_i$ es un 
t\'ermino cerrado, para cada $1\leq i\leq n$. Entonces: 
\[
\I_\nu(t)=t[x_1:=r_1,\ldots,x_n:=r_n]
\]
En particular si $t$ es un t\'ermino cerrado entonces $\I_\nu(t)=t$
}
\proof (Por inducci\'on sobre los t\'erminos). Ejercicio.\qed


\espc

El lema anterior nos dice que la interpretaci\'on de t\'erminos en un
modelo de Herbrand coincide con aplicar una sustituci\'on al t\'ermino
y que los t\'erminos cerrados se interpretan como ellos mismos.

% \defin{Sean $\vp$ una f\'ormula y $\sigma$ una sustituci\'on, decimos que
% $\vp\sigma$ es una instancia cerrada de $\vp$ si $\vp\sigma$ no
% contiene variables libres.
% }


\section{El Teorema de Herbrand}


El teorema de Herbrand es esencial para describir la sem\'antica de
programas l\'ogicos, a continuaci\'on lo estudiamos con detalle.

\defin{Sean $\L$ un lenguaje y $\K$ un conjunto de $\L$-f\'ormulas cerradas y 
universales. El conjunto de instancias cerradas $\vp\sigma$ o cerraduras 
universales, donde~$\fa x_1\ldots\fa x_n\vp\;$ pertenece a $\K$, se llama 
conjunto de instancias 
cerradas de $\K$ y se denota $IC(\K)$.
}
%Obs\'ervese que en realidad $IC(\K)=\{Cl(\chi)\sigma\mid\chi\in \K\;\mbox{y}
%\; Cl(\chi)\sigma\;\mbox{es una instancia cerrada}\}$ es decir el
%conjunto de instancias cerradas de $\K$ coincide con el conjunto de instancias
%cerradas de la forma clausular de cada f\'ormula de $\K$.
%%\bigskip\\ 
A continuaci\'on presentamos el Teorema Cl\'asico de Herbrand enunciado de 
forma adecuada a nuestras necesidades.
\teo{[de Herbrand]Sean $\L$ un lenguaje con al menos una constante y $\K$ un 
conjunto de $\L$-enunciados universales. Las siguientes condiciones son 
equivalentes:
\be
\item $\K$ tiene un modelo.
\item $\K$ tiene un modelo de Herbrand.
\item $IC(\K)$ tiene un modelo.
\item $IC(\K)$ tiene un modelo de Herbrand. 
\ee
}
\proof
Debido a que todo modelo de Herbrand es un modelo, las implicaciones
$b\Imp a$ y $d\Imp c$ son triviales.\\ La v\'alidez universal de la
f\'ormula $\fa x_1\ldots\fa x_n\vp\imp \vp\{x_1/t_1,\ldots,x_n/t_n\}$ hace
inmediatas a las implicaciones $a\Imp c$ y $b\Imp d$. 
As\'{\i} que basta probar la implicaci\'on $c\Imp b$.  \\
Sea $\M$ un modelo de $IC(\K)$. Definimos una estructura de Herbrand 
$\mathcal{N}=\pt{\H_\L,\I}$, para lo cual basta decir como se interpretan los 
s\'{\i}mbolos de predicado, puesto que por definici\'on los s\'{\i}mbolos de 
constante y de funci\'on se interpretan como ellos mismos. Si $P$ es un 
s\'{\i}mbolo de predicado $n$-ario, entonces definimos:  
$$P^\mathcal{N}=\{(t_1,\ldots,t_n)\mid \M\models P(t_1,\ldots,t_n) \} $$  
Obs\'ervese que, por construcci\'on de $\mathcal{N}$, se cumple que: 
$$
\M\models P(t_1,\ldots,t_n)\;\text{si  y s\'olo si } \mathcal{N}\models 
P(t_1,\ldots,t_n)
$$
es decir, $\M$ y $\mathcal{N}$ validan a las mismas f\'ormulas at\'omicas
cerradas. Dicho resultado se puede extender a cualquier f\'ormula cerrada
libre de cuantificadores mediante inducci\'on sobre f\'ormulas 
(ejercicio~\footnote{Hay que probar el siguiente \begin{lemma}Si $\vp$ es un 
enunciado sin cuantificadores entonces $\M\models\vp$ syss 
$\Nc\models\vp$.\end{lemma}}).
\\ Por \'ultimo veamos que $\mathcal{N}$ es modelo de $\K$. Sea $\fa 
x_1\ldots\fa
x_n\vp$ una f\'ormula de $\K$. Queremos demostrar que $\mathcal{N}\models\fa
x_1\ldots\fa x_n\vp$, lo cual es equivalente a mostrar que para cualesquiera 
$t_1,\ldots,t_n\in\;\H_\L,\;\;\mathcal{N}\models\vp[x_1:=t_1,\ldots,x_n:=t_n]$. 
Pero esta \'ultima afirmaci\'on es equivalente, por lo reci\'en observado a que
para cualesquiera
$t_1,\ldots,t_n\in \H_\L,\;\;\M\models\vp[x_1:=t_1,\ldots,x_n:=t_n]$,
lo cual es cierto puesto que $\vp[x_1:=t_1,\ldots,x_n:=t_n]$ pertenece a
$IC(\K)$ y por hip\'otesis, $\M$ es modelo de $IC(\K)$. De manera que 
$\mathcal{N}\models\fa x_1\ldots\fa x_n\vp$ por lo que $\mathcal{N}$ es un 
modelo de Herbrand de $\K$.\qed 


\medskip

Es importante observar que el teorema de Herbrand s\'olo es v\'alido para
enunciados universales. Por ejemplo si 
$\K=\{Pa,\ex x \neg Px\}$ entonces $\K$ es un conjunto de f\'ormulas cerradas
pero no universales y claramente tiene un modelo, digamos
$M=\{0,1\}$ con $\I(a)=0$ y $P^\M=\{0\}$, pero $\K$ no tiene un modelo de 
Herbrand pues el universo de Herbrand es $\H_\L=\{a\}$ de manera que los 
\'unicos modelos de Herbrand son los representados por $\vacio$ que corresponde 
a una  interpretaci\'on donde $Pa$ es falsa y por lo tanto no es modelo de 
$\K$; y el representado por $\{Pa\}$ que tampoco es modelo de $\K$ pues $\ex 
x\neg Px$ es falsa.

\smallskip

Dado que las cl\'ausulas son enunciados universales, se tiene el siguiente
\vspace*{-5pt}
\begin{corollary}
Sea $\mathcal{C}$ un conjunto de cl\'ausulas de Horn. Las siguientes 
condiciones son equivalentes:
  \begin{itemize}
  \item $\mathcal{C}$ tiene un modelo.
  \item $\mathcal{C}$ tiene un modelo de Herbrand.
  \end{itemize}
\end{corollary}

\noindent En lo que resta de esa nota presentamos algunos resultados 
sem\'anticos para programas l\'ogicos de Horn. 

\subsection{Sem\'atica declarativa de programas l\'ogicos}

Debido a las formas de sus cl\'ausulas, un programa l\'ogico~$\P$ siempre
tiene un modelo en particular debido al teorema de Herbrand basta
estudiar los modelos de Herbrand del programa~$\P$. El significado
declarativo del programa~$\P$ consiste de todas las consecuencias
l\'ogicas de~$\P$ y puede representarse con un modelo particular llamado
el modelo m\'inimo de Herbrand de $\P$.

% \defin{Dado un programa l\'ogico $\P$ decimos que la interpretaci\'on
%   $\M=\pt{M,\I}$ es un modelo de $\P$ si $\M$ es modelo de la cerradura
%   universal $\fa\C$ de $\C$ para toda cl\'ausula $\C\in\P$.
% }

% \teo{[de Herbrand para Programas Lgicos] Sean $\P$ un programa l\'ogico y
% $\M$ un modelo de $\P$. El conjunto $\M_\H=\{R\in\B_\P\mid\M\models R\}$ 
% representa a un modelo de  Herbrand de $\P$
% }
% Este resultado nos permite construir un modelo de Herbrand para un programa
% l\'ogico a partir de un modelo dado.


Construir la base de Herbrand de $\P$ es importante debido a la siguiente

\prop{Si $\P$ es un programa l\'ogico entonces $\P$ tiene un modelo de
  Herbrand}
% cuya representacin es $\B_\P$ la base de Herbrand de $\P$.
\proof
La base de Herbrand $\B_\P$ representa al modelo que hace verdaderas a todas
las f\'ormulas at\'omicas del lenguaje lo cual implica que todas las cl\'ausulas
de $\P$ son verdaderas debido a su forma (hechos o reglas). 
\qed

\medskip

De manera que cualquier programa l\'ogico tiene al menos un modelo de Herbrand, 
enseguida veremos que existe un modelo de Herbrand m\'inimo con respecto a la 
contenci\'on.
\prop{Sea $\P$ un programa l\'ogico. $\P$ tiene un modelo de Herbrand m\'inimo
  $\M_\P$ definido por:
%\[
\[
\M_\P=\bigcap\{\M\mid\M\;\mbox{ es modelo de Herbrand de }\;\P\}
\]
%\M_\P=\{G\in\B_\P\mid\P\models G\}
%\]
}
\proof
Sabemos que $\P$ tiene al menos un modelo de Herbrand representado por
la base de Herbrand $\B_\P$. Es f\'acil ver que si $\M_1,\M_2$ son modelos de 
Herbrand de $\P$ su intersecci\'on tambi\'en lo es. De manera que el modelo 
m\'inimo de Herbrand $\M_\P$ est\'a bien definido.
\qed

\prop{Sea $\P$ un programa l\'ogico. El modelo de Herbrand m\'inimo
  $\M_\P$ se representa con el conjunto de \'atomos que son consecuencia
  l\'ogica de $\P$. Es decir, %puede definirse como
\[
\M_\P=\{A\in\B_\P\mid\P\models A\}
\]
}
\proof\\
$\inc).$ Sea $A\in\M_\P$. Veamos que $\P\models A$. Si esto no sucediera 
entonces $\P\cup\{\neg A\}$ tendr\'ia un modelo y por el teorema de Herbrand 
tendr\'ia un modelo de Herbrand $\Nc$. Pero entonces como $\Nc\models\neg A$ 
mos que $A\notin\Nc$ pero por minimalidad  $\M_\P\inc\Nc$ y entonces 
tambi\'en $A\in\Nc$ lo cual es absurdo.\\
$\supseteq).$ Sea $A$ tal que $\P\models A$. Queremos ver que $A\in\M_\P$. Como 
$\M_\P$ es modelo de $\P$ y $\P\models A$ entonces $\M_\P\models A$, es decir, 
$A\in\M_\P$.
%Basta ver que el conjunto $\{A\in\B_\P\mid\A\models G\}$ representa a un 
% modelo de Herbrand de $\P$.
% \proof
% Sabemos que $\P$ tiene al menos un modelo de Herbrand. Es f\'acil ver
% que si $\M_1,\M_2$ son modelos de Herbrand de $\P$ su intersecci\'on
% tambi\'en lo es. De manera que el modelo m\'inimo de Herbrand existe y se
% define como
% \[
% \M_\P=\bigcap\{\M\mid\M\;\mbox{ es modelo de Herbrand de }\;\P\}
% \]
% \qed
% $\M_\P$ es el conjunto de todas las consecuencias l\'ogicas de $\P$. De hecho 
% $\M_\P$ es la intersecci\'on de todos los
% modelos de Herbrand de $\P$.\\

\medskip

Dado que $\M_\P$ contiene exactamente a todos los \'atomos que son 
consecuencia
l\'ogica de $\P$ entonces este modelo corresponde realmente al significado
intensional o estandar de $\P$. Es decir, la sem\'antica declarativa de $\P$ 
est\'a dada por $\M_\P$.


\subsection{Procedimiento para hallar el modelo m\'inimo de Herbrand}

Desde el punto de vista puramente matem\'atico la sem\'antica declarativa de un
programa l\'ogico ha quedado definida mediante el modelo m\'inimo. Sin embargo, 
en la pr\'actica no es claro c\'omo construir directamente a $\M_\P$, pues su
definici\'on involucra a una intersecci\'on generalmente infinita. A 
continuaci\'on veremos que existe una manera \textbf{mec\'anica} para construir 
el modelo m\'inimo.

\defin{Dado un programa l\'ogico $\P$, el operador de
  consecuencia inmediata $\T_\P$ se define como sigue
\[
\T_\P(\K)=\{P\in\B_\P\mid P\impp Q_1,\ldots,Q_m\;\mbox{es instancia cerrada de 
una   cl\'ausula de } \P \mbox{ y } Q_1,\ldots,Q_m\in\K\}
\]
donde $\K\inc\B_\P$. %es un modelo de Herbrand para $\P$.
}


\defin{Sea $\P$ un programa l\'ogico. Definimos las iteraciones del
  operador de consecuencia inmediata~$\T_\P$ para $\K\inc\B_\P$ recursivamente  
como sigue:
\[
\ba{rll}
\T_0(\K) & = & \K \\ \\
\T_{n+1}(\K) & = & \T_\P\big(\T_n(\K)\big) \\ \\
% \R_n(\P)\cup \T_\P\big(\R_n(\P)\big) \\ \\
\T_\omega(\K) & = & \bigcup_{n=0}^\infty\T_n(\K)
\ea
\]
}


El modelo m\'inimo de Herbrand de un programa l\'ogico puede obtenerse mediante
las iteraciones del operador de consecuencia inmediata iniciando en el
conjunto vac\'io, como lo asegura la siguiente

\prop{Sean $\P$ un programa l\'ogico y $\M_\P$ su m\'inimo modelo de Herbrand. 
Entonces $\M_\P=\T_\omega(\vacio)$.
}
% \teo{Sean $\P$ un programa l\'ogico y  $\vp=\ex x_1\ldots\ex 
% x_n(Q_1\land\ldots
% \land Q_m)$ una f\'ormula existencial cerrada. 
% Las siguientes condiciones son equivalentes:
% \bi
% \item $\P\models\vp$.
% \item $\P\models(Q_1\land\ldots\land Q_m)[x_1,\ldots,x_n:=t_1,\ldots,t_n]$ 
% para
%   algunos $t_1,\ldots,t_n\in\H_\P$.
% \item $\M_\P$ es un modelo de $\vp$.
% \item $\M_\P\models(Q_1\land\ldots\land Q_m)[x_1,\ldots,x_n:=t_1,\ldots,t_n]$
%   para algunos $t_1,\ldots,t_n\in\H_\P$.
% \ei
% }

% La importancia del teorema anterior se resalta al considerar el problema
% fundamental de la programacin l\'ogica:\\
% Dado un programa l\'ogico $P$ y una consulta $Q_1,\ldots,Q_m$ nos preguntamos 
% si
% $\P\models Q_1,\ldots,Q_m$ o equivalentemente si $\P\cup\{?-Q_1,\ldots,Q_m\}$
% no tiene modelo. Obs\'ervese que la cl\'ausula meta$?-Q_1,\ldots,Q_m$ es lo 
% mismo que
% $\pmi Q_1,\ldots,Q_m\equiv\neg(Q_1\land\ldots\land Q_m)$ cuya
% cerradura universal es $\fa x_1\ldots x_n(\neg(Q_1\land\ldots\land Q_m))$ que
% es una f\'ormula universal equivalente a $\neg\exists x_1\ldots
% x_n(Q_1\land\ldots\land Q_m)$. De manera que el problema original se deduce a
% decidir si $\P\models\ex x_1\ldots x_n(Q_1\land\ldots\land Q_m)$. Si
% realmente se da este caso por el teorema anterior tal consecuencia es
% atestiguada por una serie de instancias concretas para las variables
% $x_1,\ldots,x_n$ mediante t\'erminos cerrados $t_1,\ldots,t_n$. M\'as a\'un 
% tales
% t\'erminos cerrados pueden obtenerse analizando al m\'inimo modelo de Herbrand
% $\M_\P$. Por lo que $\M_\P$ es realmente el \emph{significado} del programa
% l\'ogico $\P$.


% El siguiente resultado aclara el concepto de respuesta correcta al decirnos
% que las respuestas correctas son el tipo de respuesta que esperariamos obtener
% al hacernos preguntas existenciales.

% \prop{Sea $\P$ un programa l\'ogico. Se cumple lo siguiente:
% \bi
% \item Sea $G= ?-G_1,\ldots,G_m$ una meta con variables
%   $y_1,\ldots,y_k,x_1,\ldots,x_n$. Si \[\P\models \fa y_1\ldots\fa y_k\ex
%   x_1\ldots\ex x_n(G_1\land\ldots\land G_m)\]
%  entonces existe una respuesta
%   correcta $\sigma$ para $\P\cup\{G\}$ tal que
%   $\sigma=[x_1:=t_1,\ldots,x_n:=t_n ]$ donde los t\'erminos $t_1,\ldots,t_n$
%   contienen s\'olo variables del conjunto $\{y_1,\ldots,y_k\}$.
% \item Inversamente sean $G= ?-G_1,\ldots,G_m$ una meta y
% $\sigma=[x_1:=t_1,\ldots,x_n:=t_n ]$ una respuesta correcta para
% $\P\cup\{G\}$ donde los t\'erminos $t_1,\ldots,t_n$
%   contienen s\'olo variables del conjunto $\{y_1,\ldots,y_k\}$. Si
% $y_1,\ldots,y_k,x_1,\ldots,x_n$ son variables distintas entonces 
% \[\P\models \fa y_1\ldots\fa y_k\ex
%   x_1\ldots\ex x_n(G_1\land\ldots\land G_m)\]
% \ei
% }

% La pregunta que surge inmediatamente es ?`c\'omo calcular respuestas
% correctas? lo cual responderemos mediante la sem\'antica operacional de
% los programas l\'ogicos.


Veamos un ejemplo.

\beje
Sea $\P$ el siguiente programa $\{p(X,a)\impp q(X).\; p(X,Y)\impp q(X),r(Y). 
\; q(a). \; r(b). \; q(b). \; r(c). \}$
\bi
\item El lenguaje de $\P$ es $\L(\P)=\{a,b,c,p^{(2)},q^{(1)},r^{(1)}\}$
\item El universo de Herbrand de $\P$ es $\H_\P=\{a,b,c\}$
\item La base de Herbrand de $\P$ es
\begin{small}
  \[\B_\P = \{  q(a),q(b),q(c),r(a),r(b),r(c),
               p(a,a),p(a,b),p(a,c), p(b,a),p(b,b),p(b,c), 
                p(c,a),p(c,b),p(c,c) \}\]
\end{small}
\item El modelo m\'inimo de Herbrand de $\P$ es:
\bi
\item $\T_0(\vacio)= \vacio$
\item $\T_1(\vacio)=
  \T_\P(\T_0(\vacio))=\T_\P(\vacio)=\{q(a),r(b),q(b),r(c)\}$
\item
  $\T_2(\vacio)=\T_\P(\T_1(\vacio))=\T_\P(\{q(a),r(b),q(b),r(c)\})$
\item[] 
$\;\;\;\;\;\;\;\;\;\;\;=\{q(a),r(b),q(b),r(c),p(a,a),p(b,a),p(a,b),p(a,c),p(b,b)
,p(b,c)\}$
\item $\T_3(\vacio)=\T_\P(\T_2(\vacio))=\T_2(\vacio)$
%\item $\T_4(\vacio)=\T_\P(\T_3(\vacio))=\T_\P(\vacio)$
\ei
\item En el paso $n=3$ la sucesi\'on se estabiliza y concluimos 
\[
\M_\P=\T_\omega(\vacio)=\{q(a),r(b),q(b),r(c),p(a,a),p(b,a),p(a,b),p(a,c),p(b,b)
,p(b,c)\}
\]
\ei
\eeje

\subsection{Sem\'atica operacional de programas l\'ogicos}


El significado operacional de un programa l\'ogico consiste en los
resultados obtenidos al ejecutar el programa. 

\defin{El conjunto de \'exito de un programa l\'ogico $\P$ se define como
\[
\mathcal{E}_\P=\{A\in\B_\P\mid\P\cup\{\normalfont{?-}A\}\;\mbox{tiene
    una SLD-refutaci\'on}\;\}\] 
}

\noindent
La sem\'antica operacional del programa $\P$ se define como el conjunto
de \'exitos $\mathcal{E}_\P$.  
Veamos un ejemplo

\beje
Sea $\P$ el siguiente programa $\{p(f(X))\impp p(X).,\; p(a).,\; q(b). \}$.
Entonces 
\bi
\item El lenguaje de $\P$ es $\L(\P)=\{a,b,p^{(1)},q^{(1)},f^{(1)}\}$
\item El universo de Herbrand de $\P$ es 
$\H_\P=\{a,b,f(a),f(b),f(f(a)),f(f(b)),\ldots\}$
\item La base de Herbrand de $\P$ es
  \[\B_\P  = \{  p(a),p(b),q(a),q(b),p(f(a)),p(f(b)),q(f(a)),q(f(b)),\ldots\}
\]

\item Claramente las metas $\normalfont{?-}p(a).$ y $\normalfont{?-}q(b).$ 
tienen una SLD-refutaci\'on por lo tanto pertenecen a $\mathcal{E}_\P$.
\item La meta $\normalfont{?-}p(f(a)).$ tiene \'exito pues tenemos una  
SLD-refutaci\'on a partir de las instancias $p(f(a))\impp p(a).$ y $p(a).$ de 
las cl\'ausulas del programa.
\item Similarmente podemos verificar que $\normalfont{?-}p(f^n(a)).$ tiene 
\'exito para cualquier $n\geq 0$. M\'as a\'un si una meta es \'exitosa entonces 
es de esta forma (salvo $?-q(b)$).
\item Por lo tanto el conjunto de \'exitos es:
\[
\mathcal{E}_\P=\{p(a),q(b)\}\cup\{p(f^n(a))\mid n\geq 1\}
\]
\ei
\eeje


\section{Equivalencia de ambas sem\'anticas}

Finalmente observamos la equivalencia de ambas sem\'anticas. Esta equivalencia 
es corolario de ciertos resultados te\'oricos que dependen del teorema de 
Herbrand. 

\begin{proposition}
Sea $\P$  un programa l\'ogico definido. La sem\'anticas declarativa y 
operacional para $\P$ coinciden. Es decir,
\[
\M_\P = \mathcal{E}_\P
\]
\end{proposition}

De manera que toda consecuencia l\'ogica at\'omica de $\P$ es un \'exito de 
$\P$ y viceversa, todo \'exito $G\in\mathcal{E}_\P$ es consecuencia l\'ogica de 
$\P$, es decir, cumple $\P\models G$.

% \teo{[Correctud de la SLD-resoluci\'on]
% Sean $\P$ un programa l\'ogico y $G=?-G_1,\ldots,G_m$ una meta. Cualquier
% respuesta computada $\sigma$ para $\P\cup\{G\}$ es una respuesta correcta
% para $\P\cup\{G\}$. M\'as a\'un si  $\sigma$ es una respuesta computada para
% $\P\cup\{G\}$ y $\tau$ es una sustituci\'on arbitraria entonces
% $\restr{\sigma\tau}{_G}$ es una respuesta correcta para $\P\cup\{G\}$.

% }

% \cor{Sean $\P$ un programa l\'ogico y $G=?-G_1,\ldots,G_m$ una meta. Si
%   existe una SLD-refutaci\'on de $\P\cup\{G\}$ entonces $\P\cup\{G\}$ no tiene
%   modelo y por lo tanto $\P\models\ex(G_1\land\ldots\land G_m)$.
% }

% \cor{
% $\mathcal{E}_\P\inc \M_\P$. Es decir cualquier \'exito pertenece al
% modelo m\'inimo de Herbrand de $\P$.
% }


% \teo{[Completud de la SLD-resoluci\'on] Sean $\P$ un programa l\'ogico y 
% $G=?-G_1,\ldots,G_m$ una meta. Si $\P\models\ex(G_1\land\ldots\land G_m)$
% entonces existe una SLD-refutaci\'on de $\P\cup\{G\}$.
% }


% \cor{$\M_\P\inc \mathcal{E}_\P$. Es decir, cualquier \'atomo
%   perteneciente al modelo m\'inimo de Herbrand de $\P$ pertenece a los
%   \'exitos de $\P$.}

% \teo{[Completud de la SLD-resoluci\'on con respecto a respuestas] 
% Sean $\P$ un programa l\'ogico, $G=?-G_1,\ldots,G_m$ una meta y $\sigma$ una
% respuesta correcta para $\P\cup\{G\}$. Entonces existe una respuesta
% computada $\mu$ para $\P\cup\{G\}$ y una sustituci\'on $\tau$ tales que
% $\sigma=\restr{\mu\tau}{_G}$. 
% }


\section{Incompletud e Incorrectud de {\pl}}

Debido a que las implementaciones de {\pl} no verifican las presencia de 
una variable en un t\'ermino en la unificiaci\'on de $\{X,t\}$ las propiedades 
de correctud completud no son v\'alidas como lo muestran los siguientes 
ejemplos:

\bi

\item Incorrectud: {\pl} contesta que \verb-true- a una meta que no es 
consecuencia l\'ogica del programa:
\begin{verbatim}
test :- p(X,X).

p(Y,f(Y)).

--------------------
?- test.
true.

?- p(X,X).
X = f(X).
\end{verbatim}

\item Incompletud: en el siguiente ejemplo \verb-q- es consecuencia l\'ogica 
del programa pero {\pl} se cicla y queda sin recursos sin poder contestar 
afirmativamente.
\begin{verbatim}
p :- q.
p :- r.
q :- p.
r.

--------------------------
?- q.
ERROR: Out of local stack
   Exception: (4,193,042) p ? abort
% Execution Aborted
?- 
\end{verbatim}

\ei

\end{document}
