%%% Presentaciones para Lenguajes de programacion y sus paradigmas 

\documentclass[xcolor=dvipsnames,table,handout]{beamer}
%\documentclass[xcolor=dvipsnames,table]{beamer}
\newcommand{\espc}{\vspace{0.3cm}}
\usepackage[utf8]{inputenc}
\usepackage[spanish]{babel}
\usepackage{hyperref}
\usepackage{lmodern}
\usepackage[T1]{fontenc}

%%%% paquetes matematicas
\usepackage{amssymb,amsmath,amscd}
\usepackage{extarrows}
\usepackage{stmaryrd}
\usepackage{mathabx}
\usepackage{mathrsfs}
% \usepackage{mathabx}
\usepackage{amsthm}

%%%%%
\usepackage{hyperref}
\usepackage{graphicx}
\usepackage{multicol}
\usepackage{pifont}
\usepackage{xcolor}
\usepackage{etex}
\usepackage{tikz}
\usepackage{array}
%\usepackage{pgfplots}

%%%% cosmetics
% D.Remy package for pretty display of rules
\usepackage{mathpartir}

% para insertar codigo con formato particular 
\usepackage{listing} 

% comillas 
\usepackage[autostyle=true,spanish=mexican]{csquotes}

% codigo 
\usepackage{verbatim}
\usepackage{alltt}

% footnotes
\usepackage[bottom]{footmisc}
\usepackage{setspace}

\usepackage{wrapfig}
\usepackage{caption}


\hfuzz=5.002pt %parameter to allow hbox overfulled by length before error!

% Options for presentation
% ------------------------
% \definecolor{mycolor}{RGB}{255,192,3}
\definecolor{mycolor}{RGB}{17,132,221}
\mode<presentation>
{
% \usetheme[secheader]{Boadilla}
% \usecolortheme{orchid}
\useoutertheme{infolines}
\useinnertheme{rectangles}
\setbeamertemplate{itemize items}[square]
\setbeamertemplate{enumerate items}[square]
\setbeamersize{text margin left=6mm, text margin right=6mm}

\setbeamercolor{alerted text}{fg=red,bg=red!70!white}
\setbeamercolor{background canvas}{bg=white}
\setbeamercolor{frametitle}{bg=mycolor,fg=white}
\setbeamercolor{normal text}{bg=white,fg=black}
\setbeamercolor{structure}{bg=black,fg=mycolor}
\setbeamercolor{title}{bg=mycolor,fg=white}
\setbeamercolor{subtitle}{bg=mycolor,fg=white}
\setbeamercolor{titlelike}{bg=white,fg=mycolor}

\setbeamercovered{invisible}

\setbeamercolor*{palette primary}{fg=mycolor,bg=white}
\setbeamercolor*{palette secondary}{bg=white,fg=white}
\setbeamercolor*{palette tertiary}{fg=mycolor,bg=white}
\setbeamercolor*{palette quaternary}{fg=white,bg=white}

\setbeamercolor{separation line}{bg=mycolor,fg=mycolor}
\setbeamercolor{fine separation line}{bg=white,fg=red}
\setbeamercolor{author in head/foot}{bg=mycolor!30!white,fg=mycolor!80!black}
\setbeamercolor{title in head/foot}{bg=mycolor!30!white,fg=mycolor!80!black}
\setbeamercolor{date in head/foot}{bg=mycolor!30!white,fg=mycolor!80!black}
\setbeamercolor{institute in head/foot}{bg=mycolor!30!white,fg=mycolor!80!black}
\setbeamercolor{section in head/foot}{bg=mycolor!60!white, fg=Red}
\setbeamercolor{subsection in head/foot}
{bg=mycolor!50!white,fg=mycolor!50!white}


\setbeamertemplate{headline}
{
  \leavevmode%
  \hbox{%
  \begin{beamercolorbox}[wd=.5\paperwidth,ht=2.65ex,dp=1.5ex,center]{section in 
head/foot}%
    \usebeamerfont{section in head/foot}\insertsectionhead\hspace*{2ex}
  \end{beamercolorbox}%
  \begin{beamercolorbox}[wd=.5\paperwidth,ht=2.65ex,dp=1.5ex,center]{subsection 
in head/foot}%
    \usebeamerfont{subsection in head/foot}\hspace*{2ex}\insertsubsectionhead
  \end{beamercolorbox}}%
  \vskip0pt%
}
% \beamerdefaultoverlayspecification{<+->}
\beamertemplatenavigationsymbolsempty
% \setbeamertemplate{footline}[frame number]
}

%%%% Macros para las notas de lenguajes de programacion


%%%% math

\newcommand{\vphi}{\varphi}
\newcommand{\vp}{\varphi}

% \newcommand{\dn}{\mathsf{DN}}
% \newcommand{\dnC}{\mathsf{DN_C}}
% \newcommand{\dnM}{\mathsf{DN_M}}
% \newcommand{\dnp}{\mathsf{DN_p}}
% \newcommand{\dnm}{\mathsf{DN_p^M}}
% \newcommand{\dnc}{\mathsf{DN_p^C}}

%\newcommand{\case}{\mathsf{case}}
%\renewcommand\labelitemi{$\circ$}

\newcommand{\imp}{\rightarrow}
\newcommand{\Imp}{\Rightarrow}
\renewcommand{\iff}{\leftrightarrow}
\newcommand{\Iff}{\Leftrightarrow}
\newcommand{\G}{\Gamma}
\newcommand{\D}{\Delta}


\newcommand{\De}{\mathcal{D}}
\newcommand{\F}{\mathcal{F}}
\newcommand{\Ge}{\mathcal{G}}
\newcommand{\Pe}{\mathcal{P}}
\newcommand{\I}{\mathcal{I}}
\newcommand{\C}{\mathcal{C}}
\newcommand{\K}{\mathcal{K}}
\renewcommand{\L}{\mathcal{L}}
\newcommand{\M}{\mathcal{M}}
\newcommand{\Nc}{\mathcal{N}}
%\newcommand{\E}{\mathcal{E}}
%\newcommand{\R}{\mathcal{R}}
%\newcommand{\Q}{\mathcal{Q}}
\newcommand{\Sc}{\mathcal{S}}
\newcommand{\Te}{\mathcal{T}}
\newcommand{\W}{\mathcal{W}}

\newcommand{\Db}{\mathbb{D}}
\newcommand{\Fb}{\mathbb{F}}
\newcommand{\Kb}{\mathbb{K}}
\newcommand{\Eb}{\mathbb{E}}
\newcommand{\Ebs}{\mathbb{E}^\star}
\newcommand{\Ob}{\mathbb{O}}
\newcommand{\Ib}{\mathbb{I}}
\newcommand{\Rb}{\mathbb{R}}
\newcommand{\Qb}{\mathbb{Q}}
\newcommand{\Kbb}{\mathbb{K}}
\newcommand{\T}{\mathbb{\Theta}}


\newcommand{\kb}{\bbkappa}

\newcommand{\Sf}{\mathsf{\Sigma}}

\newcommand{\fa}{\forall}
\newcommand{\ex}{\exists}

\newcommand{\inc}{\subseteq}

\newcommand{\Lb}{\Lambda}
\newcommand{\Om}{\Omega}
\newcommand{\lb}{\lambda}
\newcommand{\al}{\alpha}
\newcommand{\ga}{\gamma}


\newcommand{\mg}{\mathbb{m}}

\newcommand{\cg}{\mathbb{C}}
\newcommand{\dg}{\mathbb{D}}
\newcommand{\jg}{\mathbb{J}}
\newcommand{\Ha}{\mathcal{H}}
%\newcommand{\A}{\mathcal{A}}
\newcommand{\sg}{\mathbb{S}}

\newcommand{\Bc}{\mathcal{B}}
\newcommand{\Df}{\mathfrak{D}}
\newcommand{\Dc}{\mathcal{D}}
%\newcommand{\Tc}{\mathcal{T}}
\newcommand{\Mf}{\mathfrak{M}}

\newcommand{\Sg}{\mathbb{S}}

\newcommand{\Q}{\ensuremath{\mathbb{Q}}}
\newcommand{\Z}{\ensuremath{\mathbb{Z}}}
\newcommand{\N}{\ensuremath{\mathbb{N}}}
\newcommand{\R}{\ensuremath{\mathbb{R}}}
\renewcommand{\S}{\mathbb{\Sigma}}
\newcommand{\A}{\mathcal{A}}
\newcommand{\E}{\ensuremath{\exists}}
\newcommand{\iso}{\ensuremath{\cong}}
\newcommand{\union}{\ensuremath{\cup}}
\newcommand{\morinyec}{\ensuremath{\precapprox}}

\newcommand{\nin}{\ensuremath{\notin}}
\newcommand{\tog}{\makebox[7mm][l]}
\newcommand{\toge}{\makebox[11mm][l]}
\newcommand{\toget}{\makebox[13mm][l]}
\newcommand{\togeth}{\makebox[14mm][l]}
\newcommand{\togethe}{\makebox[15mm][l]}
\newcommand{\together}{\makebox[17mm][l]}
\newcommand{\niso}{\ensuremath{\not \cong}}


\newcommand{\Mg}{\mathbb{M}}
\newcommand{\Bg}{\mathbb{B}}
\newcommand{\Lg}{\mathbb{L}}
\newcommand{\Tg}{\mathbb{T}}

\newcommand{\sketch}{\Red{{\sc sketch}}}

\newcommand{\restr}[2]{#1\!\!\boldsymbol{\restriction}\!#2}

\newcommand{\vacio}{\varnothing}
\newcommand{\done}{\ensuremath{\checkmark}}

\newcommand{\ida}{$\Rightarrow \; )$ }
\newcommand{\regr}{$\Leftarrow \; )$ }

\newcommand{\ol}[1]{\overline{#1}}

\newcommand{\Tsf}{\mathsf{T}}

\newcommand{\inds}[1]{\index[simb]{#1}}

\newcommand{\B}{\mathbb{B}}
%\newcommand{\N}{\mathbb{N}}

\newcommand{\vx}{\vec{x}}
\newcommand{\vy}{\vec{y}}
\newcommand{\vz}{\vec{z}}
\newcommand{\vt}{\vec{t}}
\newcommand{\vf}{\vec{f}}


% \newcommand{\propo}{\ensuremath{\mathsf{PROP}}}
% \newcommand{\atom}{\ensuremath{\mathsf{ATOM}}}
\newcommand{\term}{\ensuremath{\mathsf{TERM}}}
\newcommand{\form}{\mathsf{FORM}}

\newcommand{\true}{\mathop{\mathsf{true}}}

%\newcommand{\id}{\mathsf{Id}}

%\newcommand{\uc}{\mathcal{U}}
%\newcommand{\Ic}{\mathcal{I}}
%\newcommand{\pc}{\mathcal{P}}
%\newcommand{\qc}{\mathcal{Q}}
%\newcommand{\mc}{\mathcal{M}}
\newcommand{\supc}{\supseteq}
\newcommand{\limo}{\mathop{\mathpzc{Lim}}}
\newcommand{\ord}{\mathsf{OR}}

\newcommand{\pt}[1]{\langle #1 \rangle}


%%%% frames
\newcommand{\titulos}[2]{\frametitle{#1}\framesubtitle{#2}}
\newcommand{\fot}[1]{\footnote{\scriptsize{#1}}}


%%%% ambientes

\newcommand{\cb}[2]{\colorbox{#1}{#2}}

\newcommand{\bc}{\begin{center}}
\newcommand{\ec}{\end{center}}
\newcommand{\be}{\begin{enumerate}}
\newcommand{\ee}{\end{enumerate}}
\newcommand{\bi}{\begin{itemize}}
\newcommand{\ei}{\end{itemize}}
\newcommand{\beq}{\begin{equation}}
\newcommand{\eeq}{\end{equation}}
\newcommand{\beqs}{\begin{equation*}}
\newcommand{\eeqs}{\end{equation*}}
\newcommand{\ba}{\begin{array}}
\newcommand{\ea}{\end{array}}


% \newtheorem{theorem}{Teorema}
% \newcommand{\teo}[1]{\begin{theorem} #1 \end{theorem}}
% \newtheorem{proposition}{Proposici\'on}
% \newcommand{\prop}[1]{\begin{proposition} #1 \end{proposition}}
% \newtheorem{definition}{Definici\'on}
% \newcommand{\defin}[1]{\begin{definition} #1 \end{definition}}
% \newtheorem{corollary}{Corolario}
% \newcommand{\cor}[1]{\begin{corollary} #1 \end{corollary}}
% \newtheorem{lemma}{Lema}
% \newcommand{\lema}[1]{\begin{lemma} #1 \end{lemma}}
% \newcommand{\dem}[1]{\begin{proof} #1 \end{proof}}

%\renewcommand{\qed}{\qedsymbol{$\mathbf{\dashv}$}}

%\newcommand{\proof}{\hfill\\\noindent\textbf{\textit{Demostraci\'on. }}}

\newcommand{\hint}{\emph{Sugerencia: }}


\newcounter{EjempCtr}[section]
\newenvironment{enumrom}{\renewcommand{\theenumi}{\roman{enumi}}%
\renewcommand{\theenumii}{\roman{enumii}}
\renewcommand{\theenumiii}{\roman{enumiii}}
\renewcommand{\theenumiv}{\roman{enumiv}}
\begin{enumerate}}{\end{enumerate}}
\newenvironment{Ejemplo}
        {\stepcounter{EjempCtr}%
        \begin{description}\item[Ejemplo \thesection.\arabic{EjempCtr}]}%
        {\end{description}}
\newenvironment{demostr}{%
             {\em Demostración:}
                \begin{quotation}}{\end{quotation}}

   \newcommand{\beje}{\begin{Ejemplo}}
\newcommand{\eeje}{\end{Ejemplo}}


\newtheorem{eje}{Ejemplo}[section]
\newcommand{\ejem}[1]{\begin{eje}\normalfont #1 \end{eje}}

% \renewcommand\contentsname{\'Indice}
%\renewcommand\chaptername{Cap\'itulo}
% \renewcommand\indexname{\'Indice}

%%\newcommand{\qed}{\hfill$\mathbb{Qed}$}
%\newcommand{\qed}{\hfill$\mathsf{\boldsymbol{\dashv}}$}
%\renewcommand{\qed}{\hfill$\boldsymbol{\dashv}$}


\newenvironment{prueba}{\vspace{-5mm}\noindent\textbf{Demostraci\'on}\\}{
\noindent$\blacksquare$\\}

\newcommand{\Ejercicios}{\section*{Ejercicios}}

 \newenvironment{manitas}{%
      \renewcommand{\labelitemi}{\ding{44}}%
      \vspace{-0.5cm}%
      \begin{itemize}%
      \setlength{\itemsep}{0pt}\setlength{\parsep}{0pt}\setlength{\topsep}{0pt}%
      }{\end{itemize}}
\newenvironment{malitos}{%
      \renewcommand{\labelitemi}%
            {\raisebox{1.5ex}{\makebox[0.3cm][l]{\begin{rotate}{-90}%
            \ding{43}\end{rotate}}}}%
      \vspace{-0.5cm}%
      \begin{itemize}%
      \setlength{\itemsep}{0pt}\setlength{\parsep}{0pt}\setlength{\topsep}{0pt}%
      }{\end{itemize}}
\newenvironment{ejercs}{
     \renewcommand{\labelenumi}{\thesection.\theenumi.-}
     \renewcommand{\labelenumii}{\theenumii)}
     \begin{enumerate}}
     {\end{enumerate}}

   \newcommand{\bej}{\begin{ejercs}}
\newcommand{\eej}{\end{ejercs}}


%\newenvironment{leterize}{%
%        \renewcommand{\theenumi}{\alph{enumi}}
%        \begin{enumerate}}{\end{enumerate}}

%\newenvironment{manitas}{%
%      \renewcommand{\labelitemi}{\ding{44}}%
%      \vspace{-0.5cm}%
%      \begin{itemize}%
%      
% \setlength{\itemsep}{0pt}\setlength{\parsep}{0pt}\setlength{\topsep}{0pt}%
%      }{\end{itemize}}



%%=============================================================================

\def\stackunder#1#2{\mathrel{\mathop{#2}\limits_{#1}}}


%%%% notas

\newcommand{\doubt}{\Red{{\LARGE {\sf ??}}}}

\newcommand{\coment}[1]{\hfill\\ \Big[{\bf Comentario Privado:} #1\Big]}
\newcommand{\preg}[1]{\hfill\\ \BrickRed{{\bf Pregunta:} #1}}
\newcommand{\conjet}[1]{\hfill\\ \OliveGreen{{\bf Conjecura:} #1}}

\newcommand{\pendiente}{\BrickRed{{\sc Pendiente}}}
\newcommand{\verifpendiente}{\BrickRed{{\sc Verificación pendiente}}}


%--------------------------------------------------------------------------

\DeclareMathAlphabet{\mathpzc}{OT1}{pzc}{m}{it}


\title[]{Lógica computacional}
\subtitle{Tema: Semántica de la Lógica de Primer Orden II}
\author{ Pilar Selene Linares Ar\'evalo}
\institute[UNAM-FC]{Facultad de Ciencias\\ 
Universidad Nacional Aut\'onoma de M\'exico}
\date[]{ \footnotesize{marzo 2018}
\newline{\tiny{Material desarrollado bajo el proyecto UNAM-PAPIME PE102117.}}}
 

\beamerdefaultoverlayspecification{<+->}
 
\titlegraphic{\includegraphics[width=16mm]{fc2.png}}
 

\begin{document}

\begin{frame}
\titlepage 
\end{frame}
\note{}


\frame{\titulos{Modelos}{}
Uno de los problemas fundamentales en l\'ogica es la búsqueda de modelos de
un conjunto dado de fórmulas~$\G$. \\
\pause
\espc
Sabemos que un modelo para una fórmula~$\vp$ es una interpretación~$\M=\pt{M,\I}$ tal que $\M\models\vp$, es decir tal que
$\I_\sigma(\vp)$ para cualquier estado~$\sigma$. \\
\pause
\espc

También sabemos que si bien una fórmula no necesariamente tiene un modelo, es decir, no
necesariamente es verdadera o falsa, esta propiedad sí se cumple para los enunciados, puesto que no tienen variables libres que
cambien de valor dependiendo del estado.
}

\frame{\titulos{Modelos}{}
Buscamos un modelo $\M$ para \\
$\G=\{Pb,Qb,Rb,\ex x(Px\land \neg(Qx\lor Rx),\;\fa x
(Rx \imp Px)\}$.\\
\espc
\pause

\only<2>{
\setbeamercolor{postit}{bg=yellow!50!white}
\begin{beamercolorbox}{postit}
Lo m\'as f\'acil es ir construyendo un modelo para cada f\'ormula de $\G$, si logramos que todas las f\'ormulas de $\G$ sean
verdaderas al mismo tiempo entonces habremos construído un modelo para $\G$.
\end{beamercolorbox} }    

\pause

Analicemos cada f\'ormula de $\G$: \\
\pause
\be
\item $Pb$:  para que $\M\models Pb$ se debe cumplir que $b^\I\in
P^\I$. As\'{\i} que al menos debemos tener $P^\I=\{b^\I\}$
\item $Qb$: para que $\M\models Q(b)$ se debe cumplir que $b^\I\in
Q^\I$, as\'{\i} que basta $Q^\I=\{b^\I\}$ 
\item $Rb$: para que $\M\models Q(b)$ se debe cumplir que $b^\I\in
R^\I$ por lo que basta con $R=\{b^\I\}$.
%\item $\ex x(P(x)\land \neg(Q(x)\lor R(x))$. Es decir, hay un elemento de
%$|\M|$ tal que cumple $P$ y no cumple $Q$ ni $R$. Claramente este elemento
%no puede ser $b^\I$ por lo que $P^\I$ debe tener al menos otro elemento,
%digamos $P^\I=\{b^\I, m\}$.
%\item $\fa x (R(x)\allowbreak\imp P(x))$. Debemos tener que todo elemento
%de $\G$ que cumple $R^\I$, debe cumplir $P^\I$, pero de acuerdo a como definimos
%$R^\I$ esto ya se cumple pues $b^\I\in R^\I$ y $b^\I\in P^\I$. 
\ee
%De manera que $P^\I=\{b^\I, m\}$ y $Q^\I=\{b^\I\}=R^\I$ y para
%que el modelo resulte m\'as natural podemos tomar $M=\{0,1\},\;b^\I=0$ y
%$m=1$, con lo que queda $\M=\pt{\{0,1\},\I}$ con
%$P^\I=\{0,1\}, Q^\I=\{0\}=R^\I, b^\I=0$. 
}


\frame{\titulos{Modelos}{}
Buscamos un modelo $\M$ para \\
$\G=\{Pb,Qb,Rb,\ex x(Px\land \neg(Qx\lor Rx),\;\fa x
(Rx \imp Px)\}$.\\
\espc

Analicemos cada f\'ormula de $\G$: \\
\pause
\be
\item $\ex x(Px\land \neg(Qx\lor Rx)$:  hay un elemento de
$|\M|$ que cumple $P$ y no cumple $Q$ ni $R$. Este elemento
no puede ser $b^\I$ por lo que $P^\I$ debe tener al menos otro elemento,
digamos $P^\I=\{b^\I, m\}$.
\item $\fa x (Rx \imp Px)$: todo elemento de $\G$ que cumple $R^\I$, también cumple $P^\I$, pero de acuerdo a como definimos $R^\I$ esto ya se satisface pues $b^\I\in R^\I$ y $b^\I\in P^\I$. 
\ee
%De manera que $P^\I=\{b^\I, m\}$ y $Q^\I=\{b^\I\}=R^\I$ y para
%que el modelo resulte m\'as natural podemos tomar $M=\{0,1\},\;b^\I=0$ y
%$m=1$, con lo que queda $\M=\pt{\{0,1\},\I}$ con
%$P^\I=\{0,1\}, Q^\I=\{0\}=R^\I, b^\I=0$. 
}

\frame{\titulos{Modelos}{}
Buscamos un modelo $\M$ para \\
$\G=\{Pb,Qb,Rb,\ex x(Px\land \neg(Qx\lor Rx),\;\fa x
(Rx \imp Px)\}$.\\
\espc

De lo anterior tenemos  que $P^\I=\{b^\I, m\}$ y $Q^\I=\{b^\I\}=R^\I$. \\
\espc \pause
Para que el modelo resulte m\'as natural podemos tomar $M=\{0,1\},\;b^\I=0$ y
$m=1$, con lo que queda $\M=\pt{\{0,1\},\I}$ con
$P^\I=\{0,1\}, Q^\I=\{0\}=R^\I, b^\I=0$. \\
\espc
\espc
}

\frame{\titulos{Validez universal}{}
El concepto de tautología en lógica proposicional tiene su contraparte en lógica de predicados mediante el concepto de validez universal. 
\espc
\pause
\begin{block}{Validez universal}
Decimos que una fórmula~$\vp$ es {\bf universalmente 
  válida} (o simplemente válida) si para toda interpretación~$\M$ se cumple que 
  $\M\models\vp$, es decir, si $\vp$ es verdadera en cualquier interpretación 
  posible, lo cual se denota con $\models\vp$.
\end{block}
}

\frame{\titulos{Validez universal}{}
La noción de validez universal es análoga a la noción de tautología en lógica proposicional. \\
\espc
\pause
De hecho toda fórmula cuyo esqueleto proposicional es una tautología, resulta ser una fórmula universalmente válida. \\
\espc
\pause
Por ejemplo:
$\models\fa x(Px\imp Px\lor Qxy)$  puesto que en lógica proposicional se tiene que $\models P\imp P\lor Q$.
}

\frame{\titulos{Validez Universal}{}
{\bf Ejercicio:} Mostrar que $\models\neg\ex x \vp\iff\fa x\neg \vp$. \\
\espc
\pause
Sea $\M$ una interpretaci\'on y $\sigma$ un estado de las variables, hay 2 casos:
\bi
\item $\boldsymbol{\I_\sigma(\neg\ex x\vp)=1}$. \pause Es equivalente a $\I_\sigma(\ex x\vp)=0$
cuya definición es que no existe $m\in |\M|$ tal que $\I_{\sigma[x/m]}(\vp)=1$ \\
\pause
\espc
Es decir,  para todo $m\in |\M|,\;\I_{\sigma[x/m]}(\vp)=0$ o  equivalentemente
para todo $m\in |\M|,\;\I_{\sigma[x/m]}(\neg\vp)=1$ \\
\pause
\espc
Lo anterior significa que $\I_\sigma(\fa x\neg\vp)=1$.
\ei
%Por lo tanto en cualquier caso se tiene que %$\M\models_\sigma\neg\ex x \vp$ syss
%                                %$\M\models_\sigma\fax\neg \vp$. 
%$\I_\sigma(\neg\ex x\vp)=\I_\sigma(\fa x\neg\vp)$ es decir, $\I_\sigma(\neg\ex
%x\vp\iff\fa x\neg\vp)=1$. Por lo tanto $\M\models\neg\ex x\vp\iff\fa x\neg
%\vp$. Pero como $\M$ era una interpretación arbitraria, se tiene que $\models\neg\ex x\vp\iff\fa x\neg \vp$.
}

\frame{\titulos{Validez Universal}{}
{\bf Ejercicio:} Mostrar que $\models\neg\ex x \vp\iff\fa x\neg \vp$. \\
\espc

Sea $\M$ una interpretaci\'on y $\sigma$ un estado de las variables, hay 2 casos:
\bi
\item $\boldsymbol{\I_\sigma(\neg\ex x\vp)=0}$. \pause Es decir, $\I_\sigma(\ex x\vp)=1$ syss
existe $m\in |\M|$ tal que $\I_{\sigma[x/m]}(\vp)=1$. \\
\pause
\espc
Lo anterior sucede syss existe $m\in |\M|$ tal que $\I_{\sigma[x/m]}(\neg\vp)=0$
syss no para todo $m\in |\M|,\;\I_{\sigma[x/m]}(\neg\vp)=1$ \\
\espc
\pause
Por lo tanto $\I_\sigma(\fa x\neg\vp)=0$.
\ei
%Por lo tanto en cualquier caso se tiene que %$\M\models_\sigma\neg\ex x \vp$ syss
%                                %$\M\models_\sigma\fax\neg \vp$. 
%$\I_\sigma(\neg\ex x\vp)=\I_\sigma(\fa x\neg\vp)$ es decir, $\I_\sigma(\neg\ex
%x\vp\iff\fa x\neg\vp)=1$. Por lo tanto $\M\models\neg\ex x\vp\iff\fa x\neg
%\vp$. Pero como $\M$ era una interpretación arbitraria, se tiene que $\models\neg\ex x\vp\iff\fa x\neg \vp$.
}


\frame{\titulos{Validez Universal}{}
{\bf Ejercicio:} Mostrar que $\models\neg\ex x \vp\iff\fa x\neg \vp$. \\
\espc

De lo anterior concluimos que en cualquier caso se tiene que %$\M\models_\sigma\neg\ex x \vp$ syss
%                                %$\M\models_\sigma\fax\neg \vp$. 
$\I_\sigma(\neg\ex x\vp)=\I_\sigma(\fa x\neg\vp)$, es decir, $\I_\sigma(\neg\ex
x\vp\iff\fa x\neg\vp)=1$.  \\
\espc
\pause 
Por lo tanto $\M\models\neg\ex x\vp\iff\fa x\neg
\vp$. \\
\espc
\pause 
Como además $\M$ es una interpretación arbitraria, se tiene que $\models\neg\ex x\vp\iff\fa x\neg \vp$.
}


\frame{\titulos{Equivalencia lógica}{}

\begin{block}{Equivalencia lógica}
Sean $\vp,\psi$ f\'ormulas. Decimos que $\vp$ es {\bf l\'ogicamente
equivalente} a $\psi$, denotado con $\vp\equiv\psi$, si y sólo si
$\models\vp\iff\psi$, es decir si y sólo si la fórmula $\vp\iff\psi$ es universalmente válida. 
\end{block}

}

\frame{\titulos{Equivalencia lógica}{}

Las siguientes equivalencias l\'ogicas ser\'an
de utilidad m\'as adelante: 
\espc
\bi
\item {\bf Negación de cuantificaciones } (Leyes de De Morgan generalizadas):
  \be
\item $\neg\fa x \vp\equiv\ex x\neg \vp$
\item $\neg\ex x \vp\equiv\fa x\neg \vp$ \\
  \ee
\espc
\item {\bf Eliminaci\'on de cuantificaciones m\'ultiples.}\index{eliminaci\'on!del
cuantificador m\'ultiple} 
\be
\item$\fa x\fa x\vp\equiv\fa x\vp$.
\item $\ex x\ex x\vp\equiv\ex x\vp$.
\ee

\ei
}


\frame{\titulos{Equivalencia lógica}{}

Las siguientes equivalencias l\'ogicas ser\'an
de utilidad m\'as adelante: 
\espc
\bi
\item {\bf Renombre de variables}. \\ Si $y$ no figura libre en $\vp$
entonces:\index{renombre de variables}
\be
\item $\fa x\vp\equiv\fa y(\vp[x:=y])$.
\item $\ex x\vp\equiv\ex y(\vp[x:=y])$.
\ee
\espc
\item {\bf Eliminaci\'on de cuantificaciones vacuas.} \\
 Si $x$ no figura libre en
$\vp$ entonces:\index{eliminaci\'on!del cuantificador vacuo}
\be
\item $\fa x\vp\equiv\vp$.
\item $\ex x\vp\equiv\vp$.
\ee
\ei
}


\frame{\titulos{Consecuencia lógica}{}

\begin{block}{consecuencia lógica}
Sean $\G$ un conjunto de f\'ormulas y $\vp$ una f\'ormula. Decimos que
$\vp$ es {\bf consecuencia l\'ogica} de $\G$, denotado con
$\G\models\vp$, si y sólo si todo modelo de $\G$ es un modelo de $\vp$. \\
\pause
\espc
Es decir, $\G\models\vp$ si y sólo si para toda interpretaci\'on $\M$, si $\M\models \G$ 
entonces $\M\models\vp$. \\
\pause
\espc
Si $\G\models\vp$ también decimos que $\G$ \emph{implica l\'ogicamente} a $\vp$.
\end{block}
}

\frame{\titulos{Consecuencia lógica}{}
Las siguientes observaciones son importantes:
\bi
\item Al igual que en lógica proposicional el símbolo $\models$ está
  sobrecargado y se usa para las relaciones \emph{``ser modelo de''} y
  \emph{``ser consecuencia lógica de''}.
\item Al no cambiar la definición de consecuencia lógica las
  propiedades de ésta siguen siendo válidas.
En particular el argumento con premisas $\G=\{\vp_1,\ldots,\vp_n\}$ y
conclusión $\vp$, es correcto si y sólo si $\G\models\vp$.
\ei
}

\frame{\titulos{Indecibilidad LPO}{}
Decidir validez universal, equivalencia lógica y consecuencia lógica son problemas 
son \textbf{indecidibles}, es decir, no existe un algoritmo para poder decidir  el problema en general.
\espc
\pause
\begin{exampleblock}{Teorema de Indecibilidad de Church.}
El problema de validez universal es indecidible. Es decir, no puede existir un algoritmo que reciba un enunciado 
$\vp$ como entrada y decida si $\models\vp$.
\end{exampleblock}

\begin{exampleblock}{Equivalencia lógica indecidible}
La equivalencia lógica es indecidible. Es decir, no puede existir un algoritmo que reciba como entrada dos fórmulas $\vp$ y $\psi$ y decida si $\vp\equiv\psi$.
\end{exampleblock}
}

\frame{\titulos{Indecibilidad LOP}{}
%\pause
\espc
\begin{exampleblock}{Consecuencia lógica indecidible}
La consecuencia lógica es indecidible. Es decir, no puede existir un algoritmo que reciba como entrada un conjunto finito de fórmulas $\G$ y una fórmula $\vp$ y decida si $\G\models\vp$.
\end{exampleblock}
\espc
\begin{exampleblock}{Satisfacibilidad indecidible}
Es indecidible si una fórmula es satisfacible. Es decir, dada una fórmula $\vp$ no existe un algoritmo que decida si existe un modelo $\M$ y un estado $\sigma$ tal que $\M\models_\sigma\vp$.
\end{exampleblock}
}









%\frame{\titulos{}{}
%\begin{block}{}
%
%\end{block}
%}
%
%\frame{\titulos{}{}
%\begin{exampleblock}{}
%
%\end{exampleblock}
%}
%s
%\frame{\titulos{}{}
%
%}
%



\end{document}
