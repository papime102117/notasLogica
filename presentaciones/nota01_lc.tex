%%% Presentaciones para Lenguajes de programacion y sus paradigmas 

%\documentclass[xcolor=dvipsnames,table,handout]{beamer}
\documentclass[xcolor=dvipsnames,table]{beamer}

\newcommand{\espc}{\vspace{0.3cm}}

\usepackage[utf8]{inputenc}
\usepackage[spanish]{babel}
\usepackage{hyperref}
\usepackage{lmodern}
\usepackage[T1]{fontenc}

%%%% paquetes matematicas
\usepackage{amssymb,amsmath,amscd}
\usepackage{extarrows}
\usepackage{stmaryrd}
\usepackage{mathabx}
\usepackage{mathrsfs}
% \usepackage{mathabx}
\usepackage{amsthm}

%%%%%
\usepackage{hyperref}
\usepackage{graphicx}
\usepackage{multicol}
\usepackage{pifont}
\usepackage{xcolor}
\usepackage{etex}
\usepackage{tikz}
\usepackage{array}
%\usepackage{pgfplots}

%%%% cosmetics
% D.Remy package for pretty display of rules
\usepackage{mathpartir}

% para insertar codigo con formato particular 
\usepackage{listing} 

% comillas 
\usepackage[autostyle=true,spanish=mexican]{csquotes}

% codigo 
\usepackage{verbatim}
\usepackage{alltt}

% footnotes
\usepackage[bottom]{footmisc}
\usepackage{setspace}

\usepackage{wrapfig}
\usepackage{caption}


\hfuzz=5.002pt %parameter to allow hbox overfulled by length before error!

% Options for presentation
% ------------------------
% \definecolor{mycolor}{RGB}{255,192,3}
\definecolor{mycolor}{RGB}{17,132,221}
\mode<presentation>
{
% \usetheme[secheader]{Boadilla}
% \usecolortheme{orchid}
\useoutertheme{infolines}
\useinnertheme{rectangles}
\setbeamertemplate{itemize items}[square]
\setbeamertemplate{enumerate items}[square]
\setbeamersize{text margin left=6mm, text margin right=6mm}

\setbeamercolor{alerted text}{fg=red,bg=red!70!white}
\setbeamercolor{background canvas}{bg=white}
\setbeamercolor{frametitle}{bg=mycolor,fg=white}
\setbeamercolor{normal text}{bg=white,fg=black}
\setbeamercolor{structure}{bg=black,fg=mycolor}
\setbeamercolor{title}{bg=mycolor,fg=white}
\setbeamercolor{subtitle}{bg=mycolor,fg=white}
\setbeamercolor{titlelike}{bg=white,fg=mycolor}

\setbeamercovered{invisible}

\setbeamercolor*{palette primary}{fg=mycolor,bg=white}
\setbeamercolor*{palette secondary}{bg=white,fg=white}
\setbeamercolor*{palette tertiary}{fg=mycolor,bg=white}
\setbeamercolor*{palette quaternary}{fg=white,bg=white}

\setbeamercolor{separation line}{bg=mycolor,fg=mycolor}
\setbeamercolor{fine separation line}{bg=white,fg=red}
\setbeamercolor{author in head/foot}{bg=mycolor!30!white,fg=mycolor!80!black}
\setbeamercolor{title in head/foot}{bg=mycolor!30!white,fg=mycolor!80!black}
\setbeamercolor{date in head/foot}{bg=mycolor!30!white,fg=mycolor!80!black}
\setbeamercolor{institute in head/foot}{bg=mycolor!30!white,fg=mycolor!80!black}
\setbeamercolor{section in head/foot}{bg=mycolor!60!white, fg=Red}
\setbeamercolor{subsection in head/foot}
{bg=mycolor!50!white,fg=mycolor!50!white}


\setbeamertemplate{headline}
{
  \leavevmode%
  \hbox{%
  \begin{beamercolorbox}[wd=.5\paperwidth,ht=2.65ex,dp=1.5ex,center]{section in 
head/foot}%
    \usebeamerfont{section in head/foot}\insertsectionhead\hspace*{2ex}
  \end{beamercolorbox}%
  \begin{beamercolorbox}[wd=.5\paperwidth,ht=2.65ex,dp=1.5ex,center]{subsection 
in head/foot}%
    \usebeamerfont{subsection in head/foot}\hspace*{2ex}\insertsubsectionhead
  \end{beamercolorbox}}%
  \vskip0pt%
}
% \beamerdefaultoverlayspecification{<+->}
\beamertemplatenavigationsymbolsempty
% \setbeamertemplate{footline}[frame number]
}

%%%% Macros para las notas de lenguajes de programacion


%%%% math

\newcommand{\vphi}{\varphi}
\newcommand{\vp}{\varphi}

% \newcommand{\dn}{\mathsf{DN}}
% \newcommand{\dnC}{\mathsf{DN_C}}
% \newcommand{\dnM}{\mathsf{DN_M}}
% \newcommand{\dnp}{\mathsf{DN_p}}
% \newcommand{\dnm}{\mathsf{DN_p^M}}
% \newcommand{\dnc}{\mathsf{DN_p^C}}

%\newcommand{\case}{\mathsf{case}}
%\renewcommand\labelitemi{$\circ$}

\newcommand{\imp}{\rightarrow}
\newcommand{\Imp}{\Rightarrow}
\renewcommand{\iff}{\leftrightarrow}
\newcommand{\Iff}{\Leftrightarrow}
\newcommand{\G}{\Gamma}
\newcommand{\D}{\Delta}


\newcommand{\De}{\mathcal{D}}
\newcommand{\F}{\mathcal{F}}
\newcommand{\Ge}{\mathcal{G}}
\newcommand{\Pe}{\mathcal{P}}
\newcommand{\I}{\mathcal{I}}
\newcommand{\C}{\mathcal{C}}
\newcommand{\K}{\mathcal{K}}
\renewcommand{\L}{\mathcal{L}}
\newcommand{\M}{\mathcal{M}}
\newcommand{\Nc}{\mathcal{N}}
%\newcommand{\E}{\mathcal{E}}
%\newcommand{\R}{\mathcal{R}}
%\newcommand{\Q}{\mathcal{Q}}
\newcommand{\Sc}{\mathcal{S}}
\newcommand{\Te}{\mathcal{T}}
\newcommand{\W}{\mathcal{W}}

\newcommand{\Db}{\mathbb{D}}
\newcommand{\Fb}{\mathbb{F}}
\newcommand{\Kb}{\mathbb{K}}
\newcommand{\Eb}{\mathbb{E}}
\newcommand{\Ebs}{\mathbb{E}^\star}
\newcommand{\Ob}{\mathbb{O}}
\newcommand{\Ib}{\mathbb{I}}
\newcommand{\Rb}{\mathbb{R}}
\newcommand{\Qb}{\mathbb{Q}}
\newcommand{\Kbb}{\mathbb{K}}
\newcommand{\T}{\mathbb{\Theta}}


\newcommand{\kb}{\bbkappa}

\newcommand{\Sf}{\mathsf{\Sigma}}

\newcommand{\fa}{\forall}
\newcommand{\ex}{\exists}

\newcommand{\inc}{\subseteq}

\newcommand{\Lb}{\Lambda}
\newcommand{\Om}{\Omega}
\newcommand{\lb}{\lambda}
\newcommand{\al}{\alpha}
\newcommand{\ga}{\gamma}


\newcommand{\mg}{\mathbb{m}}

\newcommand{\cg}{\mathbb{C}}
\newcommand{\dg}{\mathbb{D}}
\newcommand{\jg}{\mathbb{J}}
\newcommand{\Ha}{\mathcal{H}}
%\newcommand{\A}{\mathcal{A}}
\newcommand{\sg}{\mathbb{S}}

\newcommand{\Bc}{\mathcal{B}}
\newcommand{\Df}{\mathfrak{D}}
\newcommand{\Dc}{\mathcal{D}}
%\newcommand{\Tc}{\mathcal{T}}
\newcommand{\Mf}{\mathfrak{M}}

\newcommand{\Sg}{\mathbb{S}}

\newcommand{\Q}{\ensuremath{\mathbb{Q}}}
\newcommand{\Z}{\ensuremath{\mathbb{Z}}}
\newcommand{\N}{\ensuremath{\mathbb{N}}}
\newcommand{\R}{\ensuremath{\mathbb{R}}}
\renewcommand{\S}{\mathbb{\Sigma}}
\newcommand{\A}{\mathcal{A}}
\newcommand{\E}{\ensuremath{\exists}}
\newcommand{\iso}{\ensuremath{\cong}}
\newcommand{\union}{\ensuremath{\cup}}
\newcommand{\morinyec}{\ensuremath{\precapprox}}

\newcommand{\nin}{\ensuremath{\notin}}
\newcommand{\tog}{\makebox[7mm][l]}
\newcommand{\toge}{\makebox[11mm][l]}
\newcommand{\toget}{\makebox[13mm][l]}
\newcommand{\togeth}{\makebox[14mm][l]}
\newcommand{\togethe}{\makebox[15mm][l]}
\newcommand{\together}{\makebox[17mm][l]}
\newcommand{\niso}{\ensuremath{\not \cong}}


\newcommand{\Mg}{\mathbb{M}}
\newcommand{\Bg}{\mathbb{B}}
\newcommand{\Lg}{\mathbb{L}}
\newcommand{\Tg}{\mathbb{T}}

\newcommand{\sketch}{\Red{{\sc sketch}}}

\newcommand{\restr}[2]{#1\!\!\boldsymbol{\restriction}\!#2}

\newcommand{\vacio}{\varnothing}
\newcommand{\done}{\ensuremath{\checkmark}}

\newcommand{\ida}{$\Rightarrow \; )$ }
\newcommand{\regr}{$\Leftarrow \; )$ }

\newcommand{\ol}[1]{\overline{#1}}

\newcommand{\Tsf}{\mathsf{T}}

\newcommand{\inds}[1]{\index[simb]{#1}}

\newcommand{\B}{\mathbb{B}}
%\newcommand{\N}{\mathbb{N}}

\newcommand{\vx}{\vec{x}}
\newcommand{\vy}{\vec{y}}
\newcommand{\vz}{\vec{z}}
\newcommand{\vt}{\vec{t}}
\newcommand{\vf}{\vec{f}}


% \newcommand{\propo}{\ensuremath{\mathsf{PROP}}}
% \newcommand{\atom}{\ensuremath{\mathsf{ATOM}}}
\newcommand{\term}{\ensuremath{\mathsf{TERM}}}
\newcommand{\form}{\mathsf{FORM}}

\newcommand{\true}{\mathop{\mathsf{true}}}

%\newcommand{\id}{\mathsf{Id}}

%\newcommand{\uc}{\mathcal{U}}
%\newcommand{\Ic}{\mathcal{I}}
%\newcommand{\pc}{\mathcal{P}}
%\newcommand{\qc}{\mathcal{Q}}
%\newcommand{\mc}{\mathcal{M}}
\newcommand{\supc}{\supseteq}
\newcommand{\limo}{\mathop{\mathpzc{Lim}}}
\newcommand{\ord}{\mathsf{OR}}

\newcommand{\pt}[1]{\langle #1 \rangle}


%%%% frames
\newcommand{\titulos}[2]{\frametitle{#1}\framesubtitle{#2}}
\newcommand{\fot}[1]{\footnote{\scriptsize{#1}}}


%%%% ambientes

\newcommand{\cb}[2]{\colorbox{#1}{#2}}

\newcommand{\bc}{\begin{center}}
\newcommand{\ec}{\end{center}}
\newcommand{\be}{\begin{enumerate}}
\newcommand{\ee}{\end{enumerate}}
\newcommand{\bi}{\begin{itemize}}
\newcommand{\ei}{\end{itemize}}
\newcommand{\beq}{\begin{equation}}
\newcommand{\eeq}{\end{equation}}
\newcommand{\beqs}{\begin{equation*}}
\newcommand{\eeqs}{\end{equation*}}
\newcommand{\ba}{\begin{array}}
\newcommand{\ea}{\end{array}}


% \newtheorem{theorem}{Teorema}
% \newcommand{\teo}[1]{\begin{theorem} #1 \end{theorem}}
% \newtheorem{proposition}{Proposici\'on}
% \newcommand{\prop}[1]{\begin{proposition} #1 \end{proposition}}
% \newtheorem{definition}{Definici\'on}
% \newcommand{\defin}[1]{\begin{definition} #1 \end{definition}}
% \newtheorem{corollary}{Corolario}
% \newcommand{\cor}[1]{\begin{corollary} #1 \end{corollary}}
% \newtheorem{lemma}{Lema}
% \newcommand{\lema}[1]{\begin{lemma} #1 \end{lemma}}
% \newcommand{\dem}[1]{\begin{proof} #1 \end{proof}}

%\renewcommand{\qed}{\qedsymbol{$\mathbf{\dashv}$}}

%\newcommand{\proof}{\hfill\\\noindent\textbf{\textit{Demostraci\'on. }}}

\newcommand{\hint}{\emph{Sugerencia: }}


\newcounter{EjempCtr}[section]
\newenvironment{enumrom}{\renewcommand{\theenumi}{\roman{enumi}}%
\renewcommand{\theenumii}{\roman{enumii}}
\renewcommand{\theenumiii}{\roman{enumiii}}
\renewcommand{\theenumiv}{\roman{enumiv}}
\begin{enumerate}}{\end{enumerate}}
\newenvironment{Ejemplo}
        {\stepcounter{EjempCtr}%
        \begin{description}\item[Ejemplo \thesection.\arabic{EjempCtr}]}%
        {\end{description}}
\newenvironment{demostr}{%
             {\em Demostración:}
                \begin{quotation}}{\end{quotation}}

   \newcommand{\beje}{\begin{Ejemplo}}
\newcommand{\eeje}{\end{Ejemplo}}


\newtheorem{eje}{Ejemplo}[section]
\newcommand{\ejem}[1]{\begin{eje}\normalfont #1 \end{eje}}

% \renewcommand\contentsname{\'Indice}
%\renewcommand\chaptername{Cap\'itulo}
% \renewcommand\indexname{\'Indice}

%%\newcommand{\qed}{\hfill$\mathbb{Qed}$}
%\newcommand{\qed}{\hfill$\mathsf{\boldsymbol{\dashv}}$}
%\renewcommand{\qed}{\hfill$\boldsymbol{\dashv}$}


\newenvironment{prueba}{\vspace{-5mm}\noindent\textbf{Demostraci\'on}\\}{
\noindent$\blacksquare$\\}

\newcommand{\Ejercicios}{\section*{Ejercicios}}

 \newenvironment{manitas}{%
      \renewcommand{\labelitemi}{\ding{44}}%
      \vspace{-0.5cm}%
      \begin{itemize}%
      \setlength{\itemsep}{0pt}\setlength{\parsep}{0pt}\setlength{\topsep}{0pt}%
      }{\end{itemize}}
\newenvironment{malitos}{%
      \renewcommand{\labelitemi}%
            {\raisebox{1.5ex}{\makebox[0.3cm][l]{\begin{rotate}{-90}%
            \ding{43}\end{rotate}}}}%
      \vspace{-0.5cm}%
      \begin{itemize}%
      \setlength{\itemsep}{0pt}\setlength{\parsep}{0pt}\setlength{\topsep}{0pt}%
      }{\end{itemize}}
\newenvironment{ejercs}{
     \renewcommand{\labelenumi}{\thesection.\theenumi.-}
     \renewcommand{\labelenumii}{\theenumii)}
     \begin{enumerate}}
     {\end{enumerate}}

   \newcommand{\bej}{\begin{ejercs}}
\newcommand{\eej}{\end{ejercs}}


%\newenvironment{leterize}{%
%        \renewcommand{\theenumi}{\alph{enumi}}
%        \begin{enumerate}}{\end{enumerate}}

%\newenvironment{manitas}{%
%      \renewcommand{\labelitemi}{\ding{44}}%
%      \vspace{-0.5cm}%
%      \begin{itemize}%
%      
% \setlength{\itemsep}{0pt}\setlength{\parsep}{0pt}\setlength{\topsep}{0pt}%
%      }{\end{itemize}}



%%=============================================================================

\def\stackunder#1#2{\mathrel{\mathop{#2}\limits_{#1}}}


%%%% notas

\newcommand{\doubt}{\Red{{\LARGE {\sf ??}}}}

\newcommand{\coment}[1]{\hfill\\ \Big[{\bf Comentario Privado:} #1\Big]}
\newcommand{\preg}[1]{\hfill\\ \BrickRed{{\bf Pregunta:} #1}}
\newcommand{\conjet}[1]{\hfill\\ \OliveGreen{{\bf Conjecura:} #1}}

\newcommand{\pendiente}{\BrickRed{{\sc Pendiente}}}
\newcommand{\verifpendiente}{\BrickRed{{\sc Verificación pendiente}}}


%--------------------------------------------------------------------------

\DeclareMathAlphabet{\mathpzc}{OT1}{pzc}{m}{it}


\title[]{Lógica computacional}
\subtitle{Tema: Semántica de la Lógica Proposicional }
\author{ Pilar Selene Linares Ar\'evalo}
\institute[UNAM-FC]{Facultad de Ciencias\\ 
Universidad Nacional Aut\'onoma de M\'exico}
\date[]{ \footnotesize{febrero 2018}
\newline{\tiny{Material desarrollado bajo el proyecto UNAM-PAPIME PE102117.}}}
 

\beamerdefaultoverlayspecification{<+->}
 
\titlegraphic{\includegraphics[width=16mm]{fc2.png}}
 

\begin{document}

\begin{frame}
\titlepage 
\end{frame}
\note{}


\frame{\titulos{Significado de los conectivos lógicos}{}
\begin{block}{{\bf Negación.}}
La {\bf negación} de la fórmula $P$ es la fórmula $\mathbf{\neg P}$. \\
\pause
\espc
Corresponde en español a : {\em No, no es cierto que, es falso que, etc.}\\
\pause
\espc
Tabla de verdad: 
\begin{center}
\begin{tabular}{|c|c|}
\hline
P & $\neg P$ \\ \hline
0 & 1 \\ \hline
1 & 0 \\ \hline
\end{tabular}
\end{center}
\end{block}
}

\frame{\titulos{Significado de los conectivos lógicos}{}
\begin{block}{{\bf Disyunción.}}
La {\bf disyunción} de las fórmulas $P, Q$ es la fórmula $\mathbf{ P \lor Q}$. \\
\pause
\espc
Corresponde en español a : {\em o}\\
\pause
\espc
Tabla de verdad: 
\begin{center}
\begin{tabular}{|c|c|c|}
\hline
P & Q& $ P \lor Q$ \\ \hline
0 & 0 & 0 \\ \hline
0 & 1 & 1 \\ \hline
1 & 0 & 1 \\ \hline
1 & 1 & 1 \\ \hline
\end{tabular}
\end{center}
\end{block}

}

\frame{\titulos{Significado de los conectivos lógicos}{}
\begin{block}{{\bf Disyunción.}}
La {\bf conjunción} de las fórmulas $P, Q$ es la fórmula $\mathbf{ P \land Q}$. \\
\pause
\espc
Corresponde en español a : {\em y, pero}\\
\pause
\espc
Tabla de verdad: 
\begin{center}
\begin{tabular}{|c|c|c|}
\hline
P & Q& $ P \land Q$ \\ \hline
0 & 0 & 0 \\ \hline
0 & 1 & 0 \\ \hline
1 & 0 & 0 \\ \hline
1 & 1 &  1 \\ \hline
\end{tabular}
\end{center}

\end{block}

}


\frame{\titulos{Significado de los conectivos lógicos}{}
\begin{block}{{\bf Implicación o condicional.}}
La {\bf implicación} de las fórmulas $P, Q$ es la fórmula $\mathbf{ P \imp Q}$. Donde $P$ es el {\em antecedente} y $Q$ el {\em consecuente} de la implicación.\\
\pause
\espc
Corresponde en español a: {\em si P entonces Q; P es condición suficiente para Q; Q, si P; P sólo si Q; Q es condición necesaria para P.}\\
\pause
\espc
Tabla de verdad: 
\begin{center}
\begin{tabular}{|c|c|c|}
\hline
P & Q& $ P \imp Q$ \\ \hline
0 & 0 & 1 \\ \hline
0 & 1 & 1 \\ \hline
1 & 0 & 0 \\ \hline
1 & 1 & 1 \\ \hline
\end{tabular}
\end{center}


\end{block}

}



\frame{\titulos{Significado de los conectivos lógicos}{}
\begin{block}{{\bf Equivalencia o bicondicional.}}
La {\bf equivalencia} de las fórmulas $P, Q$ es la fórmula $\mathbf{ P \iff Q}$. \\
\pause
\espc
Corresponde en español a: {\em  P es equivalente a Q, P si y sólo si Q,  P es condición necesaria y suficiente para Q.}\\
\pause
\espc
Tabla de verdad: 
\begin{center}
\begin{tabular}{|c|c|c|}
\hline
P & Q& $ P \iff Q$ \\ \hline
0 & 0 & 1 \\ \hline
0 & 1 & 0 \\ \hline
1 & 0 & 0 \\ \hline
1 & 1 & 1 \\ \hline
\end{tabular}
\end{center}


\end{block}

}

\frame{\titulos{Formalización}{}
\begin{block}{{\bf Estado o asignación de las variables}}
Un {\bf estado o asignación} de las variables proposicionales es una función
\[ \mathcal{I} : VarP \rightarrow \{ 0,1 \} \]
\end{block}
\pause
\[  \mathcal{I}(r) = 1 \qquad  \mathcal{I}(p) = 0 \qquad \mathcal{I}(t_{16}) = 1 \]
}

\frame{\titulos{Formalización}{}
\begin{block}{{\bf Función de Interpretación}}
Dado un estado de las variables $\mathcal{I} : VarP \rightarrow \{ 0,1 \}$, definimos la {\bf interpretación de las fórmulas} con respecto a $\mathcal{I}$ como la función $\mathcal{I}^{*}: PROP \rightarrow \{ 0,1 \}$ tal que:
\begin{itemize}
\item $\mathcal{I}^{*}(p) = \mathcal{I}(p) $ con $p \in VarP$.
\item $\mathcal{I}^{*}(\top) = 1$  y $\mathcal{I}^{*}(\bot) = 0$.
\item $\mathcal{I}^{*}(\neg \varphi) = 1$ syss $\mathcal{I}^{*}(\varphi) = 0$.
\item $\mathcal{I}^{*}(\varphi_{1} \land \varphi_{2}) = 1 $ syss $\mathcal{I}^{*}(\varphi_{1}) = \mathcal{I}^{*}(\varphi_{2})=1 $.
\item $\mathcal{I}^{*}(\varphi_{1} \lor \varphi_{2}) = 0 $ syss $\mathcal{I}^{*}(\varphi_{1}) = \mathcal{I}^{*}(\varphi_{2})= 0$.
\item $\mathcal{I}^{*}(\varphi_{1} \imp \varphi_{2}) = 0 $ syss $\mathcal{I}^{*}(\varphi_{1}) = 1$ e $ \mathcal{I}^{*}(\varphi_{2})=0 $.
\item $\mathcal{I}^{*}(\varphi_{1} \iff \varphi_{2}) = 1 $ syss $\mathcal{I}^{*}(\varphi_{1}) = \mathcal{I}^{*}(\varphi_{2})$.
\end{itemize}
\end{block}
Haremos abuso de notación: escribiremos simplemente $\varphi$ en lugar de $\varphi^{*}$.
}


\frame{\titulos{Semántica }{}
\begin{exampleblock}{{\bf Lema de Coincidencia}}
Sean  $\mathcal{I}_{1}, \mathcal{I}_{2} : PROP \imp \{ 0,1 \}$, dos estados que coinciden en las variables proposicionales de la fórmula $\varphi$, es decir, $\mathcal{I}_{1}(p) = \mathcal{I}_{2}(p)$ para toda $p \in vars(\varphi)$. Entonces 
\[ \mathcal{I}_{1}(\varphi) = \mathcal{I}_{2}(\varphi)\]
\end{exampleblock}
}



\frame{\titulos{Semántica}{}
\begin{block}{{\bf Estado modificado o actualizado}}
Sean $\mathcal{I}: varp \imp \{ 0,1\}$ un estado de las variables, $p$ una variable proposicional y $v\in\{ 0,1\}$. Definimos la actualizaci\'on de $\mathcal{I}$ en $p$ por $v$, denotado $\mathcal{I}_{[p/v]}$ como sigue:
\[
\mathcal{I}_{[p/v\;]}(q) =\left\{\ba{rl}
v & \text{ si } q=p \\ \\
\mathcal{I}(q) & \text{ si } q \neq p
\ea
\right.
\]
El estado $\mathcal{I}_{[p/v]}$ se conoce como un estado modificado o una actualizaci\'on 
de $\varphi$.

\end{block}
}


\frame{\titulos{Semántica }{}
\begin{exampleblock}{{\bf Lema de Sustitución}}
Sean $\mathcal{I}$ una interpretaci\'on, $p$ una variable proposicional y $\psi$ una f\'ormula tal que $\mathcal{I}(\psi)=v$. Entonces
\[\mathcal{I}\big(\varphi[p:=\psi\;]\big) = \mathcal{I}_{[p/v\;]}(\varphi)\]
\end{exampleblock}
}


\frame{\titulos{Conceptos Semánticos básicos}{}
¿Cuántas interpretaciones hacen verdadera a $\varphi $?
\begin{block}{}
\begin{itemize}
\item Si $\mathcal{I}(\varphi)=1$ para {\bf toda interpretaci\'on} $\mathcal{I}$ decimos que $\varphi$ es una
  {\bf  tautolog\'ia o f\'ormula v\'alida} y escribimos $\models\varphi$. \\
\espc
\item Si $\mathcal{I}(\varphi)=1$ para {\bf alguna interpretaci\'on} $\mathcal{I}$ decimos que $\varphi$ es   {\bf satisfacible}, que $\varphi$ es verdadera en $\mathcal{I}$ o que $\mathcal{I}$ es {\bf modelo} de
  $\varphi$ y escribimos $\mathcal{I}\models\varphi$
\espc
\item Si $\mathcal{I}(\varphi)=0$ para {\bf alguna interpretaci\'on} $\mathcal{I}$ decimos que $\varphi$ es   {\bf falsa o insatisfacible} en $\mathcal{I}$ o que $\mathcal{I}$ no es modelo de $\varphi$ y
  escribimos $\mathcal{I}\not\models\varphi$
\espc
\item Si $\mathcal{I}(\varphi)=0$ para {\bf toda interpretaci\'on} $\mathcal{I}$ decimos que $\varphi$ es una  {\bf  contradicci\'on} o f\'ormula no satisfacible.    
\espc
\ei
\end{block}
}

\frame{\titulos{Conceptos Semánticos básicos}{} 
Similarmente si $\Gamma$ es un conjunto de f\'ormulas decimos que:
\begin{block}{{}}
\bi
\item $\Gamma$ {\bf es satisfacible} si tiene un modelo, es decir, si existe una 
interpretaci\'on $\mathcal{I}$ tal que
$\mathcal{I}(\varphi)=1$ para toda $\varphi \in\Gamma$. Lo cual denotamos a veces,
abusando de la notaci\'on, con $\mathcal{I}(\Gamma)=1$.
\espc
\item $\Gamma$ {\bf es insatisfacible }o no satisfacible si no tiene un
  modelo, es decir, si no existe una
  interpretaci\'on $\mathcal{I}$ tal que $\mathcal{I}(\varphi)=1$ para toda $\varphi\in\Gamma$.
\espc
\end{itemize}
\end{block}

}


\frame{\titulos{Conceptos Semánticos básicos}{} 
\begin{exampleblock}{{\bf Propiedades}}
Sea $\Gamma$ un conjunto de f\'ormulas, $\varphi \in \Gamma$, $\tau$ una tautología y $\chi$ una contradicción.
\bi
\item Si $\Gamma$ es satisfacible entonces :
	\bi
	\item $\Gamma \backslash \{ \varphi \}$ es satisfacible.
	\item $\Gamma \cup \{ \tau \}$ es satisfacible.
	\item $\Gamma \cup \{ \chi \}$ es insatisfacible.
	\ei
\espc
\item Si $\Gamma$ es insatisfacible entonces :
	\bi
	\item $\Gamma \cup \{ \psi \}$ es insatisfacible, para cualquier $\psi \in PROP$.
	\item $\Gamma \backslash \{ \tau \}$ es insatisfacible. 
\espc
	\ei

\ei
\end{exampleblock}
}

\frame{\titulos{Equivalencia de Fórmulas}{} 
\begin{block}{{\bf Equivalencia}}
Dos fórmulas $\varphi , \psi$ son {\bf equivalentes} si $\mathcal{I}(\varphi) = \mathcal{I}(\psi)$ para toda interpretación $\mathcal{I}$. En tal caso escribimos 
\[ \varphi \equiv \psi \]

\end{block}

\begin{exampleblock}{{\bf Proposición :}}
Sean $\varphi , \psi$ dos  fórmulas. Entonces
\[ \varphi \equiv \psi  \mbox{ si y sólo si }  \vDash \varphi \iff \psi \]
\espc
\end{exampleblock}
}

\frame{\titulos{Equivalencia de Fórmulas}{} 

\begin{exampleblock}{{\bf Regla de Leibniz}}
Sean $\varphi , \psi, \chi$ fórmulas y $p \in varP$
\[ \frac{\varphi \equiv \psi}{\chi[p:=\varphi] \equiv \chi[p:=\psi]} \]
\espc
\end{exampleblock}
}

\frame{\titulos{Consecuencia Lógica}{} 
\begin{block}{{\bf Consecuencia Lógica}}
Sean $\Gamma$ un conjunto de f\'ormulas y $\varphi$ una f\'ormula. Decimos que $\varphi$ es {\bf consecuencia l\'ogica} de $\Gamma$ si para toda interpretaci\'on $\mathcal{I}$ que sastisface a $\Gamma$, se tiene $\mathcal{I}(\varphi)=1$.  \\
\pause
\espc
Es decir, si se cumple que siempre que $\mathcal{I}$ satisface a $\Gamma$ entonces necesariamente $\mathcal{I}$ satisface a $\varphi$. En tal caso escribimos 
\[ \Gamma\models\varphi \]
\end{block}

}

\frame{\titulos{Consecuencia Lógica}{} 
\begin{exampleblock}{{}}
La relaci\'on de consecuencia l\'ogica cumple las 
siguientes propiedades:
\bi
\item Si $\varphi\in\Gamma$ entonces $\Gamma\models\varphi$. \vspace{0.2cm}
\item Principio de refutaci\'on: $\Gamma\models\varphi$ syss 
$\Gamma\cup\{\lnot\varphi\}$
  es insatisfacible. \vspace{0.2cm}
\item $\Gamma\models\varphi\imp\psi$ syss $\Gamma\cup\{\varphi\}\models\psi$. \vspace{0.2cm}
\item Insatisfacibilidad implica trivialidad: Si $\Gamma$ es insatisfacible
entonces $\Gamma\models\varphi$ para toda $\varphi\in PROP$. \vspace{0.2cm}
\item Si $\Gamma\models\bot$ entonces $\Gamma$ es insatisfacible. \vspace{0.2cm}
\item $\varphi\equiv\psi$ syss $\varphi\models\psi$ y $\psi\models\varphi$. \vspace{0.2cm}
\item $\models\varphi$ (es decir si $\varphi$ es tautolog\'ia) syss
  $\emptyset \models\varphi$ (es decir $\varphi$ es consecuencia l\'ogica del conjunto 
vac\'io). \espc
\ei 
\end{exampleblock}
}

\frame{\titulos{Consecuencia Lógica}{} 
\begin{block}{{\bf Correctud de argumentos lógicos}}
Un argumento con premisas $\varphi_{1}, ... , \varphi_{n}$ y conclusión $\psi$ es {\bf lógicamente correcto} si la conclusión se sigue de las premisas, es decir, si $\{ \varphi_{1}, ... , \varphi{n} \} \vDash \psi$. \\
\pause
\espc
Para mostrar la correctud del argumento l\'ogico 
$\varphi_1,\ldots,\varphi_n\;/\therefore\; \psi$
mediante interpretaciones, se puede proceder de alguna de las siguientes formas:

\bi
\item {\bf M\'etodo directo:} probar la consecuencia $\varphi_1,\ldots,\varphi_n\models \psi$.

\item {\bf M\'etodo indirecto (refutaci\'on):} probar que el conjunto
$\{\varphi_1,\ldots, \varphi_n,\lnot \psi\}$ es insatisfacible. 
\espc
\ei

\end{block}

}


\end{document}
