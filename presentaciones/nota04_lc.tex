%%% Presentaciones para Lenguajes de programacion y sus paradigmas 

%\documentclass[xcolor=dvipsnames,table,handout]{beamer}
\documentclass[xcolor=dvipsnames,table]{beamer}

\newcommand{\espc}{\vspace{0.3cm}}

\usepackage[utf8]{inputenc}
\usepackage[spanish]{babel}
\usepackage{hyperref}
\usepackage{lmodern}
\usepackage[T1]{fontenc}

%%%% paquetes matematicas
\usepackage{amssymb,amsmath,amscd}
\usepackage{extarrows}
\usepackage{stmaryrd}
\usepackage{mathabx}
\usepackage{mathrsfs}
% \usepackage{mathabx}
\usepackage{amsthm}

%%%%%
\usepackage{hyperref}
\usepackage{graphicx}
\usepackage{multicol}
\usepackage{pifont}
\usepackage{xcolor}
\usepackage{etex}
\usepackage{tikz}
\usepackage{array}
%\usepackage{pgfplots}

%%%% cosmetics
% D.Remy package for pretty display of rules
\usepackage{mathpartir}

% para insertar codigo con formato particular 
\usepackage{listing} 

% comillas 
\usepackage[autostyle=true,spanish=mexican]{csquotes}

% codigo 
\usepackage{verbatim}
\usepackage{alltt}

% footnotes
\usepackage[bottom]{footmisc}
\usepackage{setspace}

\usepackage{wrapfig}
\usepackage{caption}


\hfuzz=5.002pt %parameter to allow hbox overfulled by length before error!

% Options for presentation
% ------------------------
% \definecolor{mycolor}{RGB}{255,192,3}
\definecolor{mycolor}{RGB}{17,132,221}
\mode<presentation>
{
% \usetheme[secheader]{Boadilla}
% \usecolortheme{orchid}
\useoutertheme{infolines}
\useinnertheme{rectangles}
\setbeamertemplate{itemize items}[square]
\setbeamertemplate{enumerate items}[square]
\setbeamersize{text margin left=6mm, text margin right=6mm}

\setbeamercolor{alerted text}{fg=red,bg=red!70!white}
\setbeamercolor{background canvas}{bg=white}
\setbeamercolor{frametitle}{bg=mycolor,fg=white}
\setbeamercolor{normal text}{bg=white,fg=black}
\setbeamercolor{structure}{bg=black,fg=mycolor}
\setbeamercolor{title}{bg=mycolor,fg=white}
\setbeamercolor{subtitle}{bg=mycolor,fg=white}
\setbeamercolor{titlelike}{bg=white,fg=mycolor}

\setbeamercovered{invisible}

\setbeamercolor*{palette primary}{fg=mycolor,bg=white}
\setbeamercolor*{palette secondary}{bg=white,fg=white}
\setbeamercolor*{palette tertiary}{fg=mycolor,bg=white}
\setbeamercolor*{palette quaternary}{fg=white,bg=white}

\setbeamercolor{separation line}{bg=mycolor,fg=mycolor}
\setbeamercolor{fine separation line}{bg=white,fg=red}
\setbeamercolor{author in head/foot}{bg=mycolor!30!white,fg=mycolor!80!black}
\setbeamercolor{title in head/foot}{bg=mycolor!30!white,fg=mycolor!80!black}
\setbeamercolor{date in head/foot}{bg=mycolor!30!white,fg=mycolor!80!black}
\setbeamercolor{institute in head/foot}{bg=mycolor!30!white,fg=mycolor!80!black}
\setbeamercolor{section in head/foot}{bg=mycolor!60!white, fg=Red}
\setbeamercolor{subsection in head/foot}
{bg=mycolor!50!white,fg=mycolor!50!white}


\setbeamertemplate{headline}
{
  \leavevmode%
  \hbox{%
  \begin{beamercolorbox}[wd=.5\paperwidth,ht=2.65ex,dp=1.5ex,center]{section in 
head/foot}%
    \usebeamerfont{section in head/foot}\insertsectionhead\hspace*{2ex}
  \end{beamercolorbox}%
  \begin{beamercolorbox}[wd=.5\paperwidth,ht=2.65ex,dp=1.5ex,center]{subsection 
in head/foot}%
    \usebeamerfont{subsection in head/foot}\hspace*{2ex}\insertsubsectionhead
  \end{beamercolorbox}}%
  \vskip0pt%
}
% \beamerdefaultoverlayspecification{<+->}
\beamertemplatenavigationsymbolsempty
% \setbeamertemplate{footline}[frame number]
}

%%%% Macros para las notas de lenguajes de programacion


%%%% math

\newcommand{\vphi}{\varphi}
\newcommand{\vp}{\varphi}

% \newcommand{\dn}{\mathsf{DN}}
% \newcommand{\dnC}{\mathsf{DN_C}}
% \newcommand{\dnM}{\mathsf{DN_M}}
% \newcommand{\dnp}{\mathsf{DN_p}}
% \newcommand{\dnm}{\mathsf{DN_p^M}}
% \newcommand{\dnc}{\mathsf{DN_p^C}}

%\newcommand{\case}{\mathsf{case}}
%\renewcommand\labelitemi{$\circ$}

\newcommand{\imp}{\rightarrow}
\newcommand{\Imp}{\Rightarrow}
\renewcommand{\iff}{\leftrightarrow}
\newcommand{\Iff}{\Leftrightarrow}
\newcommand{\G}{\Gamma}
\newcommand{\D}{\Delta}


\newcommand{\De}{\mathcal{D}}
\newcommand{\F}{\mathcal{F}}
\newcommand{\Ge}{\mathcal{G}}
\newcommand{\Pe}{\mathcal{P}}
\newcommand{\I}{\mathcal{I}}
\newcommand{\C}{\mathcal{C}}
\newcommand{\K}{\mathcal{K}}
\renewcommand{\L}{\mathcal{L}}
\newcommand{\M}{\mathcal{M}}
\newcommand{\Nc}{\mathcal{N}}
%\newcommand{\E}{\mathcal{E}}
%\newcommand{\R}{\mathcal{R}}
%\newcommand{\Q}{\mathcal{Q}}
\newcommand{\Sc}{\mathcal{S}}
\newcommand{\Te}{\mathcal{T}}
\newcommand{\W}{\mathcal{W}}

\newcommand{\Db}{\mathbb{D}}
\newcommand{\Fb}{\mathbb{F}}
\newcommand{\Kb}{\mathbb{K}}
\newcommand{\Eb}{\mathbb{E}}
\newcommand{\Ebs}{\mathbb{E}^\star}
\newcommand{\Ob}{\mathbb{O}}
\newcommand{\Ib}{\mathbb{I}}
\newcommand{\Rb}{\mathbb{R}}
\newcommand{\Qb}{\mathbb{Q}}
\newcommand{\Kbb}{\mathbb{K}}
\newcommand{\T}{\mathbb{\Theta}}


\newcommand{\kb}{\bbkappa}

\newcommand{\Sf}{\mathsf{\Sigma}}

\newcommand{\fa}{\forall}
\newcommand{\ex}{\exists}

\newcommand{\inc}{\subseteq}

\newcommand{\Lb}{\Lambda}
\newcommand{\Om}{\Omega}
\newcommand{\lb}{\lambda}
\newcommand{\al}{\alpha}
\newcommand{\ga}{\gamma}


\newcommand{\mg}{\mathbb{m}}

\newcommand{\cg}{\mathbb{C}}
\newcommand{\dg}{\mathbb{D}}
\newcommand{\jg}{\mathbb{J}}
\newcommand{\Ha}{\mathcal{H}}
%\newcommand{\A}{\mathcal{A}}
\newcommand{\sg}{\mathbb{S}}

\newcommand{\Bc}{\mathcal{B}}
\newcommand{\Df}{\mathfrak{D}}
\newcommand{\Dc}{\mathcal{D}}
%\newcommand{\Tc}{\mathcal{T}}
\newcommand{\Mf}{\mathfrak{M}}

\newcommand{\Sg}{\mathbb{S}}

\newcommand{\Q}{\ensuremath{\mathbb{Q}}}
\newcommand{\Z}{\ensuremath{\mathbb{Z}}}
\newcommand{\N}{\ensuremath{\mathbb{N}}}
\newcommand{\R}{\ensuremath{\mathbb{R}}}
\renewcommand{\S}{\mathbb{\Sigma}}
\newcommand{\A}{\mathcal{A}}
\newcommand{\E}{\ensuremath{\exists}}
\newcommand{\iso}{\ensuremath{\cong}}
\newcommand{\union}{\ensuremath{\cup}}
\newcommand{\morinyec}{\ensuremath{\precapprox}}

\newcommand{\nin}{\ensuremath{\notin}}
\newcommand{\tog}{\makebox[7mm][l]}
\newcommand{\toge}{\makebox[11mm][l]}
\newcommand{\toget}{\makebox[13mm][l]}
\newcommand{\togeth}{\makebox[14mm][l]}
\newcommand{\togethe}{\makebox[15mm][l]}
\newcommand{\together}{\makebox[17mm][l]}
\newcommand{\niso}{\ensuremath{\not \cong}}


\newcommand{\Mg}{\mathbb{M}}
\newcommand{\Bg}{\mathbb{B}}
\newcommand{\Lg}{\mathbb{L}}
\newcommand{\Tg}{\mathbb{T}}

\newcommand{\sketch}{\Red{{\sc sketch}}}

\newcommand{\restr}[2]{#1\!\!\boldsymbol{\restriction}\!#2}

\newcommand{\vacio}{\varnothing}
\newcommand{\done}{\ensuremath{\checkmark}}

\newcommand{\ida}{$\Rightarrow \; )$ }
\newcommand{\regr}{$\Leftarrow \; )$ }

\newcommand{\ol}[1]{\overline{#1}}

\newcommand{\Tsf}{\mathsf{T}}

\newcommand{\inds}[1]{\index[simb]{#1}}

\newcommand{\B}{\mathbb{B}}
%\newcommand{\N}{\mathbb{N}}

\newcommand{\vx}{\vec{x}}
\newcommand{\vy}{\vec{y}}
\newcommand{\vz}{\vec{z}}
\newcommand{\vt}{\vec{t}}
\newcommand{\vf}{\vec{f}}


% \newcommand{\propo}{\ensuremath{\mathsf{PROP}}}
% \newcommand{\atom}{\ensuremath{\mathsf{ATOM}}}
\newcommand{\term}{\ensuremath{\mathsf{TERM}}}
\newcommand{\form}{\mathsf{FORM}}

\newcommand{\true}{\mathop{\mathsf{true}}}

%\newcommand{\id}{\mathsf{Id}}

%\newcommand{\uc}{\mathcal{U}}
%\newcommand{\Ic}{\mathcal{I}}
%\newcommand{\pc}{\mathcal{P}}
%\newcommand{\qc}{\mathcal{Q}}
%\newcommand{\mc}{\mathcal{M}}
\newcommand{\supc}{\supseteq}
\newcommand{\limo}{\mathop{\mathpzc{Lim}}}
\newcommand{\ord}{\mathsf{OR}}

\newcommand{\pt}[1]{\langle #1 \rangle}


%%%% frames
\newcommand{\titulos}[2]{\frametitle{#1}\framesubtitle{#2}}
\newcommand{\fot}[1]{\footnote{\scriptsize{#1}}}


%%%% ambientes

\newcommand{\cb}[2]{\colorbox{#1}{#2}}

\newcommand{\bc}{\begin{center}}
\newcommand{\ec}{\end{center}}
\newcommand{\be}{\begin{enumerate}}
\newcommand{\ee}{\end{enumerate}}
\newcommand{\bi}{\begin{itemize}}
\newcommand{\ei}{\end{itemize}}
\newcommand{\beq}{\begin{equation}}
\newcommand{\eeq}{\end{equation}}
\newcommand{\beqs}{\begin{equation*}}
\newcommand{\eeqs}{\end{equation*}}
\newcommand{\ba}{\begin{array}}
\newcommand{\ea}{\end{array}}


% \newtheorem{theorem}{Teorema}
% \newcommand{\teo}[1]{\begin{theorem} #1 \end{theorem}}
% \newtheorem{proposition}{Proposici\'on}
% \newcommand{\prop}[1]{\begin{proposition} #1 \end{proposition}}
% \newtheorem{definition}{Definici\'on}
% \newcommand{\defin}[1]{\begin{definition} #1 \end{definition}}
% \newtheorem{corollary}{Corolario}
% \newcommand{\cor}[1]{\begin{corollary} #1 \end{corollary}}
% \newtheorem{lemma}{Lema}
% \newcommand{\lema}[1]{\begin{lemma} #1 \end{lemma}}
% \newcommand{\dem}[1]{\begin{proof} #1 \end{proof}}

%\renewcommand{\qed}{\qedsymbol{$\mathbf{\dashv}$}}

%\newcommand{\proof}{\hfill\\\noindent\textbf{\textit{Demostraci\'on. }}}

\newcommand{\hint}{\emph{Sugerencia: }}


\newcounter{EjempCtr}[section]
\newenvironment{enumrom}{\renewcommand{\theenumi}{\roman{enumi}}%
\renewcommand{\theenumii}{\roman{enumii}}
\renewcommand{\theenumiii}{\roman{enumiii}}
\renewcommand{\theenumiv}{\roman{enumiv}}
\begin{enumerate}}{\end{enumerate}}
\newenvironment{Ejemplo}
        {\stepcounter{EjempCtr}%
        \begin{description}\item[Ejemplo \thesection.\arabic{EjempCtr}]}%
        {\end{description}}
\newenvironment{demostr}{%
             {\em Demostración:}
                \begin{quotation}}{\end{quotation}}

   \newcommand{\beje}{\begin{Ejemplo}}
\newcommand{\eeje}{\end{Ejemplo}}


\newtheorem{eje}{Ejemplo}[section]
\newcommand{\ejem}[1]{\begin{eje}\normalfont #1 \end{eje}}

% \renewcommand\contentsname{\'Indice}
%\renewcommand\chaptername{Cap\'itulo}
% \renewcommand\indexname{\'Indice}

%%\newcommand{\qed}{\hfill$\mathbb{Qed}$}
%\newcommand{\qed}{\hfill$\mathsf{\boldsymbol{\dashv}}$}
%\renewcommand{\qed}{\hfill$\boldsymbol{\dashv}$}


\newenvironment{prueba}{\vspace{-5mm}\noindent\textbf{Demostraci\'on}\\}{
\noindent$\blacksquare$\\}

\newcommand{\Ejercicios}{\section*{Ejercicios}}

 \newenvironment{manitas}{%
      \renewcommand{\labelitemi}{\ding{44}}%
      \vspace{-0.5cm}%
      \begin{itemize}%
      \setlength{\itemsep}{0pt}\setlength{\parsep}{0pt}\setlength{\topsep}{0pt}%
      }{\end{itemize}}
\newenvironment{malitos}{%
      \renewcommand{\labelitemi}%
            {\raisebox{1.5ex}{\makebox[0.3cm][l]{\begin{rotate}{-90}%
            \ding{43}\end{rotate}}}}%
      \vspace{-0.5cm}%
      \begin{itemize}%
      \setlength{\itemsep}{0pt}\setlength{\parsep}{0pt}\setlength{\topsep}{0pt}%
      }{\end{itemize}}
\newenvironment{ejercs}{
     \renewcommand{\labelenumi}{\thesection.\theenumi.-}
     \renewcommand{\labelenumii}{\theenumii)}
     \begin{enumerate}}
     {\end{enumerate}}

   \newcommand{\bej}{\begin{ejercs}}
\newcommand{\eej}{\end{ejercs}}


%\newenvironment{leterize}{%
%        \renewcommand{\theenumi}{\alph{enumi}}
%        \begin{enumerate}}{\end{enumerate}}

%\newenvironment{manitas}{%
%      \renewcommand{\labelitemi}{\ding{44}}%
%      \vspace{-0.5cm}%
%      \begin{itemize}%
%      
% \setlength{\itemsep}{0pt}\setlength{\parsep}{0pt}\setlength{\topsep}{0pt}%
%      }{\end{itemize}}



%%=============================================================================

\def\stackunder#1#2{\mathrel{\mathop{#2}\limits_{#1}}}


%%%% notas

\newcommand{\doubt}{\Red{{\LARGE {\sf ??}}}}

\newcommand{\coment}[1]{\hfill\\ \Big[{\bf Comentario Privado:} #1\Big]}
\newcommand{\preg}[1]{\hfill\\ \BrickRed{{\bf Pregunta:} #1}}
\newcommand{\conjet}[1]{\hfill\\ \OliveGreen{{\bf Conjecura:} #1}}

\newcommand{\pendiente}{\BrickRed{{\sc Pendiente}}}
\newcommand{\verifpendiente}{\BrickRed{{\sc Verificación pendiente}}}


%--------------------------------------------------------------------------

\DeclareMathAlphabet{\mathpzc}{OT1}{pzc}{m}{it}


\title[]{Lógica computacional}
\subtitle{Tema: Lógica de primer orden: Especificación Formal y Semántica.  }
\author{ Pilar Selene Linares Ar\'evalo}
\institute[UNAM-FC]{Facultad de Ciencias\\ 
Universidad Nacional Aut\'onoma de M\'exico}
\date[]{ \footnotesize{marzo 2018}
\newline{\tiny{Material desarrollado bajo el proyecto UNAM-PAPIME PE102117.}}}
 

\beamerdefaultoverlayspecification{<+->}
 
\titlegraphic{\includegraphics[width=16mm]{fc2.png}}
 

\begin{document}

\begin{frame}
\titlepage 
\end{frame}
\note{}

\frame{\titulos{Especificación Formal}{}
Algunos consejos:
\bi
\item Nuestro objetivo es extraer predicados a partir de los enunciados dados 
  en español de manera que el enunciado completo se construya al combinar 
  dichos predicados mediante conectivos y cuantificadores.\\
\espc
{\em La Luna \textcolor<2-7>{blue}{brilla}}. \\
%\pause
\espc
{\em Daniela \textcolor<3-7>{blue}{compró} \textcolor<4-7>{purple}{una película}}. \\
%\pause
\espc
{\em Los leones \textcolor<5-7>{blue}{comen} \textcolor<6-7>{purple}{carne cruda}}. \\
%\pause
\espc
{\em Todos los días \textcolor<7>{blue}{están soleados}}.
\ei
}


\frame{\titulos{Especificación Formal}{}
Algunos consejos:
\bi

\item Si en el español aparecen frases como \textit{para todos, para 
cualquier,   todos, cualquiera,los, las etc.} debe usarse el {\bf cuantificador 
universal}.


 \item Si en el español frases como \textit{para alg\'un, existe un, 
  alguno,alguna, uno, una etc.} debe usarse el {\bf cuantificador 
  existencial.}\\ 

\item En ciertas ocasiones, frases en español que
  involucran \textit{alguien, algo} deben especificarse con un cuantificador 
  universal y no un existencial. \\ \pause \espc
  Por ejemplo, el enunciado \textit{si alguien demasiado alto entra por la puerta entonces se 
  pegará con el marco}, se puede reescribir en español como
  \textit{cualquiera demasiado alto que entre por la puerta se pegará con el marco },
  lo cual nos lleva a la f\'ormula  ~$\fa x(A(x) \land E(x)\imp P(x))$.\\
\ei
}


\frame{\titulos{Especificación Formal}{}
Algunos consejos:
\bi
\item Cualquier especificación compuesta que involucre cuantificadores puede
  formarse identificando en ella alguno de los cuatro juicios aristotélicos: \\
\espc
  \bi
   \item Universal afirmativo:  \textit{Todo S es P} \\
    \hspace{1cm}$\fa x(S(x)\imp P(x))$. \espc

   \item Existencial afirmativo. \textit{Alg\'un S es P} \\
\hspace{1cm}    $\ex x(S(x)\land P(x)).$ \espc

   \item Universal negativo: \textit{Ning\'un S es P} \\ 
\hspace{1cm}    $\fa x(S(x)\imp \neg P(x))$ \espc

   \item Existencial negativo:  \textit{Alg\'un S no es P} \\
\hspace{1cm}    $\ex x(S(x)\land \neg P(x))$ \espc

  \ei
\ei
}



\frame{\titulos{Especificación Formal}{}
Algunos consejos:
\bi
\item Las fórmulas~$\exists x P(x)$ y~$\exists x\exists y(P(x)\land P(y))$ expresan 
 lo mismo: un objeto del universo cumple~$P$.  \\
\pause 
\espc
Para indicar que x, y representan elementos diferentes, se debe agregar explícitamente la propiedad ~$x\ne y$.
\ei
}



\frame{\titulos{Especificación Formal}{}
Ejemplos:
\bi
\item {\em Todo \textcolor{purple}{día} que \textcolor{blue}{está soleado} no \textcolor{blue}{está nublado}}. 
\[ \forall x (D(x) \land S(x) \imp \neg N(x) )\]
\small{ Donde $D(x)=x \, es \, día$, $S(x)=x \, está \, soleado$ y $N(x)=x \, está \, nublado$.}
\item {\em Hay una \textcolor{purple}{lanza} que \textcolor{blue}{perfora} a todos los \textcolor{purple}{escudos}}. 
\[ \ex x (L(x) \land \forall y (E(y) \imp D(x,y)))\]
\small{ Donde $L(x)=x \, es \, lanza$, $E(x)=x \, es \, escudo$ y $P(x,y)=x \, perfora \, a \, y$.}
\item {\em Hay un \textcolor<2-5>{purple}{profesor} al que ningún \textcolor<3-5>{purple}{estudiante} le ha \textcolor<4-5>{blue}{preguntado} alguna \textcolor<5>{purple}{duda}}. 
\vspace{-0.3cm}
\[ \exists w (P(w) \land \neg \exists z (E(z) \land \exists u (D(u) \land A(z,u,w))))\]
\small{ Donde $P(x)=x \, es \, profesor$, $E(x)=x \, es \, estudiante$, $D(x)=x\,es\,duda$ y $A(x,y,z)=x \, le \, pregunta \, y \, a \, z$.}
\ei
}

\frame{\titulos{Semántica Informal}{}
\begin{tabular}{|c||p{4cm}|p{4cm}|}
\hline
\textbf{Fórmula }& \textbf{¿Cuándo es verdadera?} &
\textbf{¿Cuándo es Falsa?}  \\
\hline
\textcolor<1>{purple}{$\forall x  P(x)$}&\textcolor<1>{purple}{ $P(x)$ es verdadera para todo $x$ en el universo de discurso} &\textcolor<1>{purple}{ Existe un $x$ para el cual $P(x)$ es falsa.}  \\
\hline
\textcolor<2>{purple}{$\exists x P(x)$} &\textcolor<2>{purple}{Existe un $x$ para el cual $P(x)$ es verdadera.   }& \textcolor<2>{purple}{$P(x)$ es falsa para todo $x$ en el universo de discurso.} \\
\hline
\end{tabular}
}


\frame{\titulos{Semántica Informal}{}
\begin{tabular}{|c||p{4cm}|p{4cm}|}
\hline
\textbf{Fórmula }& \textbf{¿Cuándo es verdadera?} &
\textbf{¿Cuándo es Falsa?}  \\
\hline
\textcolor<1>{purple}{$\forall x \forall y  P(x,y)$}& \textcolor<1>{purple}{$P(x,y)$ es verdadera para cualquier par de elementos $x,y$.}&\textcolor<1>{purple}{ Existe un par $x,y$ para el cual $P(x,y)$ es falsa. } \\
\hline
\textcolor<2>{purple}{$\exists x  \exists y P(x,y)$} &\textcolor<2>{purple}{Existe un par $x,y$ para el cual $P(x,y)$ es verdadera.  } &\textcolor<2>{purple}{ $P(x,y)$ es falsa para todo par de elementos $x,y$. }\\
\hline
\textcolor<3>{purple}{$\forall x \exists y  P(x,y)$}&\textcolor<3>{purple}{ Para todo elemento $x$, podemos encontrar un $y$ tal que $P(x,y)$ es verdadera.} &\textcolor<3>{purple}{ Existe un $x$ tal que $P(x,y)$ es falsa para todo $y$.  }\\
\hline
\textcolor<4>{purple}{$\exists x  \forall y P(x,y)$} &\textcolor<4>{purple}{Existe un $x$ para el cual $P(x,y)$ es verdadera para cualquier $y$.  } &\textcolor<4>{purple}{ Para todo $x$ existe un $y$ tal que $P(x,y)$ es falsa.} \\
\hline
\end{tabular}
}

\frame{\titulos{Semántica Formal}{}
\begin{block}{{\bf Interpretación o estructura}}
Sea $\L=\Pe\cup\F\cup\C$ un lenguaje de primer orden. Una {\bf estructura o
  interpretación} para $\L$ es un par $\M=\pt{M,\I}$ donde $M\neq\varnothing$ es un conjunto
  no vacío llamado el universo de la estructura, e $\I$ es una función con
  dominio $\L$ tal que: \\
\bi
\item Si $P^{(n)}\in\Pe$ entonces $\I(P)$ es una función booleana que decide si una tupla está o no en la relación deseada, es decir, $\I(P): M^n \to Bool$. \\
\espc
\item Si $f^{(n)}\in\F$ entonces $\I(f)$ es una función $\I(f):M^n\imp M$. \espc
\item Si $c\in\C$ entonces $\I(c)$ es un elemento de $M$, es decir $\I(c)\in M$. \espc
\ei 

\end{block}
}


\frame{\titulos{Semántica Formal}{Términos}
\begin{block}{{\bf Estado o Asignación}}
 Un {\bf estado, asignación o valuación de las variables} es una función $\sigma:\mathsf{Var}\imp M$. \espc
\end{block}
\pause
\begin{block}{{\bf Estado modificado o actualizado}}
Sea $\sigma:\mathsf{Var}\imp M$ un estado de las variables. \pause Dadas las
  variables $x_1,\ldots,x_n$ y los elementos del universo $m_1,\ldots,m_n\in
  M$ definimos el {\bf estado modificado o actualizado} en $x_1,\ldots,x_n$ por
  $m_1,\ldots,m_n$ denotado $\sigma[x_1,\ldots,x_n/m_1,\ldots,m_n]$ o $\sigma[\vx/\vec{m}\;]$ como sigue: \pause
\beqs
\sigma[\vx/\vec{m}\;](y) =\left\{\ba{rl}
\sigma(y) & \mbox{si}\; y\notin\{x_1,\ldots,x_n\} \\ \\
m_i      & \mbox{si}\; y = x_i\;\;\;\;1\leq i\leq n 
\ea
\right.
\eeqs
\espc
\end{block}
}

\frame{\titulos{Semántica Formal}{Términos}
\begin{block}{{\bf Interpretación de Términos}}
Sea $\sigma$ un estado de las variables. Definimos la {\bf función de interpretación} con respecto a $\sigma$, $\I_\sigma:\term\imp |\M|$ como sigue:\pause
\beqs
\ba{lll}
\I_\sigma(x) &\; = \; & \sigma(x) \\ \\ 
\I_\sigma(c) &\; = \; & \I(c) \\ \\  %= c^\I \\ 
\I_\sigma\big(f(t_1,\ldots,t_n)\big)  & \; = \; &  
f^\I_\sigma(\I(t_1),\ldots,\I_\sigma(t_n))
\ea
\eeqs
\end{block}
}

\frame{\titulos{Semántica Formal}{Términos}
\begin{exampleblock}{{\bf Lema de coincidencia para términos}}
Sean $t\in\term$ y $\sigma_1,\sigma_2$ dos estados de las variables tales que $\sigma_1(x)=\sigma_2(x)$ para toda variable $x$
  que figura en $t$. Entonces \pause
 \[\I_{\sigma_1}(t)=\I_{\sigma_2}(t) .\]
\end{exampleblock}
}


\frame{\titulos{Semántica Formal}{Términos}
\begin{exampleblock}{{\bf Lema de sustitución para términos}}
Sean $r\in\term$, $\sigma$ un estado de   las variables, $[\vx:=\vt\;]$ una sustitución y $m_1,\ldots,m_n\in
  M$ tales que $\I_\sigma(t_i)=m_i\;\;1\leq i\leq n$. Entonces \pause
\beqs
\I_\sigma\big(r[\vx:=\vt\;]\big) = \I_{\sigma[\vx/\vec{m}\;]}(r)
\eeqs
\end{exampleblock}
}

\frame{\titulos{Semántica Formal}{Fórmulas}
\begin{block}{{\bf Interpretación de Fórmulas}}
Sea $\sigma$ un estado de las variables. Definimos la {\bf función de interpretación} sobre fórmulas con respecto a $\sigma$, $\I_\sigma:\form\imp\{0,1\}$ como sigue: \pause
\[
\ba{lcl}
\I_\sigma(\bot) \; = \;  0  & \I_\sigma(\top) \; = \;  1 & \\ \\ \pause
\I_\sigma\big(P(t_1,\ldots,t_m)\big)\; = \;1 & \;\mbox{si y sólo si}\; &
\big(\I_\sigma(t_1),\ldots,\I_\sigma(t_m)\big)\in P^\I \\ \\ \pause
\I_\sigma(t_1=t_2) \; = \;  1 & \;\mbox{si y sólo si}\; & \I_\sigma(t_1)=\I_\sigma(t_2) \\ \\ \pause
\I_\sigma(\neg\vp) \; = \; 1 & \;\mbox{si y sólo si}\; & \I_\sigma(\vp)=0 \\ \\ \pause
\I_\sigma(\vp\land\psi) \; = \;  1 & \;\mbox{si y sólo si}\; &
\I_\sigma(\vp)=\I_\sigma(\psi)=1 \\ \\ \pause
\I_\sigma(\vp\lor\psi) \; = \;  0 & \;\mbox{si y sólo si}\; &
\I_\sigma(\vp)=\I_\sigma(\psi)=0 \\ \\ 
\ea
\]
\end{block}
}


\frame{\titulos{Semántica Formal}{Fórmulas}
\begin{block}{{\bf Interpretación de Fórmulas}}
continuación: 
\[
\ba{lcl}
\I_\sigma(\vp\imp\psi) \; = \;  0 & \;\mbox{si y sólo si}\; &
\I_\sigma(\vp)=1\;\mbox{e}\;\I_\sigma(\psi)=0 \\ \\ \pause
\I_\sigma(\vp\iff\psi) \; = \;  1 & \;\mbox{si y sólo si}\; &
\I_\sigma(\vp)=\I_\sigma(\psi) \\ \\ \pause
\I_\sigma(\fa x\vp) \; = \;  1 & \;\mbox{si y sólo si}\; &
\I_{\sigma[x/m]}(\vp)=1\;\;\mbox{para todo}\;m\in M \\ \\ \pause
\I_\sigma(\ex x\vp) \; = \;  1 & \;\mbox{si y sólo si}\; &
\I_{\sigma[x/m]}(\vp)=1\;\;\mbox{para algún}\;m\in M \\ \\
\ea
\]

\end{block}
}

\frame{\titulos{Semántica Formal}{Fórmulas}
\begin{exampleblock}{{\bf Lema de coincidencia para fórmulas}}
Sean $\vp$ una fórmula y 
$\sigma_1,\sigma_2$ dos estados de las variables tales que 
$\sigma_1(x)=\sigma_2(x)$ para toda variable
$x\in FV(\vp)$. Entonces  \pause
 \[ \I_{\sigma_1}(\vp)=\I_{\sigma_2}(\vp). \]
 \espc
\end{exampleblock}
}


\frame{\titulos{Semántica Formal}{Fórmulas}
\begin{exampleblock}{{\bf Lema de sustitución para fórmulas}}
Sean $\vp\in\form$, $\sigma$ un estado de las variables, $[\vx:=\vt\;]$ una sustitución y $m_1,\ldots,m_n\in M$ 
  tales que $\I_\sigma(t_i) = m_i \;\;1\leq i\leq n$. Entonces \pause
\[
\I_\sigma\big(\vp[\vx:=\vt\;]\big) = \I_{\sigma[\vx/\vec{m}\;]}(\vp) 
\]
\end{exampleblock}
}

\frame{\titulos{Semántica Formal}{Fórmulas}
\begin{block}{{\bf Verdad y Satisfacibilidad}}
Sean $\vp$ una fórmula y $\M=\pt{M,\I}$ una 
interpretación. Entonces \\
\bi
\item $\vp$ es
  {\bf satisfacible en} $\mathbf{\M}$ si existe un
  estado de las variables $\sigma$ tal que $\I_\sigma(\vp)=1$, lo cual suele
  denotarse con $\M\models\vp[\sigma]$ o con $\M\models_\sigma\vp$.  
\item $\vp$ es {\bf verdadera
  en} $\M$ si para todo estado de las variables $\sigma$  se tiene $\I_\sigma(\vp)=1$,
  es decir, si $\vp$ es satisfacible en $\M$ en todos los estados posibles. \\
  \pause
  En tal caso también decimos que $\M$ es un {\bf modelo} de $\vp$ lo cual se
  denotará con $\M\models\vp$. \espc
\ei
\end{block}
}

\frame{\titulos{Semántica Formal}{Fórmulas}
\begin{block}{{\bf Falsedad}}
Sean $\M=\pt{M,\I}$ una interpretación y $\vp$ una fórmula. Decimos que 
$\vp$ es {\bf falsa en} $\M$ si y sólo si $\M\models\neg\vp$. \pause Es decir $\vp$ es falsa 
si y sólo si su negación $\neg\vp$ es verdadera.

\end{block}
}

\frame{\titulos{Semántica Formal}{Fórmulas}
En lógica proposicional las nociones de ser falsa y no ser verdadera coinciden. Sin embargo, en lógica de predicados la equivalencia se pierde. \\
\espc
Una fórmula no verdadera es  aquella tal que es insatisfacible en algún estado de sus variables, o bien tal que su negación es satisfacible en algún estado de sus variables. Sin embargo, para poder afirmar que  $\vp$ es falsa, por definición tendríamos que mostrar que $\M\models\neg\vp$, es decir que~$\neg\vp$ es satisfacible en \textbf{todos} los estados posibles. Por lo tanto la noción de falsedad es más fuerte que la noción de no ser verdadera.
}

\frame{\titulos{Semántica Formal}{Propiedades de la relación de verdad}
La relaci\'on de verdad en lógica de predicados no tiene las mismas propiedades que su contraparte en la lógica proposicional. \\
\espc
Por ejemplo, sean $\L=\{P^{(1)},Q^{(1)}\},\;\M=\langle \mathbb{N},\I\rangle$ 
donde $P^\I$ es la propiedad \enquote{ser par} y $Q^\I$ es la
propiedad \enquote{ser impar}. \\
Entonces $\M\models P(x)\lor Q(x)$ puesto que cualquier n\'umero natural es
par o es impar. \\
\espc
Sin embargo, no se cumple que $\M\models P(x)$ ni que $\M\models Q(x)$. Puesto 
que el valor de~$x$ no puede ser siempre par o siempre impar, todo depende del 
estado de las variables. 
}


\frame{\titulos{Semántica Formal}{Propiedades de la relación de verdad}
\begin{block}{{Propiedades de la relación de verdad}}
Sean $\M$ una $\L$-interpretaci\'on y $\vp,\psi$ f\'ormulas. Entonces
\bi
\item Si $\M\models\vp$ entonces $\M\not\models\neg\vp$. \espc
\item Si $\M\models\vp\imp\psi$ y $\M\models\vp$ entonces
  $\M\models\psi$. \espc
\item Si $\M\models\vp$ entonces $\M\models\vp\lor\psi$. \espc
  \item $\M\models\vp\land\psi$ si y sólo si $\M\models\vp$ y $\M\models\psi$. \espc
\item $\M\models\vp\;\mbox{si y sólo si}\;\M\models\fa\vp.$ donde
  $\fa\vp$ denota a la cerradura universal de $\vp$, es decir a la
  fórmula obtenida al cuantificar universalmente todas las variables
  libres de $\vp$.
  
\ei
\end{block}

}

\end{document}
