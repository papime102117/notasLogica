	%%% Presentaciones para Lenguajes de programacion y sus paradigmas 

%\documentclass[xcolor=dvipsnames,table,handout]{beamer}
\documentclass[xcolor=dvipsnames,table]{beamer}
\newcommand{\espc}{\vspace{0.3cm}}

\usepackage[utf8]{inputenc}
\usepackage[spanish]{babel}
\usepackage{hyperref}
\usepackage{lmodern}
\usepackage[T1]{fontenc}

%%%% paquetes matematicas
\usepackage{amssymb,amsmath,amscd}
\usepackage{extarrows}
\usepackage{stmaryrd}
\usepackage{mathabx}
\usepackage{mathrsfs}
% \usepackage{mathabx}
\usepackage{amsthm}

%%%%%
\usepackage{hyperref}
\usepackage{graphicx}
\usepackage{multicol}
\usepackage{pifont}
\usepackage{xcolor}
\usepackage{etex}
\usepackage{tikz}
\usepackage{array}
%\usepackage{pgfplots}

%%%% cosmetics
% D.Remy package for pretty display of rules
\usepackage{mathpartir}

% para insertar codigo con formato particular 
\usepackage{listing} 

% comillas 
\usepackage[autostyle=true,spanish=mexican]{csquotes}

% codigo 
\usepackage{verbatim}
\usepackage{alltt}

% footnotes
\usepackage[bottom]{footmisc}
\usepackage{setspace}

\usepackage{wrapfig}
\usepackage{caption}


\hfuzz=5.002pt %parameter to allow hbox overfulled by length before error!

% Options for presentation
% ------------------------
% \definecolor{mycolor}{RGB}{255,192,3}
\definecolor{mycolor}{RGB}{17,132,221}
\mode<presentation>
{
% \usetheme[secheader]{Boadilla}
% \usecolortheme{orchid}
\useoutertheme{infolines}
\useinnertheme{rectangles}
\setbeamertemplate{itemize items}[square]
\setbeamertemplate{enumerate items}[square]
\setbeamersize{text margin left=6mm, text margin right=6mm}

\setbeamercolor{alerted text}{fg=red,bg=red!70!white}
\setbeamercolor{background canvas}{bg=white}
\setbeamercolor{frametitle}{bg=mycolor,fg=white}
\setbeamercolor{normal text}{bg=white,fg=black}
\setbeamercolor{structure}{bg=black,fg=mycolor}
\setbeamercolor{title}{bg=mycolor,fg=white}
\setbeamercolor{subtitle}{bg=mycolor,fg=white}
\setbeamercolor{titlelike}{bg=white,fg=mycolor}

\setbeamercovered{invisible}

\setbeamercolor*{palette primary}{fg=mycolor,bg=white}
\setbeamercolor*{palette secondary}{bg=white,fg=white}
\setbeamercolor*{palette tertiary}{fg=mycolor,bg=white}
\setbeamercolor*{palette quaternary}{fg=white,bg=white}

\setbeamercolor{separation line}{bg=mycolor,fg=mycolor}
\setbeamercolor{fine separation line}{bg=white,fg=red}
\setbeamercolor{author in head/foot}{bg=mycolor!30!white,fg=mycolor!80!black}
\setbeamercolor{title in head/foot}{bg=mycolor!30!white,fg=mycolor!80!black}
\setbeamercolor{date in head/foot}{bg=mycolor!30!white,fg=mycolor!80!black}
\setbeamercolor{institute in head/foot}{bg=mycolor!30!white,fg=mycolor!80!black}
\setbeamercolor{section in head/foot}{bg=mycolor!60!white, fg=Red}
\setbeamercolor{subsection in head/foot}
{bg=mycolor!50!white,fg=mycolor!50!white}


\setbeamertemplate{headline}
{
  \leavevmode%
  \hbox{%
  \begin{beamercolorbox}[wd=.5\paperwidth,ht=2.65ex,dp=1.5ex,center]{section in 
head/foot}%
    \usebeamerfont{section in head/foot}\insertsectionhead\hspace*{2ex}
  \end{beamercolorbox}%
  \begin{beamercolorbox}[wd=.5\paperwidth,ht=2.65ex,dp=1.5ex,center]{subsection 
in head/foot}%
    \usebeamerfont{subsection in head/foot}\hspace*{2ex}\insertsubsectionhead
  \end{beamercolorbox}}%
  \vskip0pt%
}
% \beamerdefaultoverlayspecification{<+->}
\beamertemplatenavigationsymbolsempty
% \setbeamertemplate{footline}[frame number]
}

%%%% Macros para las notas de lenguajes de programacion


%%%% math

\newcommand{\vphi}{\varphi}
\newcommand{\vp}{\varphi}

% \newcommand{\dn}{\mathsf{DN}}
% \newcommand{\dnC}{\mathsf{DN_C}}
% \newcommand{\dnM}{\mathsf{DN_M}}
% \newcommand{\dnp}{\mathsf{DN_p}}
% \newcommand{\dnm}{\mathsf{DN_p^M}}
% \newcommand{\dnc}{\mathsf{DN_p^C}}

%\newcommand{\case}{\mathsf{case}}
%\renewcommand\labelitemi{$\circ$}

\newcommand{\imp}{\rightarrow}
\newcommand{\Imp}{\Rightarrow}
\renewcommand{\iff}{\leftrightarrow}
\newcommand{\Iff}{\Leftrightarrow}
\newcommand{\G}{\Gamma}
\newcommand{\D}{\Delta}


\newcommand{\De}{\mathcal{D}}
\newcommand{\F}{\mathcal{F}}
\newcommand{\Ge}{\mathcal{G}}
\newcommand{\Pe}{\mathcal{P}}
\newcommand{\I}{\mathcal{I}}
\newcommand{\C}{\mathcal{C}}
\newcommand{\K}{\mathcal{K}}
\renewcommand{\L}{\mathcal{L}}
\newcommand{\M}{\mathcal{M}}
\newcommand{\Nc}{\mathcal{N}}
%\newcommand{\E}{\mathcal{E}}
%\newcommand{\R}{\mathcal{R}}
%\newcommand{\Q}{\mathcal{Q}}
\newcommand{\Sc}{\mathcal{S}}
\newcommand{\Te}{\mathcal{T}}
\newcommand{\W}{\mathcal{W}}

\newcommand{\Db}{\mathbb{D}}
\newcommand{\Fb}{\mathbb{F}}
\newcommand{\Kb}{\mathbb{K}}
\newcommand{\Eb}{\mathbb{E}}
\newcommand{\Ebs}{\mathbb{E}^\star}
\newcommand{\Ob}{\mathbb{O}}
\newcommand{\Ib}{\mathbb{I}}
\newcommand{\Rb}{\mathbb{R}}
\newcommand{\Qb}{\mathbb{Q}}
\newcommand{\Kbb}{\mathbb{K}}
\newcommand{\T}{\mathbb{\Theta}}


\newcommand{\kb}{\bbkappa}

\newcommand{\Sf}{\mathsf{\Sigma}}

\newcommand{\fa}{\forall}
\newcommand{\ex}{\exists}

\newcommand{\inc}{\subseteq}

\newcommand{\Lb}{\Lambda}
\newcommand{\Om}{\Omega}
\newcommand{\lb}{\lambda}
\newcommand{\al}{\alpha}
\newcommand{\ga}{\gamma}


\newcommand{\mg}{\mathbb{m}}

\newcommand{\cg}{\mathbb{C}}
\newcommand{\dg}{\mathbb{D}}
\newcommand{\jg}{\mathbb{J}}
\newcommand{\Ha}{\mathcal{H}}
%\newcommand{\A}{\mathcal{A}}
\newcommand{\sg}{\mathbb{S}}

\newcommand{\Bc}{\mathcal{B}}
\newcommand{\Df}{\mathfrak{D}}
\newcommand{\Dc}{\mathcal{D}}
%\newcommand{\Tc}{\mathcal{T}}
\newcommand{\Mf}{\mathfrak{M}}

\newcommand{\Sg}{\mathbb{S}}

\newcommand{\Q}{\ensuremath{\mathbb{Q}}}
\newcommand{\Z}{\ensuremath{\mathbb{Z}}}
\newcommand{\N}{\ensuremath{\mathbb{N}}}
\newcommand{\R}{\ensuremath{\mathbb{R}}}
\renewcommand{\S}{\mathbb{\Sigma}}
\newcommand{\A}{\mathcal{A}}
\newcommand{\E}{\ensuremath{\exists}}
\newcommand{\iso}{\ensuremath{\cong}}
\newcommand{\union}{\ensuremath{\cup}}
\newcommand{\morinyec}{\ensuremath{\precapprox}}

\newcommand{\nin}{\ensuremath{\notin}}
\newcommand{\tog}{\makebox[7mm][l]}
\newcommand{\toge}{\makebox[11mm][l]}
\newcommand{\toget}{\makebox[13mm][l]}
\newcommand{\togeth}{\makebox[14mm][l]}
\newcommand{\togethe}{\makebox[15mm][l]}
\newcommand{\together}{\makebox[17mm][l]}
\newcommand{\niso}{\ensuremath{\not \cong}}


\newcommand{\Mg}{\mathbb{M}}
\newcommand{\Bg}{\mathbb{B}}
\newcommand{\Lg}{\mathbb{L}}
\newcommand{\Tg}{\mathbb{T}}

\newcommand{\sketch}{\Red{{\sc sketch}}}

\newcommand{\restr}[2]{#1\!\!\boldsymbol{\restriction}\!#2}

\newcommand{\vacio}{\varnothing}
\newcommand{\done}{\ensuremath{\checkmark}}

\newcommand{\ida}{$\Rightarrow \; )$ }
\newcommand{\regr}{$\Leftarrow \; )$ }

\newcommand{\ol}[1]{\overline{#1}}

\newcommand{\Tsf}{\mathsf{T}}

\newcommand{\inds}[1]{\index[simb]{#1}}

\newcommand{\B}{\mathbb{B}}
%\newcommand{\N}{\mathbb{N}}

\newcommand{\vx}{\vec{x}}
\newcommand{\vy}{\vec{y}}
\newcommand{\vz}{\vec{z}}
\newcommand{\vt}{\vec{t}}
\newcommand{\vf}{\vec{f}}


% \newcommand{\propo}{\ensuremath{\mathsf{PROP}}}
% \newcommand{\atom}{\ensuremath{\mathsf{ATOM}}}
\newcommand{\term}{\ensuremath{\mathsf{TERM}}}
\newcommand{\form}{\mathsf{FORM}}

\newcommand{\true}{\mathop{\mathsf{true}}}

%\newcommand{\id}{\mathsf{Id}}

%\newcommand{\uc}{\mathcal{U}}
%\newcommand{\Ic}{\mathcal{I}}
%\newcommand{\pc}{\mathcal{P}}
%\newcommand{\qc}{\mathcal{Q}}
%\newcommand{\mc}{\mathcal{M}}
\newcommand{\supc}{\supseteq}
\newcommand{\limo}{\mathop{\mathpzc{Lim}}}
\newcommand{\ord}{\mathsf{OR}}

\newcommand{\pt}[1]{\langle #1 \rangle}


%%%% frames
\newcommand{\titulos}[2]{\frametitle{#1}\framesubtitle{#2}}
\newcommand{\fot}[1]{\footnote{\scriptsize{#1}}}


%%%% ambientes

\newcommand{\cb}[2]{\colorbox{#1}{#2}}

\newcommand{\bc}{\begin{center}}
\newcommand{\ec}{\end{center}}
\newcommand{\be}{\begin{enumerate}}
\newcommand{\ee}{\end{enumerate}}
\newcommand{\bi}{\begin{itemize}}
\newcommand{\ei}{\end{itemize}}
\newcommand{\beq}{\begin{equation}}
\newcommand{\eeq}{\end{equation}}
\newcommand{\beqs}{\begin{equation*}}
\newcommand{\eeqs}{\end{equation*}}
\newcommand{\ba}{\begin{array}}
\newcommand{\ea}{\end{array}}


% \newtheorem{theorem}{Teorema}
% \newcommand{\teo}[1]{\begin{theorem} #1 \end{theorem}}
% \newtheorem{proposition}{Proposici\'on}
% \newcommand{\prop}[1]{\begin{proposition} #1 \end{proposition}}
% \newtheorem{definition}{Definici\'on}
% \newcommand{\defin}[1]{\begin{definition} #1 \end{definition}}
% \newtheorem{corollary}{Corolario}
% \newcommand{\cor}[1]{\begin{corollary} #1 \end{corollary}}
% \newtheorem{lemma}{Lema}
% \newcommand{\lema}[1]{\begin{lemma} #1 \end{lemma}}
% \newcommand{\dem}[1]{\begin{proof} #1 \end{proof}}

%\renewcommand{\qed}{\qedsymbol{$\mathbf{\dashv}$}}

%\newcommand{\proof}{\hfill\\\noindent\textbf{\textit{Demostraci\'on. }}}

\newcommand{\hint}{\emph{Sugerencia: }}


\newcounter{EjempCtr}[section]
\newenvironment{enumrom}{\renewcommand{\theenumi}{\roman{enumi}}%
\renewcommand{\theenumii}{\roman{enumii}}
\renewcommand{\theenumiii}{\roman{enumiii}}
\renewcommand{\theenumiv}{\roman{enumiv}}
\begin{enumerate}}{\end{enumerate}}
\newenvironment{Ejemplo}
        {\stepcounter{EjempCtr}%
        \begin{description}\item[Ejemplo \thesection.\arabic{EjempCtr}]}%
        {\end{description}}
\newenvironment{demostr}{%
             {\em Demostración:}
                \begin{quotation}}{\end{quotation}}

   \newcommand{\beje}{\begin{Ejemplo}}
\newcommand{\eeje}{\end{Ejemplo}}


\newtheorem{eje}{Ejemplo}[section]
\newcommand{\ejem}[1]{\begin{eje}\normalfont #1 \end{eje}}

% \renewcommand\contentsname{\'Indice}
%\renewcommand\chaptername{Cap\'itulo}
% \renewcommand\indexname{\'Indice}

%%\newcommand{\qed}{\hfill$\mathbb{Qed}$}
%\newcommand{\qed}{\hfill$\mathsf{\boldsymbol{\dashv}}$}
%\renewcommand{\qed}{\hfill$\boldsymbol{\dashv}$}


\newenvironment{prueba}{\vspace{-5mm}\noindent\textbf{Demostraci\'on}\\}{
\noindent$\blacksquare$\\}

\newcommand{\Ejercicios}{\section*{Ejercicios}}

 \newenvironment{manitas}{%
      \renewcommand{\labelitemi}{\ding{44}}%
      \vspace{-0.5cm}%
      \begin{itemize}%
      \setlength{\itemsep}{0pt}\setlength{\parsep}{0pt}\setlength{\topsep}{0pt}%
      }{\end{itemize}}
\newenvironment{malitos}{%
      \renewcommand{\labelitemi}%
            {\raisebox{1.5ex}{\makebox[0.3cm][l]{\begin{rotate}{-90}%
            \ding{43}\end{rotate}}}}%
      \vspace{-0.5cm}%
      \begin{itemize}%
      \setlength{\itemsep}{0pt}\setlength{\parsep}{0pt}\setlength{\topsep}{0pt}%
      }{\end{itemize}}
\newenvironment{ejercs}{
     \renewcommand{\labelenumi}{\thesection.\theenumi.-}
     \renewcommand{\labelenumii}{\theenumii)}
     \begin{enumerate}}
     {\end{enumerate}}

   \newcommand{\bej}{\begin{ejercs}}
\newcommand{\eej}{\end{ejercs}}


%\newenvironment{leterize}{%
%        \renewcommand{\theenumi}{\alph{enumi}}
%        \begin{enumerate}}{\end{enumerate}}

%\newenvironment{manitas}{%
%      \renewcommand{\labelitemi}{\ding{44}}%
%      \vspace{-0.5cm}%
%      \begin{itemize}%
%      
% \setlength{\itemsep}{0pt}\setlength{\parsep}{0pt}\setlength{\topsep}{0pt}%
%      }{\end{itemize}}



%%=============================================================================

\def\stackunder#1#2{\mathrel{\mathop{#2}\limits_{#1}}}


%%%% notas

\newcommand{\doubt}{\Red{{\LARGE {\sf ??}}}}

\newcommand{\coment}[1]{\hfill\\ \Big[{\bf Comentario Privado:} #1\Big]}
\newcommand{\preg}[1]{\hfill\\ \BrickRed{{\bf Pregunta:} #1}}
\newcommand{\conjet}[1]{\hfill\\ \OliveGreen{{\bf Conjecura:} #1}}

\newcommand{\pendiente}{\BrickRed{{\sc Pendiente}}}
\newcommand{\verifpendiente}{\BrickRed{{\sc Verificación pendiente}}}


%--------------------------------------------------------------------------

\DeclareMathAlphabet{\mathpzc}{OT1}{pzc}{m}{it}


\title[]{Lógica computacional}
\subtitle{Tema: Lógica de primer orden: Introducción y sintaxis. }
\author{ Pilar Selene Linares Ar\'evalo}
\institute[UNAM-FC]{Facultad de Ciencias\\ 
Universidad Nacional Aut\'onoma de M\'exico}
\date[]{ \footnotesize{marzo 2018}
\newline{\tiny{Material desarrollado bajo el proyecto UNAM-PAPIME PE102117.}}}
 

\beamerdefaultoverlayspecification{<+->}
 
\titlegraphic{\includegraphics[width=16mm]{fc2.png}}
 

\begin{document}

\begin{frame}
\titlepage 
\end{frame}
\note{}

\frame{\titulos{Sintaxis}{}
\begin{center}
\begin{tikzpicture}
\draw [blue] (2.3,0) circle [radius=2];
\node[blue] at (2.4,1) {$x_{1}, x_{2}, ... , x_{n}$};
\node[blue] at (2.4,0) {$c_{1}, c_{2}, ...,c_{n}$};
\node[blue] at (2.4,-1) {$f(t_{1}, t_{2}, ... , t_{n})$};
\node at (2.4,-2.5) {TÉRMINOS};
\draw [purple] (6.7,0) circle [radius=2];
\node[purple] at (6.6,1.1) {$\top, \bot$};
\node[purple] at (6.6,0.6) {$P(t_{1}, ... , t_{n})$};
\node[purple] at (6.6,-0.3) {$\neg \varphi \, , \, \varphi \land \psi \, , \, \varphi \lor \psi$};
\node[purple] at (6.6,-0.8) {$ \, \varphi \imp \psi \, , \, \varphi \iff \psi$};
\node[purple] at (6.6,-1.2) {$ \, \exists x \varphi  \, , \, \forall y \varphi $};
\node at (6.4,-2.5) {FÓRMULAS};
\draw [magenta] (5.4,0.3) rectangle (7.9,1.4);
\end{tikzpicture}
\end{center}
}


\frame{\titulos{Sintaxis}{}

\begin{block}{{\bf Términos}}
Los {\bf términos} del lenguaje son aquellas expresiones que representan objetos, elementos o individuos en el universo del discurso y se generan con la siguiente gramática:
$$ t ::= x \mid c \mid f(t_1,\ldots,t_m) $$
\espc
Al conjunto de términos en lenguaje de la Lógica de Primer Oren lo denotaremos con $\term$. \\
\espc
\end{block}

}



\frame{\titulos{Sintaxis}{}
\begin{block}{{\bf Fórmulas}}
El conjunto de expresiones atómicas se denotará con $\mathsf{ATOM}$ y está formado por: \\
\bi
\item Las constantes lógicas $\bot,\top$.
\item Las expresiones de la forma: $P_1(t_1,\ldots,t_n)$ donde
  $t_1,\ldots,t_n$ son términos. 
\item Las expresiones de la forma $t_1=t_2$, si el lenguaje cuenta con igualdad. \\
\ei
\pause
Los términos se construyen bajo la siguiente gramática:
\[
\begin{array}{rcl}
  ATOM & ::= & \bot \mid \top \mid P(t_1,\ldots,t_m) \mid t_1=t_2
\end{array}
\]
\espc
\end{block}
}

\frame{\titulos{Sintaxis}{}
\begin{block}{{\bf Fórmulas}}
El conjunto $\form$ de fórmulas compuestas, llamadas usualmente {\em fórmulas}, se define recursivamente como sigue: 
\bi
\item Si $\vp\in\mathsf{ATOM}$ entonces $\vp\in\form$. Es decir, 
toda fórmula atómica es una fórmula.
\item Si $\vp\in\form$ entonces $(\neg\vp)\in\form$.
\item Si $\vp,\psi\in\form$ entonces 
$(\vp\land\psi),(\vp\lor\psi),(\vp\imp\psi),(\vp\iff\psi)\in\form$.
\item Si $\vp\in\form$ y $x\in\mathsf{Var}$ entonces 
$(\fa x\vp),(\exists x\vp)\in\form$.
\ei
\espc
\pause
La gramática correspondiente es:
\[
\begin{array}{rcl}
  F&::=&ATOM \mid (\neg F) \mid (F\, \star \, F)  \mid (\fa x \, F) \mid (\ex x \, F) \\
\star & ::= & \land \mid \lor \mid \imp \mid \iff 
\end{array}
\]
\espc
\end{block}
}


\frame{\titulos{Sintaxis}{}
{\bf Convención: }
 Los cuantificadores se aplican a la \textit{mínima} expresión sintácticamente posible delante del cuantificador. \\
\espc
\pause
De manera que
\[
\ba{rll}
\fa x\vp\imp \psi  & \text{ es} &  (\fa x\vp)\imp \psi \\
\pause
\espc
\ex y\vp \land \fa w\psi\imp\chi& \text{ es} & (\ex y\vp)\land(\fa 
w\psi)\imp\chi\\
\ea
\]
}

\frame{\titulos{Inducción y recursión}{}
\begin{block}{{\bf Definición recursiva  de funciones sobre términos}}
Para definir una funci\'on $h:\term \imp A$, basta definir $h$ como sigue:
\bi
\item Definir $h(x)$ para $x\in\mathsf{Var}$.
\espc
\item Definir $h(c)$ para cada constante $c\in\C$.
\espc
\item Suponiendo que $h(t_1),\ldots,h(t_n)$ est\'an definidas, definir \\
$h(f(t_1,\ldots,\allowbreak t_n))$ para cada s\'{\i}mbolo de funci\'on $f\in\F$ 
de índice $n$.
\ei
\espc
\end{block}


}


\frame{\titulos{Inducción y recursión}{}
\begin{block}{{\bf Definición recursiva de funciones sobre fórmulas}}
Para definir una funci\'on~$h:\form\imp A$, basta definir~$h$ como sigue:
\bi
\item Definir $h$ para cada f\'ormula at\'omica, es decir, definir $h(\bot), 
h(\top)$, $h(P(t_1,\ldots,t_n))$ y $h(t_1=t_2)$ si el lenguaje tiene igualdad.
\espc
\item Suponiendo definidas $h(\vp)$ y $h(\psi)$, definir a partir de
ellas a\\
$h(\neg\vp),\;h(\vp\lor\psi),\;h(\vp\land\psi),\;h(\vp\imp\psi),
\;h(\vp\iff\psi) 
,\;h(\fa x\vp)$ y $h(\ex x\vp)$. 
\ei
\espc
\end{block}

\espc 
}


\frame{\titulos{Inducción y recursión}{}
\begin{block}{{\bf Principio de inducción estructural para términos}}
Sea $\mathcal{P}$ una propiedad acerca de t\'erminos. Para demostrar que $\mathcal{P}$ es
v\'alida para todos los t\'erminos, basta seguir los siguientes pasos:
\bi
\item Caso base: mostrar que
\bi
\item $\mathcal{P}$ es v\'alida para $x$, con $x\in\mathsf{Var}$ .
\item $\mathcal{P}$ es v\'alida para $c$, con $c\in\mathcal{C}$.
\ei
\espc
\item Hipótesis de inducción: suponer $\mathcal{P}$ para cualesquiera 
$t_1,\ldots,t_n\in\term$.
\espc
\item Paso inductivo: usando la Hipótesis de inducción mostrar que 
\bi
\item  $f(t_1,\ldots,t_n)$ cumple $\mathcal{P}$, donde $f\in\mathcal{F}$ es un símbolo de función de índice $n$.
\ei
\espc
\ei

\end{block}
}


\frame{\titulos{Inducción y recursión}{}
\begin{block}{{\bf Principio de inducción estructural para fórmulas}}
Sea $\mathcal{P}$ una propiedad acerca de fórmulas. Para probar que toda fórmula $
\vp\in\form$ tiene la propiedad $\mathcal{P}$ basta seguir los siguientes pasos:
\espc
\bi
\item Caso base: mostrar que toda fórmula atómica tiene la propiedad $\mathcal{P}$. \\


\item Hipótesis de inducción: suponer que $\vp$ y $\psi$ cumplen $\mathcal{P}$.
\espc
\item Paso inductivo: mostrar usando la Hipótesis de inducci\'on que
\be
\item $(\neg\vp)$ también cumple $\mathcal{P}$.
\item  $(\vp\star\psi)$ tiene la propiedad $\mathcal{P}$, donde 
$\star\in\{\imp,\land,\lor,\iff\}$
\item $\fa x\vp$ y $\ex x\vp$ cumplen $\mathcal{P}$.
\espc
\ee
\ei

\end{block}
}



\frame{\titulos{Ligado y alcance}{}
\begin{block}{{\bf Ligado y alcance}}
Dada una cuantificación $\fa x \varphi$ o $\ex x \varphi$, la presencia de $x$ en $\forall x$ o $\ex x$ es la variable que {\em liga} el cuantificador correspondiente; mientras que la fórmula $\varphi$ es el {\bf alcance}, ámbito o radio del cuantificador.  \\
\pause
\espc
Una presencia de la variable $x$ en la fórmula $\varphi$ está {\bf ligada} si figura en el alcance de un cuantificador y éste es el más cercano a $x$. \\
\espc
Si una presencia de la variable $x$ no es ligada, decimos que es {\bf libre}.
\end{block}
}

\frame{\titulos{Sustitución}{}
\begin{block}{{\bf Sustitución sobre términos}}
La aplicación de una sustitución $[\vec{x}:=\vec{t}]$ a un término $r$,
denotada $r[\vec{x}:=t]$, se define como el término obtenido al reemplazar 
\textbf{simult\'aneamente} todas las presencias de $x_i$ en $r$ por $t_i$. 
Este proceso de define  recursivamente como sigue:
\pause
\[
\ba{rll}
x_i[\vec{x}:=\vec{t}\,] & = & t_i \qquad  1\leq i\leq n \\ \\ 
\pause
z[\vec{x}:=\vec{t}\,] & = & z \qquad  \text{ si } z\neq x_i\;1\leq i\leq n  \\ \\
\pause
c[\vec{x}:=\vec{t}\,] & = & c \qquad 
  \text{ si } c\in\C\text{, es decir, }c\text{ constante } \\ \\
\pause
f(t_1,\ldots,t_m)[\vec{x}:=\vec{t}\,] & = & 
f(t_1[\vec{x}:=\vec{t}\,],\ldots,t_m[\vec{x}:=\vec{t}\,]) \quad \text{ con }f^{(m)}\in\F.
\ea
\espc
\]
\end{block}
}

\frame{\titulos{Sustitución}{}
Debido a la presencia de variables libres y ligadas, la aplicación de una {\em sustitución textual} a una fórmula puede llevar a situaciones problemáticas, por ejemplo: 
\pause
\espc
\bi
\item Generar expresiones que no son fórmulas:  \[
  \big(\fa x P(y,fx)\big)[x,y:=gy,z] = \fa gy P\big(z,fgy)\big)
  \]
La expresión de la derecha no es una fórmula. 
\ei
}

\frame{\titulos{Sustitución}{}
\bi
\item Captura de variables: 
Consideremos la siguiente aplicación de sustitución
\[ 
\ba{rcl} 
\fa x\big(\ex   y(\,x\neq y)\big) [x:=y]  & -> & \\ \pause
&  -> &   \ex y\big(\,(x\neq y)\big) [x:=y]  \\ \pause
& -> &  \,(x\neq y)[x:=y]  \\ \pause
& -> &  y\neq y  \\ \pause 
& & \\ 
\textcolor{red}{ \fa x\big(\ex   y(\,x\neq y)\big) [x:=y] } & \textcolor{red}{=} & \textcolor{red}{\fa x\big(\ex   y(\,y\neq y)\big) }\\
\ea\]
\ei
\pause
Para solucionar lo anterior, vamos a preferir un método utilizado en  teoría de 
  lenguajes de programación: \textit{la aplicación de una sustitución a una 
  fórmula se  define renombrando variables ligadas de manera que siempre 
  podremos obtener  una sustitución admisible.}
}

\frame{\titulos{Sustitución}{}
\begin{block}{{\bf Sustitución sobre fórmulas}}
La aplicación de una sustitución a una fórmula $\vp[\vec{x}:=\vec{t}\,]$ se define  
\textbf{recursivamente} como sigue: \pause
\[
 \ba{rll}
  \bot[\vec{x}:=\vec{t}\;] & = & \bot \\ 
  \top[\vec{x}:=\vec{t}\;] & = & \top \\ \pause
  P(t_1,\ldots,t_m)[\vec{x}:=\vec{t}\;] & = &
    P\big(t_1[\vec{x}:=\vec{t}\;], \ldots, t_m[\vec{x}:=\vec{t}\;]\big) \\ \pause
  (t_1 = t_2)[\vec{x}:=\vec{t}\;] & = & t_1[\vec{x}:=\vec{t}\;]=t_2[\vec{x}:=\vec{t}\;]\\ \\ \pause
  (\neg \vp)[\vec{x}:=\vec{t}\;] & = & \neg \big(\vp[\vec{x}:=\vec{t}\;]\big)\\ \pause
  (\vp\star \psi)[\vec{x}:=\vec{t}\;] & = &
  \big(\vp[\vec{x}:=\vec{t}\;]\star\psi[\vec{x}:=\vec{t}\;]\big)\\ \\ \pause
  (\fa y\vp)[\vec{x}:=\vec{t}\;] & = & \fa y\big(\vp[\vec{x}:=\vec{t}\;]\big)\; 
\text{ \textbf{si} } y\notin
  \vec{x}\cup Var(\vec{t}) \\
(\ex y\vp)[\vec{x}:=\vec{t}\;] & = & \ex y\big(\vp[\vec{x}:=\vec{t}\;]\big)\; \text{ 
\textbf{si} } y\notin  \vec{x}\cup Var(\vec{t}) \\ \\
\ea
\]


\end{block}
}


\frame{\titulos{Sustitución}{}
La definición de sustitución en fórmulas cuenta con una restricción aparente
en el caso de los cuantificadores, por ejemplo, la sustitución
\[
\fa x (Q(x)\to R(z,x))[z:=f(x)]
\] 
\pause
no está definida, puesto que~$x$ figura en~$f(x)$ es decir~$x\in Var(f(x))$, 
con lo que no se cumple la condición necesaria para aplicar la sustitución. \\
\pause
\espc
\espc
Esta restricción desaparece al notar que los nombres de las 
variables ligadas no importan: por ejemplo, las fórmulas $\fa x P(x)$ y 
$\fa y P(y)$ significan exactamente lo mismo.
}


\frame{\titulos{Sustitución}{}
Por lo tanto, convenimos en identificar fórmulas que sólo difieren en sus
variables ligadas, esto se hace formalmente mediante la llamada relación de
$\alpha$-equivalencia definida como sigue: \\
\pause
\espc
\begin{block}{{\bf Alfa Equivalencia}}
Decimos que dos fórmulas~$\vp_1,\;\vp_2$ son $\alpha$-equivalentes lo cual
escribimos $\vp_1\sim_\alpha \vp_2$ si y sólo si $\vp_1$ y $\vp_2$ difieren unicamente en los nombres de sus variables ligadas.
\end{block}
}

\frame{\titulos{Sustitución}{}
Las siguientes expresiones son $\alpha$-equivalentes. \\
\pause
\espc
$\fa x P(x,y)\to \ex y R(x,y,z)\;\; \sim_\alpha\;\; $  \pause 
$ \fa w P(w,y)\to \ex v R(x,v,z) \;\;  \;\; $  \\ \pause \espc 
$ \fa w P(w,y)\to \ex v R(x,v,z) \;\; \sim_\alpha \fa z P(z,y)\to \ex u R(x,u,z) $
}




\end{document}
