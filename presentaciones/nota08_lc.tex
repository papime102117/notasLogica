%%% Presentaciones para Lenguajes de programacion y sus paradigmas 

%\documentclass[xcolor=dvipsnames,table,handout]{beamer}
\documentclass[xcolor=dvipsnames,table]{beamer}

\newcommand{\espc}{\vspace{0.3cm}}

\usepackage[utf8]{inputenc}
\usepackage[spanish]{babel}
\usepackage{hyperref}
\usepackage{lmodern}
\usepackage[T1]{fontenc}

%%%% paquetes matematicas
\usepackage{amssymb,amsmath,amscd}
\usepackage{extarrows}
\usepackage{stmaryrd}
\usepackage{mathabx}
\usepackage{mathrsfs}
% \usepackage{mathabx}
\usepackage{amsthm}

%%%%%
\usepackage{hyperref}
\usepackage{graphicx}
\usepackage{multicol}
\usepackage{pifont}
\usepackage{xcolor}
\usepackage{etex}
\usepackage{tikz}
\usepackage{array}
%\usepackage{pgfplots}

%%%% cosmetics
% D.Remy package for pretty display of rules
\usepackage{mathpartir}

% para insertar codigo con formato particular 
\usepackage{listing} 

% comillas 
\usepackage[autostyle=true,spanish=mexican]{csquotes}

% codigo 
\usepackage{verbatim}
\usepackage{alltt}

% footnotes
\usepackage[bottom]{footmisc}
\usepackage{setspace}

\usepackage{wrapfig}
\usepackage{caption}


\hfuzz=5.002pt %parameter to allow hbox overfulled by length before error!

% Options for presentation
% ------------------------
% \definecolor{mycolor}{RGB}{255,192,3}
\definecolor{mycolor}{RGB}{17,132,221}
\mode<presentation>
{
% \usetheme[secheader]{Boadilla}
% \usecolortheme{orchid}
\useoutertheme{infolines}
\useinnertheme{rectangles}
\setbeamertemplate{itemize items}[square]
\setbeamertemplate{enumerate items}[square]
\setbeamersize{text margin left=6mm, text margin right=6mm}

\setbeamercolor{alerted text}{fg=red,bg=red!70!white}
\setbeamercolor{background canvas}{bg=white}
\setbeamercolor{frametitle}{bg=mycolor,fg=white}
\setbeamercolor{normal text}{bg=white,fg=black}
\setbeamercolor{structure}{bg=black,fg=mycolor}
\setbeamercolor{title}{bg=mycolor,fg=white}
\setbeamercolor{subtitle}{bg=mycolor,fg=white}
\setbeamercolor{titlelike}{bg=white,fg=mycolor}

\setbeamercovered{invisible}

\setbeamercolor*{palette primary}{fg=mycolor,bg=white}
\setbeamercolor*{palette secondary}{bg=white,fg=white}
\setbeamercolor*{palette tertiary}{fg=mycolor,bg=white}
\setbeamercolor*{palette quaternary}{fg=white,bg=white}

\setbeamercolor{separation line}{bg=mycolor,fg=mycolor}
\setbeamercolor{fine separation line}{bg=white,fg=red}
\setbeamercolor{author in head/foot}{bg=mycolor!30!white,fg=mycolor!80!black}
\setbeamercolor{title in head/foot}{bg=mycolor!30!white,fg=mycolor!80!black}
\setbeamercolor{date in head/foot}{bg=mycolor!30!white,fg=mycolor!80!black}
\setbeamercolor{institute in head/foot}{bg=mycolor!30!white,fg=mycolor!80!black}
\setbeamercolor{section in head/foot}{bg=mycolor!60!white, fg=Red}
\setbeamercolor{subsection in head/foot}
{bg=mycolor!50!white,fg=mycolor!50!white}


\setbeamertemplate{headline}
{
  \leavevmode%
  \hbox{%
  \begin{beamercolorbox}[wd=.5\paperwidth,ht=2.65ex,dp=1.5ex,center]{section in 
head/foot}%
    \usebeamerfont{section in head/foot}\insertsectionhead\hspace*{2ex}
  \end{beamercolorbox}%
  \begin{beamercolorbox}[wd=.5\paperwidth,ht=2.65ex,dp=1.5ex,center]{subsection 
in head/foot}%
    \usebeamerfont{subsection in head/foot}\hspace*{2ex}\insertsubsectionhead
  \end{beamercolorbox}}%
  \vskip0pt%
}
% \beamerdefaultoverlayspecification{<+->}
\beamertemplatenavigationsymbolsempty
% \setbeamertemplate{footline}[frame number]
}

%%%% Macros para las notas de lenguajes de programacion


%%%% math

\newcommand{\vphi}{\varphi}
\newcommand{\vp}{\varphi}

% \newcommand{\dn}{\mathsf{DN}}
% \newcommand{\dnC}{\mathsf{DN_C}}
% \newcommand{\dnM}{\mathsf{DN_M}}
% \newcommand{\dnp}{\mathsf{DN_p}}
% \newcommand{\dnm}{\mathsf{DN_p^M}}
% \newcommand{\dnc}{\mathsf{DN_p^C}}

%\newcommand{\case}{\mathsf{case}}
%\renewcommand\labelitemi{$\circ$}

\newcommand{\imp}{\rightarrow}
\newcommand{\Imp}{\Rightarrow}
\renewcommand{\iff}{\leftrightarrow}
\newcommand{\Iff}{\Leftrightarrow}
\newcommand{\G}{\Gamma}
\newcommand{\D}{\Delta}


\newcommand{\De}{\mathcal{D}}
\newcommand{\F}{\mathcal{F}}
\newcommand{\Ge}{\mathcal{G}}
\newcommand{\Pe}{\mathcal{P}}
\newcommand{\I}{\mathcal{I}}
\newcommand{\C}{\mathcal{C}}
\newcommand{\K}{\mathcal{K}}
\renewcommand{\L}{\mathcal{L}}
\newcommand{\M}{\mathcal{M}}
\newcommand{\Nc}{\mathcal{N}}
%\newcommand{\E}{\mathcal{E}}
%\newcommand{\R}{\mathcal{R}}
%\newcommand{\Q}{\mathcal{Q}}
\newcommand{\Sc}{\mathcal{S}}
\newcommand{\Te}{\mathcal{T}}
\newcommand{\W}{\mathcal{W}}

\newcommand{\Db}{\mathbb{D}}
\newcommand{\Fb}{\mathbb{F}}
\newcommand{\Kb}{\mathbb{K}}
\newcommand{\Eb}{\mathbb{E}}
\newcommand{\Ebs}{\mathbb{E}^\star}
\newcommand{\Ob}{\mathbb{O}}
\newcommand{\Ib}{\mathbb{I}}
\newcommand{\Rb}{\mathbb{R}}
\newcommand{\Qb}{\mathbb{Q}}
\newcommand{\Kbb}{\mathbb{K}}
\newcommand{\T}{\mathbb{\Theta}}


\newcommand{\kb}{\bbkappa}

\newcommand{\Sf}{\mathsf{\Sigma}}

\newcommand{\fa}{\forall}
\newcommand{\ex}{\exists}

\newcommand{\inc}{\subseteq}

\newcommand{\Lb}{\Lambda}
\newcommand{\Om}{\Omega}
\newcommand{\lb}{\lambda}
\newcommand{\al}{\alpha}
\newcommand{\ga}{\gamma}


\newcommand{\mg}{\mathbb{m}}

\newcommand{\cg}{\mathbb{C}}
\newcommand{\dg}{\mathbb{D}}
\newcommand{\jg}{\mathbb{J}}
\newcommand{\Ha}{\mathcal{H}}
%\newcommand{\A}{\mathcal{A}}
\newcommand{\sg}{\mathbb{S}}

\newcommand{\Bc}{\mathcal{B}}
\newcommand{\Df}{\mathfrak{D}}
\newcommand{\Dc}{\mathcal{D}}
%\newcommand{\Tc}{\mathcal{T}}
\newcommand{\Mf}{\mathfrak{M}}

\newcommand{\Sg}{\mathbb{S}}

\newcommand{\Q}{\ensuremath{\mathbb{Q}}}
\newcommand{\Z}{\ensuremath{\mathbb{Z}}}
\newcommand{\N}{\ensuremath{\mathbb{N}}}
\newcommand{\R}{\ensuremath{\mathbb{R}}}
\renewcommand{\S}{\mathbb{\Sigma}}
\newcommand{\A}{\mathcal{A}}
\newcommand{\E}{\ensuremath{\exists}}
\newcommand{\iso}{\ensuremath{\cong}}
\newcommand{\union}{\ensuremath{\cup}}
\newcommand{\morinyec}{\ensuremath{\precapprox}}

\newcommand{\nin}{\ensuremath{\notin}}
\newcommand{\tog}{\makebox[7mm][l]}
\newcommand{\toge}{\makebox[11mm][l]}
\newcommand{\toget}{\makebox[13mm][l]}
\newcommand{\togeth}{\makebox[14mm][l]}
\newcommand{\togethe}{\makebox[15mm][l]}
\newcommand{\together}{\makebox[17mm][l]}
\newcommand{\niso}{\ensuremath{\not \cong}}


\newcommand{\Mg}{\mathbb{M}}
\newcommand{\Bg}{\mathbb{B}}
\newcommand{\Lg}{\mathbb{L}}
\newcommand{\Tg}{\mathbb{T}}

\newcommand{\sketch}{\Red{{\sc sketch}}}

\newcommand{\restr}[2]{#1\!\!\boldsymbol{\restriction}\!#2}

\newcommand{\vacio}{\varnothing}
\newcommand{\done}{\ensuremath{\checkmark}}

\newcommand{\ida}{$\Rightarrow \; )$ }
\newcommand{\regr}{$\Leftarrow \; )$ }

\newcommand{\ol}[1]{\overline{#1}}

\newcommand{\Tsf}{\mathsf{T}}

\newcommand{\inds}[1]{\index[simb]{#1}}

\newcommand{\B}{\mathbb{B}}
%\newcommand{\N}{\mathbb{N}}

\newcommand{\vx}{\vec{x}}
\newcommand{\vy}{\vec{y}}
\newcommand{\vz}{\vec{z}}
\newcommand{\vt}{\vec{t}}
\newcommand{\vf}{\vec{f}}


% \newcommand{\propo}{\ensuremath{\mathsf{PROP}}}
% \newcommand{\atom}{\ensuremath{\mathsf{ATOM}}}
\newcommand{\term}{\ensuremath{\mathsf{TERM}}}
\newcommand{\form}{\mathsf{FORM}}

\newcommand{\true}{\mathop{\mathsf{true}}}

%\newcommand{\id}{\mathsf{Id}}

%\newcommand{\uc}{\mathcal{U}}
%\newcommand{\Ic}{\mathcal{I}}
%\newcommand{\pc}{\mathcal{P}}
%\newcommand{\qc}{\mathcal{Q}}
%\newcommand{\mc}{\mathcal{M}}
\newcommand{\supc}{\supseteq}
\newcommand{\limo}{\mathop{\mathpzc{Lim}}}
\newcommand{\ord}{\mathsf{OR}}

\newcommand{\pt}[1]{\langle #1 \rangle}


%%%% frames
\newcommand{\titulos}[2]{\frametitle{#1}\framesubtitle{#2}}
\newcommand{\fot}[1]{\footnote{\scriptsize{#1}}}


%%%% ambientes

\newcommand{\cb}[2]{\colorbox{#1}{#2}}

\newcommand{\bc}{\begin{center}}
\newcommand{\ec}{\end{center}}
\newcommand{\be}{\begin{enumerate}}
\newcommand{\ee}{\end{enumerate}}
\newcommand{\bi}{\begin{itemize}}
\newcommand{\ei}{\end{itemize}}
\newcommand{\beq}{\begin{equation}}
\newcommand{\eeq}{\end{equation}}
\newcommand{\beqs}{\begin{equation*}}
\newcommand{\eeqs}{\end{equation*}}
\newcommand{\ba}{\begin{array}}
\newcommand{\ea}{\end{array}}


% \newtheorem{theorem}{Teorema}
% \newcommand{\teo}[1]{\begin{theorem} #1 \end{theorem}}
% \newtheorem{proposition}{Proposici\'on}
% \newcommand{\prop}[1]{\begin{proposition} #1 \end{proposition}}
% \newtheorem{definition}{Definici\'on}
% \newcommand{\defin}[1]{\begin{definition} #1 \end{definition}}
% \newtheorem{corollary}{Corolario}
% \newcommand{\cor}[1]{\begin{corollary} #1 \end{corollary}}
% \newtheorem{lemma}{Lema}
% \newcommand{\lema}[1]{\begin{lemma} #1 \end{lemma}}
% \newcommand{\dem}[1]{\begin{proof} #1 \end{proof}}

%\renewcommand{\qed}{\qedsymbol{$\mathbf{\dashv}$}}

%\newcommand{\proof}{\hfill\\\noindent\textbf{\textit{Demostraci\'on. }}}

\newcommand{\hint}{\emph{Sugerencia: }}


\newcounter{EjempCtr}[section]
\newenvironment{enumrom}{\renewcommand{\theenumi}{\roman{enumi}}%
\renewcommand{\theenumii}{\roman{enumii}}
\renewcommand{\theenumiii}{\roman{enumiii}}
\renewcommand{\theenumiv}{\roman{enumiv}}
\begin{enumerate}}{\end{enumerate}}
\newenvironment{Ejemplo}
        {\stepcounter{EjempCtr}%
        \begin{description}\item[Ejemplo \thesection.\arabic{EjempCtr}]}%
        {\end{description}}
\newenvironment{demostr}{%
             {\em Demostración:}
                \begin{quotation}}{\end{quotation}}

   \newcommand{\beje}{\begin{Ejemplo}}
\newcommand{\eeje}{\end{Ejemplo}}


\newtheorem{eje}{Ejemplo}[section]
\newcommand{\ejem}[1]{\begin{eje}\normalfont #1 \end{eje}}

% \renewcommand\contentsname{\'Indice}
%\renewcommand\chaptername{Cap\'itulo}
% \renewcommand\indexname{\'Indice}

%%\newcommand{\qed}{\hfill$\mathbb{Qed}$}
%\newcommand{\qed}{\hfill$\mathsf{\boldsymbol{\dashv}}$}
%\renewcommand{\qed}{\hfill$\boldsymbol{\dashv}$}


\newenvironment{prueba}{\vspace{-5mm}\noindent\textbf{Demostraci\'on}\\}{
\noindent$\blacksquare$\\}

\newcommand{\Ejercicios}{\section*{Ejercicios}}

 \newenvironment{manitas}{%
      \renewcommand{\labelitemi}{\ding{44}}%
      \vspace{-0.5cm}%
      \begin{itemize}%
      \setlength{\itemsep}{0pt}\setlength{\parsep}{0pt}\setlength{\topsep}{0pt}%
      }{\end{itemize}}
\newenvironment{malitos}{%
      \renewcommand{\labelitemi}%
            {\raisebox{1.5ex}{\makebox[0.3cm][l]{\begin{rotate}{-90}%
            \ding{43}\end{rotate}}}}%
      \vspace{-0.5cm}%
      \begin{itemize}%
      \setlength{\itemsep}{0pt}\setlength{\parsep}{0pt}\setlength{\topsep}{0pt}%
      }{\end{itemize}}
\newenvironment{ejercs}{
     \renewcommand{\labelenumi}{\thesection.\theenumi.-}
     \renewcommand{\labelenumii}{\theenumii)}
     \begin{enumerate}}
     {\end{enumerate}}

   \newcommand{\bej}{\begin{ejercs}}
\newcommand{\eej}{\end{ejercs}}


%\newenvironment{leterize}{%
%        \renewcommand{\theenumi}{\alph{enumi}}
%        \begin{enumerate}}{\end{enumerate}}

%\newenvironment{manitas}{%
%      \renewcommand{\labelitemi}{\ding{44}}%
%      \vspace{-0.5cm}%
%      \begin{itemize}%
%      
% \setlength{\itemsep}{0pt}\setlength{\parsep}{0pt}\setlength{\topsep}{0pt}%
%      }{\end{itemize}}



%%=============================================================================

\def\stackunder#1#2{\mathrel{\mathop{#2}\limits_{#1}}}


%%%% notas

\newcommand{\doubt}{\Red{{\LARGE {\sf ??}}}}

\newcommand{\coment}[1]{\hfill\\ \Big[{\bf Comentario Privado:} #1\Big]}
\newcommand{\preg}[1]{\hfill\\ \BrickRed{{\bf Pregunta:} #1}}
\newcommand{\conjet}[1]{\hfill\\ \OliveGreen{{\bf Conjecura:} #1}}

\newcommand{\pendiente}{\BrickRed{{\sc Pendiente}}}
\newcommand{\verifpendiente}{\BrickRed{{\sc Verificación pendiente}}}


%--------------------------------------------------------------------------

\DeclareMathAlphabet{\mathpzc}{OT1}{pzc}{m}{it}


\title[]{Lógica computacional}
\subtitle{Tema: Tableaux}
\institute[UNAM-FC]{Facultad de Ciencias\\ 
Universidad Nacional Aut\'onoma de M\'exico}
\author{ Pilar Selene Linares Ar\'evalo}
\date[]{ \footnotesize{abril 2018}
\newline{\tiny{Material desarrollado bajo el proyecto UNAM-PAPIME PE102117.}}}
 

\beamerdefaultoverlayspecification{<+->}
 
\titlegraphic{\includegraphics[width=16mm]{fc2.png}}
 

\begin{document}

\begin{frame}
\titlepage 
\end{frame}
\note{}

\frame{\titulos{Introducción}{}
Los {\bf tableaux} o \emph{tablas semánticas} son un método de demostración por refutación: dado un 
conjunto de premisas~$\Gamma$ y una conclusión~$\varphi$, para mostrar que $\G\models\vp$ se debe demostrar que el conjunto $\Gamma\cup \{ \lnot \varphi \}$ no tiene un modelo. \\
\espc
\pause
Este método se basa en la semántica más que en la sintáxis de las fórmulas. Sin
embargo, se requiere de una clasificación de las fórmulas basada en sintaxis.
}


\frame{\titulos{Esquemas}{}
Aunque el método fue introducido en los años 1950 de manera independiente por Hintikka y Beth, fue Smullyan quien estandariza la notación y realiza la clasificación de las fórmulas en literales y cuatro tipos identificando su esquema:
\espc
 \pause

\begin{block}{Literal}
Una {\bf literal} es una fórmula atómica o la negación de una fórmula atómica.
\end{block}

}

\frame{\titulos{Esquemas}{}
El conjunto de fórmulas que no son literales se clasifica en cuatro tipos, en cada tipo de fórmula se distinguen ciertas subfórmulas necesarias posteriormente:
\begin{block}{Tipo $\alpha$}
\textbf{Tipo $\boldsymbol{\alpha}$}: si la fórmula es conjuntiva o es equivalente
   a una fórmula conjuntiva. \\
   Una fórmula $\chi \in \form$ es tipo $\alpha$, si tiene alguna de las
   siguientes formas:
   \begin{enumerate} 
   \item  $\chi = \vp \land \psi$, con subfórmulas $\alpha_{1} = \vp$ y
     $\alpha_{2} = \psi$.

   \item  $\chi = \lnot(\vp \lor \psi)$, con subfórmulas 
   $\alpha_{1} = \lnot \vp$ y  $\alpha_{2} = \lnot \psi$.

   \item   $\chi = \lnot(\varphi \imp \psi)$, con subfórmulas  
   $\alpha_{1} = \vp$ y  $\alpha_{2} = \lnot \psi$.
   \end{enumerate}
\end{block}

}


\frame{\titulos{Esquemas}{}

\begin{block}{Tipo $\beta$}
\textbf{Tipo $\boldsymbol{\beta}$}: si la fórmula es disyuntiva o es equivalente
   a una fórmula disyuntiva. \\
   Una fórmula $\chi \in \form$ es tipo $\beta$, si tiene alguna de las
   siguientes formas:
   \begin{enumerate}
   \item  $\chi = \vp \lor \psi$, con subfórmulas $\beta_{1} = \varphi$ y
    $\beta_{2} = \psi$.

   \item $\chi = \lnot(\vp \land \psi)$, con subfórmulas  
    $\beta_{1} = \lnot \varphi$ y $\beta_{2} = \lnot \psi$.

   \item $\chi = \vp \imp \psi$, con subfórmulas $\beta_{1} = \lnot \vp$
    y $\beta_{2} = \psi$.

   \item $\chi = \varphi \leftrightarrow \psi$, con subfórmulas 
   $\beta_{1} = \varphi \land \psi$ y $\beta_{2} = \lnot \vp \land \lnot \psi$.

   \item $\chi = \lnot ( \varphi \leftrightarrow \psi)$, con subfórmulas  
   $\beta_{1} = \lnot \varphi \land  \psi$ y $\beta_{2} = \vp\land \lnot \psi$.
   \end{enumerate}
\end{block}
}


\frame{\titulos{Esquemas}{}


\begin{block}{Tipo $\gamma$}
\textbf{Tipo $\boldsymbol{\gamma}$}: si la fórmula está cuantificada universalmente 
  o es equivalente a una fórmula cuantificada universalmente. \\
  Una fórmula $\varphi$ es de tipo $\gamma$, si tiene alguna de las 
  siguientes formas:
  \begin{enumerate}
   \item  $\chi = \forall x \varphi$.
   \item  $\chi = \lnot \exists x \varphi$.
    \end{enumerate}
\end{block}

\espc\begin{block}{Tipo $\delta$}
\textbf{Tipo $\boldsymbol{\delta}$}: si la fórmula está cuantificada 
  existencialmente o es equivalente a una fórmula cuantificada 
  existencialmente. \\
  Una fórmula $\varphi$ es tipo $\gamma$, si tiene alguna de las  siguientes 
  formas:
  \begin{enumerate}
   \item  $\chi = \exists x \varphi$.
   \item  $\chi = \lnot \forall x \varphi$.
  \end{enumerate}
\end{block}
}

\frame{\titulos{Tableaux}{}

Un {\bf tableau} es un árbol donde cada nodo está etiquetado con una fórmula y tiene a lo más dos hijos. \\
\espc
\pause
 Inicialmente el árbol únicamente contiene una rama en donde cada uno de los nodos que la conforman
están etiquetados con una de las fórmulas del conjunto $\Gamma \cup
\{\lnot \varphi \}$, si lo que queremos probar es que 
$\Gamma \models\varphi$; o bien, $\Gamma$ si lo que queremos probar es que este 
conjunto es insatisfacible.  \\
\espc
\pause
 La construcción del árbol la dictan las siguientes reglas de extensión: 
}


\frame{\titulos{Tableau}{}
\begin{exampleblock}{Regla de extensión $\al$: }

 \begin{center}
%   \boldmath
\tikzstyle{level 1}=[level distance=10mm]
   \tikz{\node{$\vp\land \psi$}
       child{node{$\vp$}
             child{node{$\psi$}}
            }
     ;
     }
\end{center}
\end{exampleblock}

\begin{exampleblock}{Regla de extensión $\beta$:}
 \begin{center}
%   \boldmath
\tikzstyle{level 1}=[level distance=15mm]
   \tikz{\node{$\vp\lor \psi$}
     child{node{$\vp$}}
     child{node{$\psi$}}
     ;
     }
\end{center}
\end{exampleblock}
}

\frame{\titulos{Tableau}{}
\begin{exampleblock}{Regla de extensión $\ga$:}
  \begin{center}
  \tikzstyle{level 1}=[level distance=12mm]
    \tikz{\node{$\fa x\vp$}
       child{node{$\vp[x:=t]$}
            }
     ;
     }
\end{center}
donde $t$ es cualquier término cerrado. Las fórmulas
tipo $\gamma$ siempre están activas.
\end{exampleblock}

}

\frame{\titulos{Tableau}{}
\begin{exampleblock}{Regla de extensión $\delta$: }
  \begin{center}
  \tikzstyle{level 1}=[level distance=12mm]
    \tikz{\node{$\ex x\vp$}
       child{node{$\vp[x:=c]$}
            }
     ;
     }
\end{center}
donde $c$ es una constante nueva, es decir una constante que no figura
antes en la rama.
\end{exampleblock}
}

\frame{\titulos{Tableaux}{}
Ejemplos de reglas de extensión aplicadas a diferentes fórmulas:
\bi
\item Extensión de f\'ormulas equivalentes tipo~$\al$: 
 \begin{center}
\tikzstyle{level 1}=[level distance=10mm]
  \tikz{\node{$\neg(\vp\lor \psi)$}
       child{node{$\neg\vp$}
             child{node{$\neg\psi$}}
            }
     ;
     }
\qquad \qquad 
  \tikz{\node{$\neg(\vp\to\psi)$}
       child{node{$\vp$}
             child{node{$\neg\psi$}}
            }
     ;
     }
\end{center}

\ei
}


\frame{\titulos{Tableaux}{}
Ejemplos de reglas de extensión aplicadas a diferentes fórmulas:
\bi

\item Extensión de f\'ormulas equivalentes tipo~$\beta$:
\begin{center}
\tikzstyle{level 1}=[level distance=15mm]
\tikzstyle{level 1}=[sibling distance= 20mm]
   \tikz{\node{$\neg(\vp\land \psi)$}
     child{node{$\neg \vp$}}
     child{node{$\neg \psi$}}
     ;
     }
 \quad
   \tikz{\node{$\vp\to \psi$}
     child{node{$\neg \vp$}}
     child{node{$\psi$}}
     ;
     }
\espc
\quad
   \tikz{\node{$\vp\iff \psi$}
     child{node{$\vp\land \psi$}}
     child{node{$\lnot\vp\land\lnot\psi$}}
     ;
     }
\quad
   \tikz{\node{$\lnot(\vp\iff \psi)$}
     child{node{$\neg \vp \land \psi$}}
     child{node{$\vp \land \lnot \psi$}}
     ;
     }
\end{center}
\ei
}


\frame{\titulos{Tableaux}{}
Ejemplos de reglas de extensión aplicadas a diferentes fórmulas:
\bi
\item Extensión de f\'ormulas equivalentes tipo~$\gamma$:
\begin{center}
  \tikzstyle{level 1}=[level distance=12mm]
    \tikz{\node{$\lnot\exists x\vp$}
       child{node{$\lnot \vp[x:=t]$}
            }
     ;
     }
\end{center}
\espc
\item Extensión de f\'ormulas equivalentes tipo~$\delta$:
\begin{center}
  \tikzstyle{level 1}=[level distance=12mm]
    \tikz{\node{$\lnot \fa x\vp$}
       child{node{$\lnot \vp[x:=a]$}
            }
     ;
     }
\end{center}
\ei
}

\frame{\titulos{Tableaux}{}

\begin{block}{Tableaux semánticos}
Sea $\Gamma = \{\varphi_{1},\varphi_{2},\ldots , \varphi_{n}\}$ un conjunto de
 fórmulas, un {\bf tableau} para $\Gamma$ denotado $\mathcal{T}(\Gamma)$ es una 
 extensión obtenida recursivamente como sigue:
 \begin{enumerate}
 \item El árbol formado por la rama cuyos nodos son $\varphi_{1}, \varphi_{2},
   \ldots , \varphi_{n}$ es un tableau para $\Gamma$.
\espc
 \item Si $\mathcal{T}_{1}$ es un  tableau para $\Gamma$, entonces la extensión
   que se obtiene después de aplicarle alguna de las reglas $\alpha,\;
   \beta,\;\gamma,\;\delta$ es un tableau para $\Gamma$.
\espc
   \item Son todos 
   % Únicamente son tableaux de $\Gamma$ aquellos que se obtienen
   %  utilizando las reglas mencionadas en 1 y 2.
 \end{enumerate}
\end{block}
}

\frame{\titulos{Tableaux}{}
\begin{block}{Tableau cerrado}
 Una rama $\rho$ de un tableau es \emph{cerrada } si tanto una literal (fórmula  atómica) $\varphi$ como su negación figuran en $\rho$. \\ 
Un tableau es {\bf cerrado} si todas sus ramas están cerradas.
\end{block}
\espc
\begin{block}{Tableau abierto}
Una rama $\rho$ de una tableau es \emph{abierta} si  no es cerrada. Un tableau se considera {\bf abierto} si contiene al menos una rama abierta.
\end{block}
}

\frame{\titulos{Tableaux}{}

\begin{exampleblock}{Correctud}
Si hay un tableau cerrado para $\Gamma$ entonces $\Gamma$ no tiene un modelo. \\
Es decir, si un tableaux con conjunto inicial $\G$ se cierra entonces dicho conjunto es insatisfacible.
\end{exampleblock}
\espc
\begin{exampleblock}{Completud refutacional}
Si $\Gamma$ no tiene un modelo entonces existe un tableau $\mathcal{T}(\Gamma)$
cerrado. \\
Es decir, si un conjunto inicial $\G$ es insatisfacible entonces 
existe una extensión $\Te(\G)$ cerrada.\\
\end{exampleblock}
}

\frame{\titulos{Tableaux}{}
\begin{exampleblock}{Corolario 1}
$\Gamma \models \varphi$ si y s\'olo si existe una extensión
 $\mathcal{T}(\Gamma \cup \{ \lnot\varphi \})$ cerrada.
\end{exampleblock}
\espc
\begin{exampleblock}{Corolario 2}
$\models\varphi$  si y s\'olo si existe una extensión $\mathcal{T}(\lnot
 \varphi)$ cerrada.
\end{exampleblock}
}

\frame{\titulos{Indecibilidad}{}
Sabemos que la lógica de predicados es indecidible, es decir, \textit{no existe un algoritmo que reciba una fórmula~$\vp$ y decida si ésta
es o no una fórmula universalmente válida}. En particular:
\espc
\bi
\item No existe un algoritmo que reciba un conjunto de premisas $\G$ y
una fórmula $\vp$ y decida si se cumple o no $\G\models\vp$.
\espc
\item Podemos preguntarnos entonces por qu\'e el método de tableaux no nos permitirá 
  decidir la consecuencia lógica en la lógica de predicados.
\ei

}

\frame{\titulos{Indecibilidad}{}
Consideremos el siguiente argumento: \\
$$ \fa x\ex y P(x,y),\; R(a,b)\;/\;\therefore \ex x R(x,b) $$ \\
\pause
es claramente correcto pues la conclusión es consecuencia de la 
segunda premisa, la primera no aporta información. \\
\pause
\espc
Aunque existe un tableau cerrado (utilizando la segunda premisa y la negaci\'on de la conclusión) también existe un tableau 
infinito (utilizando la primera premisa). \\
\espc
\pause

\setbeamercolor{postit}{bg=yellow!50!white}
\begin{beamercolorbox}{postit}
Por lo tanto si en el análisis de un argumento nos encontramos ante un tableau que no se ha cerrado no podemos asegurar algo.
\end{beamercolorbox}    
}


\frame{\titulos{Tableaux}{}

\setbeamercolor{postit}{bg=yellow!50!white}
\begin{beamercolorbox}{postit}
Los tableaux en lógica de 
predicados únicamente pueden ser utilizados  para trata de establecer  
\emph{insatisfacibilidad, validez} ó \emph{consecuencia lógica} y  por lo 
general \textbf{no} son un método apto para construir modelos como en   el caso 
de la lógica de proposiciones. 
\end{beamercolorbox}    
}

%
%\frame{\titulos{}{}
%\begin{block}{}
%
%\end{block}
%}
%
%\frame{\titulos{}{}
%\begin{exampleblock}{}
%
%\end{exampleblock}
%}
%
%\frame{\titulos{}{}
%
%}



\end{document}
